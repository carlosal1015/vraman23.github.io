\documentclass[12pt]{scrartcl}
\usepackage[utf8]{inputenc}
\usepackage[english]{babel}

\usepackage{skak}

\begin{document}
\title{The Vienna Game}
\author{Vishal Raman}
\maketitle
\begin{abstract}
\newgame

The Vienna Game starts after 
\mainline{1. e4 e5 2. Nc3 Nf6 3. f4}.
\begin{center}
\showboard
\end{center}
Black can accept or decline the Gambit, but accepting is immediately losing for Black.  Declining leads to playable lines with many traps.  Any mistakes or typos found are my own - kindly forward any concerns to my inbox.

\end{abstract}
\tableofcontents
\pagebreak
\section{The Vienna Gambit}
\subsection{Vienna Gambit Accepted}
If Black accepts the Gambit, after \mainline{3... e5xf4 4. e5} the position is immediately losing for Black($+.60$  Stockfish 11).
\begin{center}
\showboard
\end{center}
The knight has no choice but to retreat.   Moving the queen to pin the pawn is a crushing mistake ($+1.77$ Stockfish 11) after \variation {4. e5 Qe7 5. Qe2}, since the knight still has to retreat back to it's original square.

After the knight retreats, we can play moves like   \mainline{4... Ng8 5. Nf3 d6 6. d4 dxe5 7. Qe2 Bb4 8. Qxe5 Qe7 9. Bf4}, and we have regained the lost pawn and have a massive lead in development.  We also have plans to long castle and a threat on the $g7$ pawn, which would cripple the Black kingside.
\begin{center}
\showboard
\end{center}
\pagebreak
\subsection{Vienna Gambit Declined - 1}
Black must decline the gambit, but it must be done correctly.
\newgame
\fenboard{rnbqkb1r/pppp1ppp/5n2/4p3/4PP2/2N5/PPPP2PP/R1BQKBNR b KQkq f3 0 3}

The move \mainline{3... Nc6} is tempting as it defends the pawn and develops the knight, but after \mainline{4. fxe5 Ne5 5. d4 Nc6 6. e5 Ng8}, the knight is again forced to retreat and we reach the Vienna Gambit accepted position but with an additional passed pawn. 
\begin{center}
\showboard
\end{center}
 There are many fun lines from here, one such being \mainline{7. Nf3 d6 8. Bc4 dxe5 9. O-O}, with the plan to sacrifice the bishop on $f7$.  
 \begin{center}
\showboard
\end{center}

 From here, if Black takes on $d4$, we can play an Englund Gambit style move with \mainline{9... exd4 10. Ng5 dxc3 11. Bxf7}, which has massive threats on the open king.
\pagebreak
\subsection{Vienna Gambit Declined - 2}
\newgame
\fenboard{rnbqkb1r/pppp1ppp/5n2/4p3/4PP2/2N5/PPPP2PP/R1BQKBNR b KQkq f3 0 3}
Black can also decline the gambit with \mainline{3... d6}, but this also leads to a better position for White.  After \mainline{4. Nf3 Nc6 5. Bb5 Bd7 6. d3}, we have plans to capture on e5, trade the bishop on c6, and short castle which leads to an open f file for the rook.
\begin{center}
\showboard
\end{center}
\pagebreak
\subsection{Vienna Gambit Declined - Main Line}
\newgame
\fenboard{rnbqkb1r/pppp1ppp/5n2/4p3/4PP2/2N5/PPPP2PP/R1BQKBNR b KQkq f3 0 3}
The only possible move to respond is \mainline{3... d5}, counterattacking the pawn. After \mainline{4. fxe5 Nxe4}, we have a position with a couple possibilities, Nf3, d3, and the 3rd most popular Qf3.
\begin{center}
\showboard
\end{center}
The position we would like is for \mainline{5. Qf3 Nxc3 6. bxc3 } with plans to play d4, Bf3 Ke2 and O-O.
\begin{center}
\showboard
\end{center}

\end{document}
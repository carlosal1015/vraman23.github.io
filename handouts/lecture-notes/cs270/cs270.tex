\documentclass[12pt]{scrartcl}
\usepackage[sexy]{evan}
\usepackage{graphicx}

\usepackage{answers}
\Newassociation{hint}{hintitem}{all-hints}
\renewcommand{\solutionextension}{out}
\renewenvironment{hintitem}[1]{\item[\bfseries #1.]}{}
\declaretheorem[style=thmbluebox,name={Theorem}]{thm}

 %Sets
\newcommand{\N}{\mathbb{N}}
\newcommand{\Z}{\mathbb{Z}}
\newcommand{\F}{\mathbb{F}}
\newcommand{\Q}{\mathbb{Q}}
\newcommand{\R}{\mathbb{R}}
\newcommand{\C}{\mathbb C}
\newcommand{\T}{\mathbb T}
\renewcommand{\hat}{\widehat}
\newcommand{\<}{\langle}
\renewcommand{\>}{\rangle}

\let \phi \varphi
\let \mc \mathcal
\let \ol \overline
%From Topology
\newcommand{\cT}{\mathcal{T}}
\newcommand{\cB}{\mathcal{B}}
\newcommand{\cC}{\mathcal{C}}
\newcommand{\cH}{\mathcal{H}}

\newcommand{\supp}{\text{supp }}

\newcommand{\aint}{\mathrel{\int\!\!\!\!\!\!-}}
\let \grad \nabla

\begin{document}
\title{CS 270}
\author{Vishal Raman}
\thispagestyle{empty}
$ $
\vfill
\begin{center}

\centerline{\huge \textbf{CS 270: Combinatorial Algorithms} } 
\centerline{Professor: Prasad Raghavendra, Spring 2021}
\centerline{Scribe: Vishal Raman}
\end{center}
\vfill
$ $
\newpage
\thispagestyle{empty}
\tableofcontents
\newpage
%\maketitle
\section{January 20th, 2021}
\subsection{Intro to Gradient Descent}
Problem:  Given a function $f$, we wish to minimize $f$.  Gradient Descent takes the natural approach of going in the direction of locally steepest decrease.
\subsection{Convexity and Convex Functions}
\begin{definition}{Convexity} A set $K \subset \R^n$ is  convex if and only if for all $x, y \in K$, the line segment joining $x, y$ is also in $K$.  In other words, for $\lambda \in [0, 1]$, $$\lambda x + (1 - \lambda )y \in K.$$
\end{definition}

\begin{definition}[Convex Function] A function $f: \R^n \to \R$ is convex if for all $x, y$, $lambda \in [0, 1]$
$$f(\lambda x + (1 -\lambda )y) \le \lambda f(x) + (1 - \lambda) f(y).$$
\end{definition}
\begin{remark} Visually, if we draw the line segment between two points on the graph of the function, it should lie above the function.
\end{remark}
\begin{definition} A differentiable function $f: \R^n \to \R$ is convex if 
$$f(y) \ge f(x) + \langle \nabla f(x), y-x \rangle.$$
\end{definition}
Recall for $u, v \in \R^n$.
$$\left <u, v\right > = \sum_{i=1}^n u_i v_i,$$
and 
$$\|u\|_2 = \sqrt{\sum u_i ^2}.$$

\begin{definition} A twice differentiable function $f: \R^n \to \R$ is convex if $$H_f(x) \succcurlyeq 0,$$
in other words, the Hessian is positive-semidefinite.  
\end{definition}

\subsection{Unconstrainted Optimization}
We have the following problem: for $f: \R^n \to \R$ convex, we wish to find
$\min_{x \in \R^n} f(x)$.

Algorithm: for $x_0$, the initial point, $x_{t+1} \leftarrow x_t - \eta \nabla f(x_t)$, for a parameter $\eta$.  We output the average of the points $\frac{1}{T} \sum_{i} x_i.$

If we let $x^* = \text{argmin}_x f$, then $$f(x^*) \ge  f(x) + \langle \nabla f(x), x^* - x \rangle.$$
It follows that 
$$\langle -\nabla f(x) , x^* - x \rangle \ge f(x) - f(x^*) \ge 0.$$

\begin{thm} After $t$ steps(for appropriate $\eta$), 
$$\frac{1}{t} \sum_{i=1}^t f(x_i) \le f(x^*) + O(\frac{RL}{\sqrt{t}})$$
where $R = \|x_0 - x^* \|$ and $f$ is $L$-Lipschitz: for all $x, y$ $|f(x) - f(y)| \le L \|x - y\|$.
\end{thm}

\subsection{Constrained Optimization}
Given a convex set $K \subset \R^n$, we minimize a convex function $f$.  

Algorithm: We have an initial point $x_0$, $y_{t+1} = x_t - \eta \nabla f(x_t)$. Then $x_{t+1} = \pi_K(y_{t+1})$, where $\pi_K$ is a projection onto $K$, defined by 
$$\pi_K(y) = \argmin_{z \in K} \|z - y\|.$$

For convex sets, the same theorem holds, since 
$$\|y_{t+1} - x_*\| \ge \|\pi_K(y_{t+1}) - x^*\|.$$

\begin{definition} A function is $\alpha$-strongly convex if 
$$f(\lambda x + (1 - \lambda)y) \le \lambda f(x) + (1 - \lambda )f(y) - \alpha(\lambda(1 - \lambda))\|x - y\|^2.$$
\end{definition}

\begin{definition} A function is $\beta$ smooth if 
$$f(\lambda x + (1 - \lambda )y) \ge \lambda f(x) + (1 - \lambda )f(y) - \frac{\beta}{2} (\lambda(1 - \lambda))\|x - y\|^2.$$
\end{definition}

\begin{thm} After $t$ steps, we can get convergence(for appropriate $\eta$)
$$\frac{1}{t}\sum_i f(x_i) - f(x^*) \le e^{-\alpha t/\beta}.$$

We call $\beta/\alpha$ the condition number.  
\end{thm}
\pagebreak
\section{January 25th, 2021}
\subsection{Gradient Descent, Continued}
\begin{thm} After $t$ steps(for appropriate $\eta$), 
$$\frac{1}{t} \sum_{i=1}^t f(x_i) \le f(x^*) + O(\frac{RL}{\sqrt{t}})$$
where $R = \|x_0 - x^* \|$ and $f$ is $L$-Lipschitz: for all $x, y$ $|f(x) - f(y)| \le L \|x - y\|$.
\end{thm}
\begin{proof}
We will use the fact
$$2\left <a-c, b-c\right > = \|a-c\|^2 + \|b - c\|^2 - \|a - b\|^2.$$

Note that 
\begin{align*}
f(x_t) - f(x^*) &\le \left<\grad f(x_t), x_t - x^*\right> \\
&\le \left< \frac{x_t - x_{t+1}}{\eta}, x_t - x^*\right> \\
&\le \frac{1}{2\eta} \left (\|x_t - x_{t+1}\|^2 + \|x_t - x^*\|_2 - \|x_{t+1} - x^*\|^2\right ) \\
\end{align*}

It follows that $$\sum_{i=1}^t (f(x_i) - f(x^*)) \le \frac{1}{2\eta} \sum_{i=1}^t \|x_{i+1} - x_i\|^2 + \frac{1}{2\eta} \|x_1 - x^*\|^2.$$

Then, $\|x_{i+1} - x_i\|^2 = \|\eta \grad f(x_i)\|^2 = \eta^2 L^2$ so our expression
$$\sum_{i=1}^t (f(x_i) - f(x^*)) \le \frac{\eta R^2t}{2} + \frac{R^2}{2\eta}.$$
Choosing $\eta = \frac{R}{L\sqrt{t}}$ gives the desired bound.  
\end{proof}

\subsection{Projected Gradient Descent}

\end{document}

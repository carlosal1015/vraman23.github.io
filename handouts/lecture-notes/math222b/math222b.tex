\documentclass[12pt]{scrartcl}
\usepackage[sexy]{evan}
\usepackage{graphicx}

\usepackage{answers}
\Newassociation{hint}{hintitem}{all-hints}
\renewcommand{\solutionextension}{out}
\renewenvironment{hintitem}[1]{\item[\bfseries #1.]}{}
\declaretheorem[style=thmbluebox,name={Theorem}]{thm}

 %Sets
\newcommand{\N}{\mathbb{N}}
\newcommand{\Z}{\mathbb{Z}}
\newcommand{\F}{\mathbb{F}}
\newcommand{\Q}{\mathbb{Q}}
\newcommand{\R}{\mathbb{R}}
\newcommand{\C}{\mathbb C}
\newcommand{\T}{\mathbb T}
\renewcommand{\hat}{\widehat}
\newcommand{\<}{\langle}
\renewcommand{\>}{\rangle}


\let \phi \varphi
\let \mc \mathcal
\let \ov \overline
\let \weakto \rightharpoonup 
%From Topology
\newcommand{\cT}{\mathcal{T}}
\newcommand{\cB}{\mathcal{B}}
\newcommand{\cC}{\mathcal{C}}
\newcommand{\cH}{\mathcal{H}}

\newcommand{\supp}{\text{supp }}

\newcommand{\aint}{\mathrel{\int\!\!\!\!\!\!-}}
\let \grad \nabla
\let \p \partial
\let \tensor \otimes


\begin{document}
\title{Math 222b}
\author{Vishal Raman}
\thispagestyle{empty}
$ $
\vfill
\begin{center}

\centerline{\huge \textbf{Math 222b Lecture Notes}}
\centerline{\Large \textbf{Partial Differential Equations II} } 
\centerline{Professor: Maciej Zworski, Spring 2021}
\centerline{Scribe: Vishal Raman}
\end{center}
\vfill
$ $
\newpage
\thispagestyle{empty}
\tableofcontents
\newpage
%\maketitle
\section{January 19th, 2021}
\subsection{Review of Sobolev Spaces}
\begin{definition} Given $u \in \mc D'(U)$ for $U \subseteq \R^n$ open: that means that $u : C_c^\infty(U) \to C$ and for every compact set $K \subset \subset U$, $\exists C, N$ for all $\phi \in C_0^\infty(K)$ such that
$$|u(\phi)| \le C \sup_{|\alpha| \le N} |\partial^\alpha \phi|.$$ 
\end{definition}

Examples:
\begin{itemize}
\item Take $U =(0, 1)$ and take $u = \sum_{\N} \delta_{1/n}$, where $\delta_{1/n}(\phi) = \phi(1/n)$.  
\item Take $u \in L_{\text{loc}}^1(U)$, where $u(\phi) = \int u\phi$.  Differentiation is defined formally though integration by parts as $\partial^\alpha u(\phi) = (-1)^{|\alpha|} u(\partial^\alpha \phi).$
\end{itemize}

\begin{definition} The Sobolev spaces $W^{k, p}(U) = \{u \in L_{loc}^1(U) : \partial^\alpha u \in L^p(U), \forall |\alpha | \le k\}$, for $k \in \N_0$, $1 \le p \le \infty$.  Note that differentiation is in the sense of distributions.  We write $H^k(U) = W^{k, 2}(U)$, which are Hilbert spaces with the inner product$$\langle u, v\rangle = \sum_{|\alpha| \le k} \int_U \partial^\alpha u \overline{\partial^{\alpha} v}.$$
\end{definition}

\begin{definition} $W_0^{k, p}(U) = \overline{C_c^\infty(U)}$, where the closure is with respect to the $W^{k, p}$ norm.
\end{definition}

\begin{thm}[Approximation] For $U \subset \subset \R^n$, 
$$\overline{C^\infty(U) \cap W^{k, p}(U)} = W^{k, p}(U)$$
where the closure is with respect to the $W^{k, p}$.

If $\partial U \in C^1$, then we can improve up to 
$$\overline{C^\infty(\overline{U}) \cap W^{k, p}(U)} = W^{k, p}(U)$$
\end{thm}

\begin{thm}[Extension] If $U \subset \subset \R^n$
 and $\partial U \in C^1$, for $U \subset \subset V \subset \subset \R^n$, there exists $E: W^{1, p}(U) \to W^{1, p}(\R^n)$ such that $Eu\vert_{U} = u$ and the $\supp u \subset \subset V$.
 
We can extend this to $W^{k, p}$ if the boundary is $C^k$.
\end{thm}

\begin{thm}[Traces] For $U \subset \subset \R^n$ with $\partial U \in C^1$, there exists $T: W^{1, p}(U) \to L^p(\partial U)$ which is linear and boundary such that for $u \in C(\overline{U}) \cap W^{1, p}$ $Tu = u\vert_{\partial U}$.
\end{thm}

\begin{example} For $U \subset \subset \R^n$, $\partial U$ bounded, $$H_0^1(U) = \{u \in H^1 : Tu = 0 \in L^2(\partial U)\}.$$

The converse of showing $Tu = 0$ implies $H_0^1$ is the more difficult one.  
\end{example}

\subsection{Fourier Transform}
We first review the Fourier Transform.  We define the Schwartz space: $$\mc S = \{\phi \in C^\infty(\R^n) : x^\alpha \partial^\beta \phi \in L^\infty \forall \alpha, \beta \in \N^n\}.$$

For $\phi \in \mc S$, we define 
$$\hat{\phi} (\xi) = \int \phi(x) e^{-ix \cdot \xi}\, dx.$$

Note that $\mc F$, the Fourier transform is invertible on $\S$.  The key properties of the fourier transform are
$$\mc F (1/i \partial x \phi) = \xi \mc F \phi, F(x \phi) = -1/i \partial_{\xi} \mc F \phi.$$
We also have 
$$\mc F^{-1} = \frac{R\mc F}{(2\pi)^n}, R\phi(x) = \phi(-x).$$

We define $\mc S'$ onto $\C$ so that for $u \in \mc S'$, there exists $C, N$ such that 
$$|u(\phi)| \le C \sup_{|\alpha|, |\beta| \le N} |x^\alpha \partial^\beta \phi|.$$

Note that $\mc S' \subset \mc D'$.  

\begin{definition} $\mc F: \mc S' \to \mc S'$ by $\hat{u}(\phi) = u(\hat{\phi})$.
\end{definition}
Examples:
\begin{itemize}
\item $\hat{\delta_0}(\phi) = \delta_0(\hat{\phi}) = \hat{\phi}(0) = \int \phi = 1(\phi).$
\item Take $\R^2$ and consider $u(x) = \frac{1}{|x|}$.  This function is in $L_{loc}^1$.  If we multiply by $(1 + |x|)^{-2} u \in L^1(\R^n)$, it follows that $u \in \mc S'$, since 
$$|u(\phi)| = \left |\int (1 + |x|)^{-2} u (1 + |u|)^2 \phi\right | \le C \sup (1 + |x|)^2 \phi.$$

Now, we compute $\hat{u} \in \mc S'$.  Since $\mc F$ is continuous on $\mc S'$, we approximate $u$ and hope the result converges to the desired result.  Define $u_\epsilon \to u$ in $\mc S'$ for $u_\epsilon \in L^1$.  

Try $u_\epsilon(x) = \frac{e^{-\epsilon |x|^2/2}}{|x|} \in L^1$ for $\epsilon > 0$.  We want to calculate $\hat{u}_\epsilon$ and take the limit as $\epsilon \to 0^+$.   We can evaluate the integral by converting to polar coordinates and completing the square.  Unfortunately, it reduces to an integral that is too hard, but we will learn asymptotics of the integral as $\epsilon \to 0$.  We find that $\hat{u}(\xi) = 2\pi/|\xi|$.

We can approach this differently.  Note that $u = 1/|x|$ is homogeneous: $u(tx) = t^a u(x)$ for $t > 0$, for functions.  For distributions, we have that for $\phi \in S$, $u(\phi(\cdot / t) t^{-n}) = t^a u(\phi)$ for $t > 0$.  For the Fourier Transform, if $u \in \mc S'(\R^n)$ is homogeneous of degree $a$, then $\hat{u}$ is homogeneous of degree $-n-a$.  It follows that our Fourier transform is of degree $-1$.  

Furthermore, note that $1/|x|$ is spherically symmetric, and the Fourier transform preserves spherical symmetry(note that the Jacobian factor for rotations is $1$).  It follows that the fourier transform is also spherically symmetric.  It follows that 
$$\mc F(1/|x|) = C/|\xi| + \sum_{|\alpha \le N| }c_\alpha \delta_0^{(\alpha)},$$
but delta terms have too much homogeneity.  
\end{itemize}

\pagebreak
\section{December 21st, 2021}
\subsection{Plancherel's Theorem}
Recall that the Fourier transform is an isomorphism on $\mc S$ - it is a bounded linear operator whose inverse is also bounded.  

Note that 
$$\int \hat{u}(\xi) \ov{\hat{\phi}(\xi)}d\xi = \iiint u(x)\ov{\phi(y)} e^{-i(x-y)\xi}\,dxdyd\xi$$

In the sense of distributions, $\int e^{-i(x-y)\xi}\,d\xi = (2\pi)^n \delta(x-y).$
Hence,
$$\iiint u(x)\ov{\phi(y)} e^{-i(x-y)\xi}\,dxdyd\xi = (2\pi)^n \int u(x)\ov{\phi(x)}\,dx.$$

For $u, \phi \in \mc S$, we have the following: $$\< \hat{u}, \hat{\phi}\> = (2\pi)^n \< u, \phi \>.$$

This implies that 
$$\|\hat{u}\|_2 = (2\pi)^{n/2} \|u\|_{2}, u \in \S.$$
If $u_n \to u$ in $L^2$ then $u_n \to u$ in $\mc S'$ by the Cauchy-Schwartz inequality.  It follows that $\hat{u_n} \to \hat{u}$ in $\mc S'$ but our formula shows that $\hat{u}$ is in $L^2$.  Hence, $\mc F: L^2 \to L^2$ and for $u, v \in L^2$, $\<\hat{u}, \hat{v} \> = (2\pi)^n \<u, v\>$.

Recall last time, we were finding the Fourier transform of $u(x) = 1/|x|$ in $\R^2$.    For $u \in S'(\R^n)$ homogeneous of degree $a$, $\hat{u} \in \mc S'(\R^n)$ is homogeneous of degree $-n-a$.  In our example, It follows that $\hat{u}(\xi)$ is homogeneous of degree $-1$.  We also observed that $u$ is invariant under rotations so it follows that $\hat{u}$ is invariant under rotations.  

A function is homogeneous of degree $-1$ if $v(k\theta) = \frac{a(\theta)}{r}.$  Since our function is invariant under rotations, $\hat{u}(\xi) = \frac{c}{|\xi|}$ away from zero.  It follows from our previous argument that $\hat{u}(\xi) = \frac{c}{|\xi|}$ since $\delta$ terms have homogeneity of at least $-2$.  

Note that $\<u, \phi\> = (2\pi)^2 \<\hat{u}, \hat{\phi}\>$ and we find $\hat{u}$ by choosing an appropriate $\phi$.

\begin{align*}
\int_{\R^2} \frac{\phi(x)}{|x|}\,dx &= \int_0^{2\pi} \int_{0}^\infty \phi(r) \, drd\theta \\
&= 2\pi \int_{0}^\infty \phi(r)\,dr.
\end{align*}
Choosing $\phi(r) = e^{-r^2/2}$, we find that the integral is $(2\pi)^{3/2}$.  

Evaluating the other side,  
$$\hat{\phi}(\xi) = \int_{\R^2} e^{-|x|^2/2 - ix\cdot \xi}\,dx = \int e^{-\frac{1}{2}(x + i\xi)^2 - \frac{1}{2} |\xi|^2} = 2\pi \int e^{-|\xi|^2/2} = (2\pi)^{5/2}.$$

It follows that $c = 2\pi$.
\subsection{Fourier Characterization of $H^k$ spaces}
\begin{thm} $H^k(\R^n) = \{u \in S'(\R^n) : (1 + |\xi|^2)^{k/2} \hat{u} \in L^2\}.$
\end{thm}
\begin{proof}
Suppose that $\partial^\alpha u \in L^2$ for $|\alpha| \le k$.  We know that $\|u\|_{2} = (2\pi)^{-n/2}\|\hat{u}\|.$  It follows that $\hat{\partial^\alpha u} \in L^2$.  Note that $\hat{\partial^\alpha u} = i^{|\alpha|} \\xi^{\alpha} \hat{u} \in L^2$ for all $|\alpha| \le k$.

Hence,
$$(1 + |\xi|^2)^{k/2} \le C_{n, k} \sup_{|\alpha| \le k} |\xi^\alpha|.$$

So it follows that $(1 + |\xi|^2)^{k/2} \hat{u} \in L^2$.

Now, suppose $(1 + |\xi|^2)^{k/2} \hat{u} \in L^2$.  It follows that $|\xi^\alpha| \le C_{k, \alpha}(1 + |\xi|^2)^{k/2}$ for $|\alpha| \le k$.  Hence $\xi^\alpha\hat{u} \in L^2$ so it follows that $\partial^{\alpha} u \in L^2$ by Plancherel's Theorem.  

\end{proof}

\begin{remark}We use the notation $\<\xi\> = (1 + |\xi|^2)^{1/2}$.
\end{remark}

Note that the definition does not require $k \in \N$.  
\begin{definition} $H^s(\R^n) = \{u \in \mc S' : \<\xi \>^s \hat{u} \in L^2\}, s \in \R$.
\end{definition}

\begin{thm} Suppose $u \in H^s(\R^n)$ and $s > \frac{1}{2}$.  Then $v(y) = u(0, y), y \in \R^{n-1}$ satisfies $v \in H^{s - 1/2} (\R^{n-1})$.
\end{thm}
\begin{remark} We should define $Tu(y) = u(0, y)$ for $u \in \mc S$.  Then $T: H^s(\R^n) \to H^{s - 1/2}(\R^{n-1})$ if $s > 1/2$.  
\end{remark}
\begin{proof}
Take $u \in \mc S$.  We wish to show that $\|v\|_{H^{s - 1/2}(\R^{n-1})} \le C\|u\|_{H^s(\R^n)}$.

Note that $$\hat{v}(\eta) = \int_{\R^{n-1}} u(0, y) e^{-y \cdot \eta} \,dy$$
and by the Fourier Inversion Formula
$$u(0, y) = (2\pi)^{-n}\int_{\R^n} \hat{u}(\xi_1, \xi') e^{iy \cdot \xi'}\,d\xi_1 d\xi',$$
so it follows that 
\begin{align*}
\hat{v}(\eta) &= (2\pi)^{-n}\int_{\R^{n-1}}\int_{\R^n} \hat{u}(\xi_1, \xi') e^{-iy \cdot (\eta - \xi')}\,d\xi dy \\
&= (2\pi)^{-n}\int_{\R^{n}}\int_{\R^{n-1}}\hat{u}(\xi_1, \xi') e^{iy \cdot (\xi' - \eta)}\, dy d\xi\\
&= (2\pi)^{-1} \int_{\R^n} \hat{u}(\xi_1, \xi') \delta_{\xi' = \eta}d\xi \\
&= (2\pi)^{-1} \int_{\R} \hat{u}(\xi_1, \eta) d\xi_1.
\end{align*}
 
Note that up to constants
 $$\|v\|_{H^{s-1/2}}^2 = \int_{\R^{n-1}} \<\eta\>^{2s-1} |\hat{v}(\xi)|^2\, d\eta = \int_{\R^{n-1}} \<\eta\>^{2s-1} \left |\int \hat{u}(\xi_1, \eta)\, d\xi_1\right |^2d\eta.$$
  
 Then,
 \begin{align*}
 \int_{\R^{n-1}} &\<\eta\>^{2s-1} \left |\int \hat{u}(\xi_1, \eta)\, d\xi_1\right |^2d\eta \\
 &= \int \<\eta\>^{2s-1} \left |\int \hat{u}(\xi, \eta)(1 + |\xi_1|^2 + |\eta|^2)^{s/2} (1 + |\xi_1|^2 + |\eta|^2)^{-s/2}d\xi_1  \right |^2d\eta \\
 & \le \int \<\eta\>^{2s-1} \int |\hat{u}(\xi_1, \eta)|^2 (1 + |\xi_1|^2 + |\eta|^2)^s d\xi_1 \int (1 + |\xi_1|^2 + |\eta|^2)^{-s} d\xi_1\, d\eta \\
&\le \int \<\eta\>^{2s-1} \<\eta\>^{-2s+1}  \int |\hat{u}(\xi_1, \eta)|^2 (1 + |\xi_1|^2 + |\eta|^2)^s d\xi_1\int (1 + u^2)^{-s} du d\eta\\
& = \int |\hat{u}(\xi)|^2 \<\xi\>^{2s} d\xi = \|u\|_{H^s}^2,
 \end{align*}
 
since
$$\int |\hat{u}(\xi_1, \eta)|^2 (1 + |\xi_1|^2 + |\eta|^2)^s d\xi_1d\eta = \int |\hat{u}(\xi)|^2 \<\xi\>^{2s}d\xi.$$

\end{proof}
\pagebreak
\section{January 26th, 2021}
\subsection{Sobolev Spaces, continued}
Recall, we have $U \subset \R^n$ open.  We typically assume $U$ is bounded and $\partial U \in C^1$.  For these spaces, we define $$W^{k, p}(U) = \{u \in \mc D' : \partial^{\alpha} u \in L^p(U), |\alpha| \le k\}.$$

Recall the extension property: there exists a map $E: W^{1, p}(U) \to W^{1, p}(\R^n)$ such that $Eu\vert_U = u$ and $u = 0$ for $|x| > R$ for some $R$ with $U \subset \subset B(0, R)$.

We also consider the $H^s(\R^n)$, the fractional Sobolev spaces: $\{u \in \mc S'(\R^n) : \<\xi \>^s \hat{u} \in L^2\}.$  This is a Hilbert space with the norm 
$$\|u\|_{H^s}^2 = \int \<\xi\>^{2s} |\hat{u}(\xi)|^2\,d\xi.$$

Last time, we showed that If we have $u \in H^s(\R^n)$ and $s > 1/2$, then $v(y) : u(0, y)$, $y \in \R^{n-1}$ satisfies $v \in H^{s - 1/2}(\R^{n-1})$.  Today, we will show that $H^s(\R^n) \subset C_0(\R^n)$ if $s > n/2$, where $C_0$ denotes continuous functions vanishing at infinity.  This means that there exists $T: H^s(\R^n) \to H^{s - 1/2}(\R^{n-1})$ such that for $u \in \mc S$, $Tu(y) = u(0, y)$.  

\begin{thm} $H^s(\R^n) \subset C_0(\R^n)$ if $s > n/2$.
\end{thm}
\begin{proof}
We first prove that if $\<\xi\>^s \hat{u} \in L^2, s > n/2$ then $\hat{u} \in L^1(\R^n)$.  

$$\int_{\R^n} |\hat{u}|d\xi = \int_{\R^n} \<\xi\>^{-s}\<\xi\>^2 |\hat{u}| d\xi \le \|\<\xi\>^{-s} \|_2 \|u\|_{H^s}.$$
The first term is finish precisely when $s > n/2$ [exercise: convert to polar coordinates]. This implies that $u \in L^\infty(\R^n)$, following from the Fourier Inversion formula.  

We know that $x \mapsto \hat{u}(\xi)e^{ix \xi}$ is continuous so it follows that $x \mapsto u(x)$ is continuous by the dominated convergence theorem.  Finally $u(x) \to 0$ as $|x| \to \infty$ by the Riemann-Lebesgue lemma: if $\hat{u} \in L^1(\R^n)$, then $u(x) \to 0$ as $|x| \to \infty$.
\begin{proof}
Recall $\mc S(\R^n) \subset L^1(\R^n)$ is dense.  Taking $v \in L^1$, taking $v_R = v(x) 1_{B(0, R)}(x)$.  Then $v_{R} \to v$ y the dominated convergence theorem.  Now take $\phi \in C_c^\infty$ with $\phi \ge 0$, $\int \phi = 1$ wth $\phi_\epsilon(x) = \frac{1}{\epsilon^n} \phi(x/\epsilon)$.  Taking $v_{R, \epsilon} = v_R * \phi_\epsilon \in C_c^\infty(\R^n)$ and $v_R*\phi_\epsilon \to v_R$ in $L^1$ as $\epsilon \to 0$.

Hence, we can take $v \in \mc S$ so that $\|\hat{v} - \hat{u} \|_{L^1} < \epsilon/2$.  Now, $|v(x)| < \epsilon/2$ if $|x| > R$, hence $$|u(x)| \le |u(x) - v(x)| + |v(x)| < C\epsilon + \epsilon/2$$
which goes to $0$ as we send $\epsilon \to 0$.
\end{proof}
\end{proof}

\subsection{Gagliardo-Nirenberg-Sobolev(GNS) Inequalities}
\begin{theorem}
If $1 \le p < n$ and we define $p^* = \frac{np}{n-p}$, then there exists $C = C(p, n)$ so that for all $u \in C_c^\infty(\R^n)$, $$\|u\|_{L^{p^*}} \le C \|\grad u\|_p.$$
\end{theorem}
\begin{remark} We can find the value of $p^*$ without doing the computation through scaling.Take $u_\lambda(x) = u(\lambda x)$.  We have that $\|u_\lambda \|_{p^*} \le C \| \grad (u_\lambda)\|_p$.  Then, evaluate both sides and compare the exponent on $\lambda$.

Note that the result is not true for $p = n > 1$.  It is true for $p = n = 1$.  
\end{remark}
\begin{theorem}[Morrey's Inequality] For $n < p \le \infty$, there exists $C = C(p, n)$ such that for $u \in C^1(\R^n)$, we have $$\|u\|_{C^\gamma(\R^n)} \le C(\|u\|_p + \|\grad u \|_p),$$
where $\gamma = 1 - \frac{n}{p}$, where
$$\|u\|_{C^\gamma(\R^n)} = \sup |u| + \sup_{x \ne y} \frac{|u(x) - u(y)|}{|x - y|^\gamma}.$$
\end{theorem}

\begin{thm}[General Formulation]
Take $U \subset \subset \R^n$ with $\partial U \in C^1$.  Take $n \in W^{k, p}(U)$.  
\begin{itemize}
\item if $k < n/p$, then $u \in L^q(U)$ where $1/q \ge 1/p - k/n$ and $\|u\|_{L^q(U)} \le C \|u\|_{W^{k, p}}$.
\item $k > n/p$, then $u \in C^{k - [n/p] - 1, \gamma}(\overline{U})$ where $\gamma = [n/p] + 1 - n/p$ if $n/p \not \in \N$ and $1 - \delta$ for all $\delta > 0$ if $n/p \in \N$.
\end{itemize}
\end{thm}
\subsection{Compactness}
\begin{definition} Let $B$ be a Banach space.  A subset $K \subset B$ is compact if for every sequence $\{u_n\} \subset K$ such that $\|u_n\|_B \le C$, there exists a convergence subsequence $u_{n_k} \to u \in B$.
\end{definition}
\begin{remark} If $\{u : \|u\|_B \le 1\} \subset B$ is compact, then $B$ is finite dimensional.   We can have a space $B' \subset B$ and $\{u \in B' : \|u\|_{B'} \le 1\}$ compact in $B$.  If we have a sequence $\{u_n\} \subset B'$ and $\|u_n\|_{B'} \le C$ then there exists $n_k$, $u \in B$ such that $\|u_{n_k} -u\|_B \to 0$.
\end{remark}

We will take $B = L^q(U)$ where $1 \le q < p^*$ and $B' = W^{1, p}(U)$.
\begin{thm}[Rellich-Kondrachov] The unit ball in $W^{1, p}(U)$ is compact in $L^q(U)$ for \textbf{bounded} $U$.
\end{thm}
\pagebreak
\section{January 28th, 2021}
Recall the GNS inequality: if $1 \le p < n$, $p^* = \frac{np}{n-p}$, there exists $C = C(n, p)$ for all $u \in C_c^{\infty}(\R^n)$ so that $\|u|_{L^{p^*}} \le C \|\grad u\|_{L^p}$.

If $U \Subset \R^n$, $\partial U \in C^1$, then there exists $C= C(n, p, U)$ such that $L^{q}(U) \supset W^{1, p}(U)$ for $1 \le q \le p^*$.
\subsection{Compactness}
Suppose $B$ is a Banach space and $B' \subset B$ another Banach space.  We say that the inclusion $B' \subset B$ is compact if bounded sets in $B'$ are precompact in $B$.  In other words, for a sequence $\{u_n\} \subset B'$ with $\|u_n\|_{B'} \le M$, there exists a subsequence $u_{n_k}$ and $u \in B$ such that $u_{n_k} \to u$ in $B$.

\begin{example} Take $B = C([-1, 1])$, $B' = C^1([-1, 1])$ with the supremum norm on $B$ and $\|u\|_{B'} = \sup_{|x| \le 1} (|u(x)| + |u'(x)|)$.

The inclusion is compact: if we have $\|u_n\|_{B'} \le C$, by the mean value theorem, $|u_n(x)| \le C$ and $|u_n(x) - u_{n}(y)| \le C|x - y|$.  By Arzela-Ascoli, there exists a subsequence $u_{n_k}$ and $u \in C$ so that $\|u_{n_k} - u\|_{C([-1, 1])} \to 0$.
\end{example}
\begin{example} In the previous example, take $u_n(x) = |x|1_{|x| > 1/n} + (\frac{nx^2}{2} + \frac{1}{n})1_{|x| \le 1/n}$.

Then $u_n \in C^{1}[-1, 1]$ and $\|u_n\|_{C^1[-1, 1]} \le 2$.   We can take a subsequence $n_k = k$ and $u(x)= |x| \in C[-1, 1] \setminus C^1[-1, 1]$ where $u_{n_k} \to u \in C$.
\end{example}

Given a Banach space $B$, we have the dual space $B^* = \{\text{linear } u:  B \to \C | \forall x \in B, |u(x)| \le C \|x\|_B\}$.  The is also a Banach space.  

\begin{thm}[Banach-Alaoglu] Suppose $\|u_n\|_{B^*} \le M$.  Then, there exists a subsequence $u_{n_k}$ and $u \in B^*$ such that for all $x \in B$, $u_{n_k}(x) \to u(x)$.  
\end{thm}
\subsection{Rellich-Kondrachov}
\begin{thm}[Rellich-Kondrachov] If $U \Subset \R^n$, $\partial U \in C^1$, then for $1 \le q < p^*$, $L^q(U) \supset W^{1, p}(U)$ is a compact inclusion.
\end{thm}
\begin{proof}
Take $p = 2$.  Then $p^* = \frac{2n}{n-2} > 2$.  First, suppose $\ol{U} \Subset B(0, R)$.  We can assume $R = 1$.  Suppose we have a sequence $\|v_n\|_{H^1(U)} \le 1$.  There exists a sequence $u_n \in H^1(\R^n)$ such that $u_n\vert_U = v_n$, $\|u_n\|_{H^1(\R^n)} \le 1$ and $\supp u_n \subset B(0, 1)$(this is the extension operator).

We have $u_n \in H^1(\R^n)$, $\|u_n\|_{H^1} \le 1$, $\supp u_n \subset B(0, R)$.  We want $n_k$, $u \in L^2(\R^n)$ such that $u_{n_k} \to u$ in $L^2$.  We claim that $u(x) = (2\pi)^{-n} \sum_{m \in \Z^n} \hat{u}(m) e^{im\cdot x}$ for $x\in B(0, 1) \subset [-\pi, \pi]^n$ with convergence in $\mc D'$.

Alternatively,
\begin{align*}
\int u(x) \ol{\phi(x)}\,dx &= (2\pi)^{-n}\sum_{m \in \Z^n} \hat{u}(m) \int \ol{\phi(x)} e^{im\cdot x}\,dx\\
&= (2\pi)^{-n} \sum_{m \in \Z^n} \hat{u}(m) \ol{\hat{\phi}(m)}.
\end{align*}
For $u, v\in L^2$, $\int u(x)\ol{v}(x)\,dx = (2\pi)^{-n} \sum_{m \in \Z^n} \hat{u}(m) \ol{\hat{v}(m)}$.

Recall the Poisson summation formula: for $n = 1, a \ne 0$, $$\sum_{k \in \Z} e^{ikax} = \frac{2\pi}{a}\sum_{k \in \Z} \delta(x - 2\pi k/a)$$
in the distributional sense.

Note that $(1 - e^{iax}) \sum_k e^{ikax} = \sum_{k} e^{ikax} - \sum_{k} e^{i(k+1) ax} = 0$.  We can rewrite this as $-2ie^{-iax/2} \sin{(ax/2)} \sum_k e^{ikax} = 0$.  Let $w(x) = \sum_{k} e^{ikax}$, so it follows that $\supp w \subset \{\frac{2\pi}{a}k\}_{k \in Z}$.  It follows that $w(x)$ is the sum of delta functions supported at $2\pi k/a$ for $k \in \Z$ up to constants.  

Furthermore, note that $w(x + 2\pi a) = w(x)$.  So it follows that the constants are independent of the index.  To find the constant, for some function, replace $\phi(\cdot)$ with $\phi(\cdot + x)$.  Then the right side is $c \sum_{k \in \Z} \phi(2\pi k/a + x)$.  Note that $\hat{\phi}(\cdot + x)(\xi) = e^{ix \xi} \hat{\phi}(\xi)$ .  It follows that the left hand side is $\sum_{k \in \Z} \hat{\phi}(ka) e^{ikax}$.  Now suppose $\supp \phi \in C_c^\infty((0, 2\pi/a))$.    Integrating both sides, the left side is $2\pi/a \hat{\phi}(0)$.  The right side is $c \int \phi(x)\,dx = c(a) \hat{\phi}(0)$.  Thus, $c = \frac{2\pi}{a}$.

The Poisson summation formula is more generally $\sum_{k \in \Z^k} e^{iak \cdot x} = (2\pi/a)^n \sum_{k \in \Z^d} \delta(x - 2\pi a/k)$.

Applying this to a $\phi$ gives our desired claim from earlier.  it follows that $\frac{1}{(2\pi)^n} \sum_{k \in \Z^n} \hat{u}(k)\ol{\hat{v}(k)} = \int u(x) \ol{v(x)}\,dx$ with $u, v \in L^2$, $\supp u, v \in [-\pi, \pi]^n$.

Note that for $u \in L^2$, we have the Plancherel formula,
$$\|u\|_2^2 = \int |u(x)|^2\,dx = \frac{1}{2\pi}^n \sum_{n \in \Z^n} |\hat{u}(m)|^2.$$

For $u \in H^1(\R^n)$ and $\supp u \subset B(0, 1)$, then $u \in L^2$ and $\partial^\alpha u \in L^2$, for all $|\alpha| = 1$, 
$$\|\partial^\alpha u\|_{L^2}^2 = (2\pi)^n \sum_{ m \in \Z^n} |\hat{\partial^\alpha u}(m)|^2 = (2\pi)^{-n} \sum_{m \in \Z^n} |m^\alpha\hat{u}(m)|^2.$$
Claim: Under these assumptions, $\|u\|_{H^1}^2 = \Theta( \sum_{m \in \Z^n} \<m\>^2 |\hat{u}(m)|^2)$
\end{proof}

\pagebreak
\section{February 2nd, 2021}
Recall the following:
\begin{itemize}
\item GNS inequality: For $U \Subset \R^n$, $\partial U \in C^1$, $1 \le p < n$, $p^* = \frac{np}{n-p} >p$,
$$\|u\_{p^*} \le C(\|u\|_p + \|\grad u\|_p).$$
\item R-K Theorem: For $1 \le p < n$, $1 \le q < p^*$, $$W^{1, p}(U) \subset L^q(U)$$
is compact: If we have $\{u_n\}\subset W^{1, p}(U)$ and $\|u_n\|_{W^{1, p}} \le C$, there exists a subsequence $u_{n_k}$, $u \in L^q$ such that $\|u_{n_k} - u\|_{q} \to 0$.
\end{itemize} 

\subsection{Rellich-Kondrachov, continued}
Last time, we considered the special case of $\{u_n\} \subset H^1(\R^n)$ such that $\supp u_n \subset B(0, R)$ and $\|u_n\|_{H^1} \le C$, which implies that there exists a subsequence $u_{n_k}$ and $u \in L^2(\R^n)$ such that $\|u_n-u\|_{L^2} \to 0$.  We continue the proof of the special case. 

\begin{proof}
Recall that we showed that if $u \in C_0^\infty((-\pi, \pi)^n)$, we can write $u(x) = (2\pi)^{-n} \sum \hat{u}(n) e^{in \cdot x}$.  Then
$$\int u(x)\ol{v(x)}\,dx = \frac{1}{(2\pi)^n}\sum \hat{u}(n) \ol{\hat{v}(n)}$$
and 
$$\int |\grad u(x)|^2\,dx = \frac{1}{(2\pi)^n} \sum |n|^2 |\hat{u}(n)|^2.$$

For $u \in H^1$ with $\supp u \in B(0, 1)$, $$\|u\|_{H^1}^2 = \frac{1}{(2\pi)^n} \sum \<n\>^2 |\hat{u}(n)|^2.$$


$$\|u_n\|_{H^1}^2 = \sum_{\ell \in \Z^n} \<\ell\>^2 |\hat{u_n}(\ell)|^2 \le C.$$
$$\|v\|_{L^2}^2 = \sum_{\ell \in \Z^n} |\hat{v}(\ell)|^2.$$
We want to show that there exists $n_k$ such that $\|u_{n_k} - u_{n_p}\|_{L^2} \to 0$ as $k, p \to \infty$.

We introduce an operator $\Pi_p u(x) = (2\pi)^{-n}\sum_{|\ell| \le p} \hat{u}(\ell) e^{i\ell \cdot x}$.  We can think of $\Pi_p: L^2([-\pi, \pi]^n) \to \C^{N_p}.$
$N^p$ can be found through combinatorial methods(left as an exercise)[should be $\binom{n+p}{p}$ or something like that].

We have the following estimate:
$$\|(I - \Pi_p)u\|_2 \le \<p\>^{-2} \|u\|_{H^1}^2.$$
This is because
 $$(2\pi)^{-n}\sum_{|\ell| > p} |\hat{u}(\ell)|^2 = (2\pi)^{-n}\sum_{|\ell| > p} \<\ell\>^{-2}\<\ell\>^{2}|\hat{u}(\ell)|^2\le \<p\>^{-2} \|u\|_{H^1}^2.$$
 
Now, we find the Cauchy subsequence.  
\begin{enumerate}
\item For all $p$, we have $\|\Pi_p u_n\|_{\C^{N_p}} \le \|u_n\|_2 \le \|u_n\|_{H^1} \le C$.  Then $\{|z| \le C\} \subset \C^{N_p}$ is compact.  It follows that we can choose subsequences $\{n_k^{p+1}\}\subset \{u_{k}^p\}$ such that $\Pi_p u_k^p$ converges and $\limsup_{k, \ell} \|u_{k}^p - u_\ell^p\| \le C\<p\>^{-2}$, which follows from the triangle inequality.   [let $u_a^b = u_{n_a^b}$]
\item We choose $n_k = n_k^k$.  It follows that $\limsup_{k, \ell \to \infty} \|u_{n_k} -u_{n_\ell}\|_2 = 0$, since 
\end{enumerate}
\end{proof}
\subsection{Morrey's Inequality}
\begin{thm}[Morrey's Inequality]  Suppose $u \in L^p(\R^n), \grad u \in L^p(\R^n)$ and $n < p \le \infty$.  Then there exists $u^* \in C^{0, \gamma}(\R^n)$, with $\gamma = 1 - \frac{n}{p}$ such that $u = u^*$ almost everywhere and $\|u^*\|_{C^0, \gamma} \le \|u\|_p + \|\grad u\|_{p}$.
\end{thm}
\begin{remark}
Recall $$\|u\|_{C^{0, \gamma}} = \sup |u(x)| + \sup_{x \ne y} \frac{|u(x)- u(y)|}{|x - y|^\gamma}.$$
\end{remark}

\begin{proof}
We use the Littlewood-Paley Decomposition.  
\begin{lemma}[Dyadic Partitions of Identity]  There exists a function $\psi_0 \in C_c^\infty(\R), \psi \in C_c^\infty(\R \setminus \{0\})$ such that $$\psi_0(\xi) + \sum_{j=0}^\infty \psi(2^{-j}|\xi|) = 1.$$
\end{lemma}
\begin{proof}
Choose $\phi_0 \in C_c^\infty((-1, 1))$ with $0 \le \psi_0 \le 1$ and $\phi_0(p) = 1$, $|p| \le 1/2$.  

Choose a new function $$\phi_1(p) = \sum_{j \in \Z} \phi_0(p - j) \ge 1.$$
Note that $\phi_1(p - k) = \phi_1(p)$ for $k \in \Z$.  Choose $\phi(p) = \frac{\phi_0(p)}{\phi_1(p)}.$

Then
$$\sum_{j \in \R} \phi(p - j) = \sum \frac{\phi_0(p - j)}{\phi_1(p - j)} = \frac{1}{\phi_1(p)} \sum \phi_0(p -j) = 1.$$

Define $\psi(r) = \phi(\frac{\log r}{\log 2})$ for positive $r$.  Notice that $\psi \in C_c^\infty((0, \infty)).$  This gives that 
$$\sum_{j \in \Z} \psi(2^{-j} r) = 1.$$

Define $\psi_0(r) = 1 - \sum_{j = 0}^\infty \psi(2^{-j} r)$.  Note that $\psi_0(r) = 1$ for $r < 1/2$ and $\psi_0(r) = 0$ for $r > 1$.  It follows that $\psi_0$ and $\psi$ satisfy the conditions. 
\end{proof}
We can extend the Dyadic Partitions of Identity in $\R^n$ in the natural way.  We then define the Littlewood-Paley Decomposition as 
$$u = \psi_0(D) u + \sum_{j=1}^\infty \psi(2^{-j} D)u$$
where for $a \in L^\infty(\R^n)$, $a(D)u = \mc F^{-1}(a(\xi)\hat{u}(\xi))$ where $D_x = 1/i \partial_x$ and $\hat{Du} = \xi \hat{u}$.
\end{proof}
\pagebreak
\section{February 4th, 2021}
\subsection{Morrey's Inequality, continued}
Recall the statement of the theorem.
\begin{thm}[Morrey's Inequality]  Suppose $u \in L^p(\R^n), \grad u \in L^p(\R^n)$ and $n < p \le \infty$.  Then there exists $u^* \in C^{0, \gamma}(\R^n)$, with $\gamma = 1 - \frac{n}{p}$ such that $u = u^*$ almost everywhere and $\|u^*\|_{C^0, \gamma} \le \|u\|_p + \|\grad u\|_{p}$.
\end{thm}
\begin{proof}
Recall for $u \in \mc S(\R)$, $\hat{D_{x_j} u}(\xi) = \xi_j \hat{u}(\xi)$ where $D_{x_j} = \frac{1}{i} \partial_{x_j}$.  

We define a \textbf{Fourier multipler} $a \in L^\infty(\R^n)$ so that $a(D)u = \mc{F}^{-1}(a(\xi) \hat{u}(\xi))$ for $u \in \mc S$.   Note that for $a \in L^\infty$, $\|a(D)u\|_{L^2} \le \sup |a| \|u\|_{L^2}$.  For $\psi \in \mc S(\R^n)$, if we take $u \in \mc S'$, then $\psi(D) u \in \mc S'$ , and $\psi(\xi) \hat{u}(\xi) \in \mc S'$.

Recall the Littlewood - Paley Decomposition. We had a lemma: there exists $\psi_0 \in C_c^\infty(\R)$ and $\psi \in C_c^\infty(\R \setminus \{0\})$ such that for all $\xi \in \R^n$, 
$$\psi_0(|\xi|) + \sum_{j=0}^\infty \psi(2^{-j} |\xi|) = 1.$$ 
Slightly abusing notation, we will write $\psi_0(\xi) = \psi_0(|\xi|)$ and $\psi(\xi) = \psi(|\xi|)$.  

The full L-P Decomposition is given as follows: given $u \in \mc S'$, $a = \psi_0(D) u + \sum_{j=1}^\infty \psi(2^{-j }D) u$.  We will write $h = 2^{-j}$ as a shorthand.
\begin{lemma} Suppose $\chi \in C_c^\infty(\R^n)$.  Then for $u \in \mc S(\R^n)$, $\|\chi(hD)u\|_{L^\infty} \le C h^{-n/p} \|u\|_{L^p}$ 
and $\|\chi(hD) u\|_{L^p} \le (2\pi)^{-n} \|\hat{\chi}\|_1 \|u\|_p$.
\end{lemma}
\begin{proof}


Recall the following inequalities
\begin{itemize}
\item Holder's Inequality: $\|fg\|_1 \le \|f\|_p\|g\|_q$ for $1/p + 1/q = 1$ for $1 \le p \le \infty$.  
\item Minkowski's Inequality: $\|f+g\|_p \le \|f\|_p + \|g\|_p$ and 
$$\left \|\int F(x, t)\,dt \right \|_p \le \int \|F(\cdot, t)\|_p \,dt.$$
\item Young's inequality: $\|f * g\|_p \le \|f\|_1 \|g\|_p$.
\end{itemize}
We have 

\begin{align*}
\chi(hD)u(x) &= \mc F^{-1}(\chi(h\xi) \hat{u}(\xi)) = (2\pi)^{-n} \iint e^{i(x-y)\xi}\chi(h\xi) u(y)dyd\xi \\
&= (2\pi h)^{-n} \int \hat{\chi}\left (\frac{x-y}{h}\right ) u(y)\,dy \\
&\le (2\pi h)^{-n} \frac{C}{h^n} \|\hat{\chi}(\cdot /h)\|_q \|u\|_p
\end{align*}
Then,
$$\|\hat{\chi}(\cdot /h)\|_q = \left (\int |\hat{\chi} (y/h)|^q \,dy\right )^{1/q} = h^{n/q} \|\hat{\chi}\|_q.$$
It follows that 
$$|\chi(hD)u(x) | \le C h^{-n + n/q} \|u\|_p = C h^{-n/p} \|u\|_p.$$

For the second inequality, note that $\chi(hD)u(x) = (2\pi h)^{n} \hat{\chi}(\cdot/h) * u$.  Applying Young's Inequality,
$$\| \chi(hD) u\|_p \le \frac{1}{(2\pi h)^n}\|\hat{\chi}(\cdot/h)\|_1\|u\|_p \le \frac{1}{(2\pi)^n} \|\hat{\chi}\|_1 \|u\|_p.$$
\end{proof}
\begin{theorem} For $u \in L^p$, $1\le p \le \infty$, $u \in C^{0, \gamma}(\R^n)$ if and only if for every $\chi \in C_c^\infty(\R^n \setminus 0)$, $\|\chi(hD) u\|_\infty \le C h^{\gamma}$.
\end{theorem}
\begin{proof} We start with the forward direction.  Note that 
\begin{align*}
\chi(hD) u(x) &= \frac{1}{(2\pi h)^n} \int \hat{\chi}((x - y)/h) u(y) \,dy \\
&= (2\pi )^{-n} \int \hat{\chi}(y)u(x - yh) \,dy \\
&= (2\pi)^{-n}\int \hat{\chi}(y) (u(x - yh) - u(x))\,dy \\
\end{align*}
So it follows that 
$$|\chi (hD) u(x)| \le C \|u\|_{C^{0, \gamma}} \int |\hat{\chi}(y)| (hy)^\gamma dy \le C \|u\|_{C^{0, \gamma}} h^{\gamma} \int |\hat{\chi}(y)| |y|^\gamma \,dy$$
\end{proof}
and the last integral is bounded since $\hat{\chi}$ is a Schwartz function, so it follows that $|\chi(hD)u(x)| \le C \|u\|_{C^{0, \gamma}} h^{\gamma}$.
\end{proof}

\pagebreak
\section{February 9th, 2021}
\subsection{Fourier Transform proof of Morrey's Inequality}
Recall the Littlewood-Paley decomposition: There exists $\psi_0 \in C_c^\infty(\R^n)$, $\psi \in C_c^\infty(\R^n \setminus \{0\})$ such that $$1 = \psi_0(\xi) + \sum_{j=0}^\infty \psi(2^{-j}\xi).$$  From this, we have for $u \in \mc S'$, 
$$u = \psi_0(D)u + \sum_{j=0}^\infty \psi(2^{-j}D)u.$$
More generally, for $a\in \mc S(\R^n)$, $a(D)u = \mc F^{-1}(a(\xi) \hat{u}(\xi))$.
We were proving the following theorem:
\begin{theorem} For $u \in L^p$, $1\le p \le \infty$, $u \in C^{0, \gamma}(\R^n)$ if and only if for every $\chi \in C_c^\infty(\R^n \setminus 0)$, $\|\chi(hD) u\|_\infty \le C h^{\gamma}$.
\end{theorem}
\begin{proof}
We proved the forward direction last time.  We now show the converse.  

Denote
$$\Lambda_\gamma(u) = \sup_{0 < h < 1} h^{-\gamma}(\| \psi(hD)u\|_\infty + \max \| \psi_k(hD)u\|_\infty)$$
where $\psi_k(\xi) = \xi_k \psi(\xi)$.  

We have the hypothesis: $\|u\|_p + \Lambda_\gamma(u) < \infty$.  We want to show that $\|u\|_{C^\gamma, 0} \le C(\|u\|_p + \Lambda_\gamma(u))$.  We first bound $\|u\|_\infty$.  Note that 
\begin{align*}
\|u\|_\infty &\le \|\psi_0(D)u\|_\infty + \sum_j \|\psi(2^{-j} D)u \|_\infty \\
&\le \|\psi_0(D)u\|_\infty + \sum_j 2^{-j\gamma} \Lambda_\gamma(u) \\
&\le C \|u\|_p + (2^\gamma - 1)^{-1} \Lambda_\gamma(u).
\end{align*}

Now, we bound the quotient term, $|u(x) - u(y)|/|x - y|^\gamma$.  In order words, we want
$$|u(x) - u(y)| \le C(\|u\|_p + \Lambda_\gamma(u))r^{\gamma},$$
if $|x - y| \le r$.

Note that \begin{align*}
u(x) - u(y) &= \psi_0(D) u(x) - \psi_0(D) u(y) + \sum_{j} \left (\psi(2^{-j}D) u(x) - \psi(2^{-j}D) u(y)\right ) \\
\end{align*}
It is enough to prove that $|\psi_0(D)u(x) - \psi_0(D) u(y)| \le Cr^{\gamma} \|u\|_p$ and 
$$\sum_{j=0}^\infty |\psi(2^{-j}D)u(x) - \psi(2^{-j}D)u(y)| \le Cr^{\gamma} \Lambda_\gamma(u).$$

Note that 
\begin{align*}
|\psi_0(D) u(x) - \psi_0(D)u(y) | &\le \sup(\grad(\psi_0(D)u))|x-y|  \\
&\le |x - y| \frac{1}{(2\pi)^{n}} \sup \int \grad |\hat{\psi}_0(x-y) ||u(y)|\,dy \\
&\le |x - y| \frac{1}{(2\pi)^n} \| \grad \hat{\psi}_0\|_q \|u\|_p.
\end{align*}

For the second inequality, we prove for both high frequency and low frequency estimates.  For the high ones,
$$|\psi(hD)u(x) -\psi(hD)u(y)| \le 2 \|\psi(hD)u\|_\infty \le 2 h^\gamma \Lambda_\gamma (u).$$

For low frequencies, 
\begin{align*}
|\psi(hD)u(x) - \psi(hD) u(y)|&\le Cr \max_k \|D_{x_k} \psi(hD) u\|_\infty \\
&= Cr h^{-1} \max_k \| hD_{x_k} \psi(hD)u\|_\infty \\
&= Crh^{-1}\max_k \|\psi_k(h D) u\|_\infty \\
&\le  Crh^{-1} \max_k \| \psi_k(hD)u\|_\infty \\
&\le Crh^{\gamma - 1} \Lambda_\gamma(u).
\end{align*}

Then, note that 
$$ \sum_{2^j \le s} Cr2^{-j(\gamma - 1)} \le C' r  s^{1-\gamma}$$

and 
$$\sum_{2^j > s} C2{-j\gamma} \le C'' s^{-\gamma}.$$
It follows that 
\begin{align*}
\sum_{j=0}^\infty |\psi(2^{-j}D)u(x) - \psi(2^{-j}D)u(y)| &\le C\Lambda_\gamma(u)(rs^{1 - \gamma} + s^{-\gamma}) \le Cr^\gamma\Lambda_\gamma(u) 
\end{align*}
\end{proof}
\pagebreak
\section{February 11th, 2021}
\subsection{Finishing Morrey's Inequality}
The original statement of the theorem.
\begin{thm}[Morrey's Inequality]  Suppose $u \in L^p(\R^n), \grad u \in L^p(\R^n)$ and $n < p \le \infty$.  Then there exists $u^* \in C^{0, \gamma}(\R^n)$, with $\gamma = 1 - \frac{n}{p}$ such that $u = u^*$ almost everywhere and $\|u^*\|_{C^0, \gamma} \le C(\|u\|_p + \|\grad u\|_{p})$.
\end{thm}
Last time, we showed the following theorem:
\begin{thm} For $u \in L^p$, $1\le p \le \infty$, $u \in C^{0, \gamma}(\R^n)$ if and only if for every $\chi \in C_c^\infty(\R^n \setminus 0)$, $\|\chi(hD) u\|_\infty \le C h^{\gamma}$.
\end{thm}
Recall that
$$\Lambda_\gamma(u) = \sup_{0 < h < 1} h^{-\gamma}(\| \psi(hD)u\|_\infty + \max \| \psi_k(hD)u\|_\infty)$$
where $\psi_k(\xi) = \xi_k \psi(\xi)$.  
We proved this by showing that $\|u\|_{C^{0, \gamma}} \le C(\|u\|_p + \Lambda_\gamma(u))$.  We now show the complete proof of Morrey's Inequality.
\begin{proof}
It suffices to show that $\Lambda_\gamma(u) \le C\|\grad u\|_p$.  Recall that for all $\chi \in C_c^\infty(\R^n)$, we showed that $\|\chi(hD)u\|\le  Ch^{-n/p}\|u\|_p$.  Note that 
\begin{align*}
\|\phi(hD)hD_{x_j}u\|_\infty &\le ch^{1-n/p} \|\grad u\|_p \\
\Rightarrow \| \phi_j(hD)u\|_\infty &\le Ch^\gamma \|\grad u\|_p.
\end{align*}

We would like to write $\psi \in C_c^\infty(\R^n\setminus \{0\})$, $\psi(\xi) = \sum \xi_j \chi_j(\xi)$ with $\chi_j \in C_c^\infty$.  We can do this with $\sum \xi_j \frac{\xi_j}{|\xi|^2}\psi(\xi)$.

It follows that 
$$\|\psi(hD)u\|_\infty \le \sum_{j=1}^\infty \|\xi_j \chi(hD) u\|_\infty \le Ch^\gamma \|\grad u\|_p.$$
\end{proof}

We can use this result to show regularity properties for solutions to PDEs.  For example, one statement is as follows: suppose $u \in L^1$, $\Delta u = f \in C^{k, \gamma}$ for $0 < \gamma < 1$.  We could show that $u \in C^{k+2, \gamma}$.
\subsection{Final Comments about Sobolev Spaces}
\begin{definition}
Suppose $U \subset \subset \R^n$ with $\partial U \in C^1$.  Then 
$W_0^{1,p}(U) = \ol{C_c^\infty(U)}$ where the closure is respect to the $W^{1, p}$ norm.  
\end{definition}

\begin{fact} $W_0^{1, p} = \{u \in W^{1, p}(U) : Tu = 0  \}$, where $T:W^{1, p}(U) \to L^{p}(\partial U)$ linear and bounded and for $u \in W^{1, p}(U) \cap C(\ol{U})$, $Tu = u\vert_{\partial U}$.
\end{fact}

\begin{fact}[Poincare Inequality] Suppose $1 \le p < n$ and $1 \le q \le p^* = \frac{np}{n-p}$.   Then $\|u\|_q \le C\|\grad u\|_p$.
\end{fact}
\begin{thm}[Poincare Inequality(v2)] For all $1 \le p \le \infty$, $\|u\|_p \le C\|u\|_p$.
\end{thm}
\begin{proof}
Suppose $p < n$.  This follows from the version $1$.  Suppose $\infty > p \ge n$.  In this case, take $q = n - \epsilon$.  Then $q^* = \frac{(n - \epsilon) n}{\epsilon}$.  Choose small enough $\epsilon$ so that $q^* \ge p$.  Then, we apply Poincare $1$:
$\|u\|_p \le \|u\|_{q^*} \le C\|\grad u\|_q \le C\| \grad u\|_p$.  For $p =\infty$, the result follows from Morrey's inequality.
\end{proof}

\subsection{Duality}
Recall the Riesz Representation Theorem for Hilbert Spaces:
\begin{thm}[Riesz Representation] For $\Phi: H \to \C$ with a Hilbert space $H$, if $|\Phi(u)| \le C\|u\|$, there exists $v \in H$ such that $\Phi(u) = \<u, v\>$.
\end{thm}

\begin{fact} $H^{-s}(\R^n) = (H^s(\R^n))^*$: if $u \in H^{-s}(\R^n)$ and $v \in H^s(\R^n)$, $u \in H^s(\R^n)$, then $\<u, v\>_{L^2} = \int u\ol{v}$ is well defined, and for any $\Phi: H^s(\R^n) \to \C$ such that $|\Phi(u)| \le C \|u\|_{H^s}$, there exists $v \in H^{-s}$ such that $\Phi(u) = \<u, v\>_{L^2}$.
\end{fact}
\begin{proof}
First assume $u, v \in \mc S$.  Then $$\int u\ol v = (2\pi)^n \int \hat{u}(\xi) \ol{\hat{v}(\xi)} = (2\pi)^n\int \<\xi\>^s \<\xi\>^{-s} \ol{\hat{v}(\xi)}$$
so it follows that $$|\<u, v\>|_2 \le (2\pi)^n \| u\|_{H^s} \|v\|_{H^{-s}}.$$

Conversely, suppose we have $\Phi$ as above.  Riesz implies that $$\Phi(u) = \<u, w\>_{H^s} = (2\pi)^{-n} \int\<\xi\>^{2s} \hat{u} \ol{\hat w} = (2\pi)^{-n} \int \hat{u} \ol{\<\xi\>^{2s} \hat{w}}$$
and we finish by setting $\hat{v} = \<\cdot\>^{2s} \hat{w}$.
\end{proof}

\subsection{Duality on Bounded Domains}
We define $H^{-1}(U) = \{u \in \mc D'(U) : \forall \phi \in C_c^\infty(U), |u(\phi)| \le C \|\phi_{H^1}\}$.  The norm on $H^{-1}$ is given by 
$$\|u\|_{H^{-1}(U)} = \sup\{|u(\phi)| : \phi \in H_0^1, \|\phi\|_{H^1} \le 1\},$$
which is the usual operator norm, treating $u$ as a linear functional on $H^1(U)$.

\begin{example} We claim $$H_0^1((0, \pi)) = \{u(x) = \sum_{n=1}^\infty a_n \sin{nx}: \sum |a_n|^2n^2 < \infty \sim \|u\|_{H^1}\}.$$  Then, $$H^{-1}((0, \pi)) = \{v(x) = \sum_{n=1}^\infty a_n \sin{nx}: \sum_{n=1}^\infty |a_n|^2 n^{-2} < \infty\},$$ where we take convergence in the sense of distributions.

Then, $$\<u, v\> = \sum a_n \ol{b_n} = \sum_{n=1}^\infty na_n n^{-1} \ol b_n \le \|u\|_{H^1} \|v\|_{H^{-1}}.$$
\end{example}
\pagebreak
\section{February 16th, 2021}
\subsection{Calculus of Variations: Minimizing Distance in the Plane}

We start with a motivating example. Take points $a, b$ in the $x$-axis, $c, d$ in the y-axis.  We wish to find a function $y = f(x)$ such that $f(a) = c$, $f(b) = d$ and the graph of $f$ has the shortest length.  Recall that 
$$L(f) = \int_{a}^b (1 + f'(x)^2)^{1/2}\,dx.$$
We wish to minimize $L$ over all paths from $a$ to $b$.  If $f$ is a minimizer, then for all $\phi \in C_c^\infty((a, b))$, $L(f + t\phi)$ has a minimum at $t = 0$.  This implies that $\frac{d}{dt} L(f + t\phi)\vert_{t = 0} = 0$ for all $\phi$ as above.  Then,
\begin{align*}
\frac{d}{dt} L(f + t\phi) &= \frac{d}{dt} \int_{a}^b (1 + (f + t\phi)'^2)^{1/2}\,dx \\
&\int_{a}^b \frac{\partial}{\partial t} [(1 + (f + t\phi)'^2)^{1/2}]\,dx \\
&= \int_a^b \frac{\phi'(x) (f'(x) + t\phi'(x))}{(1 +(f'(x) + t\phi'(x)^2 )^{1/2}}
\end{align*}
Applying $t = 0$, we have
\begin{align*}
 0 = \int_a^b \phi'(x) \frac{f'(x)}{(1 + f'(x)^2)^{1/2}}\,dx
\end{align*}
for all $\phi \in C_c^\infty((a, b))$.

Integrating by parts, we get that 
\begin{align*}
\int_a^b \phi(x)\left (\frac{f'(x)}{(1 + (f'(x))^2)^{1/2}}\right )' \,dx = 0.
\end{align*}

The implies that 
$$\frac{d}{dx} \left (\frac{f'(x)}{(1 + f'(x)^2)^{1/2}}\right ) = 0,$$
with $f(a) = c$, $f(b) = d$.

We find that $f'(x) = \alpha$ so $f(x) = \alpha x + \beta$. 
\subsection{Calculus of Variations: Minimizing Area in $\R^3$}
Take $U \subset \subset \R^2$, $\partial U \in C^1$.  We wish to minimize the area of the graph with the condition that $f = g$ on $\partial U$.

We have 
$$A(f) = \iint_U (1 + |\grad f(x)|^2)^{1/2}\,dx.$$
We wish to minimize $A(f)$ over $f$ satisfying $f = g$ on $\partial U$.

If $f$ is a minimizer, $t \mapsto A(f + t\phi)$, $\phi \in C_c^\infty(U)$ has a minimum at $t = 0$.  So, $$\frac{d}{dt}A(f + t\phi) \vert_{t = 0} = 0.$$

Doing the same calculation as before, we have 
\begin{align*}
\frac{d}{dt} A(f + t\phi)\vert_{t = 0} &= \int_U \frac{\partial }{\partial t} (1 + |\grad f + t\grad \phi|^2)^{1/2}\,dx \\
&= \int_U \frac{\grad \phi \cdot \grad f}{(1 + |\grad f|^2)^{1/2}}\,dx \\
&= \int_U \phi \left [\left (\frac{f_{x_1}}{(1 + |\grad f|^2)^{1/2}}\right )_{x_1} + \left (\frac{f_{x_2}}{(1 + |\grad f|^2)^{1/2}}\right )_{x_2} \right ]\,dx.
\end{align*}
As before, this implies that 
$$\left (\frac{f_{x_1}}{(1 + |\grad f|^2)^{1/2}}\right )_{x_1} + \left (\frac{f_{x_2}}{(1 + |\grad f|^2)^{1/2}}\right )_{x_2} = 0.$$
This is called the \textbf{Minimal Surface Equation}.  

We will not solve this, but how could we do it?  Consider $f \in H^1(U)$, and note that $Tf = g \in L^2(\partial U)$ is well-defined.  If we take $m = \inf \{A(f) = f \in H^1(U), f \vert_{\partial U} = g\}$.  Then, there exists $f_j \in H^1(U), f_j \vert_{\partial U} = g$ with $A(f_j) \to m$.  Could we find $f_{j_k} \to f$?

\subsection{Calculus of Variations: General Setup}
Take $U \subset \subset \R^n$.  Take $L: \R^n \times \R \times \ol{U} \to \R$ in $C^\infty$, written as $L(p, z, x)$.  We introduce the functional $I[w] = \int_U L(D_w(x), w(x), x)\,dx$, with $w \vert_{\partial U} = g$.  

\begin{example} In the minimal surface problem, $L(p, z, x) = (1 + |p|^2)^{1/2}$.
\end{example}

We first derive an equation satisfied by the minimizer.  As before, we have $I[w] = \int_U L(Dw, w, x)\, dx$, a minimizer.  This implies that $\frac{d}{dt} I[w + t\phi] \vert_{t = 0}= 0$ for all $\phi \in C_c^\infty(U)$.

Then,
\begin{align*}
\int_U \frac{d}{dt} \left [L(Dw + tD\phi, w+t\phi, x) \right ]\,dx\vert_{t = 0} &= \int_U \left (D\phi \cdot D_pL(Dw, w, x) + \phi D_z L \right ) \,dx \\
&= \int_U \left (- \sum (L_{p_j}(Dw, w, x))_{x_j} + D_zL(Dw, w, x)\right ) \phi\,dx \\
\Longrightarrow &\boxed{-\sum_{j=1}^n (L_{p_j}(Dw, w, x))_{x_j} + D_zL(Dw, w, x) = 0},
\end{align*}
the \textbf{Euler-Lagrange Equation}. 
\begin{example} Take $L(p, z, x) = |p|^2/2 - f(x)z$.  
\begin{align*}
I[w] = \int_U \left (|\grad w(x)|^2/2 - f(x)w(x) \right )\,dx.\\
\end{align*}
Since $L_{p_j} = p_j$, the Euler-Lagrange equation is given by
$$-\sum(w_{x_j})_{x_j} -f(x) = 0 \Longrightarrow -\Delta w = f, w\vert_{\partial U} = g.$$
\end{example}

We can generalize this as follows:  If we take $L = |p|^2/2 + F(z)$ and $f(z) = F'(z)$.  The Euler-Lagrange equation is then $-\grad w = f(w)$.  For example, if we take $f(z) = z^p$, $F(z) = \frac{z^{p+1}}{p+1}$.

We could also take non-constant coefficients: $L(p, z, x) = \frac{1}{2}\sum a_{ij}(x) p_i p_j - f(x)z$, where $a_{ij} = a_{ji}$.

Then, $L_{p_j} = \frac{1}{2}\sum_{i=1}^n a_{ij}(x) p_i $.  Then, the Euler-Lagrange equation is given by
$$-\sum_{i, j = 1}^n \partial_{x_j}(a_{ij }\partial_{x_i}w(x)) = f(x).$$
When $\sum a_{ij}(x) \xi_i \xi_j \ge c|\xi|^2$ for all $\xi \in \R^n$, $x \in \ol{U}$, this is solvable.  

\subsection{Existence of Minimizers}
\begin{itemize}
\item Coercivity: There exists $\alpha . 0, \beta \ge 0$ with $L(p, z, u) \ge \alpha |p|^q - \beta$, for $1 < q < \infty$, for all $z \in \R$, $x \in \ol U$.  

The condition gives the following bound: $I[w] \ge \alpha \|Dw\|_q^q - \beta \mu(U)$.  We can always set $\beta = 0$ by translating $L$ by a constant.  Taking $\mc A = \{w \in W^{1, q}(U) : u\vert_{\partial U} = g\}$, we minimize $I[w]$ over $\mc A$.
\item Lower semicontinuity: Suppose we have $u_k \rightharpoonup u$ weakly in $W^{1, q}$.  Then, $$I[u] \le \liminf I[u_k].$$
\end{itemize}
\pagebreak
\section{February 18th, 2021}
Recall, we have $L: \R^n \times \R \times \ol{U} \to \R$ for $U \Subset \R^n$, $\partial U \in C^1$.  We denote $L = L(p, z, x)$, $D_pL = (\partial_{p_1}L, \dots, \partial_{p_n}L)$, etc.  We also defined 
$$I[w] = \int_U L(Dw(x), w(x), x)\,dx, \quad w\vert_{\partial U} = g.$$

As an example, $L(p, x) = \frac{1}{2}\sum a_{ij}(x) p_ip_j - f(x)z$.  Last time, we used the principle that if $w$ is a minimizer, for every $\phi \in C_c^\infty(U)$, $t \mapsto I[w + t\phi]$ has a local minimum at $t = 0$.  This implies that if $w$ is a minimizer, $L$ satisfies the Euler-Lagrange equation:
$$-\sum_{j=1}^n (L_{p_j}(Dw, w, x))_{x_j} + D_zL(Dw, w, x) = 0.$$

\subsection{Second-Derivative Test}
If $i'(0) = 0, i''(0) > 0$, then we have a local minimum at $0$.

By definition $i(t) = I[w + t\phi]$, where $\phi \in C_c^\infty$.   Recall that
$$i'(t) = \int\left ( \sum \phi_{x_j} \partial_{p_j}L(Dw + t\phi, w + t\phi, x) + \phi \partial_zL(Dw + t\phi, w + t\phi, x)\right )\,dx.$$

Then,
\begin{align*}
i''(0) = \int \left (\sum_{i, j} \phi_{x_j} \phi_{x_i} \partial_{p_j p_i}L + \sum_j \phi \phi_{x_j} \partial_z \partial_{p_j} L + \phi^2 L_{zz} \right )\,dx
\end{align*}
If this is at least $0$ for all $\phi$, what do we get about $L$?   This makes sense for $\phi$ that is Lipschitz and $0$ at the boundary.  If we choose $\phi(x) = \epsilon \rho(\frac{x \xi}{\epsilon}) \zeta(x)$, where $\zeta \in C_c^\infty(U)$ and  $\rho$ consists of triangles with slope $\pm 1$ starting at $0$.  Then $|\rho'(x)| = 1$ almost everywhere.  

Using this $\phi$, we get $\phi_{x_j} = \epsilon \rho(x \xi / \epsilon) \zeta'(x) + \xi_j \rho'(x \xi/\epsilon) \zeta(x) = \xi_j \rho'(x \xi/\epsilon) \zeta(x) + O(\epsilon)$ and 
\begin{align*}
0 &\le \int_U \sum_{i, j} (\xi_i \xi_j \partial_{p_ip_j}^2 L)((\rho')^2 \zeta^2) + O(\epsilon) \\
&\xrightarrow{\epsilon \to 0} \int_U \sum_{i, j} (\xi_i \xi_j \partial_{p_ip_j}^2 L)( \zeta^2) 
\end{align*}
for any $\zeta \in C_c^\infty(U)$, so it follows that for all $\xi \in \R^n$, $\sum \xi_i \xi_j L_{p_ip_j}(Dw(x), w(x), x) \ge 0$.  Hence, it is useful to assume convexity:
$$\sum_{i, j}^n \xi_i, \xi_j L_{p_i p_j} L(p, z, x) \ge 0$$
for all $\xi \in \R^n$, $(p, z, x) \in \R^n \times \R \times U$.  (

\begin{align*}
L(p + t\xi) &= L(p) + t \sum \xi_j L_{p_j} L(p) + t^2 \int_{0}^1 (1 - s) \sum \xi_i \xi_j L_{p_ip_j}(p + st\xi)\,ds \\
\end{align*}
\subsection{Convexity}
For smooth $L$, convexity is the statement 
$$\sum L_{p_i p_j}(p, z, x) \xi_i \xi_j \ge c|\xi|^2 \ge 0.$$
for all $\xi \in \R^n$.  
\begin{example} for $L = 1/2 \sum a_{ij}(x) p_i p_j$, $a_{ij} = a_{ji}$, convexity is that 
$$\sum a_{ij}(x) \xi_i \xi_j \ge c |\xi|^2$$

We call this an Elliptic operator.
\end{example}
\begin{example} For the minimal surface equation, $L = (1 + |p|^2)^{1/2}$.  Note that $L_{p_i} = \frac{p_i}{(1 + |p|^2)^{1/2}}.$  

$$L_{p_ip_j} = \frac{\delta_{ij}}{(1 + |p|^2)^{1/2}} - \frac{p_ip_j}{(1 + |p|^2)^{3/2}} = \frac{\delta_{ij}(1 + |p|^2) - p_ip_j}{(1 + |p|^2)^{3/2}}.$$

Then,
\begin{align*}
\sum L_{p_ip_j} \xi_i \xi_j &= \frac{1}{(1 + |p|^2)^{3/2}}\left ( \sum |\xi|^2 (1 + |p|^2) - \sum \xi_i p_i \xi_j p_j\right )\\
&= (1 + |p|^2)^{3/2} (|\xi|^2 + |\xi|^2 |p|^2 - \<\xi, p\>^2) \ge 0
\end{align*}

This is not strictly convex, since as $p \to \infty$ our term goes to $0$.
\end{example}
\subsection{Existence of Minimizers}
Recall our conditions:
\begin{itemize}
\item Coercivity: There exists $\alpha > 0, \beta \ge 0$ with $L(p, z, u) \ge \alpha |p|^q - \beta$, for some $1 < q < \infty$, for all $z \in \R$, $x \in \ol U$.  

The condition gives the following bound: $I[w] \ge \alpha \|Dw\|_q^q - \beta \mu(U)$.  We can always set $\beta = 0$ by translating $L$ by a constant.  Taking $\mc A = \{w \in W^{1, q}(U) : u\vert_{\partial U} = g\}$, we minimize $I[w]$ over $\mc A$.
\item Lower semicontinuity: Suppose we have $u_k \rightharpoonup u$ weakly in $W^{1, q}$.  Then, $$I[u] \le \liminf I[u_k].$$
\end{itemize}

As we will see, the coercivity leads to nice compactness results via Rellich-Kondrachov.  How can we use lower semicontinuity?  Assume $\mc A$ is nonempty.  Take $m = \inf_{w \in mc A} I[w]$.  Then, we have $I[w_j] \to m$.  Assuming coercivity, we have $\|Dw_j \|_q$ is bounded.  If $w_0 \in \mc A$, then $\|w - w_0\|_q \le \|Dw - Dw_0\|_q$ by the Poincare inequality.  So it follows that $\|w_j\|_q \le C$.  From the Banach-Alaoglu Theorem, we have $w_j$ is weakly compact in $W^{1, q}$.  Passing to a subsequence, $w_j \rightharpoonup w$ in $W^{1, q}$.  From lower semicontinuity, we have $I[w] \le \liminf I[w_j] = m$.  This implies that $I[w] = m$.

\pagebreak
\section{February 23rd, 2021}
\subsection{Weak Convergence}
We have a Banach space $B$ with dual $B^*$ with $u : B \to \C$ linear in the dual if for all $x \in B$, $|u(x)| \le C\|x\|_B$.

\begin{thm}[Banach - Alaoglou] The unit ball $\{u \in B^*: \|u\|_{B^*} \le 1\}$ is weak-* compact: if we have $\|u_j\|_{B^*} \le 1$, then there exists a subsequence and $u \in B^*$ such that for every $x \in B$, $u_{j_k}(x) \to u(x)$.
\end{thm}
\begin{corollary} If $B$ is reflexive, $(B^*)^* = B$, then $\{x: \|x\|_B \le 1\}$ is weakly compact.  Given $\|x_j\| \le 1$, there exists $x \in B$ and a subsequence so that $u(x_{j_k}) \to u(x)$ for $u \in B^*$.
\end{corollary}
\begin{example} Take $B = L^q(U)$ for $1 < q < \infty$.  This is reflexive since $B^* = L^{q'}(U)$ with $q^{-1} + (q')^{-1} = 1$ for $1 < q' < \infty$.
\end{example}
\begin{remark} For $u_j \in B$, we say $x_j \rightharpoonup x \in B$ iff for all $u \in B^*$, $u(x_j) \to u(x)$.
\end{remark}
\begin{itemize}
\item If $B$ reflexive and $x_j \weakto x$, then $\|x\| \le \liminf \|x_j\|$.  This is because $|x(u)| = \lim |x_j(u)| \le \liminf \|u\|_{B^*} \|x_j\|_B$ and $\|x\|_B = \sup_{\|u\|_{B^*} = 1} |x(u)|$.
\item If $B$ reflexive and $x_j\weakto x$, there exists $C$ such that $\|x_j\|_B \le C$.  For every $u \in B^*$, $|x_j(u)| \le C(u)$, which implies by the Uniform Boundedness Principle that $\|x_j\|_B \le C$.
\item (We don't assume $B$ is reflexive) Suppose $V \subset B$ is a closed subspace.  Then $V$ is weakly closed.  This is a special case of Mazur's Theorem.  
\begin{proof}
We need to show that if $x_j \in V$, $x_j \weakto x \in B$, then $x \in V$.  For $u \in B^*$, $u(x_j) \to u(x)$.  So if $x \not \in V$, we want to construct $u \in B^*$ so that $u(x_j) = 0$ and $u(x) = 1$.  

Recall Hahn-Banach:  If we have a subspace $\tilde V \subset B$ and $\tilde{\phi} : \tilde V \to \C$ with $|\tilde{\phi}(x)| \le C \|x\|_B$, with $x \in \tilde V$, then there exists $\phi \in B^*$ so that $\phi\vert_{\tilde{V}} = \tilde{\phi}$.

Take $\tilde{V} = V + \C x$.  Define $\tilde{\phi}: \tilde{V} \to \C$ and define $\tilde{\phi}(y + \alpha x) = \alpha$, $y \in V$, $\alpha \in \C$.  It suffices to check that it is bounded.  We need $\tilde{\phi}(y + \alpha x) \le C\|y + \alpha x\|$.  Suppose not - for every $n$, there exists $y_n, \alpha_n$ such that $ |\alpha _n|= |\tilde{\phi}(y_n + \alpha_n x)| > n \|y_n + \alpha_n x\|$.  

Dividing by $\alpha_n$, we get 
$$1/n > \| y_n/\alpha_n + x\|_B.$$
But this would imply that $-y_n/\alpha_n \to x \not \in V$, but $V$ is closed.  

\end{proof}
\end{itemize}
\subsection{Calculus of Variations}
We now move back to calculus of variations.  We have $I[w] = \int_U L(Dw(x), w(x), x)\,dx$ with $U \Subset \R^n$, $\partial U C^1$.  We have $L = L(p, z, x) \in C^\infty(\R^n \times \R \times \ol{U})$.  We wish to minimize $L$ under the constraint that $w \vert_{\partial U} = g$.

Recall that we introduced $i(t) = I[w + t\phi]$, $\phi \in \C_c^\infty(U)$.  If $w$ is a minimizer of $I[w]$, then
\begin{itemize}
\item $i'(0) = 0$ for all $\phi$ implies that $-\sum_{j=1}^n \partial_{x_j} (\partial_{p_j}L(Dw, w, x)) + \partial_zL(Dw, w, x) = 0$(Euler-Lagrange equation).
\item If $i''(0) \ge 0$ for all $\phi$, then we have the convexity condition: for all $\xi \in \R^n$, $\sum_{i, j} \partial_{p_i} \partial_{p_j} L(Dw, w, x)\xi_i \xi_j \ge 0$.
\end{itemize}
We introduced the conditions:
\begin{itemize}
\item Coercivity: there exists $1 < q < \infty$, $\alpha > 0$, $\beta \ge 0$ such that $L(p, z, x) \ge \alpha |p|^q - \beta$.  This implies that $I[w] \ge \alpha\|Dw\|_q^q - \beta$.
\item (Weak) Lower semicontinuity: If $u_j \weakto u$ and $Du_j \weakto u$ weakly in $L^q(U)$, then $I[u] \le \liminf I[u_j]$.
\end{itemize}

\begin{remark} Take $\mc A = \{u \in W^{1, q}: u\vert_{\partial U} = g\} \ne \emptyset$.  If we put $m = \inf_{\mc A} I[w]$, there is a sequence $w_k \in \mc A$ such that $I[w_k] \to m$.  Using weak compactness, we have a subsequence $w_{k_j} \weakto w$ in $W^{1, q}$ with $w \in \mc A$.  Then $m \le I[w] \le \liminf I[w_{j_k}]= m$, so $I[w] = m$.
\end{remark}

\subsection{Getting around Lower Semicontinuity}
\begin{thm} Suppose $L \ge -C$ and $p \mapsto L(p, z, x)$ is convex for all $(z, x) \in \R \times \ol{U}$.  Then, for any $1 < q < \infty$, $w \mapsto I[w]$ is weakly lower semicontinuous in $W^{1, q}$: if $w_j \weakto w$ in $L^q$, $Dw_j \rightharpoonup Dw$ in $L^q$, then $I[w] \le \liminf I[w_j]$.
\end{thm}
\begin{remark} 
Convexity implies that for all $p_1, p_2$, $L(p_1) \ge L(p_2) + D_pL(p_2) \cdot (p_1-p_2)$.
\end{remark}
\begin{proof}
We assume that $u_k \weakto u$ in $L^q$ and $Du_k \weakto Du$ in $L^q$.  We have $\ell = \liminf I[u_k]$ and we want $I[u] \le \ell$.

By taking a subsequence, we can say that $\ell = \lim I[u_k]$.  By taking another subsequence, we can say $u_k \to u \in L^q$, since weak convergence implies that $\|u_k\|_{q} \le C$ and $\|Du_k\|_q \le C$ and using compactness of $W^{1, q}$ in $L^q$. By taking a subsequence we can use $u_k \to u$ almost everywhere[this is the Riesz-Fisher theorem].  By Egorov's Theorem, for every $\epsilon$, there exists a set $E_\epsilon$ such that $m(U \setminus E_\epsilon) \le \epsilon$  so that $u_k \to u$ uniformly on $E_\epsilon$.  Note that $m(U) < \infty$.

We define a set $F_\epsilon = \{x \in U : |u(x)| + |Du(x)| \le \frac{1}{\epsilon}\}$.  Then $m(U \setminus F_\epsilon) \to 0$ as $\epsilon \to 0$.  We define $G_\epsilon = E_\epsilon \cap F_\epsilon$, with $m(U \setminus G_\epsilon) \le m(U \setminus E_\epsilon) + m(U \setminus F_\epsilon) \to 0$ as $\epsilon \to 0$.  

Without loss of generality, we can assume $L \ge 0$.  Note that 
\begin{align*}
I[u_k] &= \int_U L(Du_{k}, u_k, x) \\
&\ge \int_{G_\epsilon} L(Du_k, u_k, x) \\
&\ge \int_{G_\epsilon} L(Du, u_k, x) + D_pL(Du, u_k, x)(Du_k - Du)\,dx
\end{align*}
Then, $\lim \int_{G_\epsilon} L(Du, u_k, x) = \int_{G_\epsilon} L(Du, u, x)$ since $u_k \to u$ uniformly on $G_\epsilon$ and $Du$ is uniformly bounded on $G_\epsilon$.  For the second term, $D_pL(Du, u_k, x) \to D_pL(Du, u, x)$ uniformly on $G_\epsilon$.  Then $Du_k \weakto Du$ in $L^q$.  Then writing $\int (D_pL(Du, u_k, x) - D_pL(Du, u, x))(Du_k - Du) + D_pL(Du, u, x) \cdot (Du_k - Du)$, $Du_k - Du$ is bounded in $L^q$ and $(D_pL(Du, u_k, x) - D_pL(Du, u, x))$ converges uniformly to $0$, so the first term goes to $0$.  For the second term, $D_pL(Du, u, x)$ is bounded and $Du_k - Du$ converges weakly to $0$.  

It follows that $$\ell = \liminf I[u_k] \ge \int_{G_\epsilon}(Du, u, x)\,dx  \xrightarrow{\epsilon \to 0} \int_U L(Du, u, x)\,dx = I[u].$$

Hence, $I[u] \le \liminf I[u_k]$, as desired.

\end{proof}

\pagebreak
\section{February 25th, 2021}
\subsection{Existence of Minimizers}
We proved last time that a convexity condition was sufficient for showing the weak lower semicontinuity condition.   
\begin{example} A simple example is $L(p, z, x) = \sum a_{ij}(x) p_ip_j$ where $a_ij = a_{ji}$ and $\sum a_{ij}(x) \xi_i \xi_j \ge \theta|\xi|^2.$ for all $\xi \in \R^n$, $x \in \ol{U}$.  In this case, $\mc A = \{u \in H^1(U): u\vert_{\partial U} = g\}$, if $g \in H^{1/2}(\partial U)$.  Then, we minimize $\int_U \sum a_{ij}(x)\partial_{x_j}u \partial_{x_i} u \,dx = 0$ with $u\vert_{\partial U} = g$.
\end{example}

\begin{thm}[Existence of Minimizers] Suppose $p \mapsto L(p, z, x)$ is convex and $L(p, z, x) \ge \alpha|p|^q - \beta$, $\alpha > 0$, $\beta \ge 0$, $1 < q < \infty$.  Suppose that $\mc A = \{w \in W^{1, q}(U): w \vert_{\partial U}= g\} \ne \emptyset$ with $g \in L^q(\partial U)$, then there exists $u \in \mc A$ such that $I[u] = \min_{w \in \mc A} I[w]$.
\end{thm}
\begin{proof}
We can assume without loss of generality that $\beta = 0$.  Put $m = \inf_A I[w] \ne \infty$.  Choose a sequence $u_k \in \mc A$ such that $I[u_k] \to m$.  Then $I[u_k] \ge \alpha \int |Du_k|^q$.  This implies that $\|Du_k\|_{L^q} \le C$.

Fix $w \in \mc A $ and note that $u_k - w \in W_0^{1, q}$.  Recall the Poincare Inequality: if $v \in W_0^{1, q}$, then $\|v\|_q \le C\|Dv\|_q$.  Hence $$\|u_k\|_q \le \|u_k - w\|_q + \|w\|_q \le \|Du_k - Dw\|_q + \|w\|_q \le \|Du_k\|_q + \|w\|_{W^{1, q}}\le C'.$$

This implies that $\|u_k\|_{W^{1, q}} \le C$.  Hence, there exists a subsequence $u_k \weakto u$ in $W^{1, q}$.  This means that $u_k - w \weakto u - w$ in $W^{1, q}$ but $u_k - w \in W_0^{1, q}$, a closed subspace in $W^{1, q}$ which implies that $u - w \in W_0^{1, q}$.  Hence, $u \in \mc A$. So, it suffices to show that $u$ is a minimizer.  

From the convexity of $p \mapsto L(p, z, x)$, we have $I$ is weakly lower-semicontinuous.  In other words, $I[u] \le \liminf_{k \to 0} I[u_k] = \inf_{w \in \mc A} I[w]$. Hence, $I[u] = \inf_{w \in \mc A} I[w]$.

\end{proof}

\subsection{Uniqueness of Minimizers} 
\begin{thm}[Uniqueness of Minimizers] Suppose $L = L(p, x)$ and there exists $\theta > 0$ such that for all $\xi \in \R^n$, $p \in \R^n$, $x \in \ol U$, we have $\sum L_{p_i p_j}(p, x) \xi_i \xi_j \ge \theta|\xi|^2$(uniform convexity).  Then, any minimizer of $I[u]$ is unique.   
\end{thm}
\begin{proof}
We have that $L(p, x) \ge L(q, x) \ge D_pL(q, x)(p-q) + \frac{\theta}{2}|p-q|^2$ from strict convexity(this follows from the Taylor remainder).  Let $u, \tilde{u}$ be minimizers of $I[w]$.  Take $v = \frac{u + \tilde{u}}{2}$. From strict convexity, we have
$$I[u] \ge I[v] + \int D_pL(Du/2 + D\tilde{u} / 2, x) (Du/2 - D\tilde{u}/2) + \theta/8|Du - D\tilde{u}|^2\,dx.$$


Similarly, 
$$I[\tilde u] \ge I[v] + \int D_pL(Du/2 + D\tilde{u} / 2, x) (D\tilde{u}/2 - Du/2) + \theta/8|Du - D\tilde{u}|^2\,dx.$$

Then, $$m = I[u]/2 + I[\tilde{u}]/2 \ge I[v] + \theta/8\int  |Du - D\tilde{u}|^2\,dx \ge m + \theta/8 \int |Du - D\tilde{u}|^2\,dx.$$

This would imply that $\theta/8 \int |Du - D\tilde{u}|^2\,dx \le 0$, which implies that $Du = D\tilde{u}$ almost everywhere but $u \vert_{\partial U} = \tilde{u} \vert_{\partial U}$ so $u = \tilde{u}$ almost everywhere.  
\end{proof}

\subsection{The Euler-Lagrange Equation}
Recall the example $L(p, x) = \sum a_{ij}(x) p_i p_j \ge \theta|p|^2$ with $a_{ij} = a_{ji}$.  This has the Euler-Lagrange Equation:
$$\sum_{i, j = 1}^n\p_{x_i} (a_{ij} \p_{x_j} u) = 0, \quad u\vert_{\partial U} = g.$$
For all $g \in H^{1/2}(\partial U)$ we can find $u \in H^1(U)$ such that the Euler-Lagrange equation weakly(in the sense of distributions).  Furthermore, $u$ is unique.  We will show this next time.

Today, we show that we can solve the equation:
$$\sum_{i, j = 1}^n\p_{x_i} (a_{ij} \p_{x_j} u) = f, \quad u\vert_{\partial U} = g, f \in H^{-1}(U)$$
with $u \in H^1(U)$.  We can write 
$$I[w] = \int \sum \left (a_{ij}(x) \p_{x_i} w\p_{x_j} w - f(x) w \right ) \,dx$$

In this case $L(p, z, x) = \sum a_{ij}(x) p_i p_j - f(x)z$ for $z \in\R$, so this is not bounded below.   However, $p \mapsto L(p, z, x)$ is convex.  So it suffices to deal with the coercivity issue.  

What we really need is that $I[w] \ge \int _U |Dw|^2 - \beta$.  

Recall the Peter Paul inequality: 
$$2ab \le a^2/\epsilon + \epsilon b^2, \quad \forall \epsilon > 0.$$
Hence,
$$I[w] \ge \theta \int |Dw|^2 - \int |f| |w| \ge \theta \int |Dw|^2 - \frac{1}{2\epsilon }\int |f|^2 - \frac{\epsilon}{2} \int |w|^2.$$
If we fix $w_0 \in H^1$ with $w_0 \vert_{\partial U} = g \in H^{1/2}$, we have
$$I[w] \ge \theta \int |Dw|^2 - \frac{1}{2\epsilon}\int |f|^2 - \frac{\epsilon}{2}\left ( \int |w - w_0|^2 + \int |w_0|^2\right ) \ge \theta \int |Dw|^2 - \frac{C}{2\epsilon} - \frac{\epsilon}{2} |Dw|^2.$$


It follows that 
$$I[w] \ge \theta \int |Dw|^2 - C_\epsilon - \epsilon \int |Dw|^2 = (\theta - \epsilon) \int |Dw|^2 - C_\epsilon.$$

Choosing $\epsilon < \theta/2$, we have
$$I[w] \ge \frac{\theta}{2} \int |Dw|^2 - C_\epsilon \ge \alpha \int |Dw|^2 - C.$$

One problem: we assumed $f \in L^2$.  How do we fix this? We solve 2 problems:
$$\sum (a_{ij}u_{x_j})_{x_i} = f \in H^{-1}, u \vert_{\partial U} = 0,$$
and $u \in H_0^1(U)$.  Then, $\int fu \le \|f\|_{H^{-1}} \|u\|_{H_0^1} \le \frac{1}{\epsilon} \|f\|_{H^{-1}}^2 + \epsilon \|u\|_{H_0^1}$.  For $u \in H_0^1$, $\|u\|_{H_0^1} \le C \|Du\|_{2}$.  Then, we apply the same argument.

If not, we take $v = u - \tilde{u}$ and we have $$-\sum (a_{ij} v_{x_j})_{x_i} = 0, v\in H_0^1(U).$$  Otherwise, we multiply by $v$ and we use the definition of weak solution.  This implies that $\int \sum a_{ij} v_{x_i}v_{x_j}$, but it is also at least $\theta \int |Dv|^2$, which shows that it is exactly zero.
\pagebreak
\section{March 2nd, 2021}
Today, we find conditions on the Lagrangian so that the the E-L condition holds:
$$-\sum (\partial_{p_j}L(Du(x), u(x), x))_{x_j} + \partial_z L(Du, u, x) = 0.$$
\subsection{Euler-Lagrange Equation, continued}
We will make the following assumptions:  for all $p \in \R^n$, $z \in \R$, $x \in \ol U$,
\begin{itemize}
\item $|L(p, z, x)| \le C(|p|^q + |z|^q + 1)$,
\item $|D_pL|, |D_zL| \le C(|p|^{q-1} + |z|^{q-1} + 1)$.
\end{itemize}
It is natural to consider the E-L equation in a weak sense, that is for all $v \in C_c^\infty(U)$, 
$$\int \sum \partial_{p_j} L v_{x_j} + \partial_z L v = 0.$$

The conditions imply that $|\partial_{p_j} L| \le C(|Du|^{q-1} + |u|^{q-1}) \in L^{q'}(U)$, and the same with $\partial_z L$.  This implies that our integral condition would make sense for $v \in W_0^{1, q}(U)$.  

\begin{definition} Suppose our assumptions from above hold(bounds on $|L|$ and $|DL|$) and $u \in \mc A = \{w \in W^{1, q}(U) : w\vert_{\partial U} = g\}$.  We then say that the E-L equation holds weakly if for all $v \in W_0^{1, q}(U)$, we have 
$$\int \sum \partial_{p_j} L v_{x_j} + \partial_z L v = 0.$$
\end{definition} 

\begin{thm} Suppose $u \in \mc A$ is a minimizer for $L$ satisfying the bounds.  Then $u$ is a weak solution to the Euler Lagrange equation.  
\end{thm}
\begin{proof}
Define $i(t) = I[u + t v]$ where $v \in W_0^{1, q}$.  
Let
$$\frac{i(t) - i(0)}{t} = \int_U \frac{L(Du + tDv, u + tv, x) - L(Du, u, x)}{t}\,dx$$
and call the integrand $L^t(x)$.

We have that $L^t(x) \xrightarrow{t \to 0} \sum L_{p_i}(Du, u, x) v_{x_i} + L_z v$ almost everywhere in $x$.  We want to bound $|L^t(x)|$ by a function in $L^1$ so that we can apply the dominated convergence theorem.   

Note that $f(\xi + t \eta) - f(\xi) = \eta \int_0^t f'(\xi + t\eta)\,dt$.This means that $$L(Du + tDv, u + tv, x) - L(Du, u, x) = \int_0^t \sum L_{p_j}(Du + sDv, u + sv, x) v_{x_j} + L_z v \,ds.$$

Now, we bound this using our assumptions.  Namely, recall that $|DL| \le C(|p|^{q-1} + |z|^{q-1} + 1)$.  It follows that 
\begin{align*}
|L(Du + tDv, u + tv, x) &- L(Du, u, x)| \le \int_0^t \sum |L_{p_j}(Du + sDv, u + sv, x) v_{x_j}| +| L_z v| \,ds \\
&\le C \int_0^t (|Du + sDv|^{q-1} + |u + sv|^{q-1} + 1) (|Dv| + |v|)\,ds \\
&\le C \int_0^t (1 + |Du|^{q-1} + |Dv|^{q-1} + |u|^{q-1} + |v|^{q-1})(|Dv + |v||)\,ds \\
&\le Ct \left ( |Du|^{q-1} (|Dv| + |v|) + |u|^{q-1} (|Dv| + |v|) + |Dv|^q + |v|^q + 1\right )\\
\end{align*}


We would like to say 
$$\left ( |Du|^{q-1} (|Dv| + |v|) + |u|^{q-1} (|Dv| + |v|) + |Dv|^q + |v|^q + 1\right ) \le C \left ( |Du|^{q} + |u|^{q} + |Dv|^q + |v|^q + 1\right ).$$

We do this via Young's inequality: $ab \le \frac{a^{q'}}{q'} + \frac{b^{q}}{q}$.  This implies that 
$$|Du|^{q-1}|Dv| \le C (|Du|^{(q-1)q'} + |Dv|^{q}),$$
and doing this for the other product terms gives the desired inequality.

It follows that $|L^t(x)|$ is bounded by an $L^1$ function, so we can apply the dominated convergence theorem, which gives the result.
\end{proof}

\begin{remark} The converse is not necessarily true.  However, we have the following theorem:
\end{remark}

\begin{thm} Suppose $u \in \mc A$ is a weak solution to the Euler-Lagrange equation.  If $(p, z) \mapsto L(p, z, x)$ is convex for all $x \in U$, then $u$ is a minimizer.  
\end{thm}
\begin{proof}
From convexity, we have that $L(p, z, x) + D_pL(p, z, x) \cdot (q - p) + D_zL(p, z, x)(w-z) \le L(q, w, z)$.  Upon integrating, if we set $p = Du$, $q = Dw$, $z = u$, $w = w(x)$ for $w \in \mc A$:
$$I[u] + \int_U D_p(Du, u, x)\cdot (Dw - Du) + D_zL(Du, u, x)\cdot (w - u)\, dx \le I[w].$$

But $w - u \in W_0^{1, q}$ and $u$ weakly satisfies the equation so it follows that the integral is $0$ and for every $w \in \mc A$, $I[u] \le I[w]$.
\end{proof}

%\subsection{Regularity}
%Can weak solutions be upgraded to strong solutions?  If we had $L(p, z, x) = |p|^2/2 - zf(x)$, $U$ with $C^\infty$ boundary and $g \in C^\infty(\partial U)$ and $f \in C^\infty(\ol{\partial U})$.  When we take $f \in H^{-1}(U)$, $g \in H^{1/2} (U)$, we obtain a solution to $-\Delta u = f$ with $u \in H^1(U)$ with $u\vert_{\partial U} = g$.  However, in our case, we can obtain a solution with $u \in C^\infty(\ol{U})$.


\pagebreak
\section{March 4th, 2021}
\subsection{Regularity}
We will make the assumption that $L(p, z, x) = L(p) - zf(x)$.  We also assume that $|L(p)| \le C(|p|^2 + 1)$ and $|D_pL(p)| \le C(|p| + 1)$, $|D_p^2 L(p)| \le C$.  Finally, we assume strong convexity: $\sum L_{p_i p_j} (p) \xi_i \xi_j \ge \theta |\xi|^2$ for all $p, \xi \in \R^n$.  

\begin{example} $L(p) = \frac{1}{2}|p|^2$ is an example of a function which satisfies the above conditions. 
\end{example}

Last time, we showed (with weaker assumptions) that if $I[u]$ is a minimum, then $u$ satisfies the E-L equation weakly.  With a convexity condition, we have the converse as well.   We assume that $u\vert_{\partial U} = 0$ for simplicity.
\begin{proposition}  There exists a constant $C = C(L, n, U)$ so that $\|u\|_{H^1} \le C\|f\|_2$.  
\end{proposition}
\begin{proof}
We use the weak E-L with $v = u$.  Namely,
$$ \int_U \sum L_{p_j}(Du) u_{x_j} = \int_U fu.$$
Then, strict convexity implies that $(DL(p) - DL(0)) \cdot p \ge \theta |p|^2$, with $p = Du$.    Hence,
$$\theta \int_U |Du|^2 \le \int_U D_pL(Du) \cdot Du - \int_U DL(0) \cdot Du = \int_U fu,$$
where the second term is $0$ by the divergence theorem.   
Then $$\theta \int_U |Du|^2 \le \int_U fu \le \int_U \frac{f^2}{\epsilon} + \epsilon u^2 = \|f\|^2/\epsilon + \epsilon\|u\|^2.$$

Hence, $\|Du\|_2^2 \le \frac{1}{\epsilon \theta} \|f\|^2 + \frac{\epsilon}{\theta}\|u\|^2$.  By the Poincare inequality, $\|u\|_2 \le C\|Du\|_2^2$, so it follows that by taking epsilon small enough.
\end{proof}
\subsection{Interior Regularity}
\begin{thm} Suppose that $- \sum (L_{p_j}(Du))_{x_j} = f$ weakly, $f \in L^2$, $u \in H_0^1$ with the same bounds as before.  Then $u \in H_{loc}^2(U)$.  
\end{thm} 
\begin{proof}
Take $V \Subset  W  \Subset U$ open sets.  Choose a function $\zeta \in C_c^\infty(W)$ and $\zeta = 1$ near $V$.

Choose $D_k^h u(x) = \frac{u(x + he_k) - u(x)}{h}.$  If $h$ is small enough then it is well defined on $U$.  Note that for $v \in C_c^\infty(W)$, $\int u D_k^{-h} v = -\int vD_k^h u$.  Then, we define $v = -D_k^{-h}(\zeta^2 D_k^h u) \in H_0^1(W) \subset H_0^1(U)$.

Then noting that $(D_k^{-h}v)_{x_i} = D_k^{-h}(v_{x_i})$, 
$$\int_U \sum D_k^h(L_{p_i}(Du))(\zeta^2 D_k^h u)_{x_i} = -\int_U fD_k^{-h}(\zeta^2 D_k^h u).$$
\begin{align*}
D_k^h(L_{p_i}(Du))&= \frac{1}{h}\int_0^1 \sum_{j} L_{p_ip_j}(sDu(x + he_k) + (1 - s)Du(x))\,ds \cdot (D_{x_j}u(x + he_k) - D_{x_j }u(x))\\
&= \sum_{j=1}^n a_{ij}^h(x)  D_k^h u_{x_j}(x)\\
\end{align*}
where $$a_{ij}^h(x) = \int_0^1 L_{p_i p_j} (sDu(x + he_k) + (1-s)Du(x)).$$

So, we have
$$\int_U \sum_{j=1}^n a_{ij}^h(x) D_k^h u_{x_j}(\zeta^2 D_k^h u)_{x_i} = -\int_U f D_k^{-h}(\zeta^2 D_k^h u).$$

The left hand side is 
$$\int \sum a_{ij}^h D_k^h u_{x_j} D_k^h u_{x_i} \zeta^2 + \sum a_{ij}^h D_k^h u_{x_j} 2\zeta \zeta_{x_i} D_k^h u.$$

The first term is bounded from below by $\theta |D_k^h Du|^2 \zeta^2$ by strict convexity.  The second term is bounded by $- \int \zeta |D_k^h Du||D_k^h u| \ge -\epsilon \int \zeta^2 |D_k^h Du|^2 - \frac{1}{\epsilon} |D_k^h u|^2.$ 

We end up with 
$$\int \zeta^2 |D_k^h Du|^2 \,dx \le C \int_W (f^2 + |D_k^h u|^2).$$
\end{proof}
\pagebreak
\section{March 9th, 2021}
\subsection{Inner Regularity}
Last time, we discussed regularity in the special case where $L = L(p) - f(x)z$.  We also assumed that $|D_p^k L(p)| \le C(1 + |p|^{2-k})$ and strict convexity: for all $p, \xi \in \R^n$, $\sum L_{p_i p_j}(p)\xi_i \xi_j \ge \theta|\xi|^2$.  

We know that for $U \Subset \R^n$, $\partial U$ $C^1$ and $f \in L^2$, for all $g \in H^{1/2}(\partial U)$, there exists $u$ such that for all $v \in H_0^1(U)$, 
$$\int_U (\sum L_{p_j}(Du)v_{x_j} - f(x)v)\,dx = 0, \quad u\vert_{\partial U} = g.$$

\begin{theorem} Suppose that $- \sum (L_{p_j}(Du))_{x_j} = f$ weakly, $f \in L^2$, $u \in H_0^1$ with the same bounds as before.  Then $u \in H_{loc}^2(U)$.  
\end{theorem} 
\begin{remark} The main idea in the proof is that you can estimate derivatives with cutoff functions if you only need local results.  This requires carefully choosing the width of the quotients to stay away from the boundary.  
\end{remark}
\begin{proof}
We had a function $D_k^h u(x) = \frac{u(x + he_k) - u(x)}{h}$, and we chose a $\zeta$ so that $\zeta \equiv 1$ in $V$ and $\zeta \in C_c^\infty(W)$ where $V \subset W \subset U$.  

Then, we defined $v = -D_k^{-h}(\zeta^2 D_k^h u)$.  Last time, we showed that 
$$\int_U \sum a_{ij}^h(x) D_k^h u_{x_j} (\zeta^2 D_x^h u)_{x_i} = \int_U f D_k^{-h}(\zeta^2 D_k^h u),$$
where we have the bounds
$$\theta |\xi|^2 \le \sum a_{ij}^h(x) \xi_i \xi_j \le C|\xi|^2.$$

Differentiating the expression, the LHS gives
$$ \int \sum a_{ij}^h(x)(D_k^h u_{x_{j}} D_k^h u_{x_i})\zeta^2 + \int \sum a_{ij}^h(x) D_k^h u_{x_j} 2\zeta \zeta_{x_i} D_k^h u.$$
The first term is bounded below by $\theta \int_U |D_k^h Du|^2 \zeta^2$.  The second term is bounded below by $-C\int \zeta| D_k^h Du| |D_k^h u|$.

Now, the RHS is bounded above by
$$\int_W |f| |D_k^{-h}Du|\zeta^2 + \int_W |f| \zeta |D_k^h u|.$$

Note that 
$$D_k^h u = \int_0^1 u_{x_k}(x + t h e_k)\,dt.$$

So it follows that $\int_W |D_k^h D_k^{-h} u|\zeta^2 \le \int |D_k^h Du|\zeta^2$.  It follows that 
\begin{align*}
\int_W |f| |D_k^h Du|\zeta^2 + \int_W |f| \zeta |D_k^h u| &\le \frac{2}{\epsilon} \int |f|^2 + \epsilon \int |D_k^h Du|^2 \zeta^2 + \epsilon \int \zeta^2 |D_k^h u|^2\\
&\le \frac{2}{\epsilon} \int |f|^2 + \epsilon \int \zeta^2 |D_k^h u|^2 + \epsilon \int |D_k^h Du|^2 \zeta^2.
\end{align*}

By using Peter Paul on the LHS bound and making $\epsilon << 1$, we obtain the inequality
$$\theta/2 \int |D_k^h Du|^2 \zeta^2 \le 2/\epsilon \int |f|^2 + \epsilon \int \zeta^2 |D_k^h u|^2 \le 2/\epsilon \int |f|^2 + \epsilon \int |Du|^2.$$

Finally, we claim that $\int |D_k^h w|^2 \le C \Rightarrow w_{x_k} \in L^2_{loc}$.  We know that $D_k^h w$ is bounded in $L^2$ for all $h$, so it is weakly compact.  Hence, $D_k^{h_j} w \weakto v \in L^2$.  It follows that $$\int (D_k^h w)\phi = -\int w D_k^{-h} \phi \to - \int w \phi_{x_k} = \int w_{x_k} \phi,$$
which is the weak derivative.  
\end{proof}

\subsection{Higher Regularity}
Take $f \equiv 0$ and $\int L_{p_i}(Du) v_{x_i} = 0$ for all $v \in H_0^1$.  Take $w \in C_0^\infty(U)$ and set $v = -w_{x_k}$.  We have
$$-\int \sum L_{p_i}(Du)\partial_{x_k}(w_{x_i})\,dx.$$

But $u \in H_{loc}^2$, so it follows that 
$$\int \sum_{i, j} L_{p_i p_j}(Du) u_{x_j x_k} w_{x_i}\,dx.$$

Setting $\tilde{u} = u_{x_k} \in H^1$.  We get that $\sum \partial_{x_j}(a_{ij}(x)\partial_{x_i} \tilde{u}) = 0$ where $a_{ij} = L_{p_i p_j}(Du)$.  But the bounded coefficients don't give a strong enough condition to prove the result.  

\begin{theorem} Suppose that $w \in H_{loc}^1(U)$, $\sum \partial_{x_j}(a_{ij}(x)\partial_{x_i}w) = 0$ weakly, and $\theta|\xi|^2 \le \sum a_{ij}(x) \xi_i \xi_j \le C |\xi|^2$.   Then, there exists $\gamma > 0$ such that $w \in C_{loc}^{0, \gamma}$(DeGiorgi-Nash, Moser).  Applying with $w = \tilde{u} = u_{x_k} $ gives that $u \in C_{loc}^{1, \gamma}$, which implies that $a_{ij} = L_{p_i p_j}(Du) \in C^{0, \gamma}$.  Finally, using Schauder estimates, we have that $u \in C^{2, \gamma}$.  
\end{theorem}
\pagebreak
\section{March 11th, 2021}
We follow the book \textit{Grigis-Sjostraund: Microlocal Analysis for Differential Operators}. 
\subsection{Oscillatory Integrals}
 We denote $X \subset \R^n$ as an open set and $\mc D'(X) =\{u: C_c^\infty(X) \to \C: \forall K \subset \subset X, \exists C, N, \forall \phi \in C_c^\infty(K), |u(\phi)| \le C\sup_{|\alpha| \le N}|\partial^\alpha \phi|\}$.

We wish to generalize expressions like $\delta_0(x) = \frac{1}{(2\pi)^n} \int e^{ix \cdot \xi}\,d\xi$.  This is an "oscillatory integral" in the sense that we are integrating something that oscillates rapidly.   This means that for all $\psi \in C_c^\infty$, $\delta_0(\psi) = \psi(0) = (2\pi)^{-n} \int \int e^{ix \cdot \xi} \psi(x)\,dx d\xi$.

We change the phase of $x \cdot \xi$ to a function $\phi(x, \theta)$ where $x \in X$, $\theta \in \R^N$ so that $\phi(x, \lambda \theta) = \lambda \phi(x, \theta)$ for $\lambda > 0$.

The amplitude $(2\pi)^{-n}$ is generalized to $a(x, \theta)$, and we try to consider which properties we need so that we can define $I(a, \phi) = \int_{\R^N} a(x, \theta) e^{o\phi(x, \theta)}\, d\theta$ to be a distribution.  
\begin{example} Take $X = \R^n$, $N = n$.  Define $P(\xi)$ to be a homogeneous polynomial satisfying $P(\xi) \ne 0$ whenever $\xi \ne 0$.  For example, we could take $P(\xi) = |\xi|^2$.  Define $\chi \in C_c^\infty(\R^n)$ so that $\chi \equiv 1$ near zero.  Take $E(x) = (2\pi)^{-n} \chi(x) \int_{\R^n}\frac{1 - \chi(\xi)}{P(\xi)} e^{ix \cdot \xi}\,d\xi$.  This doesn't converge, but the integrand is a smooth homogeneous function of degree $-m$ away from zero.  If we define $D_x = \partial_x/i$, then $$P(D)E(x) = \frac{1}{2\pi} \chi(x) \int_{\R^{n}} \frac{1 - \chi(\xi)}{P(\xi)} P(\xi) e^{ix \cdot \xi} + [P(D), \chi]u \int_{\R^n} \frac{1 - \chi}{R}e^{ix \xi}\,d\xi,$$
where $[P(D), \chi]$ is the commutator.  

The first term is $$\frac{1}{2\pi} \chi(x)\int e^{ix\xi}d\xi + (2\pi)^{-n} \chi(x) \int (-\chi(\xi))e^{ix\xi}d\xi.$$
The second term is 
$$\sum_{|\alpha| > 1} C \partial^\alpha X (2\pi)^{-n} \int (1-\xi)/P \xi^\beta e^{ix\xi} d\xi.$$

The first term is $\delta_0(x)$, and the second is compactly supported.  For the last term, for $x \ne 0$, $1/|x|^2 \<x, \partial_\xi\>e^{ix\ xi} = e^{ix \cdot \xi}$.  The idea is that we can integrate by parts and the $\partial_{\xi}$ derivatives will decay rapidly.  Then, we can replace this term with something that decays rapidly so that  we have $\delta_0(x) + K(x)$ for a compactly supported $K \in C_c^\infty(\R^{n})$.  This is called a "Parametrix" for $P(D)$.
\end{example}

In our above example, $\phi(x, \theta) = x \cdot \theta$ and $\theta = \xi$, with $a(x, \xi) = \frac{1 - \chi}{P(\xi)}$.  Our function satisfies the estimate
$$|\partial_{\xi}^\alpha ((1 - \chi)/P)| \le C_\alpha \<\xi\>^{-m - |\alpha|}.$$

\subsection{General Theory: Amplitudes}
\begin{definition} $S_{\rho, \delta}^m(X \times \R^{N}) = \{a \in C^\infty(X \times \R^N) : \forall K \Subset X, \alpha \in \N^n, \beta \in \N^N, \exists C = C(K, \alpha, \beta) : |\partial_X^\alpha \partial_{\xi}^\beta a| \le C\<\xi\>^{m - \rho|\beta| + \delta|\alpha|}\}.$  
\end{definition}
In our previous example, when $P$ is a homogeneous polynomial of degree $q$, $\frac{1 - \chi(\xi)}{P(\xi)} \in S_{1, 0}^{-q}$.
\begin{remark} This only makes sense for $0 \le \rho \le 1$ and $0 \le \delta \le 1$.  For $\rho > 1$.  Then $|\partial^\alpha a| \le C_{N, \alpha} \<\xi\>^{-N}$.  Suppose $\xi$ has dimension $[in]$ and $a$ has dimension $[in]^k$.  $\partial_\xi^\alpha a$ has dimensions $[in]^{k - |\alpha|}$.  On the other hand, $|\partial_{\xi}^\alpha a| \le C\<\xi\>^{m - \rho |\alpha|}$, which has dimension $[in]^{m - \rho |\alpha|}$.  Then $k - |\alpha| = m - \rho|\alpha|$ so $k = m - (\rho - 1)|\alpha|$, but since $\alpha$ is arbitrary, the units of $a$ are any negative number.  

The same analysis would work for $\delta < 0$.
\end{remark}

$S_{\rho, \delta}^m$ is a Frechet space, one that is generated by seminorms.  Namely, note that $\|a\|_{K, \alpha, \beta} = \sup_{(x, \theta)} \<\theta\>^{-m + \rho |\beta| + |\alpha| \delta}|\partial_x^\alpha \partial_\xi^\beta a|$.  Then $a \in S_{\rho, \delta}^m$ if and only if for all $K \Subset X$, $\|a\|_{K, \alpha, \beta}< \infty$.    The space is a Frechet space if it is complete with respect to this norm.  This is also a meterizable space, with a metric defined in the obvious way.  

Some properties:
\begin{itemize}
\item If $m \le m'$, $\delta \le \delta'$, $\rho \ge \rho'$, then $S_{\rho, \delta}^m \subset S_{\rho', \delta'}^{m'}$.
\item We define $S^{-\infty}(X \times \R^N) = \{a \in C^\infty(X \times \R^N) : K \Subset X, \forall N, \exists C |\partial_x^\alpha \partial_{\xi}^\beta a| \le C\<\xi\>^{-N}\}$.
\item $S^{-\infty}(X \times \R^N) = \bigcap_{m} S_{\rho, \delta}^m (X \times \R^N)$.  We call this the residual space.  
\end{itemize}

\begin{example} Take $a \in C^\infty(X \times \R^N)$ and for $|\theta| \ge 1$, $\lambda > 0$, $a(x, \lambda \theta) = \lambda^m a(x, \theta)$.  We claim that $a \in S_{1, 0}^m$.  If we differentiate, we have $\partial_\theta^\alpha a(x, \lambda \theta) = \lambda^{m - |\alpha|} \partial_\theta^\alpha a(x, \theta)$.
\end{example}
\pagebreak
\section{March 16th, 2021}
\subsection{Amplitudes of Oscillatory Integrals}
We are making sense of $I(a, \phi) = \int_{\R^N} a(x, \theta) e^{i\phi(x, \theta)}\,d\theta$ on a distribution.  We defined a class of functions for $m \in \R$, $\rho, \delta \in [0, 1]$.  $S_{\rho, \delta}^m(X \times \R^{N}) = \{a \in C^\infty(X \times \R^N) : \forall K \Subset X, \alpha \in \N^n, \beta \in \N^N, \exists C = C(K, \alpha, \beta) : |\partial_X^\alpha \partial_{\xi}^\beta a| \le C\<\xi\>^{m - \rho|\beta| + \delta|\alpha|}\}.$ These are called the \textbf{symbols} of order $m$ and type $(\rho, \delta)$.  We write $S^m = S_{1, 0}^m$, which is the case where $|\<\theta\>^{-m+|\beta|}\partial_x^\alpha \partial_\xi^\beta a | \le C_{\alpha \beta}$.

Why are these called symbols?  Suppose we have $P(x, 0)u = \sum_{|\alpha| \le  m} a_\alpha(x) D_k^\alpha u$.  We can also write this as $(2\pi)^{-n} \int \sum_{|\alpha| \le m} a_\alpha(x) \xi^\alpha e^{i(x - y)\xi} u(y)\,dy d\xi$, $u \in \S$.  The current order of integration makes sense but we can also just consider the integral in $d\xi$, which is an oscillatory integral, whose integrand is $p(x, \xi)$ is the symbol of $P(x, D)$.  Then $p \in S^m(X \times \R^n)$, which is of type $(1, 0)$.  

\subsection{A Cool Example}


Suppose $f \in C^\infty(X \times \R^n; [0, \infty))$ homogeneous - $f(x, \lambda \theta) = \lambda  f(x, \theta)$, $\lambda > 0$, $|\theta| \ge 1$.  Define $a(x, \theta) = e^{-f(x, \theta)}$.  Note that $0 \le a \le 1$.  This is also a smooth function.  

We claim that $$\partial_x^\alpha \partial_\theta^\beta (e^{-f}) = \sum_{|\tilde{\alpha}| \le |\alpha|, |\tilde{\beta}| \le |\beta|} a_{\alpha \beta}(x, \theta) (\partial_x f^){\tilde{\alpha}} (\partial_\theta f)^{\tilde{\beta}}  e^{-f}.$$

We can estimate the bad term $(\partial_x f^){\tilde{\alpha}} (\partial_\theta f)^{\tilde{\beta}}  e^{-f}$.  For this, we use Landau's inequality: If $g \in C^2(U)$, $g \ge 0$, for all $K \Subset U$, there exists $C$ such that $|\grad g(x)| \le C \sqrt{g(x)}$, $x \in K$.
\begin{proof}
Note that $0 \le g(x + y) = g(x) + \grad g(x) + y + O(|y|^2)$. This implies that $-\grad g(x) \cdot y \le g(x) + O(|y|^2)$.  Taking $y = -\epsilon \grad g(x)$, it follows that 
$$\epsilon |\grad g(x)| \le g(x) + O(\epsilon^2 |\grad g(x)|^2),$$
which implies the result.  
\end{proof}


We have that $f \ge 0$ so we have that $|\partial_x f| + |\partial_\theta f| \le C f^{1/2}$, with $1 \le \theta \le 2$, $x \in K$.  For $\lambda > 0$, note that $$\lambda^{-1} \partial_x f(x, \lambda \theta) + \partial_\theta f(x, \theta) \le C \lambda^{-1/2} f(x, \lambda \theta)^{1/2}.$$

It follows by taking $\tilde \theta = \lambda \theta$,
$$|\theta|^{-1/2} \partial_x f(x, \theta) + |\theta|^{1/2} \partial_\theta f(x, \theta) \le C(f, \theta)^{1/2}, \qquad |\theta| \le 1, x \in K.$$

Now, we estimate 
$$|(\partial_x f)^{\tilde{\alpha}} (\partial_\theta f)^{\tilde{\beta}}  e^{-f} | \le |\theta|^{|\tilde{\alpha}|/2}|\theta|^{-|\tilde{\beta}|/2} f^{1/2(|\tilde{\alpha} + \tilde{\beta}|)} e^{-f}.$$

Finally $f^k e^{-f} \le k!$, which is some constant, so it follows that our term is bounded by $C_{\tilde{\alpha}\tilde{\beta}} |\theta|^{|\tilde{\alpha}|/2} |\theta|^{-\tilde{\beta}/2} \in S_{1/2, 1/2}^0 (X \times \R^N)$.
\subsection{Topology of the Symbol Space}
\begin{proposition} Suppose $\{a_j\}$ is bounded in $S_{\rho, \delta}^m$ and $a_j(x, \theta) \to a(x, \theta)$ for all $x, \theta \in X \times \R^N$. Then, $a \in S_{\rho, \delta}^m$ and $a_j \to a$ in $S_{\rho, \delta}^{m'}$ for all $m' > m$.
\end{proposition}
\begin{proof}
We first prove a lemma.  
\begin{lemma} Suppose $f \in C^2([-\epsilon, \epsilon])$.  Then $|f'(0)| \le 2 \|f\|_{L^\infty}^{1/2} \|f''\|_{L^\infty}^{1/2} + (2/\epsilon + 1/2) \|f\|_{L^\infty}$, where $\|g\|_{L^\infty} = \sup_{|x|\le \epsilon} |g|$.
\end{lemma}
\begin{proof}
$f(x) = f(0) + xf'(0) + x^2 + \int_0^1 (1 - t)f''(tx)\,dt$. We have the estimate 
$$|x f'(0)| \le 2\|f\|_\infty + \frac{x^2}{2}\|f''\|_\infty.$$

Dividing by $x$, we we have $|f'(0)| \le 2/x\|f\|_\infty +x/2 \|f''\|_\infty$.  Then, we take $x = \min(2 \|f\|_\infty^{1/2}/\|f''\|_\infty^{1/2}, \epsilon) \le 2\|f\|_\infty^{1/2}\|f''\|_\infty^{1/2} + (\frac{2}{\epsilon} + \frac{1}{2})\|f\|_\infty,$
\end{proof}
From the lemma,
$$\|a_j' - a_k'\| \le C\|a_j - a_k\|_{\infty}^{1/2} \|a_j'' - a_k''\|_{\infty}^{1/2} + C \|a_j - a_k\|_\infty.$$
Hence, a Cauchy sequence in $L^\infty$ implies Cauchy for higher symbols.      

\begin{example} Take $a = 1$, $a_j(\theta) = \chi(\theta/j)$.  $a_j(\theta) \to a(\theta)$ for all $\theta$.  Do we have convergence of $a_j \to a$ in $S^0$?  No! because $\|a_j - a\|_\infty = 1$.  This is similar to the statement $C_0^\infty$ is dense in $L^p$ for $1 \le p < \infty$.

Now, $\|\<\theta\>^{-\delta}(a_j - a)\|_\infty \to 0$ as $j \to \infty$.  
\end{example}
Define $$b_j = \frac{\partial_x^\alpha \partial_\theta^\beta (a_j - a)}{\<\theta\>^{m' - \rho|\beta| + \delta|\alpha|}} = \frac{1}{\<\theta\>^{m' - m}} \frac{\partial_x^\alpha \partial_\theta^\beta(a_j - a)}{\<\theta\>^{m - \rho|\beta| + \delta|\alpha|}}.$$
We know that $\partial_x^\alpha \partial_\theta^\beta(a_j - a)$ goes to $0$ on compact sets, and $\<\theta\>^{m - m'} \to 0$ as $|\theta| \to \infty$.  Then, for all $\epsilon > 0$, there exists $R_\epsilon$ so that $|k_j| < \epsilon$ if $|\theta| > R_\epsilon$.  On the other hand, for $|\theta| \le R_\epsilon$, $x \in K$, $|k_j| < \epsilon$ if $j>J_\epsilon$.    These two things imply that $|k_j| < \epsilon$ for $j > J_\epsilon$.   
\end{proof}
\pagebreak
\section{March 18th, 2021}
\subsection{Topology of the Symbol Space, continued}
Recall that we wish to define a distribution in terms of the oscillatory integral $I(a, \phi) = \int_{\R^N} u(x, \theta)\, e^{i\phi(x, \theta)} d\theta$.  We defined a class $a \in S_{\rho, \delta}^m (X \times \R^N)$ if for all $K \Subset X$, $\alpha, \beta$, there exists $C$ so that $|\partial_x^\alpha \partial_\theta^\beta a(x, \theta)| \le C\<\theta\>^{m-\rho|\alpha| + \delta|\beta|}$.  This class is equipped with seminorms $\|a\|_{K, \alpha, \beta}$ in the obvious way and they generate the topology.

Last time, we proved the following:  if $\{a_j\} \subset S_{\rho, \delta}^m$ is bounded($\|a_j\|_{K, \alpha, \beta} \le C_{K, \alpha, \beta}$), for all $(x, \theta) \in X \times \R^n$, $a_j(x, \theta) \to a(x, \theta)$.  Then, $a \in S_{\rho, \delta}^m$ and $a_j \to a$ in $S_{\rho, \delta}^{m'}$ for $m' > m$.

\begin{proposition} For every $m' > m$, $S^{-\infty}(X \times \R^N)$ is dense in $S_{\rho, \delta}^m$ in the $S^{m'}_{\rho, \delta}$ topology.
\end{proposition}
\begin{proof} Write $\chi_j(\theta) = \chi(\theta/j)$ where $\chi$ is a bump function which is $1$ on open set.  Note that $\partial_{\theta}^\alpha \chi_j(\theta) = j^{-|\alpha|} \chi^{(\alpha)}(\theta/j)$, which is supported in $|\theta| \in [j, 2j]$, so we have$O(\<\theta\>^{-|\alpha|})$.  Hence, $\chi_j \in S_{1,0}^0$ and moreover, $a_j := \chi_j a \in S^{m}_{\rho, \delta}$ uniformly: $\{a_j\}$ is a bounded sequence in $S_{\rho, \delta}^m$.  On the other hand, $a_j \in S^{-\infty}$, so applying the result from last time implies the desired since $a_j \to a$ pointwise.  

\end{proof}
\begin{remark} From Poincare, we say that $a \sim \sum_{j=0}^\infty a_j h^j$(where the RHS doesn't converge) if for all $N$, there exists $C$ so that $|a - \sum_{j=0}^{N-1} a_jh^j| \le Ch^N$.  This is like a Taylor's Theorem.  
\end{remark}
\begin{theorem}
Suppose $a_j \in S_{\rho, \delta}^{m_j}$, $m_j \to -\infty$, $m_0 \ge m_1 \ge \dots$.  Then, there exists $a \in S^m_{\rho,\delta}$ such that for every $k$
$$a - \sum_{j=0}^{k-1} a_j \in S_{\rho, \delta}^{m_k}(X \times \R^N).$$

$a$ is unique in the sense that if the equality holds for another $\tilde{a}$, then $a - \tilde{a} \in S^{-\infty}$.
\end{theorem} 
We will write $a \sim \sum_{j=0}^\infty a_j$.
\begin{proof}
Suppose we have a sequence $\|\cdot\|_{k, \ell}$ is a sequence of seminorms defining the topology on $S_{\rho, \delta}^{m_k}$(for $k$ constant, $\|\cdot\|_{k, \ell}$ is a sequence of seminorms).  

For every $j$, there exists $b_j \in S^{-\infty}$ so that $\|a_j - b_j\|_{\nu, \mu}\le 2^{-j}$ for $0, \nu, \mu \le j-1$.  This is okay because $S^{-\infty}$ is dense in $S^{m_j}_{\rho, \delta}$ in the topology of $S^{m_{nu}}$ for $\nu \le j-1$.  

This implies that if we have $\sum_{j \ge k} a_j - b_j$, this converges in $S^{m_k}_{\rho, \delta}$ for all $k$.  Now, we put $a = \sum_{j=0}^\infty a_j - b_j \in S_{\rho, \delta}^{m_0}$.  Note that $a - \sum_{j < k} a_j = \sum_{j\ge k} a_k - \sum_{j=0}^\infty b_k = -\sum_{j < k} b_k + \sum_{k}^\infty a_j - b_j$.  The first term is in $S^{-\infty}$ because it is a finite sum of things in $S^{-\infty}$.  The second term is in $S_{\rho, \delta}^{m_0}$.
\end{proof} 
\subsection{Phase Functions}
We denote $\dot{\R}^N$ to be $\R^N \setminus \{0\}$.  
\begin{definition}[Non-degenerate Phase Function] A function $\phi \in C^\infty(X \times \dot{\R}^{N})$ satisfying
\begin{itemize}
\item $\phi(x, \lambda \theta) = \lambda \phi(x, \theta)$, $\lambda > 0$.
\item $Im(\phi) \ge 0$.
\item $d\phi \ne 0$: $d \phi = \sum \partial_{\theta_j}\phi d\theta_j + \sum \partial_{x_j} \phi dx_j$ or the gradient vector is different from zero for all $x, \theta$.
\end{itemize}
\end{definition}
\begin{lemma} Suppose $m + k < -N$.  Then, $a \mapsto I(a, \phi)$ defines a continuous map between $S_{\rho, \delta}^m(X \times \R^N) \to C^k(X)$.
\end{lemma}
\begin{proof}
$I(a, \phi) \in C(X)$ if $|a| \le \<\theta\>^{-N - \epsilon}$ i. e. $a \in S^m_{\rho, \delta}$ and $m < -N$.  When we differentiate this $k$ times, we get a function in $S^{m+k}_{\rho, \delta}$ which would map to $C(X)$ if $m + k < -N$.
\end{proof}
\begin{corollary} If $a \in S^{-\infty}$, this implies that $I(a, \phi) \in C^\infty(X)$.  
\end{corollary}
\begin{theorem} Suppose $0 < \rho \le 1$ and $0 \le \delta < 1$.  There exists a unique continuous map from $\bigcup_m S^{m}_{\rho, \delta} \ni a \mapsto I(a, \phi) \in \mc D'(X)$, such that for $a \in S^{m}_{\rho, \delta}$, $m < -N$, $I(a, \phi) = \int a(x, \theta) e^{i \phi(x, \theta)}\, d\theta$.  If $k \in \N$ and $m - k \min(\rho, 1-\delta) < -N$, then the map $a \mapsto I(a, \phi)$ is continuous as a map from $S^m_{\rho, \delta} \to \mc D'^{(k)}(X)$.  
\end{theorem}
\begin{remark} Recall that $u \in \mc D'^{(k)}(X)$ if for all $K \Subset X$, there exists $c$ so that for all $\phi \in C_c^\infty(K)$, $|u(\phi)| \le C \sup_{|\alpha| \le k} |\partial^\alpha \phi|$.
\end{remark}
\begin{proof}
Uniqueness: this follows from the fact that $S^{-\infty}$ is dense in $S^m_{\rho, \delta}$ in $S^{m'}_{\rho, \delta}$ topology for all $m' > m$.  

Existence: We sketch the proof and prove it next time.
\begin{itemize}
\item Use $d\phi \ne 0$ to find a differential operator $L$ so that $L^T e^{i\phi} = e^{i\phi}$.  
\item For $a \in S^{-\infty}$, from integration by parts, $I(a, \phi)v = \int L^k(a, v) e^{i\phi} d\theta dx$, which has stronger regularity.
\end{itemize}
\end{proof}
\begin{example} Take $u(x) = \int_{\R^n} \<\xi\>e^{ix \cdot \xi}\,d\xi$.  Then $u(v) = \int_{\R^n \times \R^n} \<\xi\>v(x) e^{ix\ cdot \xi} \,dx d\xi$ where $v \in C_c^\infty$.  Now, we construct $L$ so that $L^T(e^{ix\xi}) = e^{ix\xi}$.  This corresponds to $(\<A, \partial_x\> + \<B, \partial_\xi\>)e^{ix\xi} = e^{ix\xi}$.  Then, we want $i\<A, \xi\> + i\<B, x\> = 1$.  
\end{example}
\pagebreak
\section{March 30th, 2021}
\subsection{Oscillatory Integrals as Distributions}
We had formal expressions $\int_{\R^N} a(x, \theta) e^{i\phi(x, \theta)}\, d\theta$.  
\begin{itemize}
\item $a \in S_{\rho, \delta}^M(X \times \R^N)$ if and only if $\partial_x^\alpha \partial_\xi^\beta a = O(\<\xi\>^{m - |\beta|\rho + \delta|\alpha|})$.
\item For a nondegenerate phase function, we have the conditions, $\phi$ is homogeneous of degree $1$ in $\theta$, $Im(\phi) \ge 0$, and $d\phi \ne 0$.
\end{itemize}

\begin{lemma} If $a \in S_{\rho, \delta}^m$, $m+k < -N$, then $a \mapsto I(a, \phi) = \int ae^{i\phi}$ is continuous as a map from $S_{\rho, \delta}^m \mapsto C^k(X)$.  
\end{lemma}
\begin{theorem} Suppose $0 < \rho \le 1$ and $0 \le \delta < 1$.  There exists a unique continuous map from $\bigcup_m S^{m}_{\rho, \delta} \ni a \mapsto I(a, \phi) \in \mc D'(X)$, such that for $a \in S^{m}_{\rho, \delta}$, $m < -N$, it coincides with $I(a, \phi) = \int a(x, \theta) e^{i \phi(x, \theta)}\, d\theta$.  If $k \in \N$ and $m - k \min(\rho, 1-\delta) < -N$, then the map $a \mapsto I(a, \phi)$ is continuous as a map from $S^m_{\rho, \delta} \to \mc D'^{(k)}(X)$.  
\end{theorem}
\begin{proof}
We showed uniqueness last time.  Namely, for all $a \in S_{\rho, \delta}^m$ there exists $a_j \in S_{\rho, \delta}^{-\infty}$ such that $a_j \to a$ in $S_{\rho ,\delta}^{m'}$ for $m' > m$ so $I(a_j, \phi)$ are uniquely determined by $\int a_j e^{i\phi} \in C^\infty(X)$, so continuity implies the uniqueness.

For existence, we use the following lemma.
\begin{lemma} Suppose that $\phi$ is a non-degenerate phase function.  Then, there exists $a_j \in S_{1, 0}^0$, $b_\ell \in S_{1, 0}^{-1}$, $c \in S_{1, 0}^{-1}$ such that if we define $L = \sum_j a_j(x, \theta) \partial_{\theta_j} + \sum_\ell b_\ell(x, \theta)\partial_{x_\ell} + c(x, \theta)$ so that ${}^tL(e^{i\phi}) = e^{i\phi}$.
\end{lemma}
\begin{example} ${}^tL$ denotes the transpose of the operator when considered as a distribution.  Namely, $Lv(u) = v({}^t L u)$.  For example, ${}^t \partial_{x_j} = -\partial_{x_j}$.  It is also easier to understand when taking it for test functions through integration.
\end{example}
\begin{proof}
Take $\chi \in C_c^\infty(\R^N)$ so that $\chi \equiv 1$ near $0$.  Define $\Phi = \sum |\frac{\partial \phi}{\partial x_j}|^2 + |\theta|^2 \sum |\frac{\partial \phi}{\partial \theta_j}|^2$.  We have that $\Phi$ is a smooth function away from $0$.  It is homogeneous of degree $2$ in $\theta$, which can be understood by considering each component.  This expression is nonzero away from $0$.  We can define the transpose 
$${}^t L = \frac{1 - \chi(\theta)}{i\Phi} \sum |\theta|^2 \ol{\frac{\partial \phi}{\partial \theta_j}} \partial_{\theta_j} + \sum \ol{\frac{\partial \phi}{\partial x_j}} \partial_{x_j}  + \chi(\theta).$$

We claim that ${}^tL(e^{i\phi})= e^{i\phi}$. Note that $\phi \mapsto i\Phi$.  The first term is in $S^{0}$, the second is in $S^{-1}$ and $\chi \in S^{-\infty} \subset S^{-1}$.
\end{proof}
We start with $u \in C_c^\infty(X)$.  First, take $a \in S^{-\infty}$.  Then, $I(a, \phi) \in C^\infty(X)$, so we have
\begin{align*}
\<I(a, \phi), u\> &= \int_X \int_{\R^N} a(x, \theta) u(x)e^{i\phi(x, \theta)}\,d\theta dx \\
&= \int_X \int_{\R^N} au({}^t L)^k (e^{i\phi}) \,d\theta dx \\
&= \int_X \int_{\R^N} L^k(au) e^{i\phi}\, d\theta \,dx
\end{align*}
Now, suppose that $a \in S_{\rho, \delta}^m$ for $\rho > 0$ and $\delta < 1$.  Then, $L^k(au) \in S_{\rho, \delta}^{m-k \min(\rho, 1-\delta)}$, which means that for $\rho < 1$, $\delta > 1$, we gain decay.  This gives a continuous map $(a, u) \mapsto S_{\rho, \delta}^m \times C_c^\infty(X)$.

Then, $\sup_{K \times \R^N} |L^k(au) \<\theta\>^{-m + k \min(\rho, 1-\delta)}| \le C\|a\|_{\ell, K} \sum_{|\alpha| \le k} \sup |\partial^\alpha u|$.  This implies that if $a \in S_{\rho, \delta}^m$ and $k$ is such that $m-k \min(\rho, 1-\delta) < -N$, we can define
$$\<I_k(a, \phi), u\> = \int e^{i\phi} L^k(au)\,d\theta dx \le C_{a, k, K} \sum_{|\alpha| \le k} \sup |\partial^\alpha u|.$$

To define $I(a, \phi)$ for $a \in S^m$ we need to show that if $m - k' \min(\rho, 1- \delta) < -N$, then $I_k(a, \phi) = I_{k'}(a, \phi)$.  We can prove this similarly to the uniqueness argument, since we have a sequence $a_j \to a$ in $S^{m'}$ for $m' > m$ and $I_k(a_j, \phi) = I_{k'}(a_j, \phi) = I(a, \phi)$.
\end{proof}
\subsection{Some Remarks}
\begin{itemize}
\item $I(a, \phi) = \lim_{\epsilon \to 0} \int e^{i\phi(x, \theta)} a(x, \theta) \chi(\epsilon \theta)\,d\theta$.  This follows from the density of $a_\epsilon \to a$ in $S^{-\infty}$.
\item  Suppose $V = \sum a_j \partial_{\theta_j} + \sum b_\ell \partial_{x_\ell} + c$, $a_j \in S^0$, $b_{\ell}, c \in S^{-1}$.  Then, $\iint {}^t V(e^{i\phi}) au = \iint e^{i\phi} V(au)$, with the understanding that these are oscillatory integrals.  
\item We can use the above remark to prove things like smoothness of oscillatory integrals away from points.  For example, take $(2\pi)^{-n}\int e^{ix\theta}\,d\theta = \delta_0(x)$.  We can take $V = \frac{(1-\chi(x)) x\partial_\theta}{|x|^2} \in S^0$ and it follows that $\int e^{ix\theta}\,d\theta \equiv 0$ away from $0$.  
\end{itemize}
\pagebreak
\section{April 1st, 2021}
\subsection{Oscillatory Integrals, continued}
Recall that $\bigcup_m S_{\rho, \delta}^m \ni a \mapsto I(a, \phi) \in \mc D'(X)$.  We showed that for $a \in S^{-\infty}$, we had $I(a, \phi) = \int a(x, \theta) e^{i\phi(x, \theta)}\, d\theta \in C^\infty(X)$, with the assumption that $\phi$ is a non-degenerate phase function.

We mentioned last time that if we have $\iint {}^tL(e^{i\phi}) au \,d\theta dx = \iint e^{i\phi} L(au)\,d\theta dx$ for any $u \in C_c^\infty(X)$ where $L = \sum a_j\partial_{\theta_j} + \sum b_k \partial_{x_k} + c$, $a_j \in S^0, b_k \in S^{-1}, c \in S^{-1}$.
\begin{definition}Given a phase function $\phi$, define $C_\phi = \{(x, \theta): d_\theta \phi = 0\}$.
\end{definition}
If we take $\delta_0(x)$, $\phi = x\ cdot \theta$, then $C_\phi = \{(0, \theta): \theta\in \R^n\}$.
\begin{lemma} Suppose $a \in S^{m}_{\rho, \delta}$, $0 < \rho $, $\delta < 1$, $a \equiv 0$ in a conic neighborhood $C_\phi$. Then $I(a, \phi) \in C^\infty(X)$.
\end{lemma}
\begin{definition} A conic neighborhood of $\theta_0 \in \dot{\R}^N$ is a set $\theta$ satisfying $|\frac{\theta}{|\theta|} - \frac{\theta_0}{|\theta_0|}| < \epsilon$ for some $\epsilon$.
\end{definition}
\begin{proof}
Claim: there exists $L = \sum a_j \partial_{\theta_j} + c$, $a_j \in S^0$, $c \in S^{-1}$ such that ${}^t L(e^{i\phi}) = (1 - b)e^{i\phi}$, $b \in S^0$ with $\supp{b} \cap \supp{a} =  \emptyset$.

Namely, choose $b$, which is homogeneous of degree $0$, $|\theta| > 1$ in a conic neighborhood of $C_\phi$ and $b = 0$ on the support of $a$.  We can simply take $b(x, \theta)$, $\theta \in S^{N-1}$ so that $b = 1$ near $C_\phi \cap S^{N-1}$.  If $\supp{b}$ is close enough to $C_\phi$, this implies the disjoint supports by the assumption that $a \equiv 0$ on the conic neighborhood of $C_\phi$.

Then, we construct ${}^tL = (1 - b) \frac{1}{|\phi \theta|^2} \<\phi_\theta, \partial_\theta\>$.  This is fine because $b \equiv 1$ on the set where $\phi_\theta$ vanishes.  We also know that $(1-b)\partial^\alpha a = \partial^\alpha a$ and $\partial^\alpha b \partial^\beta a = 0$.  This implies that $I(a, \phi) = I(L^k a, \phi)$ for all $k$ where we have $L(au) = (La) u$.  Furthermore, note that $L^k a \in S^{m - k\min(\rho, 1-\delta)}$, so we can make this arbitrarily small for large $k$, which implies that $I(a, \phi) \in C^m(X)$ for all $m$.


\end{proof}
\begin{theorem} $\operatorname{sing} \supp I(a, \phi) \subset \pi(C_\phi)$ where $\pi: X \times \dot \R^N \to X$, $\pi(x, \theta) = x$.
\end{theorem}
\begin{remark} Recall that for $u \in \mc D'(X)$, $\operatorname{sing} \supp u = \{x: \exists U = nbhd(x), a\vert_{U} \in C^\infty(U)\}^c$.
\end{remark}
\begin{proof}
Suppose $x_0 \not \in \pi(C_\phi)$.  Then, there exists $\psi(x)$ such that $\psi(x_0) = 1$, $\supp \psi \cap \pi(C_\phi) = \emptyset$(this exists because $C_\phi$ is closed, so its projection is closed).  Then $\psi I(a, \phi) = I(\psi a, \phi)$.  Now, $\supp \psi a \cap C_\phi = \emptyset$.  Now, we apply the lemma.  
\end{proof}
Some examples
\begin{itemize}
\item For $f \in C^\infty(X)$, $Im(f) \ge 0$ and $f(x) = 0$ implies that $df(x) \ne 0$(this implies that $\{x; f(x) = 0\}$ is a $C^\infty$ hypersurface). Define $u(x) = \int_0^\infty e^{if(x) \tau} \,d\tau$.  Strictly speaking, this should be $\int_0^\infty \chi(\tau) e^{if(x)\tau} \,d\tau + \int_{\R} (1 - \chi(\tau)) e^{if(x) \tau} \,d\tau$, which is now $I(1 - \chi, f(x)\tau) \in S^0$.  

Note that $\int_0^\infty e^{i(f(x) + i\epsilon) \tau} \,d\tau = \frac{i}{f(x) + i\epsilon}$ converges to $u(x) \in \mc D'(X)$. It follows that $u(x) = i(f(x) )^{-1}$ and $C_\phi = \{(x, \tau): f(x) = 0\}$.  
\item We wish to solve $(\partial_t^2 - \Delta) u = 0$, $u\vert_{t = 0} = f$, $\partial_t u \vert_{t = 0} = 0$, $f \in S'(\R^n)$.   Then, $u(t, x) = (2\pi)^{-n} 1/2\iint  \sum_{\pm} e^{i(x - y) \xi  \pm |\xi|t} f(y) \,dy d\xi$.  We can check this is a solution to the wave equation.

Define $U(t, x, y) = (2\pi)^{-n} \int \frac{1}{2} \sum_{\pm} e^{i(x-y) \xi \pm i |\xi| t}\,d\xi \in \mc D(\R \times \R^n )$, since we can check directly that $\phi_{\pm}(t, x, y, \xi)$ is a phase function.  Note that $C_{\phi_{\pm}} = \{|x - y| = t\}$, which gives the light cone.  
\end{itemize}
\subsection{Fourier Integral Operators and Pseudodifferential Operators}
Suppose that $\phi \in C^\infty(X \times Y \times \R^N)$, $X \subset \R^n$ $Y \subset \R^m$ is a phase function in all variables.  Namely, $X \times Y$ is the "old $X$" and $a \in S_{\rho, \delta}^m(X \times Y \times \R^N)$, $\rho > 0$, $\delta < 1$.  We obtain $K(x, y) = \int a(x, y, \theta) e^{i \phi(x, y, \theta)}\, d\theta \in \mc D'(X \times Y)$.  $K$ defines $A: C_c^\infty(Y) \to \mc D'(X)$ via $u \in C_c^\infty(Y)$, $c \in C_c^\infty(X)$, and $\<Au, v\> := \<K, v \otimes u\>$, where $v \otimes u(x, y) = v(x) u(y) \in C_c^\infty(X \times Y)$.  

Formally, we write $Au(x) = \iint a(x, y, \theta) e^{i\phi(x, y, \theta)} u(y)\,dy d\theta$ which is understood as an oscillatory integral when paired with $v$.  This is called the \textbf{Fourier Integral Operator}.

For $\theta \in \R^n$, $X = Y \subset \R^n$, $\phi(x, y, \theta) = (x - y) \cdot \theta$, we have a \textbf{Pseudodifferential Operator}.

Suppose we have a differential operator $P(x, D) = \sum_{|\alpha| \le m} a_\alpha(x) D_x^\alpha$, $a-\alpha \in C^\infty(X)$.  We can write
$$Pu(x) = (2\pi)^{-n} \iint \sum a_\alpha(x) \xi^\alpha u(y) \,dy d\xi.$$
Then, $a(x, \xi) = \sum a_\alpha \xi^\alpha \in S_{1, 0}^m(X \times \R^n)$. 
 
\begin{theorem}
Suppose $A$ is an FIO, we have the following:
\begin{itemize}
\item If for every $x \in X$, $(y, \theta) \mapsto \phi(x, y, \theta)$ is a phase function, then $A: C_c^\infty(Y) \to C^\infty(X)$.
\item If for every $y \in Y$, $(x, \theta) \mapsto \phi(x, y, \theta)$ is a phase function, then $a: \mc E'(Y) \to \mc D'(X)$.
\end{itemize}
\end{theorem}
\pagebreak
\section{April 6th, 2021}
\subsection{Operators on Oscillatory Integrals}
Take $X \subset \R^n$, $Y \subset \R^m$, $a \in S_{\rho, \delta}^m(X \times Y \times \R^N)$, $\phi = \phi(x, y, \theta)$ a non-degenerate phase function.  Assuming $\rho > 0$, $\delta < 1$, this gives a distribution $K = I(a, \phi) \in \mc D'(X \times Y)$.  Furthermore, this defines an continuous operator $A: C_c^\infty(Y) \to \mc D'(X)$ given for $u \in C_c^\infty(Y)$, $v \in C_c^\infty(X)$ by
$$\<Au, v\> = \<K, u \otimes v\>$$
where $u \tensor v = u(x)v(y) \in C_c^\infty(X \times Y).$  These are called \textbf{Fourier Integral Operators}. 
\begin{itemize}
\item $K_\pm(t, x, y) = (2\pi)^{-n} \int e^{i \<\xi, x-y\>} e^{\pm i|\xi| t}$ for $t \in \R$, $x, y\in \R^n$.
\item For $X = Y$, $\phi(x, y, \xi) = \<x-y, \xi\>$, this gives the \textbf{Pseudo-differential Operators}.  
\end{itemize}
\begin{remark} The map $a \mapsto I(a, \phi) \in \mc D'(X \times Y)$ is continuous, so that for $u, v \in C_c^\infty(W), W \Subset X \times Y$,
$$|\<K, u \tensor v\>| \le C \sum_{|\alpha| \le N, |\beta|\le N} \sup |\partial_x^\alpha \partial_y^\beta (uv)|.$$
\end{remark}

\begin{theorem}
Suppose $A$ is an FIO, we have the following:
\begin{itemize}
\item If for every $x \in X$, $(y, \theta) \mapsto \phi(x, y, \theta)$ is a non-degenerate phase function($d_{y, \theta}\phi \ne 0$), then $A: C_c^\infty(Y) \to C^\infty(X)$.
\item If for every $y \in Y$, $(x, \theta) \mapsto \phi(x, y, \theta)$ is a non-degenerate phase function($d_{x, \theta} \ne 0$), then $a: \mc E'(Y) \to \mc D'(X)$.
\end{itemize}
\end{theorem}
\begin{example} For a pseudo-differential operator, we have $\phi = \<x - y, \xi\>$ and $(y, \xi) \mapsto \<x-y, \xi\>$ and $(x, \xi) \mapsto \<x-y, \xi\>$ are non-degenerate.  
\end{example}
\begin{proof}
Take $\Phi = |d_y\phi|^2 + |\theta|^2 |d_\theta \phi|^2$, which is nonzero for $X \times Y \times \dot{\R}^N$.  This is homogeneous of degree $2$.  Let $\chi$ be the usual bump function and set $${}^t L = \frac{1 - \chi}{i\Phi}(\<\partial_y \phi, \partial_y\> +|\theta|^2 \<\partial_\theta \phi, \partial_\theta\>) + \chi.$$

Then, $L = \<A, \p_y\> + \<B, \p_\theta\> + c$ for $A \in S^{-1}$, $B \in S^{0}$, $c \in S^{-\infty} \subset S^{-1}$.  We showed that for operators of this form, we can integrate by parts.  Namely,
$$\<Au, v\> = \int e^{i\phi} v(x) L^k(au) \,dxdyd\theta.$$
Then, $L^k \in S^{m - k\min(\rho, 1-\delta)}$.  It follows that $Au = \int e^{i\phi} L^k(au) \,dy d\theta \in C^\infty(X)$.

For the second part, note that for any $A: C_c^\infty(Y) \to \mc D'(X)$, we define ${}^tA: C_c^\infty(X) \to \mc D'(Y)$ by $\<{}^t A v, u\> = \<Au, v\>.$  If $A$ is defined using $K_A \in \mc D'(X \times Y)$, this implies that ${}^tA$ is defined by $K_{{}^tA}(y, x) = K_A(x, y)$.  This means that if $K_A = I(a, \phi)$, then $K_{{}^t a} = I(\tilde{a}, \tilde{\phi})$ where $\tilde{a}(y, x, \theta) = a(x, y, \theta)$ and $\tilde{\phi}(y, x, \theta) = \phi(x, y, \theta)$.  But now, the condition that $(x, \theta) \mapsto \phi(x, y, \theta)$ is non-degenerate implies that $K_{{}^t A} : C_c^\infty(X) \to \mc C^\infty(Y)$.   It follows that $A: \mc E'(Y) \to \mc D'(X)$.  For $u \in \mc E'(Y)$, $v \in C_c^\infty(X)$, we define $\<Au, v\> = \<{}^t A v, u\>$.

\end{proof}


\subsection{Method of Stationary Phase}
We defined $C_\phi = \{(x, \theta) : \phi_\theta =0 \}$.  We proved that $\operatorname{sing}\supp I(a, \phi) \subset \pi(C_\phi)$ where $\pi(x, \theta) = x$.  One could ask, when do we have equality?  To do this, we need to "evaluate" our oscillatory integrals.  This is the \textbf{Method of Stationary Phase}.



Define $$I(\lambda) = \int e^{i \lambda \phi(x)}a(x)\,dx,$$ for $a \in C_c^\infty(\R)$, $\phi \in C^\infty(\R, \R)$.  We could also take
$$J(\lambda) = \int e^{-\lambda \psi(x)} a(x)\,dx.$$
The study of the first is the stationary phase method and the study of the second is the steepest descent method.

\subsection{Steepest Descent}
Recall from lecture 1, we were stuck on the integral $\int_0^{2\pi} e^{-\frac{1}{2\epsilon} |\xi|^2 \cos^2 \theta} \,d\theta$ as $\epsilon \to 0$.  The Method of Steepest Descent will give us the asymptotic expansion of this.   

For $a \in C_c^\infty$, $\psi$ has a unique non-degenerate minimum. 

$\int e^{-\lambda \psi(x)}a(x) \,dx = e^{-\lambda \psi(x_0)}((2 \pi \lambda \psi ''(x_0) )^{-1/2} a(x_0) + b_1 \lambda^{-1/2 - 1} + b_2 \lambda^{-1/2 - 2} + \dots)$.  This means that for all $N$, there exists $C$ such that 
$$| \int e^{-\lambda \psi(x)} a(x)\,dx - e^{-\lambda \psi_0(x_0)}(2 \pi \lambda \psi ''(x_0) )^{-1/2} a(x_0) + b_1 \lambda^{-1/2 - 1} + b_2 \lambda^{-1/2 - 2} + \dots )| \le Ce^{-\lambda \psi(x_0)} \lambda^{-N - 5/2}.$$  

If we take $\chi \in C_c^\infty(\R)$, $\chi$ with support close to $x_0$, 
$$|\int e^{-\lambda \psi(x)}(1 - \chi(x)) a(x)\,dx| \le  e^{-\lambda(\psi(x_0) + \epsilon)}.$$

By Taylor's Formula, $\psi(x) = 1/2(x - x_0)^2 \psi_1(x)$, where $\psi(x_0) = \psi''(x_0)$.  Then, take $y = y(x) = (x - x_0) (\psi_1(x))^{1/2}$ for $x$ near $x_0$.  Since $y'(x_0) \ne 0$, we can write $x = x(y)$, for $y$ near 0.  Finally, $\phi(x(y)) = y^2/2$, so we have
$$\int e^{-\lambda \psi(x)} \chi(x) a(x)\,dx = \int e^{-1/2 \lambda y^2}b(y)\,dy$$
where $b(y) = \chi(x(y)) a(x(y)) |dx/dy|$.  We know that $b(0) = a(x_0)/(\psi''(x_0))^{1/2}$.  We are reduced to studying 
$$\int e^{-1/2 \lambda y^2} b(y)\,dy.$$

Then, $\int u\ol v = (2\pi)^{-1} \int \hat{u} \ol{\hat{v}}$, Plancherel's formula.  Then ,$u = b(y)$, $v = e^{-1/2 \lambda y^2}$.  The Fourier transform of $e^{-1/2 \lambda \cdot ^2} = \frac{\sqrt{2\pi}}{ \lambda^{1/2}} e^{-\xi^2/2\lambda}$, so it follows that 
$$\int e^{-1/2 \lambda y^2} b(y)\,dy = (2\pi \lambda)^{-1/2} \int e^{-\xi^2/2\lambda} \hat{b}(\xi)\,d\xi.$$

We wish to consider this as $\lambda \to \infty$.  By expanding the Taylor series of the exponential term, we have
$$(2\pi\lambda)^{-1/2} \sum \int \frac{1}{k!} (-1/2\lambda \xi^2)^k \hat{\beta}(\xi)d\xi = (2\pi\lambda)^{-1} \sum \frac{(-1/2\lambda)^k}{k!} \int \xi^{2k} \hat{b}(\xi) d\xi.$$
Recall that $\int \xi^{2k} \hat{b}(\xi) = (2\pi) D_x^{2k} b(0)$.
\pagebreak
\section{April 8th, 2021}
\subsection{Steepest Descent}
Recall for Steepest Descent, we had 
$$J(\lambda) = \int \chi(x) e^{-\lambda \psi(x)}a(x)\,dx + O(e^{-\lambda(\psi(x_0) + a)})$$
for $\chi \in C_c^\infty$ a smooth bump function near $x_0$.  We then wrote $\psi(x) = \psi(x_0) + 1/2 y(x)^2$, which gives a nice function $x \mapsto y(x)$ with $y(x_0) = 0$ and $y'(x_0) \ne 0$.

Applying this, we had $J(\lambda) = e^{-\lambda \psi(x_0)}\int b(y) e^{-\lambda y^2/2}\,dy$, $b \in C_c^\infty$ near $0$ and $b(x_0) = \frac{a(x_0)}{(\psi(x_0))^{1/2}}$.

Then, via Plancherel, we obtained
$$J(\lambda) = (2\pi)^{-1/2} \int \hat{b}(\xi) e^{-\xi^2/2\lambda} dy/\sqrt{\lambda}.$$
which has a finite expansion
$$(2\pi)^{-1/2} \int \hat{b}(\xi) \sum_{k=0}^{N-1} 1/k! (-1/2\lambda)^k \xi^{2k}\,d\xi/\sqrt{\lambda} + C \int \hat{b}(\xi) \xi^{2N} e^{-1/2(\theta \xi)^2} d\xi/\sqrt{\lambda}.$$

Then, note that 
$\int \xi^{2k} \hat{b}(\xi) = 2\pi D_y^{2k}b(0)$, so we obtain
$$(2\pi)^{1/2}/\sqrt{\lambda} \sum_0^{N-1} 1/k! (2\pi)^{-k} \partial_y^{2k}b(0) + C_N \lambda^{-N - 1/2}\| \hat{(D^{2n}b)} \|_1.$$

Then, we estimate
$$\|\hat{\partial^{2N} b}\|_1 \le c (\| \partial^{2n + 2}b \|_{L^1}  + \| \partial^{2N} b\|_{L^1}) \le C \sup_{|\alpha| \le 2N + 2} |\partial^\alpha b|.$$


$$J(\lambda) = e^{-\lambda \psi(x_0)} (2\pi)^{1/2} \sum_{k=0}^{N-1} 1/k! (2\pi)^{-k} \partial_y^{2k} b(0) + O(\lambda^{-N - 1/2}).$$

Putting this together, If $\psi$ has a unique non-degenerate minimum at $x_0$, $\psi'(x_0) = 0$, $\psi''(x_0) > 0$, $a \in C_0^\infty(\R)$, we gave
$$\int a(x) e^{-\lambda\psi(x)}\, dx = e^{-\lambda \psi(x_0)} (2\pi/\psi''(x_0))^{1/2} 1/\sqrt{\lambda} (a(x_0) + \lambda^{-1} a_1 + \dots + \lambda^{-N-1} a_{N-1} + O(\lambda^{-N - 1/2})).$$

\subsection{Stirling's Formula}
Recall the Gamma function
$$\Gamma(s) = \int_0^\infty e^{-t} t^s dt/t.$$
We wish to examine this as $s \to \infty$.

We rewrite this as 
$$e^{-t} t^s dt/t = e^{-t + s\log t} dt/t = e^{s \log x}e^{-s(x \log x)}dx/x.$$

Then, $\Gamma(s) = s^s \int_0^\infty e^{-s(x - \log x)}\frac{dx}{x}.$  The phase $\psi(x) = x - \log x$.  Note that it achieves a minimum at $x = 1$ and $\psi''(x) > 0$.

Take $\chi$ to be a bump function at $1$.  Then,
$$\int_0^1 (1 - \chi(x)) e^{-s(x - \log x)} dx/x - O(e^{-(s(1 + a))}),$$
where the $a$ is the point on the phase where $\chi$ removes the support.  Similarly,
$$\int_1^\infty (1 - \chi(x)) e^{-s(x - \log x)}dx/x = O(e^{-s(1 + b)}).$$

We see that 
$$\Gamma(s) = s^s[\int \chi(x)e^{-s(x - \log x)} \,dx/x + O(e^{-s \min(1 + a, 1 + b)})].$$
The first term is exactly suitable for the steepest descent method.  

Namely, $\psi = x - \log x$, $a(x) = 1/x$.  $\psi'(1) = 0$, $\psi''(1) = 1$, $a(1) = 1$, so we obtain the estimate
$$\Gamma(s) = s^s e^{-s}[ \sqrt{2\pi}s^{-1/2} + a_1 s^{-3/2} + \dots + O(s^{-N - 1/2})].$$
The leading term is exactly Stirling:
$$\Gamma(s) = \sqrt{2\pi} s^{s - 1/2} e^{-s}(1 + O(1/s)).$$


\subsection{Stationary Phase}
Take $I(\lambda) = \int e^{i\lambda \phi(x)} a(x)\,dx$, $\phi \in C^\infty(\R; \R)$, $a \in C_c^\infty(\R)$.  
\begin{lemma} Suppose $|\psi'(x)| > 0$ on $\supp(a)$.  Then,
$$|I(\lambda)| \le C \sup_{|\alpha| \le N} |\partial^\alpha a| \lambda^{-N}.$$
\end{lemma}
\begin{proof}
Define ${}^t L = \frac{1}{i|\phi'|^2} \psi' \cdot \partial_x$.  Note that $1/\lambda {}^t L (e^{- \lambda \phi}) = e^{- \lambda \phi}$.  It follows that $$I(\lambda) = \lambda^{-N} \int e^{i \lambda \phi} (L^N a)(x)\,dx \le |\supp a| \lambda^{-N} \sup|L^N a| \le C_N \lambda^{-N} \sup_{|\alpha| \le N} |\partial^\alpha a|.$$
\end{proof}
\begin{remark} The converse is not known for higher dimensions.  
\end{remark}
This means that the contributions to the integral come from stationary points.
\begin{theorem}[Stationary Phase(dim $1$)] Suppose $\supp a \subset (\alpha, \beta)$, $a \in C_0^\infty$ and $\phi$ has a unique critical point in $(\alpha, \beta)$, $\phi'(x_0) = 0$, $\phi''(x_0) \ne 0$.  Then $$I(\lambda) = \frac{\sqrt{2\pi}}{|\phi''(x_0)|^{1/2}} e^{i\pi/4 sgn \phi''(x_0)} \lambda^{-1/2}(a_0 + \lambda^{-1} a_1 + \lambda^{-2}a_2 + \dots + \lambda^{-N-1} a_{N - 1}) + \lambda^{N - 1/2}S_N,$$
where $a_0 = a(x_0)$ and $|S_n| \le C_N \sup_{|\alpha| \le 2N + 2} |\partial^\alpha a|$.

\end{theorem}
\end{document}







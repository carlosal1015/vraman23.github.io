\documentclass[12pt]{scrartcl}
\usepackage[sexy]{evan}
\usepackage{graphicx}

\usepackage{answers}
\Newassociation{hint}{hintitem}{all-hints}
\renewcommand{\solutionextension}{out}
\renewenvironment{hintitem}[1]{\item[\bfseries #1.]}{}
\declaretheorem[style=thmbluebox,name={Theorem}]{thm}

 %Sets
\newcommand{\N}{\mathbb{N}}
\newcommand{\Z}{\mathbb{Z}}
\newcommand{\F}{\mathbb{F}}
\newcommand{\Q}{\mathbb{Q}}
\newcommand{\R}{\mathbb{R}}
\newcommand{\C}{\mathbb C}
\newcommand{\T}{\mathbb T}
\renewcommand{\hat}{\widehat}
\newcommand{\<}{\langle}
\renewcommand{\>}{\rangle}


\let \phi \varphi
\let \mc \mathcal
\let \ov \overline
%From Topology
\newcommand{\cT}{\mathcal{T}}
\newcommand{\cB}{\mathcal{B}}
\newcommand{\cC}{\mathcal{C}}
\newcommand{\cH}{\mathcal{H}}

\newcommand{\supp}{\text{supp }}

\newcommand{\aint}{\mathrel{\int\!\!\!\!\!\!-}}
\let \grad \nabla


\begin{document}
\title{Math 222b}
\author{Vishal Raman}
\thispagestyle{empty}
$ $
\vfill
\begin{center}

\centerline{\huge \textbf{Math 222b Lecture Notes}}
\centerline{\Large \textbf{Partial Differential Equations II} } 
\centerline{Professor: Maciej Zworski, Spring 2021}
\centerline{Scribe: Vishal Raman}
\end{center}
\vfill
$ $
\newpage
\thispagestyle{empty}
\tableofcontents
\newpage
%\maketitle
\section{January 19th, 2021}
\subsection{Review of Sobolev Spaces}
\begin{definition} Given $u \in \mc D'(U)$ for $U \subseteq \R^n$ open: that means that $u : C_c^\infty(U) \to C$ and for every compact set $K \subset \subset U$, $\exists C, N$ for all $\phi \in C_0^\infty(K)$ such that
$$|u(\phi)| \le C \sup_{|\alpha| \le N} |\partial^\alpha \phi|.$$ 
\end{definition}

Examples:
\begin{itemize}
\item Take $U =(0, 1)$ and take $u = \sum_{\N} \delta_{1/n}$, where $\delta_{1/n}(\phi) = \phi(1/n)$.  
\item Take $u \in L_{\text{loc}}^1(U)$, where $u(\phi) = \int u\phi$.  Differentiation is defined formally though integration by parts as $\partial^\alpha u(\phi) = (-1)^{|\alpha|} u(\partial^\alpha \phi).$
\end{itemize}

\begin{definition} The Sobolev spaces $W^{k, p}(U) = \{u \in L_{loc}^1(U) : \partial^\alpha u \in L^p(U), \forall |\alpha | \le k\}$, for $k \in \N_0$, $1 \le p \le \infty$.  Note that differentiation is in the sense of distributions.  We write $H^k(U) = W^{k, 2}(U)$, which are Hilbert spaces with the inner product$$\langle u, v\rangle = \sum_{|\alpha| \le k} \int_U \partial^\alpha u \overline{\partial^{\alpha} v}.$$
\end{definition}

\begin{definition} $W_0^{k, p}(U) = \overline{C_c^\infty(U)}$, where the closure is with respect to the $W^{k, p}$ norm.
\end{definition}

\begin{thm}[Approximation] For $U \subset \subset \R^n$, 
$$\overline{C^\infty(U) \cap W^{k, p}(U)} = W^{k, p}(U)$$
where the closure is with respect to the $W^{k, p}$.

If $\partial U \in C^1$, then we can improve up to 
$$\overline{C^\infty(\overline{U}) \cap W^{k, p}(U)} = W^{k, p}(U)$$
\end{thm}

\begin{thm}[Extension] If $U \subset \subset \R^n$
 and $\partial U \in C^1$, for $U \subset \subset V \subset \subset \R^n$, there exists $E: W^{1, p}(U) \to W^{1, p}(\R^n)$ such that $Eu\vert_{U} = u$ and the $\supp u \subset \subset V$.
 
We can extend this to $W^{k, p}$ if the boundary is $C^k$.
\end{thm}

\begin{thm}[Traces] For $U \subset \subset \R^n$ with $\partial U \in C^1$, there exists $T: W^{1, p}(U) \to L^p(\partial U)$ which is linear and boundary such that for $u \in C(\overline{U}) \cap W^{1, p}$ $Tu = u\vert_{\partial U}$.
\end{thm}

\begin{example} For $U \subset \subset \R^n$, $\partial U$ bounded, $$H_0^1(U) = \{u \in H^1 : Tu = 0 \in L^2(\partial U)\}.$$

The converse of showing $Tu = 0$ implies $H_0^1$ is the more difficult one.  
\end{example}

\subsection{Fourier Transform}
We first review the Fourier Transform.  We define the Schwartz space: $$\mc S = \{\phi \in C^\infty(\R^n) : x^\alpha \partial^\beta \phi \in L^\infty \forall \alpha, \beta \in \N^n\}.$$

For $\phi \in \mc S$, we define 
$$\hat{\phi} (\xi) = \int \phi(x) e^{-ix \cdot \xi}\, dx.$$

Note that $\mc F$, the Fourier transform is invertible on $\S$.  The key properties of the fourier transform are
$$\mc F (1/i \partial x \phi) = \xi \mc F \phi, F(x \phi) = -1/i \partial_{\xi} \mc F \phi.$$
We also have 
$$\mc F^{-1} = \frac{R\mc F}{(2\pi)^n}, R\phi(x) = \phi(-x).$$

We define $\mc S'$ onto $\C$ so that for $u \in \mc S'$, there exists $C, N$ such that 
$$|u(\phi)| \le C \sup_{|\alpha|, |\beta| \le N} |x^\alpha \partial^\beta \phi|.$$

Note that $\mc S' \subset \mc D'$.  

\begin{definition} $\mc F: \mc S' \to \mc S'$ by $\hat{u}(\phi) = u(\hat{\phi})$.
\end{definition}
Examples:
\begin{itemize}
\item $\hat{\delta_0}(\phi) = \delta_0(\hat{\phi}) = \hat{\phi}(0) = \int \phi = 1(\phi).$
\item Take $\R^2$ and consider $u(x) = \frac{1}{|x|}$.  This function is in $L_{loc}^1$.  If we multiply by $(1 + |x|)^{-2} u \in L^1(\R^n)$, it follows that $u \in \mc S'$, since 
$$|u(\phi)| = \left |\int (1 + |x|)^{-2} u (1 + |u|)^2 \phi\right | \le C \sup (1 + |x|)^2 \phi.$$

Now, we compute $\hat{u} \in \mc S'$.  Since $\mc F$ is continuous on $\mc S'$, we approximate $u$ and hope the result converges to the desired result.  Define $u_\epsilon \to u$ in $\mc S'$ for $u_\epsilon \in L^1$.  

Try $u_\epsilon(x) = \frac{e^{-\epsilon |x|^2/2}}{|x|} \in L^1$ for $\epsilon > 0$.  We want to calculate $\hat{u}_\epsilon$ and take the limit as $\epsilon \to 0^+$.   We can evaluate the integral by converting to polar coordinates and completing the square.  Unfortunately, it reduces to an integral that is too hard, but we will learn asymptotics of the integral as $\epsilon \to 0$.  We find that $\hat{u}(\xi) = 2\pi/|\xi|$.

We can approach this differently.  Note that $u = 1/|x|$ is homogeneous: $u(tx) = t^a u(x)$ for $t > 0$, for functions.  For distributions, we have that for $\phi \in S$, $u(\phi(\cdot / t) t^{-n}) = t^a u(\phi)$ for $t > 0$.  For the Fourier Transform, if $u \in \mc S'(\R^n)$ is homogeneous of degree $a$, then $\hat{u}$ is homogeneous of degree $-n-a$.  It follows that our Fourier transform is of degree $-1$.  

Furthermore, note that $1/|x|$ is spherically symmetric, and the Fourier transform preserves spherical symmetry(note that the Jacobian factor for rotations is $1$).  It follows that the fourier transform is also spherically symmetric.  It follows that 
$$\mc F(1/|x|) = C/|\xi| + \sum_{|\alpha \le N| }c_\alpha \delta_0^{(\alpha)},$$
but delta terms have too much homogeneity.  
\end{itemize}

\pagebreak
\section{December 21st, 2021}
\subsection{Plancherel's Theorem}
Recall that the Fourier transform is an isomorphism on $\mc S$ - it is a bounded linear operator whose inverse is also bounded.  

Note that 
$$\int \hat{u}(\xi) \ov{\hat{\phi}(\xi)}d\xi = \iiint u(x)\ov{\phi(y)} e^{-i(x-y)\xi}\,dxdyd\xi$$

In the sense of distributions, $\int e^{-i(x-y)\xi}\,d\xi = (2\pi)^n \delta(x-y).$
Hence,
$$\iiint u(x)\ov{\phi(y)} e^{-i(x-y)\xi}\,dxdyd\xi = (2\pi)^n \int u(x)\ov{\phi(x)}\,dx.$$

For $u, \phi \in \mc S$, we have the following: $$\< \hat{u}, \hat{\phi}\> = (2\pi)^n \< u, \phi \>.$$

This implies that 
$$\|\hat{u}\|_2 = (2\pi)^{n/2} \|u\|_{2}, u \in \S.$$
If $u_n \to u$ in $L^2$ then $u_n \to u$ in $\mc S'$ by the Cauchy-Schwartz inequality.  It follows that $\hat{u_n} \to \hat{u}$ in $\mc S'$ but our formula shows that $\hat{u}$ is in $L^2$.  Hence, $\mc F: L^2 \to L^2$ and for $u, v \in L^2$, $\<\hat{u}, \hat{v} \> = (2\pi)^n \<u, v\>$.

Recall last time, we were finding the Fourier transform of $u(x) = 1/|x|$ in $\R^2$.    For $u \in S'(\R^n)$ homogeneous of degree $a$, $\hat{u} \in \mc S'(\R^n)$ is homogeneous of degree $-n-a$.  In our example, It follows that $\hat{u}(\xi)$ is homogeneous of degree $-1$.  We also observed that $u$ is invariant under rotations so it follows that $\hat{u}$ is invariant under rotations.  

A function is homogeneous of degree $-1$ if $v(k\theta) = \frac{a(\theta)}{r}.$  Since our function is invariant under rotations, $\hat{u}(\xi) = \frac{c}{|\xi|}$ away from zero.  It follows from our previous argument that $\hat{u}(\xi) = \frac{c}{|\xi|}$ since $\delta$ terms have homogeneity of at least $-2$.  

Note that $\<u, \phi\> = (2\pi)^2 \<\hat{u}, \hat{\phi}\>$ and we find $\hat{u}$ by choosing an appropriate $\phi$.

\begin{align*}
\int_{\R^2} \frac{\phi(x)}{|x|}\,dx &= \int_0^{2\pi} \int_{0}^\infty \phi(r) \, drd\theta \\
&= 2\pi \int_{0}^\infty \phi(r)\,dr.
\end{align*}
Choosing $\phi(r) = e^{-r^2/2}$, we find that the integral is $(2\pi)^{3/2}$.  

Evaluating the other side,  
$$\hat{\phi}(\xi) = \int_{\R^2} e^{-|x|^2/2 - ix\cdot \xi}\,dx = \int e^{-\frac{1}{2}(x + i\xi)^2 - \frac{1}{2} |\xi|^2} = 2\pi \int e^{-|\xi|^2/2} = (2\pi)^{5/2}.$$

It follows that $c = 2\pi$.
\subsection{Fourier Characterization of $H^k$ spaces}
\begin{thm} $H^k(\R^n) = \{u \in S'(\R^n) : (1 + |\xi|^2)^{k/2} \hat{u} \in L^2\}.$
\end{thm}
\begin{proof}
Suppose that $\partial^\alpha u \in L^2$ for $|\alpha| \le k$.  We know that $\|u\|_{2} = (2\pi)^{-n/2}\|\hat{u}\|.$  It follows that $\hat{\partial^\alpha u} \in L^2$.  Note that $\hat{\partial^\alpha u} = i^{|\alpha|} \\xi^{\alpha} \hat{u} \in L^2$ for all $|\alpha| \le k$.

Hence,
$$(1 + |\xi|^2)^{k/2} \le C_{n, k} \sup_{|\alpha| \le k} |\xi^\alpha|.$$

So it follows that $(1 + |\xi|^2)^{k/2} \hat{u} \in L^2$.

Now, suppose $(1 + |\xi|^2)^{k/2} \hat{u} \in L^2$.  It follows that $|\xi^\alpha| \le C_{k, \alpha}(1 + |\xi|^2)^{k/2}$ for $|\alpha| \le k$.  Hence $\xi^\alpha\hat{u} \in L^2$ so it follows that $\partial^{\alpha} u \in L^2$ by Plancherel's Theorem.  

\end{proof}

\begin{remark}We use the notation $\<\xi\> = (1 + |\xi|^2)^{1/2}$.
\end{remark}

Note that the definition does not require $k \in \N$.  
\begin{definition} $H^s(\R^n) = \{u \in \mc S' : \<\xi \>^s \hat{u} \in L^2\}, s \in \R$.
\end{definition}

\begin{thm} Suppose $u \in H^s(\R^n)$ and $s > \frac{1}{2}$.  Then $v(y) = u(0, y), y \in \R^{n-1}$ satisfies $v \in H^{s - 1/2} (\R^{n-1})$.
\end{thm}
\begin{remark} We should define $Tu(y) = u(0, y)$ for $u \in \mc S$.  Then $T: H^s(\R^n) \to H^{s - 1/2}(\R^{n-1})$ if $s > 1/2$.  
\end{remark}
\begin{proof}
Take $u \in \mc S$.  We wish to show that $\|v\|_{H^{s - 1/2}(\R^{n-1})} \le C\|u\|_{H^s(\R^n)}$.

Note that $$\hat{v}(\eta) = \int_{\R^{n-1}} u(0, y) e^{-y \cdot \eta} \,dy$$
and by the Fourier Inversion Formula
$$u(0, y) = (2\pi)^{-n}\int_{\R^n} \hat{u}(\xi_1, \xi') e^{iy \cdot \xi'}\,d\xi_1 d\xi',$$
so it follows that 
\begin{align*}
\hat{v}(\eta) &= (2\pi)^{-n}\int_{\R^{n-1}}\int_{\R^n} \hat{u}(\xi_1, \xi') e^{-iy \cdot (\eta - \xi')}\,d\xi dy \\
&= (2\pi)^{-n}\int_{\R^{n}}\int_{\R^{n-1}}\hat{u}(\xi_1, \xi') e^{iy \cdot (\xi' - \eta)}\, dy d\xi\\
&= (2\pi)^{-1} \int_{\R^n} \hat{u}(\xi_1, \xi') \delta_{\xi' = \eta}d\xi \\
&= (2\pi)^{-1} \int_{\R} \hat{u}(\xi_1, \eta) d\xi_1.
\end{align*}
 
Note that up to constants
 $$\|v\|_{H^{s-1/2}}^2 = \int_{\R^{n-1}} \<\eta\>^{2s-1} |\hat{v}(\xi)|^2\, d\eta = \int_{\R^{n-1}} \<\eta\>^{2s-1} \left |\int \hat{u}(\xi_1, \eta)\, d\xi_1\right |^2d\eta.$$
  
 Then,
 \begin{align*}
 \int_{\R^{n-1}} &\<\eta\>^{2s-1} \left |\int \hat{u}(\xi_1, \eta)\, d\xi_1\right |^2d\eta \\
 &= \int \<\eta\>^{2s-1} \left |\int \hat{u}(\xi, \eta)(1 + |\xi_1|^2 + |\eta|^2)^{s/2} (1 + |\xi_1|^2 + |\eta|^2)^{-s/2}d\xi_1  \right |^2d\eta \\
 & \le \int \<\eta\>^{2s-1} \int |\hat{u}(\xi_1, \eta)|^2 (1 + |\xi_1|^2 + |\eta|^2)^s d\xi_1 \int (1 + |\xi_1|^2 + |\eta|^2)^{-s} d\xi_1\, d\eta \\
&\le \int \<\eta\>^{2s-1} \<\eta\>^{-2s+1}  \int |\hat{u}(\xi_1, \eta)|^2 (1 + |\xi_1|^2 + |\eta|^2)^s d\xi_1\int (1 + u^2)^{-s} du d\eta\\
& = \int |\hat{u}(\xi)|^2 \<\xi\>^{2s} d\xi = \|u\|_{H^s}^2,
 \end{align*}
 
since
$$\int |\hat{u}(\xi_1, \eta)|^2 (1 + |\xi_1|^2 + |\eta|^2)^s d\xi_1d\eta = \int |\hat{u}(\xi)|^2 \<\xi\>^{2s}d\xi.$$

\end{proof}

\end{document}

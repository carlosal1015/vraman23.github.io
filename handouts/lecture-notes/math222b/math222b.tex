\documentclass[12pt]{scrartcl}
\usepackage[sexy]{evan}
\usepackage{graphicx}

\usepackage{answers}
\Newassociation{hint}{hintitem}{all-hints}
\renewcommand{\solutionextension}{out}
\renewenvironment{hintitem}[1]{\item[\bfseries #1.]}{}
\declaretheorem[style=thmbluebox,name={Theorem}]{thm}

 %Sets
\newcommand{\N}{\mathbb{N}}
\newcommand{\Z}{\mathbb{Z}}
\newcommand{\F}{\mathbb{F}}
\newcommand{\Q}{\mathbb{Q}}
\newcommand{\R}{\mathbb{R}}
\newcommand{\C}{\mathbb C}
\newcommand{\T}{\mathbb T}
\renewcommand{\hat}{\widehat}
\newcommand{\<}{\langle}
\renewcommand{\>}{\rangle}


\let \phi \varphi
\let \mc \mathcal
\let \ov \overline
%From Topology
\newcommand{\cT}{\mathcal{T}}
\newcommand{\cB}{\mathcal{B}}
\newcommand{\cC}{\mathcal{C}}
\newcommand{\cH}{\mathcal{H}}

\newcommand{\supp}{\text{supp }}

\newcommand{\aint}{\mathrel{\int\!\!\!\!\!\!-}}
\let \grad \nabla


\begin{document}
\title{Math 222b}
\author{Vishal Raman}
\thispagestyle{empty}
$ $
\vfill
\begin{center}

\centerline{\huge \textbf{Math 222b Lecture Notes}}
\centerline{\Large \textbf{Partial Differential Equations II} } 
\centerline{Professor: Maciej Zworski, Spring 2021}
\centerline{Scribe: Vishal Raman}
\end{center}
\vfill
$ $
\newpage
\thispagestyle{empty}
\tableofcontents
\newpage
%\maketitle
\section{January 19th, 2021}
\subsection{Review of Sobolev Spaces}
\begin{definition} Given $u \in \mc D'(U)$ for $U \subseteq \R^n$ open: that means that $u : C_c^\infty(U) \to C$ and for every compact set $K \subset \subset U$, $\exists C, N$ for all $\phi \in C_0^\infty(K)$ such that
$$|u(\phi)| \le C \sup_{|\alpha| \le N} |\partial^\alpha \phi|.$$ 
\end{definition}

Examples:
\begin{itemize}
\item Take $U =(0, 1)$ and take $u = \sum_{\N} \delta_{1/n}$, where $\delta_{1/n}(\phi) = \phi(1/n)$.  
\item Take $u \in L_{\text{loc}}^1(U)$, where $u(\phi) = \int u\phi$.  Differentiation is defined formally though integration by parts as $\partial^\alpha u(\phi) = (-1)^{|\alpha|} u(\partial^\alpha \phi).$
\end{itemize}

\begin{definition} The Sobolev spaces $W^{k, p}(U) = \{u \in L_{loc}^1(U) : \partial^\alpha u \in L^p(U), \forall |\alpha | \le k\}$, for $k \in \N_0$, $1 \le p \le \infty$.  Note that differentiation is in the sense of distributions.  We write $H^k(U) = W^{k, 2}(U)$, which are Hilbert spaces with the inner product$$\langle u, v\rangle = \sum_{|\alpha| \le k} \int_U \partial^\alpha u \overline{\partial^{\alpha} v}.$$
\end{definition}

\begin{definition} $W_0^{k, p}(U) = \overline{C_c^\infty(U)}$, where the closure is with respect to the $W^{k, p}$ norm.
\end{definition}

\begin{thm}[Approximation] For $U \subset \subset \R^n$, 
$$\overline{C^\infty(U) \cap W^{k, p}(U)} = W^{k, p}(U)$$
where the closure is with respect to the $W^{k, p}$.

If $\partial U \in C^1$, then we can improve up to 
$$\overline{C^\infty(\overline{U}) \cap W^{k, p}(U)} = W^{k, p}(U)$$
\end{thm}

\begin{thm}[Extension] If $U \subset \subset \R^n$
 and $\partial U \in C^1$, for $U \subset \subset V \subset \subset \R^n$, there exists $E: W^{1, p}(U) \to W^{1, p}(\R^n)$ such that $Eu\vert_{U} = u$ and the $\supp u \subset \subset V$.
 
We can extend this to $W^{k, p}$ if the boundary is $C^k$.
\end{thm}

\begin{thm}[Traces] For $U \subset \subset \R^n$ with $\partial U \in C^1$, there exists $T: W^{1, p}(U) \to L^p(\partial U)$ which is linear and boundary such that for $u \in C(\overline{U}) \cap W^{1, p}$ $Tu = u\vert_{\partial U}$.
\end{thm}

\begin{example} For $U \subset \subset \R^n$, $\partial U$ bounded, $$H_0^1(U) = \{u \in H^1 : Tu = 0 \in L^2(\partial U)\}.$$

The converse of showing $Tu = 0$ implies $H_0^1$ is the more difficult one.  
\end{example}

\subsection{Fourier Transform}
We first review the Fourier Transform.  We define the Schwartz space: $$\mc S = \{\phi \in C^\infty(\R^n) : x^\alpha \partial^\beta \phi \in L^\infty \forall \alpha, \beta \in \N^n\}.$$

For $\phi \in \mc S$, we define 
$$\hat{\phi} (\xi) = \int \phi(x) e^{-ix \cdot \xi}\, dx.$$

Note that $\mc F$, the Fourier transform is invertible on $\S$.  The key properties of the fourier transform are
$$\mc F (1/i \partial x \phi) = \xi \mc F \phi, F(x \phi) = -1/i \partial_{\xi} \mc F \phi.$$
We also have 
$$\mc F^{-1} = \frac{R\mc F}{(2\pi)^n}, R\phi(x) = \phi(-x).$$

We define $\mc S'$ onto $\C$ so that for $u \in \mc S'$, there exists $C, N$ such that 
$$|u(\phi)| \le C \sup_{|\alpha|, |\beta| \le N} |x^\alpha \partial^\beta \phi|.$$

Note that $\mc S' \subset \mc D'$.  

\begin{definition} $\mc F: \mc S' \to \mc S'$ by $\hat{u}(\phi) = u(\hat{\phi})$.
\end{definition}
Examples:
\begin{itemize}
\item $\hat{\delta_0}(\phi) = \delta_0(\hat{\phi}) = \hat{\phi}(0) = \int \phi = 1(\phi).$
\item Take $\R^2$ and consider $u(x) = \frac{1}{|x|}$.  This function is in $L_{loc}^1$.  If we multiply by $(1 + |x|)^{-2} u \in L^1(\R^n)$, it follows that $u \in \mc S'$, since 
$$|u(\phi)| = \left |\int (1 + |x|)^{-2} u (1 + |u|)^2 \phi\right | \le C \sup (1 + |x|)^2 \phi.$$

Now, we compute $\hat{u} \in \mc S'$.  Since $\mc F$ is continuous on $\mc S'$, we approximate $u$ and hope the result converges to the desired result.  Define $u_\epsilon \to u$ in $\mc S'$ for $u_\epsilon \in L^1$.  

Try $u_\epsilon(x) = \frac{e^{-\epsilon |x|^2/2}}{|x|} \in L^1$ for $\epsilon > 0$.  We want to calculate $\hat{u}_\epsilon$ and take the limit as $\epsilon \to 0^+$.   We can evaluate the integral by converting to polar coordinates and completing the square.  Unfortunately, it reduces to an integral that is too hard, but we will learn asymptotics of the integral as $\epsilon \to 0$.  We find that $\hat{u}(\xi) = 2\pi/|\xi|$.

We can approach this differently.  Note that $u = 1/|x|$ is homogeneous: $u(tx) = t^a u(x)$ for $t > 0$, for functions.  For distributions, we have that for $\phi \in S$, $u(\phi(\cdot / t) t^{-n}) = t^a u(\phi)$ for $t > 0$.  For the Fourier Transform, if $u \in \mc S'(\R^n)$ is homogeneous of degree $a$, then $\hat{u}$ is homogeneous of degree $-n-a$.  It follows that our Fourier transform is of degree $-1$.  

Furthermore, note that $1/|x|$ is spherically symmetric, and the Fourier transform preserves spherical symmetry(note that the Jacobian factor for rotations is $1$).  It follows that the fourier transform is also spherically symmetric.  It follows that 
$$\mc F(1/|x|) = C/|\xi| + \sum_{|\alpha \le N| }c_\alpha \delta_0^{(\alpha)},$$
but delta terms have too much homogeneity.  
\end{itemize}

\pagebreak
\section{December 21st, 2021}
\subsection{Plancherel's Theorem}
Recall that the Fourier transform is an isomorphism on $\mc S$ - it is a bounded linear operator whose inverse is also bounded.  

Note that 
$$\int \hat{u}(\xi) \ov{\hat{\phi}(\xi)}d\xi = \iiint u(x)\ov{\phi(y)} e^{-i(x-y)\xi}\,dxdyd\xi$$

In the sense of distributions, $\int e^{-i(x-y)\xi}\,d\xi = (2\pi)^n \delta(x-y).$
Hence,
$$\iiint u(x)\ov{\phi(y)} e^{-i(x-y)\xi}\,dxdyd\xi = (2\pi)^n \int u(x)\ov{\phi(x)}\,dx.$$

For $u, \phi \in \mc S$, we have the following: $$\< \hat{u}, \hat{\phi}\> = (2\pi)^n \< u, \phi \>.$$

This implies that 
$$\|\hat{u}\|_2 = (2\pi)^{n/2} \|u\|_{2}, u \in \S.$$
If $u_n \to u$ in $L^2$ then $u_n \to u$ in $\mc S'$ by the Cauchy-Schwartz inequality.  It follows that $\hat{u_n} \to \hat{u}$ in $\mc S'$ but our formula shows that $\hat{u}$ is in $L^2$.  Hence, $\mc F: L^2 \to L^2$ and for $u, v \in L^2$, $\<\hat{u}, \hat{v} \> = (2\pi)^n \<u, v\>$.

Recall last time, we were finding the Fourier transform of $u(x) = 1/|x|$ in $\R^2$.    For $u \in S'(\R^n)$ homogeneous of degree $a$, $\hat{u} \in \mc S'(\R^n)$ is homogeneous of degree $-n-a$.  In our example, It follows that $\hat{u}(\xi)$ is homogeneous of degree $-1$.  We also observed that $u$ is invariant under rotations so it follows that $\hat{u}$ is invariant under rotations.  

A function is homogeneous of degree $-1$ if $v(k\theta) = \frac{a(\theta)}{r}.$  Since our function is invariant under rotations, $\hat{u}(\xi) = \frac{c}{|\xi|}$ away from zero.  It follows from our previous argument that $\hat{u}(\xi) = \frac{c}{|\xi|}$ since $\delta$ terms have homogeneity of at least $-2$.  

Note that $\<u, \phi\> = (2\pi)^2 \<\hat{u}, \hat{\phi}\>$ and we find $\hat{u}$ by choosing an appropriate $\phi$.

\begin{align*}
\int_{\R^2} \frac{\phi(x)}{|x|}\,dx &= \int_0^{2\pi} \int_{0}^\infty \phi(r) \, drd\theta \\
&= 2\pi \int_{0}^\infty \phi(r)\,dr.
\end{align*}
Choosing $\phi(r) = e^{-r^2/2}$, we find that the integral is $(2\pi)^{3/2}$.  

Evaluating the other side,  
$$\hat{\phi}(\xi) = \int_{\R^2} e^{-|x|^2/2 - ix\cdot \xi}\,dx = \int e^{-\frac{1}{2}(x + i\xi)^2 - \frac{1}{2} |\xi|^2} = 2\pi \int e^{-|\xi|^2/2} = (2\pi)^{5/2}.$$

It follows that $c = 2\pi$.
\subsection{Fourier Characterization of $H^k$ spaces}
\begin{thm} $H^k(\R^n) = \{u \in S'(\R^n) : (1 + |\xi|^2)^{k/2} \hat{u} \in L^2\}.$
\end{thm}
\begin{proof}
Suppose that $\partial^\alpha u \in L^2$ for $|\alpha| \le k$.  We know that $\|u\|_{2} = (2\pi)^{-n/2}\|\hat{u}\|.$  It follows that $\hat{\partial^\alpha u} \in L^2$.  Note that $\hat{\partial^\alpha u} = i^{|\alpha|} \\xi^{\alpha} \hat{u} \in L^2$ for all $|\alpha| \le k$.

Hence,
$$(1 + |\xi|^2)^{k/2} \le C_{n, k} \sup_{|\alpha| \le k} |\xi^\alpha|.$$

So it follows that $(1 + |\xi|^2)^{k/2} \hat{u} \in L^2$.

Now, suppose $(1 + |\xi|^2)^{k/2} \hat{u} \in L^2$.  It follows that $|\xi^\alpha| \le C_{k, \alpha}(1 + |\xi|^2)^{k/2}$ for $|\alpha| \le k$.  Hence $\xi^\alpha\hat{u} \in L^2$ so it follows that $\partial^{\alpha} u \in L^2$ by Plancherel's Theorem.  

\end{proof}

\begin{remark}We use the notation $\<\xi\> = (1 + |\xi|^2)^{1/2}$.
\end{remark}

Note that the definition does not require $k \in \N$.  
\begin{definition} $H^s(\R^n) = \{u \in \mc S' : \<\xi \>^s \hat{u} \in L^2\}, s \in \R$.
\end{definition}

\begin{thm} Suppose $u \in H^s(\R^n)$ and $s > \frac{1}{2}$.  Then $v(y) = u(0, y), y \in \R^{n-1}$ satisfies $v \in H^{s - 1/2} (\R^{n-1})$.
\end{thm}
\begin{remark} We should define $Tu(y) = u(0, y)$ for $u \in \mc S$.  Then $T: H^s(\R^n) \to H^{s - 1/2}(\R^{n-1})$ if $s > 1/2$.  
\end{remark}
\begin{proof}
Take $u \in \mc S$.  We wish to show that $\|v\|_{H^{s - 1/2}(\R^{n-1})} \le C\|u\|_{H^s(\R^n)}$.

Note that $$\hat{v}(\eta) = \int_{\R^{n-1}} u(0, y) e^{-y \cdot \eta} \,dy$$
and by the Fourier Inversion Formula
$$u(0, y) = (2\pi)^{-n}\int_{\R^n} \hat{u}(\xi_1, \xi') e^{iy \cdot \xi'}\,d\xi_1 d\xi',$$
so it follows that 
\begin{align*}
\hat{v}(\eta) &= (2\pi)^{-n}\int_{\R^{n-1}}\int_{\R^n} \hat{u}(\xi_1, \xi') e^{-iy \cdot (\eta - \xi')}\,d\xi dy \\
&= (2\pi)^{-n}\int_{\R^{n}}\int_{\R^{n-1}}\hat{u}(\xi_1, \xi') e^{iy \cdot (\xi' - \eta)}\, dy d\xi\\
&= (2\pi)^{-1} \int_{\R^n} \hat{u}(\xi_1, \xi') \delta_{\xi' = \eta}d\xi \\
&= (2\pi)^{-1} \int_{\R} \hat{u}(\xi_1, \eta) d\xi_1.
\end{align*}
 
Note that up to constants
 $$\|v\|_{H^{s-1/2}}^2 = \int_{\R^{n-1}} \<\eta\>^{2s-1} |\hat{v}(\xi)|^2\, d\eta = \int_{\R^{n-1}} \<\eta\>^{2s-1} \left |\int \hat{u}(\xi_1, \eta)\, d\xi_1\right |^2d\eta.$$
  
 Then,
 \begin{align*}
 \int_{\R^{n-1}} &\<\eta\>^{2s-1} \left |\int \hat{u}(\xi_1, \eta)\, d\xi_1\right |^2d\eta \\
 &= \int \<\eta\>^{2s-1} \left |\int \hat{u}(\xi, \eta)(1 + |\xi_1|^2 + |\eta|^2)^{s/2} (1 + |\xi_1|^2 + |\eta|^2)^{-s/2}d\xi_1  \right |^2d\eta \\
 & \le \int \<\eta\>^{2s-1} \int |\hat{u}(\xi_1, \eta)|^2 (1 + |\xi_1|^2 + |\eta|^2)^s d\xi_1 \int (1 + |\xi_1|^2 + |\eta|^2)^{-s} d\xi_1\, d\eta \\
&\le \int \<\eta\>^{2s-1} \<\eta\>^{-2s+1}  \int |\hat{u}(\xi_1, \eta)|^2 (1 + |\xi_1|^2 + |\eta|^2)^s d\xi_1\int (1 + u^2)^{-s} du d\eta\\
& = \int |\hat{u}(\xi)|^2 \<\xi\>^{2s} d\xi = \|u\|_{H^s}^2,
 \end{align*}
 
since
$$\int |\hat{u}(\xi_1, \eta)|^2 (1 + |\xi_1|^2 + |\eta|^2)^s d\xi_1d\eta = \int |\hat{u}(\xi)|^2 \<\xi\>^{2s}d\xi.$$

\end{proof}
\pagebreak
\section{January 26th, 2021}
\subsection{Sobolev Spaces, continued}
Recall, we have $U \subset \R^n$ open.  We typically assume $U$ is bounded and $\partial U \in C^1$.  For these spaces, we define $$W^{k, p}(U) = \{u \in \mc D' : \partial^{\alpha} u \in L^p(U), |\alpha| \le k\}.$$

Recall the extension property: there exists a map $E: W^{1, p}(U) \to W^{1, p}(\R^n)$ such that $Eu\vert_U = u$ and $u = 0$ for $|x| > R$ for some $R$ with $U \subset \subset B(0, R)$.

We also consider the $H^s(\R^n)$, the fractional Sobolev spaces: $\{u \in \mc S'(\R^n) : \<\xi \>^s \hat{u} \in L^2\}.$  This is a Hilbert space with the norm 
$$\|u\|_{H^s}^2 = \int \<\xi\>^{2s} |\hat{u}(\xi)|^2\,d\xi.$$

Last time, we showed that If we have $u \in H^s(\R^n)$ and $s > 1/2$, then $v(y) : u(0, y)$, $y \in \R^{n-1}$ satisfies $v \in H^{s - 1/2}(\R^{n-1})$.  Today, we will show that $H^s(\R^n) \subset C_0(\R^n)$ if $s > n/2$, where $C_0$ denotes continuous functions vanishing at infinity.  This means that there exists $T: H^s(\R^n) \to H^{s - 1/2}(\R^{n-1})$ such that for $u \in \mc S$, $Tu(y) = u(0, y)$.  

\begin{thm} $H^s(\R^n) \subset C_0(\R^n)$ if $s > n/2$.
\end{thm}
\begin{proof}
We first prove that if $\<\xi\>^s \hat{u} \in L^2, s > n/2$ then $\hat{u} \in L^1(\R^n)$.  

$$\int_{\R^n} |\hat{u}|d\xi = \int_{\R^n} \<\xi\>^{-s}\<\xi\>^2 |\hat{u}| d\xi \le \|\<\xi\>^{-s} \|_2 \|u\|_{H^s}.$$
The first term is finish precisely when $s > n/2$ [exercise: convert to polar coordinates]. This implies that $u \in L^\infty(\R^n)$, following from the Fourier Inversion formula.  

We know that $x \mapsto \hat{u}(\xi)e^{ix \xi}$ is continuous so it follows that $x \mapsto u(x)$ is continuous by the dominated convergence theorem.  Finally $u(x) \to 0$ as $|x| \to \infty$ by the Riemann-Lebesgue lemma: if $\hat{u} \in L^1(\R^n)$, then $u(x) \to 0$ as $|x| \to \infty$.
\begin{proof}
Recall $\mc S(\R^n) \subset L^1(\R^n)$ is dense.  Taking $v \in L^1$, taking $v_R = v(x) 1_{B(0, R)}(x)$.  Then $v_{R} \to v$ y the dominated convergence theorem.  Now take $\phi \in C_c^\infty$ with $\phi \ge 0$, $\int \phi = 1$ wth $\phi_\epsilon(x) = \frac{1}{\epsilon^n} \phi(x/\epsilon)$.  Taking $v_{R, \epsilon} = v_R * \phi_\epsilon \in C_c^\infty(\R^n)$ and $v_R*\phi_\epsilon \to v_R$ in $L^1$ as $\epsilon \to 0$.

Hence, we can take $v \in \mc S$ so that $\|\hat{v} - \hat{u} \|_{L^1} < \epsilon/2$.  Now, $|v(x)| < \epsilon/2$ if $|x| > R$, hence $$|u(x)| \le |u(x) - v(x)| + |v(x)| < C\epsilon + \epsilon/2$$
which goes to $0$ as we send $\epsilon \to 0$.
\end{proof}
\end{proof}

\subsection{Gagliardo-Nirenberg-Sobolev(GNS) Inequalities}
\begin{theorem}
If $1 \le p < n$ and we define $p^* = \frac{np}{n-p}$, then there exists $C = C(p, n)$ so that for all $u \in C_c^\infty(\R^n)$, $$\|u\|_{L^{p^*}} \le C \|\grad u\|_p.$$
\end{theorem}
\begin{remark} We can find the value of $p^*$ without doing the computation through scaling.Take $u_\lambda(x) = u(\lambda x)$.  We have that $\|u_\lambda \|_{p^*} \le C \| \grad (u_\lambda)\|_p$.  Then, evaluate both sides and compare the exponent on $\lambda$.

Note that the result is not true for $p = n > 1$.  It is true for $p = n = 1$.  
\end{remark}
\begin{theorem}[Morrey's Inequality] For $n < p \le \infty$, there exists $C = C(p, n)$ such that for $u \in C^1(\R^n)$, we have $$\|u\|_{C^\gamma(\R^n)} \le C(\|u\|_p + \|\grad u \|_p),$$
where $\gamma = 1 - \frac{n}{p}$, where
$$\|u\|_{C^\gamma(\R^n)} = \sup |u| + \sup_{x \ne y} \frac{|u(x) - u(y)|}{|x - y|^\gamma}.$$
\end{theorem}

\begin{thm}[General Formulation]
Take $U \subset \subset \R^n$ with $\partial U \in C^1$.  Take $n \in W^{k, p}(U)$.  
\begin{itemize}
\item if $k < n/p$, then $u \in L^q(U)$ where $1/q \ge 1/p - k/n$ and $\|u\|_{L^q(U)} \le C \|u\|_{W^{k, p}}$.
\item $k > n/p$, then $u \in C^{k - [n/p] - 1, \gamma}(\overline{U})$ where $\gamma = [n/p] + 1 - n/p$ if $n/p \not \in \N$ and $1 - \delta$ for all $\delta > 0$ if $n/p \in \N$.
\end{itemize}
\end{thm}
\subsection{Compactness}
\begin{definition} Let $B$ be a Banach space.  A subset $K \subset B$ is compact if for every sequence $\{u_n\} \subset K$ such that $\|u_n\|_B \le C$, there exists a convergence subsequence $u_{n_k} \to u \in B$.
\end{definition}
\begin{remark} If $\{u : \|u\|_B \le 1\} \subset B$ is compact, then $B$ is finite dimensional.   We can have a space $B' \subset B$ and $\{u \in B' : \|u\|_{B'} \le 1\}$ compact in $B$.  If we have a sequence $\{u_n\} \subset B'$ and $\|u_n\|_{B'} \le C$ then there exists $n_k$, $u \in B$ such that $\|u_{n_k} -u\|_B \to 0$.
\end{remark}

We will take $B = L^q(U)$ where $1 \le q < p^*$ and $B' = W^{1, p}(U)$.
\begin{thm}[Rellich-Kondrachov] The unit ball in $W^{1, p}(U)$ is compact in $L^q(U)$ for \textbf{bounded} $U$.
\end{thm}
\pagebreak
\section{January 28th, 2021}
Recall the GNS inequality: if $1 \le p < n$, $p^* = \frac{np}{n-p}$, there exists $C = C(n, p)$ for all $u \in C_c^{\infty}(\R^n)$ so that $\|u|_{L^{p^*}} \le C \|\grad u\|_{L^p}$.

If $U \Subset \R^n$, $\partial U \in C^1$, then there exists $C= C(n, p, U)$ such that $L^{q}(U) \supset W^{1, p}(U)$ for $1 \le q \le p^*$.
\subsection{Compactness}
Suppose $B$ is a Banach space and $B' \subset B$ another Banach space.  We say that the inclusion $B' \subset B$ is compact if bounded sets in $B'$ are precompact in $B$.  In other words, for a sequence $\{u_n\} \subset B'$ with $\|u_n\|_{B'} \le M$, there exists a subsequence $u_{n_k}$ and $u \in B$ such that $u_{n_k} \to u$ in $B$.

\begin{example} Take $B = C([-1, 1])$, $B' = C^1([-1, 1])$ with the supremum norm on $B$ and $\|u\|_{B'} = \sup_{|x| \le 1} (|u(x)| + |u'(x)|)$.

The inclusion is compact: if we have $\|u_n\|_{B'} \le C$, by the mean value theorem, $|u_n(x)| \le C$ and $|u_n(x) - u_{n}(y)| \le C|x - y|$.  By Arzela-Ascoli, there exists a subsequence $u_{n_k}$ and $u \in C$ so that $\|u_{n_k} - u\|_{C([-1, 1])} \to 0$.
\end{example}
\begin{example} In the previous example, take $u_n(x) = |x|1_{|x| > 1/n} + (\frac{nx^2}{2} + \frac{1}{n})1_{|x| \le 1/n}$.

Then $u_n \in C^{1}[-1, 1]$ and $\|u_n\|_{C^1[-1, 1]} \le 2$.   We can take a subsequence $n_k = k$ and $u(x)= |x| \in C[-1, 1] \setminus C^1[-1, 1]$ where $u_{n_k} \to u \in C$.
\end{example}

Given a Banach space $B$, we have the dual space $B^* = \{\text{linear } u:  B \to \C | \forall x \in B, |u(x)| \le C \|x\|_B\}$.  The is also a Banach space.  

\begin{thm}[Banach-Alaoglu] Suppose $\|u_n\|_{B^*} \le M$.  Then, there exists a subsequence $u_{n_k}$ and $u \in B^*$ such that for all $x \in B$, $u_{n_k}(x) \to u(x)$.  
\end{thm}
\subsection{Rellich-Kondrachov}
\begin{thm}[Rellich-Kondrachov] If $U \Subset \R^n$, $\partial U \in C^1$, then for $1 \le q < p^*$, $L^q(U) \supset W^{1, p}(U)$ is a compact inclusion.
\end{thm}
\begin{proof}
Take $p = 2$.  Then $p^* = \frac{2n}{n-2} > 2$.  First, suppose $\ol{U} \Subset B(0, R)$.  We can assume $R = 1$.  Suppose we have a sequence $\|v_n\|_{H^1(U)} \le 1$.  There exists a sequence $u_n \in H^1(\R^n)$ such that $u_n\vert_U = v_n$, $\|u_n\|_{H^1(\R^n)} \le 1$ and $\supp u_n \subset B(0, 1)$(this is the extension operator).

We have $u_n \in H^1(\R^n)$, $\|u_n\|_{H^1} \le 1$, $\supp u_n \subset B(0, R)$.  We want $n_k$, $u \in L^2(\R^n)$ such that $u_{n_k} \to u$ in $L^2$.  We claim that $u(x) = (2\pi)^{-n} \sum_{m \in \Z^n} \hat{u}(m) e^{im\cdot x}$ for $x\in B(0, 1) \subset [-\pi, \pi]^n$ with convergence in $\mc D'$.

Alternatively,
\begin{align*}
\int u(x) \ol{\phi(x)}\,dx &= (2\pi)^{-n}\sum_{m \in \Z^n} \hat{u}(m) \int \ol{\phi(x)} e^{im\cdot x}\,dx\\
&= (2\pi)^{-n} \sum_{m \in \Z^n} \hat{u}(m) \ol{\hat{\phi}(m)}.
\end{align*}
For $u, v\in L^2$, $\int u(x)\ol{v}(x)\,dx = (2\pi)^{-n} \sum_{m \in \Z^n} \hat{u}(m) \ol{\hat{v}(m)}$.

Recall the Poisson summation formula: for $n = 1, a \ne 0$, $$\sum_{k \in \Z} e^{ikax} = \frac{2\pi}{a}\sum_{k \in \Z} \delta(x - 2\pi k/a)$$
in the distributional sense.

Note that $(1 - e^{iax}) \sum_k e^{ikax} = \sum_{k} e^{ikax} - \sum_{k} e^{i(k+1) ax} = 0$.  We can rewrite this as $-2ie^{-iax/2} \sin{(ax/2)} \sum_k e^{ikax} = 0$.  Let $w(x) = \sum_{k} e^{ikax}$, so it follows that $\supp w \subset \{\frac{2\pi}{a}k\}_{k \in Z}$.  It follows that $w(x)$ is the sum of delta functions supported at $2\pi k/a$ for $k \in \Z$ up to constants.  

Furthermore, note that $w(x + 2\pi a) = w(x)$.  So it follows that the constants are independent of the index.  To find the constant, for some function, replace $\phi(\cdot)$ with $\phi(\cdot + x)$.  Then the right side is $c \sum_{k \in \Z} \phi(2\pi k/a + x)$.  Note that $\hat{\phi}(\cdot + x)(\xi) = e^{ix \xi} \hat{\phi}(\xi)$ .  It follows that the left hand side is $\sum_{k \in \Z} \hat{\phi}(ka) e^{ikax}$.  Now suppose $\supp \phi \in C_c^\infty((0, 2\pi/a))$.    Integrating both sides, the left side is $2\pi/a \hat{\phi}(0)$.  The right side is $c \int \phi(x)\,dx = c(a) \hat{\phi}(0)$.  Thus, $c = \frac{2\pi}{a}$.

The Poisson summation formula is more generally $\sum_{k \in \Z^k} e^{iak \cdot x} = (2\pi/a)^n \sum_{k \in \Z^d} \delta(x - 2\pi a/k)$.

Applying this to a $\phi$ gives our desired claim from earlier.  it follows that $\frac{1}{(2\pi)^n} \sum_{k \in \Z^n} \hat{u}(k)\ol{\hat{v}(k)} = \int u(x) \ol{v(x)}\,dx$ with $u, v \in L^2$, $\supp u, v \in [-\pi, \pi]^n$.

Note that for $u \in L^2$, we have the Plancherel formula,
$$\|u\|_2^2 = \int |u(x)|^2\,dx = \frac{1}{2\pi}^n \sum_{n \in \Z^n} |\hat{u}(m)|^2.$$

For $u \in H^1(\R^n)$ and $\supp u \subset B(0, 1)$, then $u \in L^2$ and $\partial^\alpha u \in L^2$, for all $|\alpha| = 1$, 
$$\|\partial^\alpha u\|_{L^2}^2 = (2\pi)^n \sum_{ m \in \Z^n} |\hat{\partial^\alpha u}(m)|^2 = (2\pi)^{-n} \sum_{m \in \Z^n} |m^\alpha\hat{u}(m)|^2.$$
Claim: Under these assumptions, $\|u\|_{H^1}^2 = \Theta( \sum_{m \in \Z^n} \<m\>^2 |\hat{u}(m)|^2)$
\end{proof}

\pagebreak
\section{February 2nd, 2021}
Recall the following:
\begin{itemize}
\item GNS inequality: For $U \Subset \R^n$, $\partial U \in C^1$, $1 \le p < n$, $p^* = \frac{np}{n-p} >p$,
$$\|u\_{p^*} \le C(\|u\|_p + \|\grad u\|_p).$$
\item R-K Theorem: For $1 \le p < n$, $1 \le q < p^*$, $$W^{1, p}(U) \subset L^q(U)$$
is compact: If we have $\{u_n\}\subset W^{1, p}(U)$ and $\|u_n\|_{W^{1, p}} \le C$, there exists a subsequence $u_{n_k}$, $u \in L^q$ such that $\|u_{n_k} - u\|_{q} \to 0$.
\end{itemize} 

\subsection{Rellich-Kondrachov, continued}
Last time, we considered the special case of $\{u_n\} \subset H^1(\R^n)$ such that $\supp u_n \subset B(0, R)$ and $\|u_n\|_{H^1} \le C$, which implies that there exists a subsequence $u_{n_k}$ and $u \in L^2(\R^n)$ such that $\|u_n-u\|_{L^2} \to 0$.  We continue the proof of the special case. 

\begin{proof}
Recall that we showed that if $u \in C_0^\infty((-\pi, \pi)^n)$, we can write $u(x) = (2\pi)^{-n} \sum \hat{u}(n) e^{in \cdot x}$.  Then
$$\int u(x)\ol{v(x)}\,dx = \frac{1}{(2\pi)^n}\sum \hat{u}(n) \ol{\hat{v}(n)}$$
and 
$$\int |\grad u(x)|^2\,dx = \frac{1}{(2\pi)^n} \sum |n|^2 |\hat{u}(n)|^2.$$

For $u \in H^1$ with $\supp u \in B(0, 1)$, $$\|u\|_{H^1}^2 = \frac{1}{(2\pi)^n} \sum \<n\>^2 |\hat{u}(n)|^2.$$


$$\|u_n\|_{H^1}^2 = \sum_{\ell \in \Z^n} \<\ell\>^2 |\hat{u_n}(\ell)|^2 \le C.$$
$$\|v\|_{L^2}^2 = \sum_{\ell \in \Z^n} |\hat{v}(\ell)|^2.$$
We want to show that there exists $n_k$ such that $\|u_{n_k} - u_{n_p}\|_{L^2} \to 0$ as $k, p \to \infty$.

We introduce an operator $\Pi_p u(x) = (2\pi)^{-n}\sum_{|\ell| \le p} \hat{u}(\ell) e^{i\ell \cdot x}$.  We can think of $\Pi_p: L^2([-\pi, \pi]^n) \to \C^{N_p}.$
$N^p$ can be found through combinatorial methods(left as an exercise)[should be $\binom{n+p}{p}$ or something like that].

We have the following estimate:
$$\|(I - \Pi_p)u\|_2 \le \<p\>^{-2} \|u\|_{H^1}^2.$$
This is because
 $$(2\pi)^{-n}\sum_{|\ell| > p} |\hat{u}(\ell)|^2 = (2\pi)^{-n}\sum_{|\ell| > p} \<\ell\>^{-2}\<\ell\>^{2}|\hat{u}(\ell)|^2\le \<p\>^{-2} \|u\|_{H^1}^2.$$
 
Now, we find the Cauchy subsequence.  
\begin{enumerate}
\item For all $p$, we have $\|\Pi_p u_n\|_{\C^{N_p}} \le \|u_n\|_2 \le \|u_n\|_{H^1} \le C$.  Then $\{|z| \le C\} \subset \C^{N_p}$ is compact.  It follows that we can choose subsequences $\{n_k^{p+1}\}\subset \{u_{k}^p\}$ such that $\Pi_p u_k^p$ converges and $\limsup_{k, \ell} \|u_{k}^p - u_\ell^p\| \le C\<p\>^{-2}$, which follows from the triangle inequality.   [let $u_a^b = u_{n_a^b}$]
\item We choose $n_k = n_k^k$.  It follows that $\limsup_{k, \ell \to \infty} \|u_{n_k} -u_{n_\ell}\|_2 = 0$, since 
\end{enumerate}
\end{proof}
\subsection{Morrey's Inequality}
\begin{thm}[Morrey's Inequality]  Suppose $u \in L^p(\R^n), \grad u \in L^p(\R^n)$ and $n < p \le \infty$.  Then there exists $u^* \in C^{0, \gamma}(\R^n)$, with $\gamma = 1 - \frac{n}{p}$ such that $u = u^*$ almost everywhere and $\|u^*\|_{C^0, \gamma} \le \|u\|_p + \|\grad u\|_{p}$.
\end{thm}
\begin{remark}
Recall $$\|u\|_{C^{0, \gamma}} = \sup |u(x)| + \sup_{x \ne y} \frac{|u(x)- u(y)|}{|x - y|^\gamma}.$$
\end{remark}

\begin{proof}
We use the Littlewood-Paley Decomposition.  
\begin{lemma}[Dyadic Partitions of Identity]  There exists a function $\psi_0 \in C_c^\infty(\R), \psi \in C_c^\infty(\R \setminus \{0\})$ such that $$\psi_0(\xi) + \sum_{j=0}^\infty \psi(2^{-j}|\xi|) = 1.$$
\end{lemma}
\begin{proof}
Choose $\phi_0 \in C_c^\infty((-1, 1))$ with $0 \le \psi_0 \le 1$ and $\phi_0(p) = 1$, $|p| \le 1/2$.  

Choose a new function $$\phi_1(p) = \sum_{j \in \Z} \phi_0(p - j) \ge 1.$$
Note that $\phi_1(p - k) = \phi_1(p)$ for $k \in \Z$.  Choose $\phi(p) = \frac{\phi_0(p)}{\phi_1(p)}.$

Then
$$\sum_{j \in \R} \phi(p - j) = \sum \frac{\phi_0(p - j)}{\phi_1(p - j)} = \frac{1}{\phi_1(p)} \sum \phi_0(p -j) = 1.$$

Define $\psi(r) = \phi(\frac{\log r}{\log 2})$ for positive $r$.  Notice that $\psi \in C_c^\infty((0, \infty)).$  This gives that 
$$\sum_{j \in \Z} \psi(2^{-j} r) = 1.$$

Define $\psi_0(r) = 1 - \sum_{j = 0}^\infty \psi(2^{-j} r)$.  Note that $\psi_0(r) = 1$ for $r < 1/2$ and $\psi_0(r) = 0$ for $r > 1$.  It follows that $\psi_0$ and $\psi$ satisfy the conditions. 
\end{proof}
We can extend the Dyadic Partitions of Identity in $\R^n$ in the natural way.  We then define the Littlewood-Paley Decomposition as 
$$u = \psi_0(D) u + \sum_{j=1}^\infty \psi(2^{-j} D)u$$
where for $a \in L^\infty(\R^n)$, $a(D)u = \mc F^{-1}(a(\xi)\hat{u}(\xi))$ where $D_x = 1/i \partial_x$ and $\hat{Du} = \xi \hat{u}$.
\end{proof}
\pagebreak
\section{February 4th, 2021}
\subsection{Morrey's Inequality, continued}
Recall the statement of the theorem.
\begin{thm}[Morrey's Inequality]  Suppose $u \in L^p(\R^n), \grad u \in L^p(\R^n)$ and $n < p \le \infty$.  Then there exists $u^* \in C^{0, \gamma}(\R^n)$, with $\gamma = 1 - \frac{n}{p}$ such that $u = u^*$ almost everywhere and $\|u^*\|_{C^0, \gamma} \le \|u\|_p + \|\grad u\|_{p}$.
\end{thm}
\begin{proof}
Recall for $u \in \mc S(\R)$, $\hat{D_{x_j} u}(\xi) = \xi_j \hat{u}(\xi)$ where $D_{x_j} = \frac{1}{i} \partial_{x_j}$.  

We define a \textbf{Fourier multipler} $a \in L^\infty(\R^n)$ so that $a(D)u = \mc{F}^{-1}(a(\xi) \hat{u}(\xi))$ for $u \in \mc S$.   Note that for $a \in L^\infty$, $\|a(D)u\|_{L^2} \le \sup |a| \|u\|_{L^2}$.  For $\psi \in \mc S(\R^n)$, if we take $u \in \mc S'$, then $\psi(D) u \in \mc S'$ , and $\psi(\xi) \hat{u}(\xi) \in \mc S'$.

Recall the Littlewood - Paley Decomposition. We had a lemma: there exists $\psi_0 \in C_c^\infty(\R)$ and $\psi \in C_c^\infty(\R \setminus \{0\})$ such that for all $\xi \in \R^n$, 
$$\psi_0(|\xi|) + \sum_{j=0}^\infty \psi(2^{-j} |\xi|) = 1.$$ 
Slightly abusing notation, we will write $\psi_0(\xi) = \psi_0(|\xi|)$ and $\psi(\xi) = \psi(|\xi|)$.  

The full L-P Decomposition is given as follows: given $u \in \mc S'$, $a = \psi_0(D) u + \sum_{j=1}^\infty \psi(2^{-j }D) u$.  We will write $h = 2^{-j}$ as a shorthand.
\begin{lemma} Suppose $\chi \in C_c^\infty(\R^n)$.  Then for $u \in \mc S(\R^n)$, $\|\chi(hD)u\|_{L^\infty} \le C h^{-n/p} \|u\|_{L^p}$ 
and $\|\chi(hD) u\|_{L^p} \le (2\pi)^{-n} \|\hat{\chi}\|_1 \|u\|_p$.
\end{lemma}
\begin{proof}


Recall the following inequalities
\begin{itemize}
\item Holder's Inequality: $\|fg\|_1 \le \|f\|_p\|g\|_q$ for $1/p + 1/q = 1$ for $1 \le p \le \infty$.  
\item Minkowski's Inequality: $\|f+g\|_p \le \|f\|_p + \|g\|_p$ and 
$$\left \|\int F(x, t)\,dt \right \|_p \le \int \|F(\cdot, t)\|_p \,dt.$$
\item Young's inequality: $\|f * g\|_p \le \|f\|_1 \|g\|_p$.
\end{itemize}
We have 

\begin{align*}
\chi(hD)u(x) &= \mc F^{-1}(\chi(h\xi) \hat{u}(\xi)) = (2\pi)^{-n} \iint e^{i(x-y)\xi}\chi(h\xi) u(y)dyd\xi \\
&= (2\pi h)^{-n} \int \hat{\chi}\left (\frac{x-y}{h}\right ) u(y)\,dy \\
&\le (2\pi h)^{-n} \frac{C}{h^n} \|\hat{\chi}(\cdot /h)\|_q \|u\|_p
\end{align*}
Then,
$$\|\hat{\chi}(\cdot /h)\|_q = \left (\int |\hat{\chi} (y/h)|^q \,dy\right )^{1/q} = h^{n/q} \|\hat{\chi}\|_q.$$
It follows that 
$$|\chi(hD)u(x) | \le C h^{-n + n/q} \|u\|_p = C h^{-n/p} \|u\|_p.$$

For the second inequality, note that $\chi(hD)u(x) = (2\pi h)^{n} \hat{\chi}(\cdot/h) * u$.  Applying Young's Inequality,
$$\| \chi(hD) u\|_p \le \frac{1}{(2\pi h)^n}\|\hat{\chi}(\cdot/h)\|_1\|u\|_p \le \frac{1}{(2\pi)^n} \|\hat{\chi}\|_1 \|u\|_p.$$
\end{proof}
\begin{theorem} For $u \in L^p$, $1\le p \le \infty$, $u \in C^{0, \gamma}(\R^n)$ if and only if for every $\chi \in C_c^\infty(\R^n \setminus 0)$, $\|\chi(hD) u\|_\infty \le C h^{\gamma}$.
\end{theorem}
\begin{proof} We start with the forward direction.  Note that 
\begin{align*}
\chi(hD) u(x) &= \frac{1}{(2\pi h)^n} \int \hat{\chi}((x - y)/h) u(y) \,dy \\
&= (2\pi )^{-n} \int \hat{\chi}(y)u(x - yh) \,dy \\
&= (2\pi)^{-n}\int \hat{\chi}(y) (u(x - yh) - u(x))\,dy \\
\end{align*}
So it follows that 
$$|\chi (hD) u(x)| \le C \|u\|_{C^{0, \gamma}} \int |\hat{\chi}(y)| (hy)^\gamma dy \le C \|u\|_{C^{0, \gamma}} h^{\gamma} \int |\hat{\chi}(y)| |y|^\gamma \,dy$$
\end{proof}
and the last integral is bounded since $\hat{\chi}$ is a Schwartz function, so it follows that $|\chi(hD)u(x)| \le C \|u\|_{C^{0, \gamma}} h^{\gamma}$.
\end{proof}

\pagebreak
\section{February 9th, 2021}
\subsection{Fourier Transform proof of Morrey's Inequality}
Recall the Littlewood-Paley decomposition: There exists $\psi_0 \in C_c^\infty(\R^n)$, $\psi \in C_c^\infty(\R^n \setminus \{0\})$ such that $$1 = \psi_0(\xi) + \sum_{j=0}^\infty \psi(2^{-j}\xi).$$  From this, we have for $u \in \mc S'$, 
$$u = \psi_0(D)u + \sum_{j=0}^\infty \psi(2^{-j}D)u.$$
More generally, for $a\in \mc S(\R^n)$, $a(D)u = \mc F^{-1}(a(\xi) \hat{u}(\xi))$.
We were proving the following theorem:
\begin{theorem} For $u \in L^p$, $1\le p \le \infty$, $u \in C^{0, \gamma}(\R^n)$ if and only if for every $\chi \in C_c^\infty(\R^n \setminus 0)$, $\|\chi(hD) u\|_\infty \le C h^{\gamma}$.
\end{theorem}
\begin{proof}
We proved the forward direction last time.  We now show the converse.  

Denote
$$\Lambda_\gamma(u) = \sup_{0 < h < 1} h^{-\gamma}(\| \psi(hD)u\|_\infty + \max \| \psi_k(hD)u\|_\infty)$$
where $\psi_k(\xi) = \xi_k \psi(\xi)$.  

We have the hypothesis: $\|u\|_p + \Lambda_\gamma(u) < \infty$.  We want to show that $\|u\|_{C^\gamma, 0} \le C(\|u\|_p + \Lambda_\gamma(u))$.  We first bound $\|u\|_\infty$.  Note that 
\begin{align*}
\|u\|_\infty &\le \|\psi_0(D)u\|_\infty + \sum_j \|\psi(2^{-j} D)u \|_\infty \\
&\le \|\psi_0(D)u\|_\infty + \sum_j 2^{-j\gamma} \Lambda_\gamma(u) \\
&\le C \|u\|_p + (2^\gamma - 1)^{-1} \Lambda_\gamma(u).
\end{align*}

Now, we bound the quotient term, $|u(x) - u(y)|/|x - y|^\gamma$.  In order words, we want
$$|u(x) - u(y)| \le C(\|u\|_p + \Lambda_\gamma(u))r^{\gamma},$$
if $|x - y| \le r$.

Note that \begin{align*}
u(x) - u(y) &= \psi_0(D) u(x) - \psi_0(D) u(y) + \sum_{j} \left (\psi(2^{-j}D) u(x) - \psi(2^{-j}D) u(y)\right ) \\
\end{align*}
It is enough to prove that $|\psi_0(D)u(x) - \psi_0(D) u(y)| \le Cr^{\gamma} \|u\|_p$ and 
$$\sum_{j=0}^\infty |\psi(2^{-j}D)u(x) - \psi(2^{-j}D)u(y)| \le Cr^{\gamma} \Lambda_\gamma(u).$$

Note that 
\begin{align*}
|\psi_0(D) u(x) - \psi_0(D)u(y) | &\le \sup(\grad(\psi_0(D)u))|x-y|  \\
&\le |x - y| \frac{1}{(2\pi)^{n}} \sup \int \grad |\hat{\psi}_0(x-y) ||u(y)|\,dy \\
&\le |x - y| \frac{1}{(2\pi)^n} \| \grad \hat{\psi}_0\|_q \|u\|_p.
\end{align*}

For the second inequality, we prove for both high frequency and low frequency estimates.  For the high ones,
$$|\psi(hD)u(x) -\psi(hD)u(y)| \le 2 \|\psi(hD)u\|_\infty \le 2 h^\gamma \Lambda_\gamma (u).$$

For low frequencies, 
\begin{align*}
|\psi(hD)u(x) - \psi(hD) u(y)|&\le Cr \max_k \|D_{x_k} \psi(hD) u\|_\infty \\
&= Cr h^{-1} \max_k \| hD_{x_k} \psi(hD)u\|_\infty \\
&= Crh^{-1}\max_k \|\psi_k(h D) u\|_\infty \\
&\le  Crh^{-1} \max_k \| \psi_k(hD)u\|_\infty \\
&\le Crh^{\gamma - 1} \Lambda_\gamma(u).
\end{align*}

Then, note that 
$$ \sum_{2^j \le s} Cr2^{-j(\gamma - 1)} \le C' r  s^{1-\gamma}$$

and 
$$\sum_{2^j > s} C2{-j\gamma} \le C'' s^{-\gamma}.$$
It follows that 
\begin{align*}
\sum_{j=0}^\infty |\psi(2^{-j}D)u(x) - \psi(2^{-j}D)u(y)| &\le C\Lambda_\gamma(u)(rs^{1 - \gamma} + s^{-\gamma}) \le Cr^\gamma\Lambda_\gamma(u) 
\end{align*}
\end{proof}
 \end{document}

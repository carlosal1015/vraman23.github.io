\documentclass[11pt]{scrartcl}
\usepackage[sexy]{evan}
\usepackage{graphicx}
\usepackage{mathtools}
\usepackage{listings}

 %Sets
\newcommand{\N}{\mathbb{N}}
\newcommand{\Z}{\mathbb{Z}}
\newcommand{\F}{\mathbb{F}}
\newcommand{\Q}{\mathbb{Q}}
\newcommand{\R}{\mathbb{R}}
\newcommand{\C}{\mathbb C}
\newcommand{\T}{\mathbb T}
\newcommand{\PP}{\mathbb P}
\newcommand{\supp}{\text{supp }}
\newcommand{\E}{\mathbb E}
\newcommand{\cov}{\operatorname{cov}}
\renewcommand{\Re}{\operatorname{Re}}

\DeclareMathOperator*{\Var}{Var}
\newcommand{\Hol}{\operatorname{Hol}}

\let \phi \varphi
\let \hat \widehat
\let \mc \mathcal
\let \mb \mathbb
\let \p \partial
\let \bar \overline
\let \eps \varepsilon
\newcommand\at[2]{\left.#1\right|_{#2}}

%From Topology
\newcommand{\cT}{\mathcal{T}}
\newcommand{\cB}{\mathcal{B}}
\newcommand{\cC}{\mathcal{C}}
\newcommand{\cH}{\mathcal{H}}

%Indicators 
\newcommand{\1}{\textbf{1}} % vector of all 1's
\newcommand{\I}[1]{\mathbb{I}{\left\{#1\right\}}} % indicator function


\usepackage{answers}
\Newassociation{hint}{hintitem}{all-hints}
\renewcommand{\solutionextension}{out}
\renewenvironment{hintitem}[1]{\item[\bfseries #1.]}{}
\declaretheorem[style=thmbluebox,name={Problem}, numberwithin=section]{prob}

\begin{document}
\title{EE222}
\author{Vishal Raman}
\thispagestyle{empty}
$ $
\vfill
\begin{center}

\centerline{\huge \textbf{EE222, Lecture Notes}}
\centerline{\Large \textbf{Nonlinear Systems} }
\centerline{Professor: Shankar Sastry, Koushil Sreenath; Spring 2022}
\centerline{Scribe: Vishal Raman}
\end{center}
\vfill
$ $
\newpage
\thispagestyle{empty}
\tableofcontents
\newpage
%\maketitle

\section{Lecture 1: 1/18/2022}
To dive into analysis, we first discuss how nonlinear systems differ from linear systems.
\subsection{Multiple Equilibria}
Write $\dot x = Ax$, $x \in \R^n$, $x = \theta_n = 0 \in \R^n$ is the only equilibrium if $A$ is nonsingular.  However, if $A$ is singular, $\null(A)$ are all equilibrium points.  

\begin{example}[Euler's Buckling Beam] This happens when we take a beam and apply a stress to each side. This is governed by the equation
$$m\ddot{x} + d\dot x - \mu x + \lambda x + x^3 = 0.$$
We can equivalently solve the system
$$\dot x = y,$$
$$\dot y = -\frac{d}{m}y + \frac{\mu - \lambda}{m}x - \frac{x^3}{m}.$$
If $\mu > \lambda$, we have $y = 0$, $x = \sqrt{\mu - \lambda}, x = -\sqrt{\mu - \lambda}, x = 0$ as equilibrium points. If $\mu \le \lambda$, we only have $(0, 0)$.
\end{example}

\begin{example}[Nonlinear Damping] When we bow a violin string, we can think of it as a mass on a spring sitting on a conveyor belt moving at a constant velocity.  The nonlinearity comes from the friction which is a function of $F(\dot{x} - b)$.  We have sticky friction:
$$m\ddot{x} + kx + F(\dot{x} - b) = 0.$$
\end{example}

\subsection{Qualitative Analysis: Equilibrium Points}
Take $\dot{x} = f(x)$, $x \in \R^2$, $x_0$ and equilibrium point of $f(x_0) = 0$, $x(0) = x_0, x(t) = x_0$ for all $t \ge 0$.

Recall the Jacobian
$$Df = \begin{pmatrix}
\frac{\p f_1}{\p x_1} & \frac{\p f_1}{\p x_2} \\
\frac{\p f_2}{\p x_1} & \frac{\p f_2}{\p x_2} \\
\end{pmatrix}.$$

We take $A = Df(x_0)$.  Consider the equation $\dot{z} = Az$, with $z \in \R^2$, $A = Df(x_0)$.  
\begin{theorem}[Hartmann Grobman Theorem] Under certain conditions, if no eigenvalues of $A$ are imaginary or zero, then the flow of $\dot{x} = f(x)$ can be mapped to the flow of $\dot{z} = Az$.
\end{theorem}
\end{document}

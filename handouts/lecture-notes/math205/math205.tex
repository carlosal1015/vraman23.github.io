\documentclass[12pt]{scrartcl}
\usepackage[sexy]{evan}
\usepackage{graphicx}

\usepackage{answers}
\Newassociation{hint}{hintitem}{all-hints}
\renewcommand{\solutionextension}{out}
\renewenvironment{hintitem}[1]{\item[\bfseries #1.]}{}
\declaretheorem[style=thmbluebox,name={Theorem}]{thm}

 %Sets
\newcommand{\N}{\mathbb{N}}
\newcommand{\Z}{\mathbb{Z}}
\newcommand{\F}{\mathbb{F}}
\newcommand{\Q}{\mathbb{Q}}
\newcommand{\R}{\mathbb{R}}
\newcommand{\C}{\mathbb C}
\newcommand{\T}{\mathbb T}
\renewcommand{\hat}{\widehat}
\let \phi \varphi
\let \mc \mathcal
\let \ol \overline
%From Topology
\newcommand{\cT}{\mathcal{T}}
\newcommand{\cB}{\mathcal{B}}
\newcommand{\cC}{\mathcal{C}}
\newcommand{\cH}{\mathcal{H}}

\newcommand{\supp}{\text{supp }}

\newcommand{\aint}{\mathrel{\int\!\!\!\!\!\!-}}
\let \grad \nabla

\begin{document}
\title{Math 205}
\author{Vishal Raman}
\thispagestyle{empty}
$ $
\vfill
\begin{center}

\centerline{\huge \textbf{Math 205: Complex Variables} } 
\centerline{Professor: Dan-Virgil Voiculescu, Spring 2021}
\centerline{Scribe: Vishal Raman}
\end{center}
\vfill
$ $
\newpage
\thispagestyle{empty}
\tableofcontents
\newpage
%\maketitle
\section{January 20th, 2021}
\subsection{Intro to Riemann Mapping Theorem}
Our first goal is to proof a fundamental theorem of Riemann on conformal mappings.  We start with several preparations, including some detours.  The theorem essentially says that lots of open sets in $\C$ are holomorphically isomorhpic, given that they satisfy some simple topological conditions.  

\subsection{Cauchy's Integral Formula}
Recall Cauchy's formula:
$$f(z_0) = \frac{1}{2\pi i} \int_{\Gamma} \frac{f(z)}{z - z_0} \,dz$$
where $\Gamma$ is a simple closed curve, piecewise differentiable, $z_0 \in \text{Int}(\Gamma)$, and $f : \Omega \to \C$ is a holomorphic function, with $\Omega$ is open, $\Omega \supset \Gamma \cup \text{Int}(\Gamma)$.

If $\Gamma$ is the circle $|z - z_0| = R$, we parameterize with $z = Re^{i\theta} + z_0$ with $\theta \in [0, 2\pi)$.  This gives
$$f(z_0) = \frac{1}{2\pi} \int_{0}^{2\pi} f(z_0 + Re^{i\theta})\,d\theta,$$
which represents the average of $f$ on the circle.  

It follows that 
$$|f(z_0)| \le \max_{\partial B_R(z_0)} |f(z)|,$$
with equality if and only if $f$ is constant.  

If $f: \Omega \to \C$ is holomorphic for $\Omega$ connected, open and $z_0 \in \Omega$, then
$$|f(z_0)|\le \sup_{z \in \Omega} |f(z)|$$
with equality if and only if $f$ is constant.  

\subsection{Schwarz Lemma}
 \begin{thm}[Schwarz Lemma] For $f: B_1(0) \to \C$ holomorphic with $|f(z)|\le 1$ for all $z$ and $f(0) = 0$.  Then
 $$|f(z)| \le |z|, |f'(0)| \le 1.$$
 If for some $z_0 \ne 0$, $|f(z_0)| = |z_0|$ or if $|f'(0)| = 1$ then $f(z) = cz$ for some $|c| = 1$.
 \end{thm}
 \begin{proof}
 Define a function
 $$g(z) = \begin{cases} f(z)/z, \text{if } 0 \le |z| \le 1 \\
 f'(0), \text{if } z = 0
 \end{cases}.$$
 
 Note that $g(z)$ is continuous since at zero,
 $$\lim_{z \to 0} \frac{f(z)}{z} = \lim_{z \to 0} \frac{f(z) - f(0)}{z - 0} = f'(0).$$
 
 Hence, $|g(z)| \le C < \infty$ using the Weierstrass Extreme Value theorem.    If $0 < \epsilon < |w| < r < 1$, note that taking a Keyhole Contour, we have
 $$g(w) = \frac{1}{2\pi i} \left (\int_{|z| = r} - \int_{|z| = \epsilon}\right ) \frac{g(z)}{z - w}\,dz.$$
 
Note that
 $$\left |\int_{|z| = \epsilon} \frac{g(z)}{z-w}\,dz\right | \le (2 \pi \epsilon) \cdot C \frac{1}{|w| - \epsilon} \xrightarrow{\epsilon \to 0} 0.$$
 
 It follows that $$g(w) = \frac{1}{2\pi i} \int_{|z| = r} \frac{g(z)}{z - w}\, dz$$
 for $0 < |w| < r$.  The right side is holomorphic in $w$ if $|w| < r$, so it follows that 
 $$g(w) = \frac{1}{2\pi i} \int_{|z| = r} \frac{g(z)}{z - w}\, dz$$
 is holomorphic in $|z| < 1$.  
 
 This can also be proved by taking a Taylor series about the origin.  Since there is no constant term, we can divide by $z$ to still have a convergent Taylor series.  
 
 If $r < 1$,
 $$\sup_{|z| \le r} |g(z)| = \sup_{|z| = r} |g(z)| \le \sup_{|z| = r} \frac{|f(z)|}{|z|} \le \frac{1}{r}.$$
 
 If we let $r \uparrow 1$, then we get $\sup_{|z| < 1} |g(z) |\le 1$.  It follows that $|f(z)| \le |z|,$ $|f'(0)| \le 1$.
 
 If $|f(z_0)| = z_0$ for some $0 < |z_0| < 1$ then $|g(z_0)| = 1$ and $g$ is constant by the maximum principle so $g(z) = c$, $f(z) = cz$.  If $|f'(0)| = 1$, then $|g(0)| = 1$ so $g$ is constant and $f = cz$.
 \end{proof}
\subsection{Maximum Principles} 
In the above proof, we used the maximum principle.  Some other versions we will use are the following:
 
 If $K \subset \C$ compact and $f:K \to \C$ continuous, and the restriction of $f$ to the interior of $K$ is holomorphic, then 
 $$\sup_{z \in K} |f(z)| = \sup_{z \in \partial K} |f(z)|.$$
 
If $\Omega$ is open and connected, $f: \Omega \to \C$, $z_0 \in \Omega$, and $|f(z_0)| = \sup_{z \in \Omega} |f(z)|$, then $f$ is constant.  Applying this to $e^f$ and using that $|e^f| = e^{\text{Re }f}$, we find that 
$$\text{Re }f(z_0) = \sup_{z \in \Omega} \text{Re }f(z),$$
implies that $f$
 is constant.   We have the same result for $\text{Im }f$ by replacing $f$ with $-if$.  
 
 \subsection{Homework I }
Show the Automorphisms of the unit disk are fractional linear transformations.
\begin{proof}
Following the hint, define $h = g \circ f$.  Note that $h: \{|z| < 1\} \to \{|z| < 1\}$ is a holomorphic bijection as a composition of holomorphic bijections.  Furthermore, note that $$h(0) = \frac{f(0) - z_0}{1 - \ol{z_0} f(0)} = \frac{z_0 - z_0}{1 - \ol{z_0} f(0)} = 0.$$

Note that we can apply the Schwarz lemma to $h$ and $h^{-1}$ since the ranges of both functions are $\{|z| < 1\}$.  It follows that $|h'(0)| \le 1$ and 
$$1 \ge |(h^{-1})'(0)| = \left |\frac{1}{h'(h^{-1}(0))}\right | = \left |\frac{1}{h'(0)}\right |.$$
It follows that $|h'(0)| = 1$, so by the equality case of the Schwarz lemma, $h(z) = cz$ for some $c \in \C$.

It follows that $$h(z) = \frac{f(z) -z_0}{1 - \ol{z_0} f(z)} = cz \Rightarrow f(z) = \frac{z_0 + cz}{1 + c\ol{z_0}z} = \frac{a + bz}{c + dz}$$
for $a, b, c, d \in \C$.
\end{proof}
\pagebreak
\section{January 25th, 2021}
\subsection{Uniform Convergence}
\begin{remark} They sometimes call open connected sets "regions".
\end{remark}

\begin{definition}[Uniform Convergence] Let $\Omega \subset \C$ be open.  Let $f_n: \Omega \to \C$ be holomorphic and $f: \Omega \to \C$ a function so that $\lim_{n \to \infty} \sup_{z \in K} |f(z) - f_n(z)| = 0$ for all $K \subset \Omega$ compact(also denoted $K \subset \subset \Omega$).  
\end{definition}
\begin{remark} Recall from real analysis that $f$ is a continuous function.  
\end{remark}

Some further remarks:
\begin{itemize}
\item It suffices to check the result for a sequence of compact subsets $K_m$ so that $\bigcup_m K_m^\circ = \Omega$, the it suffices to check those.  If $K \subset \subset \Omega$, then $K$ is compact and covered by the union of the subsets so there exists a finite subcovering, and uniform convergence on the subcovering implies uniform convergence on $K$.
\item It is often convenient to introduce $\|g\|_K = \sup_{z \in K} |g(z)|$.  Uniform convergence can be restated as $\|f_n - f\|_K \to 0$ for all $K \subset \subset \Omega$.
\item If $\|f_n - f\|_K \to 0$ for all $K \subset \subset \Omega$, then $f$ is also holomorphic.  It follows by passing to the limit in the Cauchy Integral formula.  Namely, take $\{z : |z - z_0| \le R\} \subset \Omega$ and consider the points in $|z_0 - \zeta| < R$.  

\begin{align*}
\left |f_n(\zeta) - \frac{1}{2\pi i} \int_{|z - z_0| = R} \frac{f(z)}{z - \zeta}\,dz\right | &= \left |  \frac{1}{2\pi i} \int_{|z - z_0| = R} \frac{f_n(z)}{z - \zeta}\,dz  - \frac{1}{2\pi i} \int_{|z - z_0| = R} \frac{f(z)}{z - \zeta}\,dz\right | \\
&\le \frac{1}{2\pi} \frac{1}{R - |z_0 - \zeta|} \cdot (2\pi R) \|f_n - f\|_{|z - z_0| = R} \to 0.
\end{align*}
So it follows that 
$$f(\zeta) = \lim_{n \to \infty} f_n(\zeta) = \frac{1}{2\pi i} \int_{|z - z_0|} \frac{f(z)}{z - \zeta} \,dz.$$

It follows that $f$ continuous on $|z - z_0| = R$ is holomorphic in $\zeta\in \{|z - z_0| < R\}$, so it follows that $f$ is holomorphic.  
\item We can similarly show that 
$$f_n^(j)(\zeta) = \frac{n!}{2\pi i} \int_{|z - z_0| = R} \frac{f_n(z)}{(z - \zeta)^{n+1}}\,dz$$
and $\|f_n^{(j)} - f^(j)\|_K \to 0$.  
\end{itemize}
From the last item, we have the following theorem.
\begin{thm} If $f_n \to f$ on compact subsets of $\Omega$, the if $f_n$ is holomorphic we find that $f$ is holomorphic and $f_n^(j) \to f^{(j)}$ uniformly on compact subsets of $\Omega$.
\end{thm}

\begin{thm}[Hurwitz] Let $\Omega$ be a region, $f : \Omega \to \C$ and $f_n: \Omega \to \C$ holomorphic with $f_n(\Omega) \subset \C \setminus \{0\}$, $n \in \N$ and $\|f_n - f\|_K \to 0$
 for all compact subsets.  Then either $f \equiv 0$ or $f(\Omega) \subset \C \setminus \{0\}$.
 \end{thm}
\begin{proof}
If $f$ is not identically zero on $\omega$, then since $f$ is holomorphic, its zeros are isolated.  If $z_0 \in \Omega$, $f(z_0) = 0$, then there is $\epsilon > 0$ so that when $0 < |z - z_0| <\epsilon$, $f(z) \ne 0$. 

Since $f(z) \ne 0$ for $|z - z_0| = \epsilon/2$, by the Weierstrass theorem applied to $|f|$ on $|z - z_0| = \epsilon$, we have $|f(z)| \ge m > 0$ on $\{|z - z_0| = \epsilon/2\} = \Gamma$.  If $\|f_n - f\|_\Gamma \le m/2$ for $n \ge N$, then $$|f_n(z)| \ge |f(z)| - m/2 \ge m - m/2 = m/2$$ for $z \in \Gamma$.  Hence, it follows that $\|1 /f_n - 1/f\|_\Gamma \to 0$(we leave this as an exercise).

Since $\|f_n' - f'\|_\Gamma \to 0$, we find that $\|f_n'/f_n - f'/f\| \to 0$(another exercise) and hence
$$\frac{1}{2\pi i} \int_\Gamma \frac{f_n'}{f_n}\,dz \to \frac{1}{2\pi i}\int_\Gamma \frac{f'}{f}\,dz.$$
The integrand of the left hand side is $(\log f_n)'$, whose integral is $0$, and the right side is the order of the zero of $f$ at $z_0$ by the argument principle.    It follows that the order of $z_0$ as a possible zero is $0$, so $f(z_0) \ne 0$.
\end{proof}

\begin{thm} For $\Omega \subset \C$ open, $\mc F$
 a set of holomorphic functions, the following are equivalent:
 \begin{itemize}
 \item for every $K \subset \subset \Omega$ $\sup_{f \in \mc F} \|f\|_K < \infty$
 \item for every sequence $(f_n)_{n \in \N} \subset \mc F$, there is a subsequence $(f_{n_j})_{j \in \N}$ with $n_1 < n_2 < \dots$ so that $(f_{n_j})_{j \in \N}$ is uniformly convergent on compact subsets of $\Omega$.
 \end{itemize}
 \end{thm}
 \begin{proof}
 We first show $2$ implies $1$.  If $\sup_{f \in \mc F} \|f\|_K = \infty$, then we can find for each $n \in \N$ $f_n \in \mc F$ so that $\|f_n\|_K \ge n.$  If we abstract a convergence subsequence, then $\|f_{n_j} - f\|_K \le C < \infty$ and $\|f_{n_j}\|_K \le \|f\|_K + C$, while $\|f_{n_j}\|_K \to \infty$, a contradiction.
 \end{proof}
 \pagebreak
 \section{January 27th, 2021}
 
 \begin{thm} For $\Omega \subset \C$ open, $\mc F$
 a set of holomorphic functions, the following are equivalent:
 \begin{itemize}
 \item for every $K \subset \subset \Omega$ $\sup_{f \in \mc F} \|f\|_K < \infty$
 \item for every sequence $(f_n)_{n \in \N} \subset \mc F$, there is a subsequence $(f_{n_j})_{j \in \N}$ with $n_1 < n_2 < \dots$ so that $(f_{n_j})_{j \in \N}$ is uniformly convergent on compact subsets of $\Omega$.
 \end{itemize}
 \end{thm}
 I missed the beginning of the class, but I will add the proof of the theorem once notes are posted.
 
 One can put a metric on holomorphic functions so that convergence in the metric is uniform convergence on compact sets.  For $f: \Omega \to \C$, but $K_n \Subset \Omega$ so that $\bigcup_{n} K_n^\circ = \Omega$ and take 
 $$d(f, g) = \sum_{n=1}^\infty \frac{\|f - g\|_{K_n}}{1 + \|f - g\|_{K_n}} 2^{-n}.$$
 
 \subsection{Riemann Sphere}
 On the set $\C \cup \{\infty\}$, we consider the topology which makes it the Alexandroff(one-point) compactification of $\C$.  If $z \in \C$, a neighborhood is one that contains a neighborhood in $\C$ and a neighborhood of $\infty$
 is of the form $\{\infty\} \cup (\C \setminus K)$ for $K \Subset \C$.
 
Let $U_+ = \C \subset \C \cup \{\infty\}$ and $U_- = (C \setminus \{0\}) \cup \{\infty\}$.  Note that the union of the two sets covers the Riemann Sphere.  Define $\psi_+:U_+ \to \C$ by $\psi_+(z) = z$ and $\psi_i: U_- \to \C$ is given by $\psi_-(w) = 1/w$ if $w \in \C \setminus \{\infty\}$ and $0$ if $w = \infty$.  Notice that these two functions are bijections.

If $V \subset \C \cup \{\infty\}$ is open, a function $f: V \to \C$ is holomorphic if 
$$f \vert_{V \cup U_\pm}\circ (\psi_{\pm}\vert_{V \cup U_\pm})^{-1}: \psi_\pm(V \cup U_\pm) \to \C$$
is holomorphic.  In this way, we know what holomorphic functions are on open sets of $\C \cup \{\infty\}$.

More generally, we can describe a Riemann surface in the following way - Let $X$ be a topological space.  Take $\{(U_\alpha, z_\alpha)\}_{\alpha \in I}$ where $U_\alpha \subset X$ is open, and $\bigcup_{\alpha \in I} U_\alpha = X$ and $z_\alpha: U_\alpha \to \C$ is continuous, $z_\alpha(U_\alpha)$ is open and $z_\alpha$ is a homeomorphism.  The key requirement is that the maps $z_\alpha \circ z_\beta^{-1}: z_\beta(U_\alpha \cup U_\beta) \to z_\alpha(U_\alpha \cup U_\beta)$ are holomorphic.

Then, if $U \subset X$ is open, $f: U \to \C$ is holomorphic if for all $\alpha \in I$,
$$f\vert_{U \cup U_\alpha} \circ (z_\alpha \vert_{u \cup U_\alpha})^{-1}$$
is holomorphic.  Two such atlases give the same Riemann surface if put together, we get an atlas.
 \end{document}

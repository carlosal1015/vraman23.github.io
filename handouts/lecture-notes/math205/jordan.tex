\documentclass[11pt]{scrartcl}
\usepackage[sexy]{evan}
\usepackage{graphicx}

\newcommand{\N}{\mathbb{N}}
\newcommand{\Z}{\mathbb{Z}}
\newcommand{\F}{\mathbb{F}}
\newcommand{\Q}{\mathbb{Q}}
\newcommand{\R}{\mathbb{R}}
\newcommand{\C}{\mathbb C}
\newcommand{\T}{\mathbb T}
\newcommand{\PP}{\mathbb P}
\newcommand{\supp}{\text{supp }}
\renewcommand{\hat}{\widehat}
\renewcommand{\tilde}{\widetilde}

\renewcommand{\Re}{\operatorname{Re}}
\renewcommand{\Im}{\operatorname{Im}}


\let \phi \varphi

%From Topology
\newcommand{\cT}{\mathcal{T}}
\newcommand{\cB}{\mathcal{B}}
\newcommand{\cC}{\mathcal{C}}
\newcommand{\cH}{\mathcal{H}}

\usepackage{answers}
\Newassociation{hint}{hintitem}{all-hints}
\renewcommand{\solutionextension}{out}
\renewenvironment{hintitem}[1]{\item[\bfseries #1.]}{}
\declaretheorem[style=thmbluebox,name={Theorem}]{thm}

\begin{document}
\title{Jordan Curve Theorem}
\author{Vishal Raman}
\maketitle
\begin{abstract}
We used the Jordan Curve Theorem in order to present the Caratheodory Extension theorem for conformal maps.  The following is a proof of the theorem as a corollary of a more general theorem using Homology and the Mayer-Vietoris Theorem.  Any mistakes and typos are my own - kindly direct them to my inbox.
\end{abstract}
\tableofcontents
\section{Statement of Theorem}
\begin{theorem}
Let $n, k, i \in \N$ be arbitrary proved that $k \le n-1$.  
\begin{itemize}
\item Suppose $h: \mathbb D^k \to \mathbb S^n$ is a topological embedding.  Then $$\tilde{H}_i(\mathbb S^n \setminus h(\mathbb D^k); \Z) = 0.$$
\item If $h: \mathbb S^k \to \mathbb S^n$ is a topological embedding, then $$\tilde{H}_i(\mathbb S^n\setminus h(\mathbb S^k); \Z) = \tilde{H}_i(\mathbb S^{n-k-1}; \Z).$$
\end{itemize}
\end{theorem}
\begin{corollary}[Jordan Curve Theorem] Taking $n = 2$, $k = 1$ in the above theorem, $\tilde{H}_0(\mathbb S^2 \setminus h(\mathbb S^1)) = \tilde{H}_0(\mathbb S^0; \Z) \cong \Z$, so $H_0(\mathbb S^{2} \setminus h(\mathbb S^1)) = \Z^2$, which implies that $\mathbb S^2 \setminus h(\mathbb S^1)$ has two path-connected components.  
\end{corollary}
\pagebreak
\section{Proof of the Theorem}
\begin{proof}
We proceed by induction on $k$:  the $k = 0$ case is clear.  Now, consider $\mathbb D^k \cong [0, 1]^k$ and setting $I = [0, 1]$, define $A_+ = \mathbb S^n \setminus(I^{k-1} \times [0, 1/2])$ and $A_- = \mathbb S^n \setminus(I^{k-1} \times [1/2, 1])$.  It is easy to see that $A_+ \cup A_- = \mathbb S^n\setminus h(I^k)$ and $A_+ \cup A_- = \mathbb S^n \setminus (I^{k-1})$.  By induction, we know that the homologies of $A_+ \cup A_-$ are zero.  

By Mayer-Vietoris, we have the sequence
We have the sequence
$$\dots \to \tilde H_{i+1}(A_+ \cup A_-) \to \tilde H_{i}(A_+ \cap A_-) \to H_i(A_+) \oplus H_i(A_-) \to \tilde H_i(A_+ \cup A_-)\to \dots$$
Since $ \tilde H_{i+1}(A_+ \cup A_-) = \tilde H_i(A_+ \cup A_-) = 0$, it follows that $\tilde H_i(\mathbb S^n\setminus h(I^k)) \cong \tilde H_i(A_+) \oplus \tilde H_i(A_-)$.  It suffices to check that one of these is zero.  Suppose that $\tilde {H_i}(A_+ \cap A_i) \ne 0$.  Then, one of $\tilde H_i(A_+)$ or $\tilde H_i(A_-)$ is nonzero.  Now, suppose $\alpha \in \tilde H_i(\mathbb S^n \setminus h(I^k))$ is not a boundary.  Then it is not a boundary in $A_+$ or $A_-$. We use the same principle for further subdivisions of the interval $[0, 1]$, as in the proof of the Mayer-Vietoris Theorem.  By iteration, we obtain a nested sequence of intervals
$$I_1 \supset I_2 \supset \dots \supset I_j \supset \dots$$
and it follows that there exists $p \in \bigcup I_j$ with $\alpha$ not a boundary in $\mathbb S^n \setminus h(I^{k-1} \times I_j)$.  However, we must have $\alpha$ as a boundary in $\mathbb S^n \setminus h(I^{k-1} \times \{p\})$, a contradiction.  So we must have $\alpha$ as a boundary in some finite step.  This concludes the proof of the first statement.

Now, we prove the second statement.  We again proceed by induction on $k$.  If $k = 0$
 then $\tilde H_i(\mathbb S^n \setminus \mathbb S^0) \cong \tilde H_i(\R^n \setminus \{p\}) \cong \tilde H_i(\mathbb S^{n-1})$.  Decompose $\mathbb S^k = \mathbb D_+^k \cup D_-^k$ as the $\epsilon$-neighborhood of the upper and lower hemisphere.  We denote the decomposition of the subsets as $B_+ = \mathbb S^n \setminus h(\mathbb D_+^k)$ and $B_- = \mathbb S^n \setminus h(\mathbb D_-^k)$.

By the first part, we know that $\tilde H_i(B_\pm) = 0$.  By Mayer-Vietoris, we know that $B_+ \cap B_- \sim \mathbb S^n \setminus h(\mathbb S^{n-1})$($\sim$ denotes homotopy equivalence), so we obtain
$$\tilde H_i(\mathbb S^n \setminus h(\mathbb S^k)) \cong \tilde H_{i+1}(\mathbb S^n \setminus h(\mathbb S^{k-1})) = H_{i+1}(\mathbb S^{n - k + 1 - 1}) \cong \tilde H_{i+1}(\mathbb S^{n-k}) \cong \tilde H_i(\mathbb S^{n - k - 1}).$$
\end{proof}
\end{document}

\documentclass[12pt]{scrartcl}
\usepackage[sexy]{evan}
\usepackage{graphicx}

\usepackage{answers}
\Newassociation{hint}{hintitem}{all-hints}
\renewcommand{\solutionextension}{out}
\renewenvironment{hintitem}[1]{\item[\bfseries #1.]}{}
\declaretheorem[style=thmbluebox,name={Theorem}]{thm}

 %Sets
\newcommand{\N}{\mathbb{N}}
\newcommand{\Z}{\mathbb{Z}}
\newcommand{\F}{\mathbb{F}}
\newcommand{\Q}{\mathbb{Q}}
\newcommand{\R}{\mathbb{R}}
\newcommand{\C}{\mathbb C}
\newcommand{\T}{\mathbb T}
\renewcommand{\hat}{\widehat}
\let \phi \varphi
\let \mc \mathcal
\let \ol \overline
%From Topology
\newcommand{\cT}{\mathcal{T}}
\newcommand{\cB}{\mathcal{B}}
\newcommand{\cC}{\mathcal{C}}
\newcommand{\cH}{\mathcal{H}}

\newcommand{\supp}{\text{supp }}

\newcommand{\aint}{\mathrel{\int\!\!\!\!\!\!-}}
\let \grad \nabla

\begin{document}
\title{Math 205}
\author{Vishal Raman}
\thispagestyle{empty}
$ $
\vfill
\begin{center}

\centerline{\huge \textbf{Math 205: Complex Variables} } 
\centerline{Professor: Dan-Virgil Voiculescu, Spring 2021}
\centerline{Scribe: Vishal Raman}
\end{center}
\vfill
$ $
\newpage
\thispagestyle{empty}
\tableofcontents
\newpage
%\maketitle
\section{January 20th, 2021}

 \subsection{Homework}
Show the Automorphisms of the unit disk are fractional linear transformations.

\begin{proof}
Following the hint, define $h = g \circ f$.  Note that $h: \{|z| < 1\} \to \{|z| < 1\}$ is a holomorphic bijection as a composition of holomorphic bijections.  Furthermore, note that $$h(0) = \frac{f(0) - z_0}{1 - \ol{z_0} f(0)} = \frac{z_0 - z_0}{1 - \ol{z_0} f(0)} = 0.$$

Note that we can apply the Schwarz lemma to $h$ and $h^{-1}$ since the ranges of both functions are $\{|z| < 1\}$.  It follows that $|h'(0)| \le 1$ and 
$$1 \ge |(h^{-1})'(0)| = \left |\frac{1}{h'(h^{-1}(0))}\right | = \left |\frac{1}{h'(0)}\right |.$$
It follows that $|h'(0)| = 1$, so by the equality case of the Schwarz lemma, $h(z) = cz$ for some $c \in \C$.

It follows that $$h(z) = \frac{f(z) -z_0}{1 - \ol{z_0} f(z)} = cz \Rightarrow f(z) = \frac{z_0 + cz}{1 + c\ol{z_0}z} = \frac{a + bz}{c + dz}$$
for $a, b, c, d \in \C$.
\end{proof}
 \end{document}

\documentclass[11pt]{scrartcl}
\usepackage[sexy]{evan}
\usepackage{graphicx}
\usepackage{mathtools}
\usepackage{listings}

 %Sets
\newcommand{\N}{\mathbb{N}}
\newcommand{\Z}{\mathbb{Z}}
\newcommand{\F}{\mathbb{F}}
\newcommand{\Q}{\mathbb{Q}}
\newcommand{\R}{\mathbb{R}}
\newcommand{\C}{\mathbb C}
\newcommand{\T}{\mathbb T}
\newcommand{\PP}{\mathbb P}
\newcommand{\supp}{\text{supp }}
\newcommand{\E}{\mathbb E}
\newcommand{\cov}{\operatorname{cov}}
\renewcommand{\Re}{\operatorname{Re}}

\DeclareMathOperator*{\Var}{Var}
\newcommand{\Hol}{\operatorname{Hol}}

\let \phi \varphi
\let \hat \widehat
\let \mc \mathcal
\let \p \partial
\let \bar \overline
\let \eps \varepsilon
\newcommand\at[2]{\left.#1\right|_{#2}}

%From Topology
\newcommand{\cT}{\mathcal{T}}
\newcommand{\cB}{\mathcal{B}}
\newcommand{\cC}{\mathcal{C}}
\newcommand{\cH}{\mathcal{H}}

%Indicators 
\newcommand{\1}{\textbf{1}} % vector of all 1's
\newcommand{\I}[1]{\mathbb{I}{\left\{#1\right\}}} % indicator function


\usepackage{answers}
\Newassociation{hint}{hintitem}{all-hints}
\renewcommand{\solutionextension}{out}
\renewenvironment{hintitem}[1]{\item[\bfseries #1.]}{}
\declaretheorem[style=thmbluebox,name={Problem}, numberwithin=section]{prob}

\begin{document}
\title{Math 212}
\author{Vishal Raman}
\thispagestyle{empty}
$ $
\vfill
\begin{center}

\centerline{\huge \textbf{Math 212, Lecture Notes}}
\centerline{\Large \textbf{Several Complex Variables } }
\centerline{Professor: Maciej Zworski, Fall 2021}
\centerline{Scribe: Vishal Raman}
\end{center}
\vfill
$ $
\newpage
\thispagestyle{empty}
\tableofcontents
\newpage
%\maketitle

\section{Lecture 1: 8/26/2021}
\subsection{Review of 1D Complex Analysis}
\begin{definition}[Holomorphic] Let $D \subset \C$ be an open connected domain and take $u \in C^1(D)$.  The function $u$ is \textbf{holomorphic} if $\partial_{\ol{z}} u = 0$ where $\partial_{\ol{z}} = (\partial_x + i\partial_y)$.
\end{definition}

We also have the equivalent conditions that 
$$u \in \Hol(D) \Leftrightarrow \p_{\ol{z}} = 0 \Leftrightarrow \lim_{h \to 0} \frac{u(z + h) - u(z)}{h} \text{ exists and is continuous.}$$

\begin{fact}[Green's Theorem] For $\Omega \subset \C, \partial \Omega \in C^1$, we have 
$$\int_{\p \Omega} u\,dz = \iint_{\Omega} \p_{\ol{z}} u \, d\ol{z} \wedge dz.$$
\end{fact}

\begin{theorem}[Cauchy-Pompieu Formula] Let $u \in C^1(\ol{\Omega})$.  For all $\zeta \in \Omega$, 
$$u(\zeta) = \frac{1}{2\pi i} \left (\int_{\partial \Omega} \frac{u(z)}{z - \zeta}\,dz + \iint_{\Omega} \frac{\partial_{\ol{z}} u(z)}{z - \zeta}\, dz \wedge d\ol{z} \right)$$
\end{theorem}
\begin{proof}
Let $\Omega_{\eps} = \Omega \setminus \ol{D(\zeta, \epsilon)}$, where $0 < \epsilon << 1$.  Applying Green's Theorem to $w(z) = \frac{u(z)}{z - \zeta} \in C^1(\ol{\Omega_\eps})$ and noting that $\p_{\ol{z}} w = \frac{\p_{\ol{z}} u(z)}{z - \zeta}$, we have 
$$\iint_{\Omega_{\eps}} \frac{\p_{\ol{z}} u(z)}{z - \zeta} d\ol{z} \wedge dz = \int_{\p \Omega} \frac{u(z)}{z - \zeta} \,dz - \int_{\p D(\zeta, \epsilon)} \frac{u(z)}{z - \zeta} \,dz.$$

The left-hand side converges to $\iint_\Omega \frac{\partial_{\ol{z}} u(z)}{z - \zeta} d\ol{z} \wedge dz$ by the dominated convergence theorem.  Parameterizing the disc via polar coordinates, we can write 
$$\int_{\p D(\zeta, \epsilon)} \frac{u(z)}{z - \zeta} \,dz = \int_0^{2\pi} u(\zeta + \epsilon e^{i\theta})\,d\theta \to 2\pi i u(\zeta).$$

The desired formula follows from rearranging the terms upon taking the limit as $\epsilon \to 0$.
\end{proof}
\begin{remark} We also have a partial converse: let $\phi \in C_c^k(\C)$ with $k \ge 1$ and $u(z) = \iint \frac{\phi(z)}{z - \zeta} dz \wedge d\ol{z}$.  Then $u \in C^k(\C)$ and $\p_{\ol{z}} u = \phi$.
\end{remark}
Some other notable corollaries that follow from Cauchy's Theorem:
\begin{itemize}
\item $u \in \Hol(D) \Rightarrow u \in C^\infty(D)$.
\item For all $K \Subset \Omega \Subset D$, $k$, there exists $C$ such that for all $u \in \Hol(D)$, we have 
$$\sup_{K} |u^{(j)}(z)| \le C \|u \|_{L^1(\Omega)}.$$
\item $u_j \in \Hol(D)$, $u_j \to u$ uniformly on bounded sets, then $u \in \Hol(D)$.
\end{itemize}
\pagebreak
\section{Lecture 2: 8/31/2021}
We introduce the notation $u \in \mc O(D)$ to mean that $u$ is holomorphic.
We continue with corollaries following from Cauchy's Theorem:
\begin{itemize}
\item Let $\{u_j\} \subset \mc O(D)$.  If for all $K \Subset D$, there exists $C$ such that $|u_j| \le C$, then there exists $u \in \mc O(D)$ and a subsequence $u_{j_k}$ such that $u_{j_k} \to u$ uniformly on compact sets.  
\begin{proof}
Recall the Arzela-Ascoli Theorem:
\begin{theorem}[Arzela-Ascoli] Suppose $K \Subset \C$, $\{w_j\} \subset C(K)$ and there exists $C$ such that $|w_j| \le C$ and equicontinuous: for all $\epsilon > 0$, there exists $\delta$ such that for all $z, \zeta \in K$,
$$\|z - \zeta\| < \delta \Rightarrow \|w_j(z) - w_j(\zeta) \| < \epsilon.$$
Then, there exists $j_k$ and $w \in C(K)$ such that $w_{j_k} \to w$ in $C(K)$.
\end{theorem}
Let $D = \bigcup K_j$, $K_j \subset K_{j + 1} \Subset D$.  For example, we could take
$$K_j = \{z \in D : d(z m \C \setminus D) \ge 1/j, |z| \le j\}.$$
Then, $|u_j| \le C_{K_j}$ on ${K_j}$ so it follows that $|u_{j}'| \le C_k'$ on any $K_j$.

By Arzela-Ascoli, we have a subsequence $\{u_{k}^{j + 1}\} \subset \{u_k^j\}$ such that $u_{n_k^j} \to u^j$ uniformly on $K_j$.  Then, since $u^{j + 1} \vert_{K_j} = u^j$, and $u^j \to u$ uniformly on compact sets.  
\end{proof}
\item Maximum Principle: $u \in \mc O(D(z_0, r))$ and $|u(z)| \le |u(z_0)|$, $z \in D(z_0, r)$, then $u$ is identically a constant.  
\begin{proof}
Suppose $u(z_0) \ne 0$(otherwise the problem is trivial).  
\begin{align*}
u(z_0) &= \frac{1}{2\pi i} \int_{\partial D(z_0, \rho), \rho < r} \frac{u(z)}{z - z_0} \,dz \\
&= \frac{1}{2\pi} \int_0^{2\pi} u(z_0 + \rho e^{i \theta})\,dz 
\end{align*}
It follows that we have 
$$0 = \int_0^{2\pi} \left (1 - \frac{u(z_0 + \rho e^{i \theta})}{u(z_0)} \right)$$
Taking real parts, it follows that 
$$1 = \frac{\Re\{u(z_0 + \rho e^{i \theta})\} \bar{u(z_0)}}{|u(z_0)|^2},$$
which implies the result.
\end{proof}
\item Maximum Principle for Bounded Domain: $\ol{D} \Subset \C$, $u \in \mc O(D) \cap C(\ol{D})$ then $\max_{\ol{D}} |u|$ is attained on the boundary.
\begin{proof}
Suppose $\argmax |u| = z_0$ with $z_0 \in D$.  It follows that $u$ is constant in $D(z_0, r)$.  We will show later that this implies that $u$ is constant on $D$.  
\end{proof}
\item Let $u \in \mc O(D(0, r))$ then $u(z) = \sum \frac{u^{(n)}(0)}{n!}z^n$ with the series converging uniformly in $\ol{D(0, \rho)}$ for $\rho < r$. 
\end{itemize}
\pagebreak
\section{Lecture 3: 9/2/2021}

\end{document}
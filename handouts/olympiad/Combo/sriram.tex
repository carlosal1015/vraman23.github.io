\documentclass[11pt]{article}
\usepackage[sexy]{evan}

\usepackage{answers}
\Newassociation{hint}{hintitem}{all-hints}
\renewcommand{\solutionextension}{out}
\renewenvironment{hintitem}[1]{\item[\bfseries #1.]}{}
\declaretheorem[style=thmbluebox,name={Theorem}]{thm}

 %Sets
\newcommand{\N}{\mathbb{N}}
\newcommand{\Z}{\mathbb{Z}}
\newcommand{\F}{\mathbb{F}}
\newcommand{\Q}{\mathbb{Q}}
\newcommand{\RP}{\mathbb{RP}}
\newcommand{\CP}{\mathbb{CP}}
\newcommand{\R}{\mathbb{R}}
\newcommand{\C}{\mathbb C}
\newcommand{\T}{\mathbb T}
\renewcommand{\hat}{\widehat}
\renewcommand{\Im}{\text{Im }}
\renewcommand{\>}{\rangle}
\newcommand{\<}{\langle}


\renewcommand{\Ker}{\operatorname{Ker}}
\renewcommand{\Im}{\operatorname{Im}}
\renewcommand{\Ext}{\operatorname{Ext}}

\newcommand{\res}{\operatorname{res}}
\let \phi \varphi
\let \mc \mathcal
\let \ms \mathscr
\let \mb \mathbb
\let \ol \overline
\let \subset \subseteq
\let \subsetneq \subset
%From Topology
\newcommand{\cT}{\mathcal{T}}
\newcommand{\cB}{\mathcal{B}}
\newcommand{\cC}{\mathcal{C}}
\newcommand{\cH}{\mathcal{H}}

\newcommand{\supp}{\text{supp }}

\newcommand{\aint}{\mathrel{\int\!\!\!\!\!\!-}}
\let \grad \nabla

\begin{document}
\title{Sriram, Olympiad Combinatorics}
\author{Vishal Raman}
\maketitle
\begin{abstract}
Notes and selected solutions from \textit{Sriram, Olypiad Combinatorics}.  I will generally leave out tedious computations but will refer to the text whenever possible.
\end{abstract}
\tableofcontents
\pagebreak

\section{Algorithms}
\subsection{Greedy Algorithms}
\begin{example} In a graph $G$ with $n$ vertices, no vertex has degree greater than $\Delta$.  Show that one can color the vertices using at most $\Delta + 1$ colors, such that no two neighboring vertices are the same color.
\end{example}
\begin{proof}
Use a greedy coloring algorithm with the vertices in any order: color the first vertex with $1$.  For the second vertex, use the smallest available color so that no two neighboring vertices are the same color.  Since no vertex has degree greater than $\Delta$, we will be able to color each vertex with one of $\{1, 2, \dots, \Delta + 1\}$.
\end{proof}

\begin{example} In a graph $G$ with $V$ vertices and $E$ edges, show that there exists an induced subgraph $H$ with each vertex having degree at least $E/V$. 
\end{example}
\begin{proof}

\end{proof}


 \end{document}


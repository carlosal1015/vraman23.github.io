\documentclass[11pt]{article}
\usepackage[sexy]{evan}

\usepackage{answers}
\Newassociation{hint}{hintitem}{all-hints}
\renewcommand{\solutionextension}{out}
\renewenvironment{hintitem}[1]{\item[\bfseries #1.]}{}
\declaretheorem[style=thmbluebox,name={Theorem}]{thm}

 %Sets
\newcommand{\N}{\mathbb{N}}
\newcommand{\Z}{\mathbb{Z}}
\newcommand{\F}{\mathbb{F}}
\newcommand{\Q}{\mathbb{Q}}
\newcommand{\RP}{\mathbb{RP}}
\newcommand{\CP}{\mathbb{CP}}
\newcommand{\R}{\mathbb{R}}
\newcommand{\C}{\mathbb C}
\newcommand{\T}{\mathbb T}
\renewcommand{\hat}{\widehat}
\renewcommand{\Im}{\text{Im }}
\renewcommand{\>}{\rangle}
\newcommand{\<}{\langle}


\renewcommand{\Ker}{\operatorname{Ker}}
\renewcommand{\Im}{\operatorname{Im}}
\renewcommand{\Ext}{\operatorname{Ext}}

\newcommand{\res}{\operatorname{res}}
\let \phi \varphi
\let \mc \mathcal
\let \ms \mathscr
\let \mb \mathbb
\let \ol \overline
\let \subset \subseteq
\let \subsetneq \subset
%From Topology
\newcommand{\cT}{\mathcal{T}}
\newcommand{\cB}{\mathcal{B}}
\newcommand{\cC}{\mathcal{C}}
\newcommand{\cH}{\mathcal{H}}

\newcommand{\supp}{\text{supp }}

\newcommand{\aint}{\mathrel{\int\!\!\!\!\!\!-}}
\let \grad \nabla

\begin{document}
\title{Generating Functions}
\author{Vishal Raman}
\maketitle
\begin{abstract}
Notes and selected solutions from \textit{Wilf, generatingfunctionology}(third edition).  I will generally leave out tedious computations but will refer to the text whenever possible.  I have also included problems from various math olympiads.  
\end{abstract}
\tableofcontents
\pagebreak

\section{Introductory Ideas and Examples}
As a motivating example, take the Fibonacci numbers $F_0, F_1, \dots$ with the recurrence relation $F_{n+1} = F_n + F_{n-1}$ for $n \ge 0$, with $F_0 = 0, F_1 = 1$.  There are exact formulas for $F_n$, but another useful representation is as follows:  the $n$-th Fibonacci number $F_n$, is the coefficient of $x^n$ in the expansion of 
$$F(x) = \frac{x}{1 - x - x^2}$$
as a power series about the origin.  

Generating functions are typically used for the following:
\begin{itemize}
\item Finding an exact formula for members of a sequence.  This is not always possible, depending on the sequence, but it is often a good starting point.
\item Finding a recurrence formula.  Even though we mostly obtain generating functions from recurrences, we can sometimes use them to generate new recurrence formulas, which can potentially provide new insights.
\item Find statistical properties of sequences.  These are typically called Moment Generating Functions in statistics and have many use cases.
\item Finding asymptotic formulas for sequences.  An important example of this is the \textit{Prime Number Theorem}.
\item Proving analytic properties of the sequence(convexity, unimodality).
\item Proving combinatorial identities.  
\end{itemize}
 
\subsection{Example: Two-Term Recurrences}
\begin{example}
Take the recurrence defined by $a_{n+1} = 2a_n + 1$ for $n \ge 0$, $a_0 = 0$.  We show that $a_n = 2^n - 1$ using generating functions.
\end{example}
\begin{proof}
Define $A(x) = \sum_{n \ge 0} a_n x^n$.  By multiplying the recurrence relation by $x^n$ and summing over $n$, we have 
$$\sum_{n \ge 0} a_{n+1}x^n = 2\sum_{n \ge 0}a_n x^n + \sum_{n \ge 0} x^n = 2A(x) + \frac{1}{1-x}.$$

Then, note that 
$$\sum_{n \ge 0} a_{n+1} x^n = \frac{A(x) - a_0}{x} = \frac{A(x)}{x}.$$
We obtain
$$\frac{A(x)}{x} = 2A(x) + \frac{1}{1-x} \Longrightarrow A(x) = \frac{x}{(1-x)(1-2x)}.$$

To find a formula for the coefficients, we can use a partial fractions decomposition and expand the corresponding Taylor series.  In this case, we have 
\begin{align*}
\frac{x}{(1-x)(1-2x)} &=  \left(\frac{2x}{1-2x} - \frac{x}{1-x} \right) \\
&= \sum_{n \ge 1} (2x)^n - \sum_{n \ge 1} x^n \\
&= \sum_{n \ge 1} (2^n - 1) x^n.
\end{align*}
\end{proof} 
We now handle a more challenging two term recurrence.
 \begin{example}
Take the recurrence defined by $a_{n+1} = 2a_n + n$ for $n \ge 0$, $a_0 = 1$.  Find the generating function and determine a closed formula for the coefficients.
\end{example}
\begin{proof}
As before, define $A(x) = \sum_{n \ge 0} a_n x^n$.  From the recurrence relation, we have 
$$\frac{A(x) - 1}{x} = 2A(x) + \sum_{n \ge 0} n x^n.$$
Note that 
$$\sum_{n \ge 0} nx^n = \sum_{n \ge 0} x D(x^n) = xD\sum_{n \ge 0} x^n = xD \left(\frac{1}{1-x} \right) = \frac{x}{(1-x)^2}$$
\end{proof}
where $D$ denotes the differentiation operator.  We are assuming absolute convergence of the sums in these computations so the exchanging of the sum and the differentiation operator is justified.  

Plugging this in and solving for $A(x)$, we obtain the generating function
$$A(x) = \frac{1 - 2x + 2x^2}{(1-x)^2(1-2x)}= \frac{-1}{(1-x)^2} + \frac{2}{1-2x}.$$

To compute the coefficient of $x^n$, note that the coefficient of $\frac{-1}{(1-x)^2}$ is $-n-1$ and the coefficient from $\frac{2}{1-2x}$ is $2^{n+1}$.  From this, we obtain that 
$$a_n = 2^{n+1} - n - 1.$$
\subsection{The Method of Generating Functions}
\begin{definition} Given a power series $f(x)$, the symbol $[x^n]f(x)$ denotes the coefficient of $x^n$ in the series $f(x)$.
\end{definition}
\begin{fact} $[x^n]\{x^a f(x)\} = [x^{n-a}]f(x)$.
\end{fact}
\begin{fact} $[\beta x^n]f(x) = 1/\beta [x^n]f(x)$ for $\beta \in \R$.
\end{fact}
Given a recurrence formula, we have the following steps:
\begin{enumerate}
\item Note the set of values that are taken by the free variable(it is generally $n \ge 0$ or $n \ge 1$).
\item Define a generating function, $A(x) = \sum_{n} a_n x^n$.
\item Multiply both sides of the recurrence by $x^n$ and sum over $n$.
\item Express both sides of the equation in terms of $A(x)$.
\item Solve for $A(x)$.  If an exact formula is needed, expand $A(x)$ in a power series.  
\end{enumerate}

\subsection{Example: Fibonacci Numbers}
We return to the example of Fibonacci numbers, calculating the generating function.  Define $F(x) = \sum_{n \ge 0} F_n x^n$.  

We have 
\begin{align*}
\sum_{n \ge 1} F_{n+1} x^n &= \sum_{n \ge 1} F_n x^n + \sum_{n \ge 1} F_{n-1} x^n, \\
\frac{F(x) - x}{x} &= F(x) + xF(x), \\
F(x) &= \frac{x}{1 - x - x^2}.
\end{align*}

We can write $1 - x - x^2 = (1-xr_+)(1 - xr_-)$ with $r_{\pm} = \frac{1\pm \sqrt{5}}{2}$.  It follows that 
\begin{align*}
\frac{x}{1-x-x^2} &= \frac{1}{r_+ - r_-} \left( \frac{1}{1-xr_+} - \frac{1}{1 - xr_-}\right) \\
&= \frac{1}{\sqrt{5}} \sum_{n \ge 0} (r_+^n - r_-^n)x^n.
\end{align*}
\subsection{Two Independent Variables}
Let $n$ and $k$ be integers with $0 \le k \le n$.  How many ways can we choose a subset of $k$ objects from $\{1, 2, \dots, n\}$?  We know that this is $\binom{n}{k}$, but we derive this using generating functions.
\begin{proof}
Let $f(n, k)$ be the answer to the question. In the possible collection of subsets, we can divide them into two piles: subsets containing $n$ and subsets not containing $n$.  In the first case, there are $f(n-1, k-1)$ possible subjects and the latter case there are $f(n-1, k)$ subsets.  From this, we obtain the recurrence
$$f(n, k) = f(n-1, k) + f(n-1, k-1).$$
Note the obvious initial condition $f(n, 0) = 1$.  Define the generating function
$$B_n(x) = \sum_{k \ge 0} f(n, k) x^k.$$

Using the recurrence relation, we obtain
$$B_n(x) - 1 = (B_{n-1}(x) - 1) + xB_{n-1}(x), \quad B_0(x) = 1$$
which gives
$$B_n(x) = (1 + x)B_{n-1}(x).$$
This is an easy recurrence to solve: namely $B_n(x) = (1 + x)^n$.  We obtain the desired result that $[x^k]B_n(x) = \binom{n}{k}$ by the binomial theorem.  We could also use Taylor's formula: $f(n, k)$ will be the $k$-th derivative of $(1 + x)^n$ evaluated at $x = 0$ divided by $k!$.
\end{proof}
An important thing to note is that our computation holds for arbitrary $n \in \C$, provided $k \in \N$.  From this, we obtain a general formula for binomial coefficients, given by 
$$\binom{n}{k} = \frac{n(n-1) \dots (n-k + 1)}{k!}.$$

\subsection{Exponential Generating Functions}
The choice of $x^n$ was somewhat arbitrary for our power series.  In some cases, other choices can be helpful.
\begin{definition}[Exponential Generating Function] The EGF of a sequence $\{a_n\}_{n \ge 0}$ is given by 
$$\sum_{n \ge 0} a_n \frac{x^n}{n!}.$$
\end{definition}
We call our vanilla generating function an "Ordinary Generating Function"(OGF).

A useful operation in the case of EGFs is the $x(D)\log$.  This goes as follows:
\begin{enumerate}
\item Take the logarithm of both sides of the equation.
\item Differentiate both sides and multiply through by $x$.
\item Clear the fractions.
\item For each $n$, find the coefficients of $x^n$ on both sides and equate them.  
\end{enumerate}
\pagebreak
\section{Series}
\subsection{Formal Power Series}
We now consider generating functions as an object in the algebraic ring of formal power series.  
\begin{definition} A formal power series is an expression of the form $\sum_{n \ge 0} a_n x^n$.  We can identify a formal power series with a sequence of coefficients $\{a_n\}$.  The ring of formal power series over a ring $R$ is denoted $R[[x]]$.  Addition and subtraction are as usual and multiplication is given by the Cauchy product rule:
$$\sum_n a_n x^n \sum_n b_n x^n = \sum_n \left(\sum_k a_kb_{n-k}\right) x^n.$$
\end{definition}

\begin{proposition} A formal power series $f = \sum_{n \ge 0} a_nx^n$ has a reciprocal if and only if $a_0 \ne 0$.  In this case, the reciprocal is unique.  
\end{proposition}
\begin{proof}
Let $f$ have a reciprocal, $1/f = \sum_{n \ge 0} b_n x^n$.  By definition, $c_0 = 1 = a_0b_0$ which implies that $a_0 \ne 0$.  Furthermore, note that $c_n = 0 = \sum_{k} a_kb_{n-k}$ so it follows that 
$$b_n = (-1/a_0) \sum_{k \ge 1} a_kb_{n-k},$$
which implies that the coefficients are uniquely determined.  

Conversely, if $a_0 \ne 0$, we can determine the coefficients of the reciprocal $1/f$ via the above formula.  
\end{proof}

\begin{remark}
In order to handle convergence issues, we define the composition of two formal power series $f \circ g$ if and only if $g_0 = 0$ or $f$ is a polynomial.  
\end{remark} 
 \begin{proposition} Suppose we have two formal power series $f, g$ satisfying $f \circ g (x) = g \circ f (x) =x $ and $f(0) = 0$.  Then $f = f_1 x + f_2 x^2 + \dots$ and $g = g_1 x + g_2 x^2 + \dots$ with $f_1, g_1 \ne 0$.
 \end{proposition}
 
 We could also define derivatives of formal power series in the sense of usual calculus.  It is easy to show that these satisfies the usual rules calculus rules, as well as identities such as $f' = 0 \Rightarrow f = a_0$ or $f' = f \Rightarrow f = ce^x$.
 \pagebreak
 \subsection{Calculus of Formal Ordinary Power Series Generating Functions}
 We will use the notation $f \leftrightarrow^o \{a_n\}$ to denote $f = \sum_n a_n x^n$. 
We have the following rules:
 \begin{fact} If $f \leftrightarrow^o \{a_n\}$, then for $h \in \N$. 
 $$\{a_{n + h}\} \leftrightarrow^o \frac{f - a_0 - \dots - a_{h-1}x^{h-1}}{x^h}.$$
 \end{fact}
  \begin{fact} If $f \leftrightarrow^o \{a_n\}$, and $P$ is a polynomial
 $$P(xD) f \leftrightarrow^o \{P(n) a_n\}.$$
 \end{fact}
 \begin{fact}  If $f \leftrightarrow^o \{a_n\}$, $g \leftrightarrow^o \{b_n\}$, 
 $$fg \leftrightarrow^o \left \{\sum_{r=0}^n a_r b_{n-r}\right \}.$$
 \end{fact}
  \begin{fact} If $f \leftrightarrow^o \{a_n\}$, then for $k \in \N$. 
  $$f^k \leftrightarrow^o \left \{\sum_{n_1 + \dots + n_k = n} a_{n_1} \dots a_{n_k}\right \}.$$
 \end{fact}
   \begin{fact} If $f \leftrightarrow^o \{a_n\}$, then
  $$\frac{f}{1-x}\leftrightarrow^o \left \{\sum_{j=0}^n a_j   \right \}.$$
 \end{fact}
 
  \subsection{Calculus of Formal Exponential Power Series Generating Functions}
 We will use the notation $f \leftrightarrow^e \{a_n\}$ to denote $f = \sum_n a_n x^n/n!$. 
We have the following rules:
 \begin{fact} If $f \leftrightarrow^e \{a_n\}$, then for $h \in \N$. 
 $$\{a_{n + h}\} \leftrightarrow^e D^hf.$$
 \end{fact}
  \begin{fact} If $f \leftrightarrow^e \{a_n\}$, and $P$ is a polynomial
 $$P(xD) f \leftrightarrow^e \{P(n) a_n\}.$$
 \end{fact}
 \begin{fact}  If $f \leftrightarrow^e \{a_n\}$, $g \leftrightarrow^e \{b_n\}$, 
 $$fg \leftrightarrow^e \left \{\sum_{r=0}^n \binom{n}{r}a_r b_{n-r}\right \}.$$
 \end{fact}
  \begin{fact} If $f \leftrightarrow^e \{a_n\}$, then for $k \in \N$. 
  $$f^k \leftrightarrow^e \left \{\sum_{n_1 + \dots + n_k = n}\frac{n!}{n_1! \dots n_k!} a_{n_1} \dots a_{n_k}\right \}.$$
 \end{fact}
 \pagebreak
 \section{The Exponential Formula}
 \subsection{Notation and Definitions}
 \begin{definition} A \textit{card} $\mc C(S, p)$ is a pair consisting of a finite set $S$ and a picture $p \in P$.  The \textit{weight} of $\mc C$ is $n = |S|$.  A card of weight $n$ is called standard if its label set is $[n]$.
 \end{definition}
 \begin{definition} A \textit{hand} is a set of cards whose label forms a partition of $[n]$, for some $n$.  The weight of a hand is the sum of the weights of the cards in the hand.  
 \end{definition}
 \begin{definition} A relabeling of a card $\mc C(S, p)$ with $S'$ is defined if $|S| = |S'|$ and it is the card $C(S', p)$.  If $S' = [|S|]$, we have the standard relabeling of the card.
 \end{definition}
 \begin{definition} A \textit{deck} $\mc D$ is a finite set of standard cards whose weights are all the same and whose pictures are different.  The weight of the deck is the common weight of all the cards in the deck.
 \end{definition} 
 
 \begin{definition} 
 An \textit{exponential family} $\mc F$ is a collection of decks $\mc D_1$, $\mc D_2, \dots$, where for each $n = 1, 2, \dots$, the deck $\mc D_n$ is of weight $n$.
\end{definition}
\subsection{Examples of Exponential Families}
\begin{example}[Undirected Graphs $\mc F_1$] Let $G$ be an undirected graph of $n$ vertices that are labeled with a set $S$ of labels.  The standard relabeling is as follows : we relabel the vertices with $[n]$, preserving the order of the vertices.  A card $\mc C(S, p)$ has a picture $p$ which corresponds to a standard relabeling of $G$.

A hand is a collection of cards whole label sets partition $[n]$, the weight of the hand.  Each hand $H$ corresponds to a not-necessarily-connected graph with standard labels.  The individual components of the graph may have nonstandard labels, but the overall graph has a label set of $[n]$.

Using the above definitions, the set of vertex labeld graphs forms an exponential family, where each card is a labeled connected graph, each deck $\mc D_n$ is the set of connected standard labeled graphs of $n$ vertices, and each hand is a standard labeled graph.  
\end{example}

\begin{example}[Permutations $\mc F_2$] The cards are of the form $\mc C(S, p)$ where $S$ is a set of integers and $p$ is a cyclic permutation of $[|S|]$.  A deck of these cards is one sample of every distinct standard card of a given weight.  For example, the $n$-th deck $\mc D_n$ contains exactly $(n-1)!$ cards, one for each cyclic permutation of $[n]$.  A hand is a collection of cards whose label sets partition $[n]$ for some $n$. This would consist of $k$ label sets $S_1, \dots, S_k$ whose union is $[n]$ and a cyclic permutation for each $S_i$.
\end{example}

\subsection{Main Counting Theorems}
Suppose we have two exponential families $\mc F', \mc F''$ with disjoint picture sets $P', P''$ respectively.  We can form a third family $\mc F = \mc F' \oplus \mc F''$ as follows:  fix $n \ge 1$.  From $\mc F'$ we take all the cards from $\mc D_n'$ and put them in a new pile.  From $\mc F''$, we add the cards from $\mc D_n''$ to the new pile.  Then, we repeat this for each $n \ge 1$.

\begin{lemma}[Fundamental Lemma of Labeled Counting]
Let $\mc F', \mc F''$ be exponential families and $\mc F = \mc F' \oplus \mc F''$ the merged exponential family.  Let $\mc H'(x, y), \mc H''(x, y)$ and $\mc H(x, y)$ denote the corresponding two-variable generating functions of the families.  Then
$$\mc H(x, y) = \mc H'(x, y)\mc H''(x, y).$$
\end{lemma}
\begin{proof}
\begin{align*}
h(n, k) &= \sum_{n', k'} \binom{n}{n'} h'(n', k') h''(n-n', k-k') \\
&= \left[\frac{x^n}{n!} y^k \right]\mc H'(x, y) \mc H''(x, y).
\end{align*}
\end{proof}

\begin{theorem}[The Exponential Formula]
Let $\mc F$ be an exponential family whose deck and hand enumerators and $\mc D(x)$ and $\mc H(x, y)$, respectively.  Then, 
$$\mc H(x, y) = e^{y\mc D(x)}.$$
In particular, 
$$h(n, k) = \left[\frac{x^n}{n!} \right] \left\{ \frac{\mc D(x)^k}{k!} \right\}.$$
\end{theorem}

\begin{corollary} Let $\mc F$ be an exponential family, let $\mc D(x)$ be the egf of the sequence $\{d_n\}$ of sizes of the decks, and let $\mc H(x) \leftrightarrow^e \{h_n\}$, where $h_n$ is the number of hands of weight $n$.  Then,
$$\mc H(x) = e^{\mc D(x)}.$$
\end{corollary}

\begin{corollary} Let $T$ be a set of positive integers, let $e_T(x) = \sum_{n \in T} x^n/n!$, and let $h_n(t)$ be the number of hands whose weight is $n$ and whose number of cards belongs to the allowable set $T$. Then,
$$\{h_n(T)\} \leftrightarrow^e e_T(\mc D(x)).$$
\end{corollary}

The remaining section on the exponential family are a bit technical and seem beyond my use cases.  I may return to this upon completing Chapter 4.
%\subsection{Permutations and Cycles}
\pagebreak
\section{Applications of Generating Functions}
\subsection{Statistical Moments}
Suppose $f(n)$ is the number of objects, in a set $S$ of $N$ objects that have $n$ properties, with $\sum_n f(n) = N$.  Recall the mean $\mu$ is defined by 
$$\mu = \frac{1}{N} \sum_n nf(n).$$
\begin{fact}
Suppose we had $F(x) \leftrightarrow^o  \{f(n)\}$.   Then, $\mu = \frac{F'(1)}{F(1)}$.
\end{fact}

We can do the same for the variance $\sigma^2$, which is  defined by 
$$\sigma^2 = \frac{1}{N} \sum_{\omega \in s} (n(\omega) - \mu)^2.$$
\begin{fact}
Suppose we had $F(x) \leftrightarrow^o  \{f(n)\}$.    Then $\sigma^2 = \{ (\log F)' + (\log F)''\}_{x = 1}.$
\end{fact}

We can also use these to derive statistics for exponential families.  Namely, we have the following theorem:
\begin{theorem} In an exponential family $\mc F$, the average number of cards in hands of weight $n$ is 
$$\mu(n) = [h(n)x^n/n!] \mc D(x) \mc H(x) = \frac{1}{h(n)} \sum_{r} \binom{n}{r} d_rh(n-r).$$
\end{theorem}

\begin{example} Applying the formula to cycles of permutations, with $h(n) = n!$, we have 
$$\mu (n) = \frac{1}{n!} \sum_{r} \binom{n}{r} (r-1)! (n-r)! = \sum_{r=1}^n \frac{1}{r} = H_n.$$
\end{example}
\pagebreak
\subsection{The Sieve Method}
If $S \subseteq P$ is a set of properties, let $N_S$ be the number of objects whose set of properties contains $S$.  For a fixed, $r \ge 0$, consider the sum
$$N_r = \sum_{|S| = r} N_S$$
Let the symbol $P(\omega)$ denote the set of properties that $\omega$ has.  Then, we have 
\begin{align*}
N_r &= \sum_{|S| = r} N_S \\
&= \sum_{|S| = r} \sum_{\omega\in \Omega, S \subseteq P(\omega)} 1 \\
&= \sum_{\omega \in \Omega}  \sum_{|S| = r, S \subseteq P(\omega)} 1 \\
&=  \sum_{\omega \in \Omega} \binom{|P(\omega)|}{r}
\end{align*}

Suppose there are $e_t$ objects that have exactly $t$ properties.  The above formula simplifies to
$$N_r = \sum_{t \ge 0} \binom{t}{r} e_t.$$

We wish to be able to calculate $\{e_t\}$ from $\{N_t\}$, which are often much easier to compute.  To do this, let $N(x) \leftrightarrow \{N_r\},  E(x)\leftrightarrow \{e_t\}$.  We have 
\begin{align*}
N(x) &= \sum_r \sum_t \binom{t}{r} e_t x^r \\
&= \sum_t  e_t \sum_r \binom{t}{r} x^r \\
&= \sum_t e_t(x+1)^t \\
&= E(x + 1).
\end{align*}

For an explicit formula, using $E(x) = N(x-1)$, we obtain
$$e_j = \sum_t (-1)^{t-j} \binom{t}{j} N_t.  $$
\begin{example}[Fixed points of permutations] Of the $n!$ permutations of $n$ letters, how many have exactly $r$ fixed points?  

First, we can compute $N_S = (n-|S|)!$, since we keep the letters in $S$ fixed and permute the other letters.  

Next,
$$N_r = \sum_{|S|  = r} N_S = \sum_{|S| = r} (n - |S|)! =  \binom{n}{r} (n-r)! = \frac{n!}{r!}.$$

The ordinary generating function $N(x)$ is given by 
$$N(x) = \sum_{r=0}^n \frac{n!}{r!} x^r = n! \sum_{r=0}^n \frac{x^r}{r!}.$$

Finally,
$$E(x) = \sum_t e_tx^t = n! \sum_{r=0}^n \frac{(x-1)^r}{r!}.$$
\end{example}

\begin{example}[Subset Combinatorial Identity] For fixed $n \in \N$, take $\Omega$ to be the $\binom{2n}{n}$ ways of choosing an $n$-subset of $[2n]$.  For the set $P$ of properties, we take the list of $n$ properties; an $n$-subset $Q$ has the property $i$ if $i \not \in Q$ for each $i \in [n]$.

Note that 
$$N_S = \binom{2n - |S|}{n},$$
Then,
$$N_r = \sum_{|S| = r} N_S = \binom{n}{r} \binom{2n-r}{n}.$$

It follows that 
$$\sum_j e_j t^j = \sum_{r} \binom{n}{r} \binom{2n-r}{n} (t-1)^r.$$

Alternatively, note that $e_j = \binom{n}{j}^2$, since we can choose $j$ elements that are missing in $\binom{n}{j}$ ways and we can choose the other $j$ elements from $n+1, \dots, 2n$ in $\binom{n}{j}$ ways.  It follows that 
$$\sum_j \binom{n}{j}^2 t^j = \sum_r \binom{n}{r} \binom{2n-r}{n} (t-1)^r.$$
\end{example}

\subsection{Snake Oil Method}
The method handles a large number of combinatorial identities.  It goes as follows:
\begin{enumerate}
\item Identify the free variable, say $n$.  Give a name to the sum $f(n)$.
\item Let $F(x)$ be the opsgf whose $[x^n]$ is $f(n)$.
\item Multiply the sum by $x^n$, and sum on $n$.
\item Interchange the order of summation and evaluate the inner one in a simple closed form.
\item Identify the coefficients of the generating function of the answer.  
\end{enumerate}

Some useful closed form expressions we will use are 
$$\sum_{r \ge 0} \binom{r}{k} x^r = \frac{x^k}{(1-x)^{k+1}},$$
$$\sum_r \binom{n}{r} x^r = (1 + x)^n,$$
$$\sum_n \frac{1}{n+1} \binom{2n}{n} x^n = \frac{1}{2x} (1 - \sqrt{1 - 4x}).$$

\begin{example} Compute $$\sum_{k \ge 0} \binom{k}{n-k}.$$

\end{example}
\begin{proof}
Define $f(n) = \sum_{k \ge 0} \binom{k}{n-k}$.  We have 
\begin{align*}
F(x) &= \sum_n x^n \sum_{k \ge 0} \binom{k}{n-k} \\
&= \sum_{k \ge 0} \sum_{n} \binom{k}{n-k} x^n \\
&= \sum_{k \ge 0} x^{k} \sum_{n} \binom{k}{n-k} x^{n-k} \\
&= \sum_{k \ge 0} x^k \sum_{r} \binom{k}{r} x^r \\
&= \sum_{k \ge 0} x^k (1 + x)^k \\
&= \sum_{k \ge 0} (x + x^2)^k \\
&= \frac{1}{1 - x -x^2}.
\end{align*}
This is exactly the generating function for the Fibonacci numbers, $f(n) = F_{n + 1}$.
\end{proof}
\begin{example} Compute 
$$\sum_k \binom{n + k}{m + 2k} \binom{2k}{k} \frac{(-1)^k}{k + 1}.$$
\end{example}
\begin{proof}
\begin{align*}
F(x) &= \sum_{n \ge 0} x^n \sum_{k} \binom{n + k}{m + 2k} \binom{2k}{k} \frac{(-1)^k}{k + 1} \\
&= \sum_k \binom{2k}{k} \frac{(-1)^k}{k + 1} x^{-k} \sum_{n \ge 0} \binom{n + k}{m + 2k} x^{n + k} \\
&= \sum_k \binom{2k}{k} \frac{(-1)^k}{k+1} x^{-k} \sum_{r\ge k} \binom{r}{mm + 2k} x^r \\
&= \sum_k \binom{2k}{k} \frac{(-1)^k}{k+1} x^{-k} \frac{x^{m + 2k}}{(1 - x)^{m + 2k + 1}} \\
&= \frac{x^k}{(1 - x)^{m + 1} }\sum_k \binom{2k}{k} \frac{1}{k + 1} \left(\frac{-x}{(1-x)^2} \right)^k \\
&= \frac{-x^{m-1}}{2(1-x)^{m-1}} \left (1 - \sqrt{1 + \frac{4x}{(1-x)^2}} \right) \\
&= \frac{-x^{m-1}}{2(1-x)^{m-1}} \left (1 - \frac{1+x}{1-x} \right) \\
&= \frac{x^m}{(1-x)^m}.
\end{align*}
It follows that $f(n, m) = \binom{n-1}{m-1}$.
\end{proof}
\pagebreak
\section{Problem Solving}
\begin{problem}[Putnam 2020 A2] Let $k$ be a non-negative integer.  Evaluate 
$$\sum_{j=0}^k 2^{k-j} \binom{k+j}{j}.$$
\begin{proof}
We claim the sum evaluates to $4^k$.  Note that $\binom{k+j}{j} = \binom{k+j}{k}$.  It follows that the sum is the coefficient of $x^k$ in the power series $\sum_{j=0}^n 2^{k-j} (1 + x)^{k + j}$.  Evaluating this, we find
\begin{align*}
\sum_{j=0}^n 2^{k-j} (1 + x)^{k + j} &= 2^{k}(1 + x)^{k} \sum_{j=0}^k 2^{-j} (1 + x)^j \\
&= 2^k (1 + x)^k \frac{1 - (1 + x)^{k+1}/2^{k+ 1}}{1 - (1 + x)/2} \\
&= \frac{2^{k + 1}(1 + x)^k - (1 + x)^{2k + 1}}{1 - x} \\
&= 2^{k + 1}(1 + x)^k - (1 + x)^{2k + 1} \sum_{n \ge 0} x^n.
\end{align*}
It follows that the coefficient of $x^k$ is given by 
$$2^{k +1} \sum_{j=0}^k \binom{k}{j} - \sum_{j=0}^k \binom{2k+1}{j} = 2^{2k+1} - 2^{2k} = 4^k.$$
\end{proof}
\end{problem} 
\begin{problem}(CJMO 2020/1) Let $N$ be a positive integer, and let $S$ be the set of all tuples with positive integer elements and a sum of $N$.  For all tuples $t$, let $p(t)$ denote the product of all the elements of $t$.  Evaluate 
$$\sum_{t \in S} p(t).$$
\end{problem}
\begin{proof}
We claim the sum evaluates to $F_{2N}$, where $F_k$ denotes the $k$-th Fibonacci number.  Note that the sum can be represented as the coefficient of $x^N$ in $\sum_{k=1}^N \left (\sum_{n \ge 0} nx^n \right)^k$.  Evaluating this, we find 
\begin{align*}
\sum_{k=1}^N \left (\sum_{n \ge 0} nx^n \right)^k &=  \sum_{k=1}^N \left ( \frac{x}{(1 - x)^2} \right)^k \\
&= \sum_{k=1}^N \frac{x^k}{(1 - x)^{2k}} \\
&= \sum_{k=1}^N \sum_{j \ge 0} \binom{2k - 1 + j}{2k - 1} x^{j + k} .
\end{align*}
The coefficient of $x^N$ is given by  $$\sum_{k=1}^N \binom{N + k - 1}{2k - 1} = \sum_{k=1}^N \binom{N + k - 1}{N - k} = \sum_{j \ge 0} \binom{2N - 1 - j}{j} = F_{2N}.$$
\end{proof}
\begin{problem}[IMO 1995/6] Let $p$ be an odd prime number. How many $p$-element subsets $ A$ of $ \{1,2,\dots,2p\}$ are there, the sum of whose elements is divisible by $p$?
\end{problem}
\begin{proof}
Define $f(x, y) = \prod_{k=1}^{2p} (1 + x^k y)$.  We wish to find the sum of the coefficients of terms of the form $x^{p \ell} y^p$.  We do this by first considering $f$ as a generating function in $x$ using the root of unity filter associated to $\omega = e^{\frac{2\pi i }{p}}$.   Then, we read off the coefficient of $y^p$ to find the desired expression.

Note that for $1 \le k \le p-1$, 
$$f(\omega^k, y) = \prod_{k=1}^{2p} (1 + \omega^k y) = \prod_{k=1}^{p} (1 + \omega^k y)^2 = (1 + y^p)^2.$$

It follows that 
\begin{align*}
\frac{1}{p} \sum_{i=0}^{p-1} f(\omega^k, y) &= \frac{1}{p} \left ((1 +y)^{2p} + \sum_{i=1}^{p-1} f(\omega^k, y) \right) \\
&= \frac{(1 + y)^{2p} + (p-1)(1 + y^p)^2}{p}.
\end{align*}
Finally, the coefficient of $y^p$ is given by 
$$\frac{\binom{2p}{p} + 2(p-1)}{2}.$$
\end{proof}
 \end{document}


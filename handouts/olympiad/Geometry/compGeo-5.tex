\documentclass[11pt]{scrartcl}
\usepackage[sexy]{evan}
\usepackage{graphicx}

\newcommand{\N}{\mathbb{N}}
\newcommand{\Z}{\mathbb{Z}}
\newcommand{\F}{\mathbb{F}}
\newcommand{\Q}{\mathbb{Q}}
\newcommand{\R}{\mathbb{R}}
\newcommand{\C}{\mathbb C}
\newcommand{\T}{\mathbb T}
\newcommand{\PP}{\mathbb P}
\newcommand{\supp}{\text{supp }}

\renewcommand{\Re}{\operatorname{Re}}
\renewcommand{\Im}{\operatorname{Im}}


\let \phi \varphi

%From Topology
\newcommand{\cT}{\mathcal{T}}
\newcommand{\cB}{\mathcal{B}}
\newcommand{\cC}{\mathcal{C}}
\newcommand{\cH}{\mathcal{H}}

\usepackage{answers}
\Newassociation{hint}{hintitem}{all-hints}
\renewcommand{\solutionextension}{out}
\renewenvironment{hintitem}[1]{\item[\bfseries #1.]}{}
\declaretheorem[style=thmbluebox,name={Theorem}]{thm}

\begin{document}
\title{Analytic Methods in Geometry}
\author{Vishal Raman}
\maketitle
\begin{abstract}
We present several computation results and theorems that prove to be useful in geometry problems at the Olympiad level.  This includes Cartesian coordinates, complex numbers, barycentric coordinates, etc.  
\end{abstract}
\tableofcontents
\pagebreak
\section{Computational Geometry}
\subsection{Cartesian Coordinates}
\begin{itemize}
\item (Shoelace Formula) Given three points $A = (x_1, y_1), B = (x_2, y_2), C = (x_3, y_3)$, the signed area of triangle $ABC$ is given by
$$\begin{vmatrix}
x_1 &y_1  &1 \\ 
 x_2&y_2  &1 \\ 
 x_3&y_3  &1 
\end{vmatrix}.$$
The area of a triangle $ABC$ is positive if $A, B, C$ appear in counterclockwise order.
\item Three points are collinear if and only if the area of the triangle they determine is zero.  This gives a symmetric formula for determining collinearity.
\item (Point-to-Line Distance) If $\ell$ is the line determined by $Ax + By + C = 0$, the shortest distance from $P=(x_1, y_1)$ to $\ell$ is given by 
$$d(P, \ell) = \frac{|Ax_1 + By_1 + C|}{\sqrt{A^2 + B^2}}.$$
\item (Point-to-Plane Distance) If $H$ is the plane determined by  $Ax + By + Cz + D = 0$  the shortest distance between the point $P = (x_0, y_0, z_0)$ and the plane is given by
$$d(P, H) = \frac{|Ax_0 + By_0 + Cz_0 +D |}{\sqrt{A^2 + B^2 + C^2}}$$
\end{itemize}
Problems that are most effectively solved using Cartesian coordinates usually have defining characteristics, such as
\begin{itemize}
\item a prominent right angle at the origin.
\item many intersections or perpendicular lines.
\end{itemize}
\subsection{Areas}
The area of a triangle $ABC$ is given by
\begin{itemize}
\item $\frac{1}{2}ab \sin C =\frac{1}{2}bc \sin A = \frac{1}{2}ca \sin B$
\item $\frac{a^2 \sin B \sin C}{2 \sin A}$
\item $\frac{abc}{4R}$, $R$ is the circumradius.
\item $sr$, $s$ is the semi-perimeter, $r$ is the inradius.
\item $\sqrt{s(s-a)(s-b)(s-c)}$.
\end{itemize}
A useful result from trignometry:
\begin{lemma} If $x, y, z$ satisfy $x + y + z = 180^\circ$ and $0^\circ < x , y, z < 90^\circ$, then $\tan x + \tan y + \tan z = \tan x \tan y \tan z$.
\end{lemma}
\begin{example} In $ABC$, $AB = 13$, $BC = 14$, $CA = 15$.  Find the length of the altitude from $A$ onto $\overline{BC}$.
\end{example}
\begin{proof}
Note that $s = (13 + 14 + 15)/2 = 21$.  It follows that 
$$[ABC] = \sqrt{(21)(21-13)(21-14)(21-15)} = \sqrt{(21)(8)(7)(6)} = 84.$$

Then, $$[ABC] = \frac{1}{2}h_A \cdot BC  = 84$$
so it follows that $h_A = \frac{2 \cdot 84}{BC} = 12$.
\end{proof}
\subsection{Trignometry}
\begin{itemize}
\item (Law of Sines) 
$$\frac{a}{ \sin A} = \frac{b}{\sin B} = \frac{c}{ \sin C} = 2R.$$
\item (Law of Cosines) Given a triangle $ABC$, we have 
$$a^2 = b^2 + c^2 - 2bc \cos A.$$
\item (Product-Sum)
\begin{align*}
2 \cos \alpha \cos \beta &= \cos(\alpha - \beta) + \cos (\alpha + \beta) \\
2 \sin\alpha \sin \beta &= \cos(\alpha - \beta) - \cos(\alpha + \beta) \\
2 \sin \alpha \cos \beta &= \sin(\alpha - \beta) + \sin(\alpha + \beta)
\end{align*}
\end{itemize}
\subsection{Ptolemy's Theorem}
\begin{thm}[Ptolemy's Theorem] Let $ABCD$ be a cyclic quadrilateral.  Then $$AB \cdot CD + BC \cdot DA = AC \cdot BD.$$
\end{thm}
\begin{proof}
We prove the result using trignometry.  Let $\alpha_1, \alpha_2, \alpha_3, \alpha_4$ be the angles $\angle ADB$, $\angle BAC$, $\angle CBD$, $\angle DCA$ respectively.  WLOG let $(ABCD)$ have unit diameter.  It follows from the law f sines that 
$$AB = \sin \alpha_1, BC = \sin \alpha_2, CD = \sin \alpha_3, DA = \sin \alpha_4.$$

Furthermore,
$$AC = \sin \angle ABC = \sin(\alpha_3 + \alpha_4),$$
$$CD = \sin \angle DAB = \sin(\alpha_2 + \alpha_3).$$

It suffices to show that for $\alpha_1 + \alpha_2 + \alpha_3 + \alpha_4 = 180^\circ$, 
$$\sin \alpha_1 \sin \alpha_3 + \sin \alpha_2 \sin \alpha_4 = \sin(\alpha_3 + \alpha_4) \sin (\alpha_2 + \alpha_3).$$

This result can be easily verified through the product to sum identities.
\end{proof}

\begin{thm}[Strong Form of Ptolemy's Theorem] In a cyclic quadrilateral $ABCD$ with $AB = a$, $BC = b$, $CD = c$, $DA = d$, we have 
$$AC^2 = \frac{(ac + bd)}{ab+cd}$$
and 
$$BD^2 = \frac{(ac + bd)(ab + cd)}{ad + bc}.$$
\end{thm}
\begin{proof}
The proof follows from setting
$$AC^2 = a^2 + b^2 - 2ab\cos \angle ABC = c^2 + d^2 - 2cd \cos \angle ADC$$
and noting that $\angle ADC + \angle ABC = 180^\circ$.
\begin{thm}[Stewart's Theorem] Let $ABC$ be a triangle.  Let $D$ be on $\overline{BC}$ and $m = DB$, $n = DC$, $d = AD$.  Then
$$a(d^2 + mn) = b^2m + c^2n.$$
\end{thm}

\end{proof}
\subsection{Problems}
\end{document}

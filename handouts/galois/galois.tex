\documentclass[11pt]{scrartcl}
\usepackage[sexy]{evan}
\usepackage{graphicx}

\newcommand{\N}{\mathbb{N}}
\newcommand{\Z}{\mathbb{Z}}
\newcommand{\F}{\mathbb{F}}
\newcommand{\Q}{\mathbb{Q}}
\newcommand{\R}{\mathbb{R}}
\newcommand{\C}{\mathbb C}
\newcommand{\T}{\mathbb T}
\newcommand{\PP}{\mathbb P}
\newcommand{\supp}{\text{supp }}

\renewcommand{\Re}{\operatorname{Re}}
\renewcommand{\Im}{\operatorname{Im}}


\let \phi \varphi

%From Topology
\newcommand{\cT}{\mathcal{T}}
\newcommand{\cB}{\mathcal{B}}
\newcommand{\cC}{\mathcal{C}}
\newcommand{\cH}{\mathcal{H}}

\usepackage{answers}
\Newassociation{hint}{hintitem}{all-hints}
\renewcommand{\solutionextension}{out}
\renewenvironment{hintitem}[1]{\item[\bfseries #1.]}{}
\declaretheorem[style=thmbluebox,name={Theorem}]{thm}

\begin{document}
\title{Galois Theory}
\author{Vishal Raman}
\maketitle
\begin{abstract}
These notes are part of my in-depth review of Galois Theory.  I will be following the lectures of Professor Richard Borcherds along with supplemental notes from the classic Galois Theory text from E. Artin.  Any mistakes and typos are my own - kindly direct them to my inbox.
\end{abstract}
\tableofcontents
\pagebreak
\section{Introduction}
The main use of Galois Theory is to take problems about polynomials and translate them into problems in Group theory.  The corresponding group is called the \textbf{Galois group}.

\subsection{Example: Solving Polynomials}
Can a polynomial be solved be radicals?  This is well known for polynomials of degree $2$, given by the quadratic formula.  Can we do this for higher degree polynomials?  Yes,  for polynomials up to degree $4$.  However, no for general polynomials of degree at least $5$.  The corresponding problem using Galois theory is a Galois group which is a subgroup of $S_n$.  A polynomial can be solved by radicals if the subgroup is \textbf{solvable}, that is, it can be split into a product of Abelian groups.

\subsection{Example: Geometric Constructions}
We try to trisect the angle of $60^\circ$ using ruler and compass.  This would correspond to factoring a polynomial which has roots $\cos 20^\circ, \dots$.  This turns out to correspond to a Galois group $\Z/3\Z$.  It turns out that being able to construct an object is related to the Galois group being of an order of a power of $2$, so it fails in this case.

Another famous example is Gauss's construction of the heptadecagon, which has $17$ sides.  Though it is very difficult to construct by hand, the Galois group of the corresponding polynomial is $(\Z/17\Z)^* = \Z/16\Z$, so it is constructable.  

\subsection{Main Idea of Galois Theory}
Given a polynomial $a_nx^n + \dots + a_0$ over $\Q[x]$, we look at the field generated by the roots of the polynomial, $\alpha_1, \alpha_2, \dots, \alpha_n$.  The \textbf{Galois group} is the permutations of $\alpha_1, \dots, \alpha_n$ preserving all algebraic relations between the roots.  It is clear that the Galois group is always some subgroup of $S_n$.

For $x^5 - 2$, we have roots $2^{1/5}, 2^{1/5}\zeta, 2^{1/5}\zeta^2, 2^{1/5}\zeta^3, 2^{1/5}\zeta^4$, where $\zeta$ is a fifth root of unity.  We have algebraic relations $\alpha_1\alpha_3 = \alpha_2^2$, $\alpha_2/\alpha_1 = \alpha_4/\alpha_3, \dots$.  It turns out the subgroup of permutations has order $20$, rather than 120.

If we consider a field extension $K \subseteq L$, the Galois group of the extension is the symmetries of $L$ fixing all elements of $K$.  For example, if we take $\R \subseteq \C$, the Galois group consists of $1$ and complex conjugation.  

\begin{thm} Suppose $K \subseteq L$ is a Galois extension.  That is, the order of the Galois group is $L$, which is the dimension of $L$ as a vector space over $K$.  
Then, subfields $M$ with $K \subseteq M \subseteq L$ correspond exactly to the subgroups of the Galois group.
\end{thm}

\subsection{Applications}
\begin{itemize}
\item The Langlands program: Galois groups of fields $L$, $\Q \subseteq L$ are related to modular forms.  An example of a modular form is the discriminant function
$$\Delta(q) = q \prod_{n \ge 1} (1 - q^n) = q- 24q + 252q^2 + \dots$$
where the coefficients of Ramanujan's $\tau(n)$ function. 

Wiles used this to prove Fermat's Last theorem, namely he related the solution of Fermat's Last theorem to an Elliptic curve and he related this to a Modular Form by considering the action of the Galois group.  Then, Ken Ribet showed that the modular form is not possible, which proves the theorem.
\item The Galois group is related to the Fundamental group.  If we have an extension $K \subseteq L$, this corresponds to a covering space.  The Galois group corresponds to the fundamental group of the base space.  The algebraic closure of a field corresponds to a universal covering space.
\end{itemize}

Inverse Galois Problem: Given a finite group $G$, is there an extension $K$ of $\Q$ so that the Galois group of $K$ is $G$?  This is true for many groups, but the problem is completely open in general.
\pagebreak
\section{Field Extensions}
\subsection{Algebraic Numbers}
\begin{definition} A field extension is a pair of fields $K \subseteq L$, denoted $L/K$.
\end{definition}
In Galois Theory, it is sometimes helpful to start with smaller fields and build up to the full field.
The most basic example is given by $\Q \subseteq \R \subseteq \C$.

The degree of a field extension $L/K$, denoted by $[L:K]$  is the dimension of $L$ as a vector space over $K$.  The field extension is called finite if it has finite degree.

\begin{definition}
Suppose we have $K \subseteq L$ and $\alpha \in L$.   $\alpha$ is called \textbf{algebraic} over $K$ if it is a root of a polynomial $p(x) \in K[x]$.  Otherwise, it is called  \textbf{transcendental}.  The degree of $\alpha$is the degree of an irreducible polynomial $p$ where $\alpha$ is a root.
\end{definition}
\begin{itemize}
\item It is clear that $2^{1/5}$ is algebraic since it is the root of $x^5 - 2 = 0$.  
\item The reals $\pi, e$ are transcendental.  
\item The real $\alpha = \cos{2\pi/7}$ is algebraic.   Note that $\alpha = \frac{\zeta + \zeta^{-1}}{2}$ where $\zeta = e^{2\pi i/7}$.  We can apply a clever transformation to $1 + \zeta + \zeta^2 + \dots + \zeta^6 = 0$ to find a polynomial relation in terms of $\alpha$.
\item Is $e + \pi$ transcendental?  Is $e \pi$ transcendental?  We can show that $e + \pi$ or $e\pi$ is transcendental.

Consider $x^2 - (e + \pi)x + e\pi$.  If $e + \pi$ and $e \pi$ are both algebraic then $e, \pi$ are algebraic, a contradiction.
\end{itemize}

\subsection{Algebraic Extensions}
\begin{thm} If we have $K \subseteq M$, $\alpha \in M$ is algebraic over $K$ if and only if it is contained in a finite extension of $K$.
\end{thm}
\begin{proof}
If $\alpha$ is contained in a finite extension $L$, with $[L : K] = n < \infty$, then consider $1, \alpha, \alpha^2 , \dots, \alpha^n$, which are $n+1$ elements of an $n$-dimensional vector space.  So we have a non-trivial linear relation in terms of these elements which is zero, which gives a polynomial relation of $\alpha$.

Suppose $p(x)$ is an irreducible polynomial in $K[x]$.  Then, if we consider $K[x]\setminus (p)$, this is a field.  It is obviously a ring, but the existence of inverses needs to be verified.  Suppose we have an element $q(x) \in K[x]\setminus (p)$, with $q \ne 0$.  Then $q, p$ are coprime in $K[x]$ since $p$ is irreducible.  Then, we can find $a(x)q(x) + b(x)p(x) = 1$ using the Euclidean algorithm.  Hence, $a(x)$ is the inverse of $q(x)$.

If $\alpha$ is algebraic, then $\alpha$ is a root of $p(x) \in K[x]$ with $p$ irreducible.  Consider the field $\frac{K[x]}{(p)}$.  This contains $K$ and we can map this field in $L$ with the map $x \mapsto \alpha$, which also contains $K$.  The image is a field in $L$ containing $\alpha$, and the field has dimension of the degree of $p$ so it is finite.
\end{proof}

Consider a tower of extensions $K \subseteq L \subseteq M$.  Then, $[M:K] = [M:L][L:K]$.  Pick a basis $x_1, \dots, x_m$ of $L/K$ and pick a basis $y_1, \dots, y_n$ for $M/L$ and check that $x_iy_j$ for $1 \le i \le m$ and $1 \le j \le n$ form a basis for the vector space $M$ over $K$.

\begin{thm} If $\alpha, \beta \in L$ are algebraic over $K$, then so are $\alpha + \beta, \alpha \beta, \alpha/\beta, \alpha-  \beta$.
\end{thm}
\begin{example} Consider $2^{1/2} + 2^{1/3} + 2^{1/5}$.  It is hard to come up with a polynomial relation with this element, but is algebraic as the sum of algebraic elements.
\end{example}
\begin{proof}
Consider $K \subseteq K(\alpha) \subseteq K(\alpha, \beta)$, which are finite extensions, which gives the desired result, since the elements are algebraic relations of $\alpha$ and $\beta$.
\end{proof}

\begin{thm} ($K \subseteq L$) If $\alpha$ is the root of a polynomial with algebraic coefficients, then $\alpha$ is algebraic.
\end{thm}
\begin{proof}
For a polynomial $\alpha^n + a_{n-1}\alpha^{n-1} + \dots + a_0 = 0$, take
$$K \subseteq K(a_0) \subseteq K(a_0, a_1) \subseteq \dots \subseteq K(a_0, \dots, a_{n-1}) \subseteq K(a_0, \dots, a_{n-1}, \alpha),$$
which are all finite extensions.  It follows that $\alpha$ is algebraic.
\end{proof}
\pagebreak
\section{Splitting Fields}
\subsection{Definition and Examples}
Suppose we have a field $K$ and a polynomial $p(x) \in K[x]$.  Problem: Find an extension $K \subseteq L$ such that $p$ factors into linear factors over $L[x]$, and $L$ is generated by the roots of $p$ over $K$(L is a minimal field extension).  Then, $L$ is called a \textbf{splitting field} of $K$.

\begin{itemize}
\item $p(x) = x - a_0$.  Then, the splitting field is $K$.
\item $p(x) = x^2 + a_1x + a_0 = 0$.  If this is reducible, then the splitting field is $K$.  Otherwise, take
$$L = \frac{K[x]}{(p)},$$
which is a maximal ideal since $p$ is ireeducible, so it is a field.  It contains a root of $p$ since the image of $x$ is a root by construction.  Furthermore, it contains all roots, since $p = (x - \alpha)(x - \beta)$, $\alpha + \beta = -a_1 \in K$, so $L$ is a splitting field.
\item $p(x) = x^3 - 2$ over $Q[x]$.  Take $L = Q(\sqrt[3]{2}) \cong \frac{Q[x]}{(x^3 - 2)}.$  The roots are $\sqrt[3]{2}$ multiplied by the 3rd roots of unity, so it is clear that $L$ does not contain all the roots of $p$. Adjoining one of the complex roots gives the proper splitting field:
$$\Q \subseteq \Q(\sqrt[3]{2}) \subseteq \Q(\sqrt[3]{2}, \omega)$$
where $\omega $ is a third root of unity.  Note that the degree of the splitting field extension is $6$.
\item $8x^3 +4x^2 - 4x - 1$, with roots $\cos(2\pi/7), \cos(4\pi/7), \cos(6\pi/7)$.

If we take $\Q \subseteq Q[x]/(p)$, note that $\cos(4\pi/7) = 2(\cos(2\pi/7))^2 - 1$, so the degree of the splitting field over $\Q$ is $3$.
\item Take $x^4 + 1$ over $\Q$.  If $\alpha$ is a root, so it $\alpha^3, \alpha^5, \alpha^7$.  The splitting field is $Q[\alpha] \cong Q[x]/(x^4 + 1)$, which has degree $4$.
\end{itemize}
\subsection{Existence and Uniqueness}
We first prove the existence of splitting fields.  Given a field $K_0 = K$ and a polynomial $p = p_1p_2 \dots p_n$, where $p_i$ are irreducible in $K[x]$.  If we take $K_1 = K_0[x]/(p_1)$, this is a field since $p_1$ is irreducible.  We repeat the construction until all of the terms are of degree $1$.  

Any 2 splitting fields of $p$ over $K$ are isomorphic as extensions: given $L, L'$ over $K$, there is an isomorphism of $L$ and $L'$ commuting with the embeddings of $K$ into $L$ and $L'$.

Suppose we have a field $K$ and an isomorphism to a field $K'$ sending $p$ to $p'$ and a splitting field of $K$ and $K'$ given by $L$ and $L'$ respectively.  We wish to show that $L$ is isomorhpic to $L'$.  More generally, assume $L$ is generated by roots of $p$ and $L'$ contains all roots of $p'$.

Factor $p = p_1 p_2 \dots$ with $p_i$ irreducible.  Let $\alpha$ be a root of $p_1$.  Consider the field $K_1(\alpha)$.  Over $K'$, $p' = p_1'p_2', \dots$, $\alpha$ is mapped to a root of $p_1'$.  We continue like this with the other factors of $p$ and we construct a map of fields from $L$ to $L'$.  In particular, $[L:K] \le [L': K']$. Reversing the argument, we find that $[L':K'] \le [L:K]$ so it follows that they are equal, and the map is an isomorphism.  

While we do have an isomorphism, one can ask whether there is a canonical splitting field for a given field.  This is not entirely true, and one needs to keep this in mind for certain problems.
\pagebreak
\section{Algebraic Closure}
\subsection{Definitions}

Suppose $K$ is countable.  List the polynomials in $K[x]$, $p_1, p_2, p_3, \dots$ and take the field $K = K_0 \subseteq K_1 \subset K_2 \subset \dots$ where $K_i$ is the splitting field of $p_1$.  Then, we set $\overline{K} = \bigcup_{i=1}^\infty K_i$.  If $K$ is uncountable, we well-order the polynomials $p_i$ and repeat the construction.

A field is called \textbf{algebraicly closed} if all polynomials in $L[x]$ have roots in $L$.  

Any polynomial in $K[x]$ has a root in $\overline{K}$, but we want to show that any polynomial in $\overline{K}[x]$ has a root in $\overline{K}$.  For $p(x) \in \overline{K}[x]$ with coefficients $a_0, a_1, a_2, \dots$ and $\alpha$ is a root, then
$$K \subseteq K[a_0, a_1, \dots, a_{n-1}] \subseteq K[a_0, \dots, a_{n-1}, \alpha].$$

Since all the extensions are finite, $\alpha$ is algebraic over $K$, so $\alpha$ is the root of a polynomial which factors linearly in $\overline{K}$, so it follows that $\overline{K}$ is algebraicly closed.

\begin{example} Find the smallest possible field $L$ with $K \subseteq L$ so that all elements of $L$ have a square root in $L$. 

Take $K = K_0 \subseteq K_1$, where we adjoin all square roots of elements of $K_0$.  Then, we extend $K_1 \subseteq K_2$ where adjoin all the square roots of $K_1$, and repeat the construction.  Then set $L$ to be the union of all the fields and the result follows.  
 \end{example}

\subsection{Fundamental Theorem of Algebra}
If we take $K = \R$, then $\overline{K} = \C$.  
 \begin{proof}
 Suppose $p(x) = x^n + a_{n-1}x^{n-1} + \dots + a_0 \in \C[x]$.  If $|x| = R$ large, then $|x^n| > |a_{n-1}x^{n-1 } + \dots + a_0|$.  As we contract $R$ to $0$, the number of times the polynomial goes to $0$ starts at $n$ and goes to $0$.  It follows that there is some point so that it passes through the origin.  The number of times it goes around is sometimes called the Winding number.  So it follows that $x^n + f(x)$ has a zero if $|f(x)| < |x^n|$ for $|x| = R$.  
 \end{proof}
 
 \subsection{Uniqueness of $\overline{K}$}
 Any 2 algebraic closures of $K$ are isomorhpic.  However, the isomorphic is not unique, and there is no canonical choice for the isomorphic.  
 
 One may ask if $K \to \overline{K}$ is a functor.  This is not easy, because we cannot choose the natural map in an easy way, without maybe invoking the Axiom of Choice many times.
 
 The uniqueness of the algebraic closure is analogous to finding the Fundamental group of $X$.  It is not a functor unless we choose a base point.  Using some argument with the Etale topology, we can argue that $\text{Aut}(\overline{K})$, the absolute Galois Group is not a functor, but if we choose some algebraic closure, then we do obtain a functor.
 
 A Groupoid is a category such that all morhpisms are isomorhpisms.  We can construct a fundamental groupoid that doesn't depend on a base point and an Absolute Galois Groupoid that does not depend on an Algebraic closure.  We choose points as the objects and homotopy classes of paths from $x_0 \to x_1$ as the morphisms.  Similarly, the points are algebraic closures of $K$ and the morhpisms are the isomorphisms between closures.
\end{document}

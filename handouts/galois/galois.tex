\documentclass[11pt]{scrartcl}
\usepackage[sexy]{evan}
\usepackage{graphicx}

\newcommand{\N}{\mathbb{N}}
\newcommand{\Z}{\mathbb{Z}}
\newcommand{\F}{\mathbb{F}}
\newcommand{\Q}{\mathbb{Q}}
\newcommand{\R}{\mathbb{R}}
\newcommand{\C}{\mathbb C}
\newcommand{\T}{\mathbb T}
\newcommand{\PP}{\mathbb P}
\newcommand{\supp}{\text{supp }}

\renewcommand{\Re}{\operatorname{Re}}
\renewcommand{\Im}{\operatorname{Im}}


\let \phi \varphi

%From Topology
\newcommand{\cT}{\mathcal{T}}
\newcommand{\cB}{\mathcal{B}}
\newcommand{\cC}{\mathcal{C}}
\newcommand{\cH}{\mathcal{H}}

\usepackage{answers}
\Newassociation{hint}{hintitem}{all-hints}
\renewcommand{\solutionextension}{out}
\renewenvironment{hintitem}[1]{\item[\bfseries #1.]}{}
\declaretheorem[style=thmbluebox,name={Theorem}]{thm}

\begin{document}
\title{Galois Theory}
\author{Vishal Raman}
\maketitle
\begin{abstract}
These notes are part of my in-depth review of Galois Theory.  I will be following the lectures of Professor Richard Borcherds along with supplemental notes from the classic Galois Theory text from E. Artin.  Any mistakes and typos are my own - kindly direct them to my inbox.
\end{abstract}
\tableofcontents
\pagebreak
\section{Introduction}
The main use of Galois Theory is to take problems about polynomials and translate them into problems in Group theory.  The corresponding group is called the \textbf{Galois group}.

\subsection{Example: Solving Polynomials}
Can a polynomial be solved be radicals?  This is well known for polynomials of degree $2$, given by the quadratic formula.  Can we do this for higher degree polynomials?  Yes,  for polynomials up to degree $4$.  However, no for general polynomials of degree at least $5$.  The corresponding problem using Galois theory is a Galois group which is a subgroup of $S_n$.  A polynomial can be solved by radicals if the subgroup is \textbf{solvable}, that is, it can be split into a product of Abelian groups.

\subsection{Example: Geometric Constructions}
We try to trisect the angle of $60^\circ$ using ruler and compass.  This would correspond to factoring a polynomial which has roots $\cos 20^\circ, \dots$.  This turns out to correspond to a Galois group $\Z/3\Z$.  It turns out that being able to construct an object is related to the Galois group being of an order of a power of $2$, so it fails in this case.

Another famous example is Gauss's construction of the heptadecagon, which has $17$ sides.  Though it is very difficult to construct by hand, the Galois group of the corresponding polynomial is $(\Z/17\Z)^* = \Z/16\Z$, so it is constructable.  

\subsection{Main Idea of Galois Theory}
Given a polynomial $a_nx^n + \dots + a_0$ over $\Q[x]$, we look at the field generated by the roots of the polynomial, $\alpha_1, \alpha_2, \dots, \alpha_n$.  The \textbf{Galois group} is the permutations of $\alpha_1, \dots, \alpha_n$ preserving all algebraic relations between the roots.  It is clear that the Galois group is always some subgroup of $S_n$.

For $x^5 - 2$, we have roots $2^{1/5}, 2^{1/5}\zeta, 2^{1/5}\zeta^2, 2^{1/5}\zeta^3, 2^{1/5}\zeta^4$, where $\zeta$ is a fifth root of unity.  We have algebraic relations $\alpha_1\alpha_3 = \alpha_2^2$, $\alpha_2/\alpha_1 = \alpha_4/\alpha_3, \dots$.  It turns out the subgroup of permutations has order $20$, rather than 120.

If we consider a field extension $K \subseteq L$, the Galois group of the extension is the symmetries of $L$ fixing all elements of $K$.  For example, if we take $\R \subseteq \C$, the Galois group consists of $1$ and complex conjugation.  

\begin{thm} Suppose $K \subseteq L$ is a Galois extension.  That is, the order of the Galois group is $L$, which is the dimension of $L$ as a vector space over $K$.  
Then, subfields $M$ with $K \subseteq M \subseteq L$ correspond exactly to the subgroups of the Galois group.
\end{thm}

\subsection{Applications}
\begin{itemize}
\item The Langlands program: Galois groups of fields $L$, $\Q \subseteq L$ are related to modular forms.  An example of a modular form is the discriminant function
$$\Delta(q) = q \prod_{n \ge 1} (1 - q^n) = q- 24q + 252q^2 + \dots$$
where the coefficients of Ramanujan's $\tau(n)$ function. 

Wiles used this to prove Fermat's Last theorem, namely he related the solution of Fermat's Last theorem to an Elliptic curve and he related this to a Modular Form by considering the action of the Galois group.  Then, Ken Ribet showed that the modular form is not possible, which proves the theorem.
\item The Galois group is related to the Fundamental group.  If we have an extension $K \subseteq L$, this corresponds to a covering space.  The Galois group corresponds to the fundamental group of the base space.  The algebraic closure of a field corresponds to a universal covering space.
\end{itemize}

Inverse Galois Problem: Given a finite group $G$, is there an extension $K$ of $\Q$ so that the Galois group of $K$ is $G$?  This is true for many groups, but the problem is completely open in general.
\end{document}

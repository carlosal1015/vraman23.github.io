\documentclass[12pt]{scrartcl}
\usepackage[sexy]{evan}
\usepackage{graphicx}

 %Sets
\newcommand{\N}{\mathbb{N}}
\newcommand{\Z}{\mathbb{Z}}
\newcommand{\F}{\mathbb{F}}
\newcommand{\Q}{\mathbb{Q}}
\newcommand{\R}{\mathbb{R}}
\newcommand{\C}{\mathbb C}
\newcommand{\T}{\mathbb T}

\let \phi \varphi
\let \hat \widehat

\newcommand{\<}{\langle}
\renewcommand{\>}{\rangle}


%From Topology
\newcommand{\cT}{\mathcal{T}}
\newcommand{\cB}{\mathcal{B}}
\newcommand{\cC}{\mathcal{C}}
\newcommand{\cH}{\mathcal{H}}
\renewcommand{\arc}{\text{arc}}


\usepackage{answers}
\Newassociation{hint}{hintitem}{all-hints}
\renewcommand{\solutionextension}{out}
\renewenvironment{hintitem}[1]{\item[\bfseries #1.]}{}
\declaretheorem[style=thmbluebox,name={Theorem}]{thm}
\declaretheorem[style=thmbluebox,name={Problem}]{Prob}


\begin{document}
\title{Chapter 1: Preliminaries}
\author{Vishal Raman}
\maketitle
\begin{abstract}
My solutions to the problems and exercises from the first chapter of Stein/Shakarchi, \textit{Complex Analysis}, "Preliminaries to Complex Analysis".  Any typos or errors found are my own - kindly direct any concerns to my inbox.
\end{abstract}
\tableofcontents
\pagebreak
\section{Chapter 1: }
\subsection{Exercise 1}
Describe geometrically the sets of points $z$ in the complex plane defined by the following relations:
\begin{itemize}
 \item $|z - z_1| = |z - z_2|$, where $z_1, z_2 \in \C$.
 \item $1/z = \overline{z}$.
 \item $\text{Re}(z) = 3$.
 \item $\text{Re}(z) > c$ where $c \in \R$.
 \item $\text{Re}(az + b) > 0$ where $a , b \in \C$.
 \item $|z| = \text{Re}(z) + 1$.
 \item $\text{Im}(z) = c$.
 \end{itemize} 
 \begin{proof}
 \begin{itemize}
 \item This describes the perpendicular bisector between $z_1, z_2$, the set of points that are equidistant from both points.
 \item Equivalently, $|z| = 1$, the circle of radius $1$.
 \item A vertical line through $3$.
 \item The half plane to the right of $c$(excluding the boundary).
 \item This the half plane below a given line from the components of $a, b$.
 \item A horizontal parabola with vertex at $-i$.
 \item A horizontal line through $c$.
 \end{itemize}
 \end{proof}
 
\subsection{Exercise 2}
Let $\langle\cdot, \cdot \rangle$ denote the usual inner product in $\R^2$.  We have a Hermitian inner product in $\C$ by $(z, w) = z \overline{w}$.  Show that 
 $$\langle z, w \rangle = \frac{1}{2} \left ((z,w) + (w, z)\right ) = \text{Re}(z, w).$$
 \begin{proof}
 Let $z = a + bi, w = c+di$.
 \begin{align*}
 (z, w) + (w, z) &= z\overline{w} + w \overline{z} \\
 &= (a + bi)(c - di) + (a - bi)(c + di) \\
&= (ac + bd) + (bc - ad)i + (ac +bd) +(bc - ad)i
&= 2(ac + bd)\\
 &= 2\left <z, w\right > = 2\text{Re}(z, w).
 \end{align*}
 \end{proof}
 
\subsection{Exercise 3}
With $\omega = se^{i\phi}$, where $s \ge 0$ and $\phi \in \R$, solve the equation $z^n = \omega$ in $\C$ where $n \in \N$.  How many solutions are there?
\begin{proof}
We have $$z^n = \omega \Longrightarrow z = (\omega)^{1/n} e^{\frac{2\pi i m}{n}},$$
where $m \in \Z/n\Z$.  
$$(\omega^{1/n})^n = se^{i\phi} \Longrightarrow  \omega^{1/n} = s^{1/n} e^{i\phi/n + \frac{2\pi i k}{n}},$$
where $k \in \Z/n\Z$.  It follows that 
$$z = s^{1/n} e^{i\phi/n} e^{\frac{2\pi i(k+m)}{n}} $$
and $k+m$ is uniformly distributed in $\Z/n\Z$, so we have $n$ possible solutions if $s \ne 0$.  Otherwise, we have one solution, namely $0$.
\end{proof}

\subsection{Exercise 4}
 Show that it is impossible to define a total ordering on $\C$.  
 \begin{proof}
 Suppose $i \succ 0$.  Then, we have $i^2 = -1 \succ 0$, which is impossible.  Similarly, if $0 \succ i$, then $$0 + (-i) = -i \succ i + (-i) = 0.$$
 But then,
 $$0 \cdot (-i) = 0 \succ i\cdot (-i) = 1,$$
 a contradiction.

 \end{proof}
 
\subsection{Exercise 5} A set $\Omega$ is said to be \textbf{pathwise connected} if any two points in $\Omega$ can be joined by a curve entirely contained in $\Omega$.  The purpose of this exercise is to prove that an \textit{open} set $\Omega$ is pathwise connected if and only if $\Omega$ is connected.
 \begin{description}
\item{(a)} Suppose first that $\Omega$ is open and pathwise connected, and that it can be written as $\Omega = \Omega_1 \cup \Omega_2$ where $\Omega_1$ and $\Omega_2$ are disjoint non-empty open sets.  Choose two points $w_1 \in \Omega_1, w_2 \in \Omega_2$ and let $\gamma$ denote a curve in $\Omega$ joining $w_1$ to $w_2$.  Consider a parameterization $z:[0, 1] \to \Omega$ of this curve with $z(0) = w_1$ and $z(1) = w_2$.  

Let
 $$t^* = \sup_{0 \le t \le 1} \{t: z(s) \in \Omega_1 \forall 0 \le s < t\}.$$
 Arrive at a contradiction by considering the point $z(t^*).$
 \item{(b)} Conversely, suppose that $\Omega$ is open and connected. Fix a point $w \in \Omega$ and let $\Omega_1 \subset \Omega$ denote the set of all points that can be joined to $w$ by a curve contained in $\Omega$. Also, let $\Omega_2 \subset \Omega$ denote the set of all points that cannot be joined to $w$ by a curve in $\omega$. Prove that both $\Omega_1$ and $\Omega_2$ are open, disjoint and their union is $\Omega$. Finally, since $\Omega_1$ is non-empty (why?) conclude that $\Omega = \Omega_1$ as desired.
 \end{description}
 \begin{proof}(Part A)
 Consider the point $z(t^*)$.  Suppose $t^* < 1$.  We cannot have $z(t^*) \in \Omega_1$, since this implies there is an open ball $B$ containing $z(t^*)$ in $\Omega_1$.  It follows that $z^{-1}(B)$ is an open subset of $[0, 1]$ since $z$ is continuous, so contains points to the right of $t^*$, a contradiction.  If $t^* = 1$, then, there is a sequence of points in $\Omega_1$ converging to $z(1) \in \Omega_2$, contradicting the assumption that $\Omega \setminus \Omega_2$ is closed.  
 
 If $z(t^*) \in \Omega_2$, then $z(t) \in \Omega_2$ if and only if $t > t^*$.  Hence, $t^*$ is the infumum of values of $t$ with $z(t) \in \Omega_2$ and we repeat the argument from above.
 \end{proof}
 \begin{proof}(Part B)
It is clear that $\Omega_1 \cup \Omega_2 = \Omega$ and that they are disjoint.  
 
 Since $\Omega$ is open, we can find an open ball $B$ around $v \in \Omega_1 \subset \Omega$ which is contained in $\Omega$.  If $x \in B$ then there is a path from $v$ to $x$.  There is also a path from $w$ to $v$ so by the gluing lemma, we can find a path from $w$ to $x$.  This implies that $B \subset \Omega_1$, which shows that $\Omega_1$ is open.
 
Take $y \in \Omega_2$.  There exists an open ball $C$ around $y$ contained in $\Omega$.  For any $t \in C$, if a path exists from $w$ to $C$, then we can find a path from $w$ to $y$ by the gluing lemma.  It follows that $C \subset \Omega_2$ which shows that $\Omega_2$ is open.
 
 Since $\Omega$ is connected, we must have either $\Omega = \Omega_1$ or $\Omega = \Omega_2$.  However, $w \in \Omega_1$ implies that $\Omega = \Omega_1$.
 \end{proof}
 \subsection{Exercise 6}
 Let $\Omega$ be an open set in $\C$ and $z \in \Omega$.  The connected component of $\Omega$ containing $z$ is the set of points $\mathcal C_z$ of all points $w$ in $\Omega$ that can be joined to a curve entirely contained in $\Omega$.
  \begin{description}
\item{(a)} Check that $\mathcal C_z$ is open and connected.  Then, show that $w \in C_z$ defines an equivalence relation.
 \item{(b)} Show that $\Omega$ can have only countably many distinct connected components.
 \item{(c)} Prove that if $\Omega$
 is the complement of a compact set, then $\Omega$ has only one unbounded component.
  \end{description}
 \begin{proof}(Part A)
Note that $C_z$ is a pathwise connected set which is open if and only if it is connected.  For any $x \in C_z$, there exists a ball $B \subset \Omega$ containing $x$.  Then, gluing the path from $z$ to $x$ and $x$ to $y \in B$ shows that $B \subset C_z$, which implies that $C_z$ is open.  By Exercise $5$, $C_z$ is connected.

We now show that $w \in C_z$ is an equivalence relation.  It is clear that $z \in C_z$.  If $w \in C_z$, then there is a path from $z$ to $w$, the reverse of which is a path from $w$ to $z$, so $z \in C_w$.  Finally, If $a \in B_b$ and $b \in B_c$, then gluing the paths from $a$ to $b$ and $b$ to $c$ gives a path from $a$ to $c$, so $a \in B_c$.
 \end{proof}
 \begin{proof}(Part B)
 Suppose not.  Then there is an uncountable collection of disjoint open balls of $\Omega$.  From the density of $\Q$ in $\R$, each of these balls contains a unique rational, which is a contradiction since there are only countably many rationals.
 \end{proof}
 \begin{proof}(Part C) 
Let $C \subset \Omega$ be the unbounded component.  Since $\overline{\Omega}$ is compact, there exists an open ball $B \supset \overline{\Omega}$.  Then $B^c \subset \Omega$, and note that $B^c$ is unbounded and connected.    Since $C \cap B = \emptyset$, we must have $C \cap B^c \ne \emptyset$ so it follows that $B^c = C$.  Hence, we have exactly one unbounded component.
 \end{proof}
 
 \subsection{Exercise 7}
 We introduce mappings called \textbf{Blaschke factors}.
   \begin{description}
\item{(a)} Let $z, w$ be two complex numbers such that $\overline{z}w \ne 1$.  Prove that 
$$\left |\frac{w - z}{1 - \overline{w}z}\right| < 1$$
if $|z| < 1$ and $|w| < 1$ and also that 
$$\left |\frac{w - z}{1 - \overline{w}z}\right| = 1$$
if $|z| = 1$ or $|w| = 1$. 
\item{(b)} Prove that for a fixed $w$ in the unit disc $\mathcal D$, the mapping $F: z \mapsto \frac{w-z}{1 - \overline{w}z}$ satisfies the following conditions
\begin{itemize}
\item $F$ maps the unit disk to itself and is holomorphic.
\item $F$ interchanges $0$ and $w$.
\item $|F(z)| = 1$ if $|z| = 1$.
\item $F: \mathcal D\to \mathcal D$ is bijective.
\end{itemize}
  \end{description}
  \begin{proof}(Part A)
We have 
\begin{align*}
\left |\frac{w - z}{1 - \overline{w}z}\right| < 1 &\Leftrightarrow (w - z)  \overline{(w - z)} \le (1 - \overline{w} z)\overline{(1 - \overline{w}z)}\\
&\Leftrightarrow (w - z)(\overline{w} - \overline{z}) \le (1 - z\overline{w} ) (1 - w\overline{z}) \\
&\Leftrightarrow |w|^2 + |z|^2 - z\overline{w} - w\overline{z} \le 1 - z\overline{w} - w\overline{z} + |z|^2|w|^2 \\
&\Leftrightarrow (1  - |w|^2)(1 - |z|^2) \ge 0,
\end{align*}
which gives both results.


  \end{proof}
\begin{proof}(Part B)
From the result above if $|z| \le 1$ then since $|w| \le 1$, we have $|F(z)| \le 1$, which implies that $F(\mathcal D) \subset \mathcal D$.  Then, for any $y \in \mathcal D$, note that $F(F(y)) = y$(this can be easily verified), so it follows that $\mathcal D \subset F(\mathcal D)$, which shows that $F$ maps $\mathcal D$ to $\mathcal D$, as desired.  The function is holomorphic by the quotient rule.  It is easy to see that $F(0) = w$ and $F(w) = 0$.  Then, $|F(z)| = 1$ if $|z| = 1$ by Part A.  From $F(F(y)) = y$ it is clear that $F$ is surjective.  Then, if $F(x) = y$ then $x = F(F(x)) = F(y)$, so $x = y$ implies that $F(x) = F(y)$.  Therefore, $F$ is bijective, as desired.
\end{proof}

\subsection{Exercise 8}
Suppose $U$ and $V$ are open sets in the complex plane.  Prove that if $f: U \to V$ and $g : V \to \C$ are two functions that are differentiable(in the real sense), and $h = g \circ f$, then 
$$\frac{\partial h}{\partial z} = \frac{\partial g}{\partial z} \frac{\partial f}{\partial z} + \frac{\partial g}{ \partial \overline{z}} \frac{\partial \overline{f}}{\partial z}$$
and 
$$\frac{\partial h}{\partial \overline {z}} = \frac{\partial g}{\partial z} \frac{\partial f}{\partial \overline{z}} + \frac{\partial g}{ \partial \overline{z}} \frac{\partial \overline{f}}{\partial \overline{z}}.$$

\begin{proof}
Recall that 
$$\frac{\partial}{\partial z} = \frac{1}{2} \left (\frac{\partial}{\partial x} -i \frac{\partial}{\partial y}\right )$$
$$\frac{\partial}{\partial \overline{z}} = \frac{1}{2} \left (\frac{\partial}{\partial x} +i \frac{\partial}{\partial y}\right ).$$

Then
$$\frac{\partial h}{\partial x} = \frac{\partial f}{\partial g} \frac{\partial g}{\partial x} $$
and 
$$\frac{\partial h}{\partial y} = \frac{\partial f}{\partial g} \frac{\partial g}{\partial y} $$

so it follows that 
$$\frac{\partial h}{\partial z} = \frac{\partial f}{\partial g} \frac{\partial g}{\partial z}$$
\end{proof}

\subsection{Exercise 9}
Show that in polar coordinates, the Cauchy-Riemann Equations take the form
$$\frac{\partial u}{\partial r} = \frac{1}{r} \frac{\partial v}{\partial \theta}$$
and 
$$\frac{1}{r}\frac{\partial u}{\partial \theta} = -\frac{\partial v}{\partial r}.$$
Use these equations to show that the logarithm function defined by
$$\log z = \log r + i \theta$$
where $z = re^{i\theta}$ with $-\pi < \theta < \pi$ is holomorphic in the region $r > 0$ and $-\pi < \theta < \pi$.
\begin{proof}
If we let $z = x + iy = r(\cos \theta + i\sin \theta)$, it follows that 
\begin{align*}
u_r &= u_x \cos \theta + u_y \sin \theta \\
&= \frac{1}{r} \left (rv_y \cos \theta - rv_x \sin \theta\right ) \\
&= \frac{1}{r}v_\theta.
\end{align*}

Similarly,
\begin{align*}
v_r &= v_x \cos \theta + v_y \sin \theta \\
&= -\frac{1}{r} \left (ru_y \cos \theta - ru_x \sin \theta\right ) \\
&= -\frac{1}{r} u_\theta.
\end{align*}

For $\log z = \log r + i\theta$, we have $u(r, \theta) = \log r$ and $v(r, \theta) = \theta$, so 
$$u_r = \frac{1}{r}= \frac{1}{r} v_\theta$$
and 
$$v_r = 0 = -\frac{1}{r} u_\theta$$
so it follows that the function is holomorphic as desired.
 \end{proof}
 
 \subsection{Exercise 10}Show that 
 $$4 \frac{\partial}{\partial z} \frac{\partial }{\partial \overline{z}} = 4\frac{\partial}{\partial \overline{z}} \frac{\partial }{\partial z} = \Delta, $$
 where 
 $$\Delta = \frac{\partial^2}{\partial x^2} + \frac{\partial^2}{\partial y^2}.$$
 
 \begin{proof}
 Trivially follows from the definitions of the operators.
 \end{proof}
 \subsection{Exercise 11} Show that if $f$ is holomorhpic in the open set $\Omega$, then the real and imaginary parts of $f$ are harmonic.
 \begin{proof}
 If $f$ is holomorphic on $\Omega$ then $\frac{\partial}{\partial \overline{z}}f = 0$, and the result follows from applying the previous exercise.
 \end{proof}
 
 \subsection{Exercise 12} Consider the function defined by $f(x + iy) = \sqrt{|x||y|}$ whenever $x, y \in \R$.  Show that $f$ satisfies the Cauchy-Riemann equations at the origin, yet $f$ is not holomorphic at $0$.
 \begin{proof}
Note that $v_x, v_y = 0$, since $f$ is a real-valued function. 
 
 $$u_x(0, 0) = \lim_{h \to 0} \frac{u(h, 0) - h(0, 0)}{h} = 0,$$
 and 
 $$u_y(0, 0) = \lim_{h \to 0} \frac{u(0, h) - h(0, 0)}{h} = 0$$
 but
 $$\frac{f(t(1 + i)) - f(0)}{t(1 + i)} = \frac{|t|}{t(1 + i)},$$
 which has no limit.
 \end{proof}
 
 \subsection{Exercise 13} Suppose that $f$ is holomorphic in an open set $\Omega$.  Prove that in any one of the following cases
 \begin{itemize}
 \item Re$(f)$ is constant;
 \item Im$(f)$ is constant;
 \item $|f|$ is constant;
 \end{itemize}
 one can conclude that $f$ is constant.
 \begin{proof}
The first and second cases are equivalent, following from the C-R equations.  It follows that $Re(f) + iIm(f) = f$ is constant in these cases.

Let $f = u + iv$.  If $|f|$ is constant, then $|f|^2 = u^2 + v^2$ is constant, so it follows that 
$$\frac{\partial }{\partial x}(u^2 + v^2) = \frac{\partial }{\partial y}(u^2 + v^2) = 0.$$

Therefore,
$$uu_x + vv_x = uu_y + vv_y = 0.$$

By C-R, we have 
$$uv_y  + vv_x = -uv_x - vv_y = 0.$$

It follows that 
$$u^2 v_y + uvv_x = -u v v_x - v^2 v_y = 0 \Longrightarrow (u&2 + v^2) v_y = 0.$$

If $u^2 + v^2 = 0$, then $f = u = v = 0$.  If $v_y = 0$, then it follows that $Im(f) = 0$, and it reduces to (b)
. \end{proof}
\end{document}

\documentclass[11pt]{article}
\usepackage[sexy]{evan}

\usepackage{quiver}
\usepackage{answers}
\Newassociation{hint}{hintitem}{all-hints}
\renewcommand{\solutionextension}{out}
\renewenvironment{hintitem}[1]{\item[\bfseries #1.]}{}
\declaretheorem[style=thmbluebox,name={Theorem}]{thm}

 %Sets
\newcommand{\N}{\mathbb{N}}
\newcommand{\Z}{\mathbb{Z}}
\newcommand{\F}{\mathbb{F}}
\newcommand{\Q}{\mathbb{Q}}
\newcommand{\RP}{\mathbb{RP}}
\newcommand{\CP}{\mathbb{CP}}
\newcommand{\R}{\mathbb{R}}
\newcommand{\C}{\mathbb C}
\newcommand{\T}{\mathbb T}
\renewcommand{\hat}{\widehat}
\renewcommand{\Im}{\text{Im }}
\renewcommand{\>}{\rangle}
\newcommand{\<}{\langle}


\renewcommand{\Ker}{\operatorname{Ker}}
\renewcommand{\Im}{\operatorname{Im}}
\renewcommand{\Ext}{\operatorname{Ext}}

\newcommand{\res}{\operatorname{res}}
\let \phi \varphi
\let \mc \mathcal
\let \ms \mathscr
\let \mb \mathbb
\let \ol \overline
\let \subset \subseteq
\let \subsetneq \subset
%From Topology
\newcommand{\cT}{\mathcal{T}}
\newcommand{\cB}{\mathcal{B}}
\newcommand{\cC}{\mathcal{C}}
\newcommand{\cH}{\mathcal{H}}

\newcommand{\supp}{\text{supp }}

\newcommand{\aint}{\mathrel{\int\!\!\!\!\!\!-}}
\let \grad \nabla

\begin{document}
\title{Vakil, Algebraic Geometry}
\author{Vishal Raman}
\maketitle
\begin{abstract}
Selected solutions to problems from Vakil, \textit{The Rising Sea, Foundations of Algebraic Geometry}.  The numberings may not match up with the current edition(I used November 18th, 2017), but I have tried to include problem numbers when possible.   Any typos or mistakes are my own - kindly direct them to my inbox.
\end{abstract}
\tableofcontents
\pagebreak
\section{Sheaves}
\subsection{Morphisms of Sheaves}
\begin{problem}[2.3C] Suppose $\ms F, \ms G$ are two sheaves of sets on $X$.  Let $\mathcal Hom(\ms F, \ms G)(U):= \Mor(\ms F \vert_U, \ms G \vert_U)$. Show that $\mathcal Hom(\ms F, \ms G)$ is a sheaf of sets on $X$.
\end{problem}
\begin{proof}
For $U \subseteq V$, define $\res_{V, U}: \mc Hom(\ms F, \ms G)(V) \to \mc Hom(\ms F, \ms G)(U)$ by the map $f \mapsto f \vert_U$.  It is easy to verify that this satisfies the requirements for a presheaf of sets on $X$.

 Let $\{U_i\}_{i \in I}$ be an open cover for the open subset $U \subseteq X$.  We first show the identity axiom - let $f_1, f_2 \in \mc Hom(\ms F, \ms G)(U)$.  Suppose that $\res_{U, U_i} f_1 = \res_{U, U_i} f_2$ for each $i \in I$.  Let $V \subseteq U$ be an open subset.  For any $\tau \in \ms G(V)$, we have $f_1\vert_{U_i}(V \cap U_i)(\tau) = f_2\vert_{U_i}(V \cap U_i)(\tau)$ for each $i \in I$.  It suffices to show that $f_1(V) = f_2(V)$.

From the definition of a sheaf morphism, for each $i \in I, j \in \{1, 2\}$, we obtain the following commutative diagram:

\[\begin{tikzcd}
	{\mathscr F(V)} && {\mathscr G(V)} \\
	{\mathscr F(V \cap U_i)} && {\mathscr G(V \cap U_i)}
	\arrow["{f_j}", from=1-1, to=1-3]
	\arrow["{f_j}", from=2-1, to=2-3]
	\arrow["{\operatorname{res}_{V, V \cap U_i}}"', from=1-1, to=2-1]
	\arrow["{\operatorname{res}_{V, V \cap U_i}}", from=1-3, to=2-3]
\end{tikzcd}\]

It follows that for $\sigma \in \ms F(V)$,
\begin{align*}
\res_{V, V \cap U_i} f_1(V)(\sigma) &= f_1(V \cap U_i) (\res_{V, V \cap U_i}(\sigma)) \\
&= f_2(V \cap U_i) (\res_{V, V \cap U_i}(\sigma)) \\
&= \res_{V, V \cap U_i} f_2(V)(\sigma).
\end{align*}
By the identity axiom from $\ms G$, we find that $f_1(V) = f_2(V)$ as desired.

Next, we show the gluability axiom - take $f_i \in \mc \Hom(\ms F, \ms G)(U_i)$ satisfying $\res_{U_i, U_i \cap U_j} f_i = \res_{U_j, U_i \cap U_j} f_j$.  It suffices to show that there exists $f \in \mc Hom(\ms F, \ms G)(U)$ such that $\res_{U, U_i} f = f_i$.  Take $V \subset U$ and define $V_j := V \cap U_j$.  From the definition of a sheaf morphism, we have the following commutative diagrams(leaving out the labling of the restriction maps):

\[\begin{tikzcd}
	{\mathscr F\vert_U(V)} && {\mathscr G\vert_U(V)} & {\mathscr F\vert_U(V)} && {\mathscr G\vert_U(V)} \\
	{\mathscr F\vert_{U_i}(V_i)} && {\mathscr G\vert_{U_i}(V_i)} & {\mathscr F\vert_{U_i}(V_i)} && {\mathscr G\vert_{U_i}(V_i)} \\
	{\mathscr F\vert_{U_i \cap U_j}(V_i \cap V_j)} && {\mathscr G\vert_{U_i \cap U_j}(V_i \cap V_j)} & {\mathscr F\vert_{U_i \cap U_j}(V_i \cap V_j)} && {\mathscr G\vert_{U_i \cap U_j}(V_i \cap V_j)} \\
	&&& {}
	\arrow["{f_i(V)}", from=1-1, to=1-3]
	\arrow["{f_i(V_i)}", from=2-1, to=2-3]
	\arrow["{f_i(V_i \cap V_j)}", from=3-1, to=3-3]
	\arrow[from=1-1, to=2-1]
	\arrow[from=2-1, to=3-1]
	\arrow[from=1-3, to=2-3]
	\arrow[from=2-3, to=3-3]
	\arrow[from=1-4, to=2-4]
	\arrow[from=2-4, to=3-4]
	\arrow["{f_j(V)}", from=1-4, to=1-6]
	\arrow["{f_j(V_j)}", from=2-4, to=2-6]
	\arrow["{f_j(V_i\cap V_j)}", from=3-4, to=3-6]
	\arrow[from=1-6, to=2-6]
	\arrow[from=2-6, to=3-6]
\end{tikzcd}\]

It follows that for $\sigma \in \ms F\vert_U(V)$, we have
\begin{align*}
\res_{U_i, U_i \cap U_j}((f_i(V_i) \circ \res_{U, U_i} )(\sigma)) &= f_i(V_i \cap V_j)( \res_{U, U_i \cap U_j} (\sigma) )\\
&=  f_j(V_i \cap V_j) (\res_{U, U_i \cap U_j} (\sigma))\\
&= \res_{U_j, U_i \cap U_j}((f_j(V_j) \circ \res_{U, U_j} )(\sigma)).
\end{align*}
By gluability applied to $\ms G\vert_U(V)$, there exists $g \in \ms G\vert_U(V)$ so that $\res_{U, U_j} g = f_j(V_j) \circ \res_{U, U_j}$.  Then, we can define $f(V): \ms F\vert_U(V) \to \ms G\vert_U(V)$ by $\sigma \mapsto g$, which satisfies $\res_{U, U_j} f = f_j$ by construction.
\end{proof}

\begin{problem}[2.3E] Let $\Phi: \ms F \to \ms G$ be a morphism of presheaves.  Define the presheaf kernel by $(\ker_{pre}\Phi)(U) = \ker \Phi(U)$.  Show that the presheaf kernel is a presheaf.
\end{problem}
\begin{proof}
We first define the restriction map on $\ker_{pre}\Phi$.  For $U \hookrightarrow V$, consider the following diagram(where $\iota$ denotes the inclusion map and the vertical arrows denote the corresponding restriction maps):
% https://q.uiver.app/?q=WzAsOCxbMCwwLCIwIl0sWzEsMCwiXFxrZXJfe3ByZX1cXFBoaShWKSJdLFsyLDAsIlxcbWF0aHNjciBGKFYpIl0sWzMsMCwiXFxtYXRoc2NyIEcoVikiXSxbMCwxLCIwIl0sWzEsMSwiXFxrZXJfe3ByZX1cXFBoaShVKSJdLFsyLDEsIlxcbWF0aHNjciBGKFUpIl0sWzMsMSwiXFxtYXRoc2NyIEcoVSkiXSxbMCwxXSxbMSwyLCJcXGlvdGEiXSxbMiwzLCJcXFBoaShWKSJdLFs2LDcsIlxcUGhpKFUpIl0sWzUsNiwiXFxpb3RhIl0sWzIsNl0sWzMsN10sWzQsNV1d
\[\begin{tikzcd}
	0 & {\ker_{pre}\Phi(V)} & {\mathscr F(V)} & {\mathscr G(V)} \\
	0 & {\ker_{pre}\Phi(U)} & {\mathscr F(U)} & {\mathscr G(U)}
	\arrow[from=1-1, to=1-2]
	\arrow["\iota", from=1-2, to=1-3]
	\arrow["{\Phi(V)}", from=1-3, to=1-4]
	\arrow["{\Phi(U)}", from=2-3, to=2-4]
	\arrow["\iota", from=2-2, to=2-3]
	\arrow[from=1-3, to=2-3]
	\arrow[from=1-4, to=2-4]
	\arrow[from=2-1, to=2-2]
\end{tikzcd}\]
Note that by definition of the presheaf kernel, both rows of the diagram are exact.  Take $\sigma \in \ker_{pre}\Phi(V)$.  Note that
$$0=\res_{V, U} \circ \Phi(V) \circ \iota(\sigma) = \Phi(U) \circ \res_{V, U} \circ \iota(\sigma).$$
It follows that $\res_{V, U} \circ \iota(\sigma) \in \ker \Phi(U) = \Im \iota(\ker_{pre}\Phi(U))$, so there exists a unique element $\tau \in \ker_{pre} \Phi(U)$ with $\iota(\tau) = \tau = \res_{V, U} \circ \iota(\sigma)$.  This defines a unique map $\res_{V, U}: \ker_{pre}\Phi(V) \to \ker_{pre}\Phi(U)$ by $\sigma \mapsto \tau$.  It is easy to verify that the restriction map inherits the properties of a restriction map from $\ms F$.
\end{proof}

\begin{problem}[2.3I] Suppose $\Phi: \ms F \to \ms G$ is a sheaf morphism.  Show that $\ker_{pre}\Phi$ is a sheaf, and that it satisfies the universal property of kernels.
\end{problem}
\begin{proof}
We already showed that $\ker_{pre}\Phi$ is a presheaf, so it suffices to verify the identity and gluability axioms.

Let $U \subset X$ and $\{U_i\}_{i \in I}$ be an open cover of $U$.  Let $f_1, f_2 \in \ker_{pre}\Phi(U)$ so that $\res_{U, U_i} f_1 = \res_{U, U_i} f_2$ for all $i \in I$.  From(2.3E), for each $i \in I$, we have the following commutative diagram(where $\iota$ denotes the inclusion map and the vertical arrows denote the corresponding restriction maps):
\[\begin{tikzcd}
	0 & {\ker_{pre}\Phi(U)} & {\mathscr F(U)} & {\mathscr G(U)} \\
	0 & {\ker_{pre}\Phi(U_i)} & {\mathscr F(U_i)} & {\mathscr G(U_i)}
	\arrow[from=1-1, to=1-2]
	\arrow["\iota", from=1-2, to=1-3]
	\arrow["{\Phi(U)}", from=1-3, to=1-4]
	\arrow["{\Phi(U_i)}", from=2-3, to=2-4]
	\arrow["\iota", from=2-2, to=2-3]
	\arrow[from= 1-2, to=2-2]
	\arrow[from=1-3, to=2-3]
	\arrow[from=1-4, to=2-4]
	\arrow[from=2-1, to=2-2]
\end{tikzcd}\]

For each $i \in I$, we have
\begin{align*}
\res_{U, U_i} \circ \iota(f_1) &= \iota \circ \res_{U, U_i} f_1 \\
&= \iota \circ \res_{U, U_i} f_2 \\
&= \res_{U, U_i} \circ \iota(f_2).
\end{align*}
By the identity axiom applied to $\ms F$, we have $\iota(f_1) = \iota(f_2)$.  By the injectivity of $\iota$, we have $f_1 = f_2$, proving the identity axiom.

Now, suppose we have $f_i \in \ms F(U_i)$ for each $i \in I$ so that $\res_{U_i, U_i \cap U_j} f_i = \res_{U_j, U_i \cap U_j} f_j$ for all $i, j \in I$.  For each $i, j \in I$, we have the following commutative diagram(where $\iota$ denotes the inclusion map and the vertical arrows denote the corresponding restriction maps):
\[\begin{tikzcd}
	0 & {\ker_{pre}\Phi(U_i)} & {\mathscr F(U_i)} & {\mathscr G(U_i)} \\
	0 & {\ker_{pre}\Phi(U_i\cap U_j)} & {\mathscr F(U_i \cap U_j)} & {\mathscr G(U_i)} \\
	\arrow[from=1-1, to=1-2]
	\arrow["\iota", from=1-2, to=1-3]
	\arrow["{\Phi(U)}", from=1-3, to=1-4]
	\arrow["{\Phi(U_i)}", from=2-3, to=2-4]
	\arrow["\iota", from=2-2, to=2-3]
	\arrow[from= 1-2, to=2-2]
	\arrow[from=1-3, to=2-3]
	\arrow[from=1-4, to=2-4]
	\arrow[from=2-1, to=2-2]
\end{tikzcd}\]
We have
\begin{align*}
\res_{U_i, U_i\cap U_j} \circ \iota(f_i) &= \iota \circ \res_{U_i, U_i \cap U_j} f_i \\
&= \iota \circ \res_{U_j, U_i \cap U_j} f_j \\
&= \res_{U_j, U_i \cap U_j} \circ \iota(f_j).
\end{align*}

By the gluability axiom applied to $\ms F$, there exists $f \in \ms F(U)$ so that $\res_{U, U_i} h = \iota(f_i)$.  It suffices to show that $f \in \ker \Phi(U)$.  This follows from the fact that for each $i \in I$,
$$\res_{U, U_i} \circ \Phi(U) (f) = \Phi(U) \circ \res_{U, U_i} \circ \iota (f_i) = 0,$$
so the identity axiom applied to $\ms F$ implies that $\Phi(U) (f) = 0$, which proves the result.
\end{proof}

\subsection{Properties determined at the level of stalks; Sheafification}
\begin{problem}[2.4A] Show that the natural map $\ms F(U) \to \prod_{p \in U} \ms F_p$ is injective.
\end{problem}
\begin{proof}
If $f, g$ map to the same image $(s_p)_{p \in U}$, we can take a trivial open cover $\{U\}$ so that for each $p \in U$
$$\res_{U, U} f(p) =  f(p) = s_p = g(p) = \res_{U, U} g(p),$$
which implies that $f = g$ by the identity axiom.
\end{proof}

\begin{problem}[2.4C] Show that any choice of compatible germs for a sheaf of sets $\ms F$ over $U$ is the image of a section of $\ms F$ over $U$.
\end{problem}
\begin{proof}
Let $\prod_{p \in U} s_p$ consist of compatible germs.  Then, there is an open cover $\{U_i\}$ of $U$ and sections $f_i \in \ms F(U_i)$ so that if $p \in U_i$, then $s_p = [(f_i, U_i)]$, where the square brackets denote the equivalence class of germs.

For each pair $(i, j)$, note that for $q \in U_i \cap U_j$, we have $[(f_i, U_i)] = s_q = [(f_j, U_j)] $
so it follows that
$$[(f_i \vert_{U_i \cap U_j}, U_i \cap U_j)] = [(f_j \vert_{U_i \cap U_j}, U_i \cap U_j)].$$
By applying (2.4A)  to $U_i \cap U_j$, we have that $f_i \vert_{U_i \cap U_j} = f_j \vert_{U_i \cap U_j}$.  By the glubaility property of the sheaf, there exists $f \in \ms F(U)$ so that $f \vert_{U_i} = f_i$.  It follows that
$$f \mapsto \prod_{p \in U} [(f, U)] = \prod_{p \in U} [(f_i, U_i)] = \prod_{p \in U} s_p.$$
\end{proof}

\subsection{Sheaves on a Base}
\begin{problem}[2.5C] Suppose that $\{B_i\}$ is a base for the topology of $X$.  We have a morphism $F \to G$ of sheaves on a base given by a collection of maps $F(B_k) \to G(B_k)$ so that for $B_j \subset B_i$, the corresponding diagram with restriction maps commutes.
\begin{itemize}
\item[(a)] Verify that a morphism of sheaves is determined by the induced morphism of sheaves on a base.
\item[(b)] Show that a morphism of sheaves on the base gives a morphism of the induced sheaves.
\end{itemize}

\end{problem}
\begin{proof}(Part a) Let $\ms F, \ms G$ be the corresponding extended sheaves for $F$ and $G$ respectively.  Suppose we have morphisms of sheaves $\psi_1, \psi_2: \ms F \to \ms G$ with induced morphisms of sheaves on a base $\phi_1, \phi_2: F \to G$ respectively.  Suppose that $\phi_1(B_i) = \phi_2(B_i)$ for each set in the base.  Let $U \subset X$ be an open subset.  By definition of a base, we can write $U = \bigcup_{i \in I} B_i$.

Note that by definition, we have $F(B_i) = \ms F(B_i)$ and $G(B_i) = \ms G(B_i)$ for each $i \in I$.  It follows that $\psi_j(B_i) = \phi_j(B_i)$ for $j \in \{1, 2\}$, $i \in I$.  It follows that for all $i \in I$,
$$\res_{U, B_i} (\psi_1(U)) = \psi_1(B_i) = \phi_1(B_i) = \phi_2(B_i) = \res_{U, B_i}(\psi_2(U)).$$
By the identity axiom applied to $\ms G$, we have $\psi_1(U) = \psi_2(U)$, and since $U$ is arbitrary, we have $\psi_1 = \psi_2$, as desired.
\end{proof}
\begin{proof}(Part b) It follows from the result shown in 2.4D.  Namely, there is a natural map from $F(U)$ onto the product of stalks $\prod_{q \in U} F_q$ and similarly with $G(U)$.  Furthermore, the corresponding commutative diagram commutes.  It suffices to show that that the corresponding images are contained in the extended sheaves, which is an easy exercise.
\end{proof}

\section{Affline Schemes; Zariski Topology}
\subsection{Underlying structure of Affline Schemes}
\begin{problem}[3.2.C] Describe $\mb A_{\Q}^1$.
\end{problem}
\begin{proof}
Since $\Q$ is a field, $\Q[x]$ is a PID.  We have obvious prime ideals $(0)$ and $(f(x))$ for $f(x)$ irreducible.  Following a similar proof for $\C[x]$, we can show that the prime ideals of the form $(f(x))$ are either of the form $(x-a)$ for $a \in \Q$ or $(x^2 + ax + b)$ for $a, b \in \Q$ so that $x^2 + ax + b$ is irreducible in $\Q[x]$.  The main idea for this is that we can apply the division algorithm to repeatedly break a polynomial into smaller polynomial factors.  We could either have a rational root giving a prime ideal of the form $(x-a)$ or a pair of reals/complex roots which is of the form $(x^2 + ax + b)$ for $x^2 + ax + b$ irreducible in $\Q[x]$.

A rough picture is as follows.  We take the complex plane with all the rational points on the real axis labeled.  We identify pairs of complex roots under conjugation and also pairs of real roots under the radical conjugation ($a + \sqrt{b}$ and $a - \sqrt{b}$).  It roughly looks like the complex plane folded on itself with the axis also glued onto itself.
\end{proof}

 \end{document}

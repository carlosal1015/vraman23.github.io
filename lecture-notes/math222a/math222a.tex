\documentclass[11pt]{scrartcl}
\usepackage[sexy]{evan}
\usepackage{graphicx}

\usepackage{answers}
\Newassociation{hint}{hintitem}{all-hints}
\renewcommand{\solutionextension}{out}
\renewenvironment{hintitem}[1]{\item[\bfseries #1.]}{}
\declaretheorem[style=thmbluebox,name={Theorem}]{thm}

 %Sets
\newcommand{\N}{\mathbb{N}}
\newcommand{\Z}{\mathbb{Z}}
\newcommand{\F}{\mathbb{F}}
\newcommand{\Q}{\mathbb{Q}}
\newcommand{\R}{\mathbb{R}}
\newcommand{\C}{\mathbb C}
\newcommand{\T}{\mathbb T}

\let \phi \varphi

%From Topology
\newcommand{\cT}{\mathcal{T}}
\newcommand{\cB}{\mathcal{B}}
\newcommand{\cC}{\mathcal{C}}
\newcommand{\cH}{\mathcal{H}}



\begin{document}
\title{Math 222a}
\author{Vishal Raman}
\thispagestyle{empty}
$ $
\vfill
\begin{center}

\centerline{\huge \textbf{Math 222a Lecture Notes, Fall 2020}}
\centerline{\Large \textbf{Partial Differential Equations} } 
\centerline{Professor: Daniel Tataru}
\centerline{Vishal Raman}
\end{center}
\vfill
$ $
\newpage
\thispagestyle{empty}
\tableofcontents
\newpage
%\maketitle
\section{September 1st, 2020}
\subsection{Introduction}
Partial differential equations apply to functions $u: \R^n \rightarrow \R(\C)$, where $u$ refers to the space dimension.  Usually, $n \ge 2$($n=1$ corresponds to ODEs).   

We present the following notation: \begin{itemize}
\item $\frac{\partial}{\partial x_i} u = \partial_i u$
\item There is also multi-index notation, where $\alpha = (\alpha_1, \dots, \alpha_n)$ and $\partial^\alpha u = \partial_1^{\alpha_1}\partial_2^{\alpha_2} \dots \partial_n^{\alpha_n} u$.  The size of $\alpha$ is given by $|\alpha| = \sum_{i=1}^n \alpha_i$.
\item $C(\R^n)$, continuous functions in $\R^n$.
\item $C(\Omega)$, $\Omega \subset \R^n$, continuous functions in $\Omega$.
\item $C^1(\R^n), C^1(\Omega)$, continuously differentiable functions.
\item $C^k(\R^n), C^k(\Omega)$, $k$-times differentiable.
\item $C^\infty(\R^n) = \bigcap_{k=0}^{\infty} C^k(\R^n)$.
\end{itemize}

We consider an example PDE, 
$$F(u, \partial u, \partial^2 u, \dots, \partial^k u) = 0.$$
In the above, $k \ge 1$ and $k$ is the \textbf{order} of the equation.  We also have the shorthand $F(\partial ^{\le k}u) = 0$.  

\subsection{Classification of PDE's}
\begin{definition}[Linear PDE] The PDE is a linear function of its arguments.  We can apply multi-index notation, as follows:
$$\sum_{|\alpha| < k} c_\alpha \partial^{\alpha}u = f(x).$$
If $f(x) = 0$, the PDE is \textbf{homogeneous}, otherwise it is \textbf{inhomogeneous}.
\end{definition}
This can be separated into linear PDEs with constant coefficients, $c_{\alpha } \in \R, \C$ and variables coefficients, $c_{\alpha} = c_{\alpha}(x)$.  [In this class, we focus on constant coefficient PDEs, but many of the techniques can be extended to variable coefficient PDEs.]
\begin{definition}[Nonlinear PDE]We look at a function $F = F(u, \partial u, \dots, \partial^k u)$.  The highest order terms are take the \textit{leading role}. 
\begin{itemize}
\item Semilinear PDE's: $F$ is linear, with constant or variable coefficients in $\partial^k u$: $$\sum_{|\alpha| = k} c_{\alpha}(x)\partial^\alpha u = N(\partial^{\le k-1}u).$$
The LHS is called the principal part, and the RHS is the perturbative role.
\item Quasilinear PDE's: 
$$\sum_{|\alpha|=k} c_{\alpha}(\partial^{\le k-1} u) \partial^{\alpha}u = N(\partial^{\le k-1}u).$$
\item Fully Nonlinear PDE's: $F(\partial^{\le k} u) = 0$, with a nonlinear dependence on $\partial^k u$.  
\end{itemize}
\end{definition}
Some examples:
\begin{itemize}
\item Linear, homogeneous, variable coefficients, order 1:$$\sum_{k=1}^u c_k(x)\partial_k(u) = 0.$$
\item Define $\Delta = \partial_1^2 + \dots + \partial_n^2$, the Laplacian operator.  We have a linear, constant coefficients, inhomogeneous, order 2:$$\Delta u = f.$$
\item Semilinear, order 2: $$\Delta u = u^3.$$ [Note that translation invariance makes homogeneous vs inhomogeneous not useful for classification in the case of nonlinear PDE's.]
\item Harmonic Map Equation:
$$\Delta u = u |\nabla u|^2.$$
It is still semilinear, but with a stronger nonlienarity.
\item Monge Ampere Equation:
$$\R^2, \partial_1^2 u \partial_2^2 u - (\partial_1 \partial_2 u)^2 = 0.$$
It is a fully nonlinear equation.
\end{itemize}

\subsection{Initial Value Problems}
We have various types of problems:
\begin{itemize}
\item (Stationary Problems) With $u : \R^n \rightarrow \R$,$$F(\partial^{\le k} u) = 0,$$ might describe an equilibrium configuration of a physical system.
\item (Evolution Equations) With $u : \R\times \R^n\rightarrow \R$, $u(t, x)$ describes the state at time $t$.  We can think about the order in $x$ or in $t$. 
\end{itemize}
\begin{definition}[Initial Value Problem/Cauchy Problem] A PDE with initial conditions.
\end{definition}
\begin{example} Consider the heat equation:
$$\partial_t u = \Delta_x u,$$
$$ u(t = 0, x) = u_o(x).$$
The equation is first order in $t$, but second order in $x$.
\end{example} 
\begin{example}
In $[\R \times \R]$, the vibrating string:
$$\partial_t^2 u = \partial_x^2 u,$$
$$u(t=0, x) = u_0(x),$$
$$\partial_t u(t=0, x) = u_1(x).$$
Note that this equation is second order in time, and requires 2 pieces of initial data.

An easier problem: Compute the Taylor series of $u$ at some point $(0, x_0)$. It requires $\partial_t^{\alpha} \partial_x^{\beta} u(0, x_0)$.  
\begin{itemize}
\item This is obvious if we have no time derivative or exactly 1.  
\item Second order time derivatives come from the equation.
\item Third order or higher time derivatives come from differentiating the equation:
$$\partial_t^3 u = \partial_x^2 \partial_t u.$$
\end{itemize}
\end{example}
\subsection{Boundary Value Problems}
We begin with an example.
\begin{example} Take $\Delta u = f$ in $\Omega \subset \R^3$, which represents equilibrium for temperature in a solid.  To solve, we need information about the boundary of $\Omega$.  For example,
$$\Delta u = f \in \Omega,$$
$$u = g \in \partial \Omega.$$
\end{example}
\subsection{Fluid Classification}
We take $u : \R^n \rightarrow \R(\C)$, and 
$$F(\partial^{\le k} u) =  0.$$
This is considered to be a \textbf{scalar equation}.

We could also take a \textbf{system} of equations, where $u: \R^n \rightarrow \R^m(\C^m)$, where $u = [u_i]$ a column of equations.  These are often more difficult than scalar equation.  We should have 
$$F(\partial^{\le k} u) = 0,$$
but $F : \R^{(\cdot)} \rightarrow \R^m(\C^m).$
\begin{example} A 2-system:
$$\Delta u = v,$$
$$\Delta v = -u.$$
\end{example}
We can often reduce the order of a scalar equation by turning it into a system:
\begin{example} Consider the vibrating string, $$\partial_t^2 u = \partial_x^2 u.$$
If we take $v = \partial_t u$, the it suffices to solve the system,
$$\partial_t u = v,$$
$$\partial_t v = \partial_x^2 u.$$
We van reduce it further by saying $u_1 = \partial_x u, u_2 = \partial_t u$ for the system,
$$\partial_t u_1 = \partial_x u_2,$$
$$\partial_t u_2 = \partial_x u_1.$$
\end{example}
\end{document}

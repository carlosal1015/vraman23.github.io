\documentclass[11pt]{scrartcl}
\usepackage[sexy]{evan}
\usepackage{graphicx}

 %Sets
\newcommand{\N}{\mathbb{N}}
\newcommand{\Z}{\mathbb{Z}}
\newcommand{\F}{\mathbb{F}}
\newcommand{\Q}{\mathbb{Q}}
\newcommand{\R}{\mathbb{R}}
\newcommand{\C}{\mathbb C}
\newcommand{\T}{\mathbb T}

\let \phi \varphi
\let \hat \widehat

\newcommand{\<}{\langle}
\renewcommand{\>}{\rangle}

%From Topology
\newcommand{\cT}{\mathcal{T}}
\newcommand{\cB}{\mathcal{B}}
\newcommand{\cC}{\mathcal{C}}
\newcommand{\cH}{\mathcal{H}}


\usepackage{answers}
\Newassociation{hint}{hintitem}{all-hints}
\renewcommand{\solutionextension}{out}
\renewenvironment{hintitem}[1]{\item[\bfseries #1.]}{}
\declaretheorem[style=thmbluebox,name={Theorem}]{thm}

\begin{document}
\title{Math 258}
\author{Vishal Raman}
\thispagestyle{empty}
$ $
\vfill
\begin{center}

\centerline{\huge \textbf{Math 258 Lecture Notes, Fall 2020}}
\centerline{\Large \textbf{Harmonic Analysis} } 
\centerline{Professor: Michael Christ}
\centerline{Vishal Raman}
\end{center}
\vfill
$ $
\newpage
\thispagestyle{empty}
\tableofcontents
\newpage
%\maketitle
\section{August 27th, 2020}
\subsection{Introduction}
We begin by considering the problem of conduction of heat in a circle.  We use the map $x \mapsto e^{ix}, x \in [0, 2\pi)$.  Where $u$ is the temperature, $t$ is the time, we believed that $u_t = \gamma u_{xx}$, where subscripts denote partial derivatives.  We also have an initial condition, $f(x) = u(x, 0)$.

There are some simple solutions $e^{inx}e^{-\gamma n^2t}|_{t=0} = e^{inx}$.  The product of solutions, the sum of solutions, and scalar multiple of solutions are all solutions, so he wrote the solution as $$f(x) = \sum_{n = -\infty}^{\infty} a_n e^{inx}, u(x, t) = \sum_{n} a_n e^{-\gamma n^2t}e^{inx}.$$

\subsection{Fourier Analysis}
We take a circle $\{z \in \C : |z = 1|\}$, which can also be thought of as $\R/(2\pi\Z)$, with the map $x \mapsto e^{ix}$. 
 Suppose we have $G$ a finite abelian group, and $\hat{G} = \{\text{hom }\phi:G \rightarrow \R/\Z \}$, the dual group.  $\hat{G}$ is also a group, formally known as the set of characters.

\begin{example} If we take $G = \Z_N = \Z/N\Z$, with the map $x \mapsto e^{2\pi i xn/N}$, for $n \in Z_n$.

 Similarly, taking $G \cong Z_{N_1} \times \Z_{N_2} \times \dots$, we take $x \mapsto \prod e^{2\pi i xn/N_i}$.
\end{example}

Take $e_\xi(x) = e^{2\pi i \xi(x)}$, where $\xi: G \mapsto \R/\Z$.  Working in $L^2(G)$, we note the following:

\begin{fact} If $\xi \ne \varphi$, then $\left<e_\xi, e_\varphi \right> = 0$.
\end{fact}
\begin{proof}
$$\sum_{x \in G} \xi(x) \overline{\varphi(x)} = \sum_{u} \xi(u+y)\overline{\varphi(u+y)} - \left (\sum_u \xi(u)\overline{\varphi(u)}\right ) \xi(y) \overline{\varphi(u)}.$$
Hence, either $\left < \xi, \phi\right > = 0$ or $\xi(y)\overline{\phi}(y) = 1$ for all $y \in G$, which implies $\xi = \phi$.
\end{proof}
If follows that $\{e_f : f \in \hat{G}\}$ is an orthonormal set in $L^2(G)$ Then, the dimension is $|\hat{G}| = |G| = \dim(L^2(G))$.  Hence, the set forms an orthonormal basis for $L^2(G)$.

Then, for all $f \in L^2(G)$,  we have $$\|f\|_{L^2(G)}^2 = \sum_{\phi \in \hat{G}} |\left <f, e_\xi\right >|^2,$$
$$f = \sum_{e_{\xi} \in \hat{G}} \left <f , e_{\xi}\right >e_\phi.$$

\subsection{On Tori of Arbitrary Dimension}
We define $\T = \R/2\pi\Z$, from $[0, 2\pi]$.  We then work on $\T^d$, $d \ge 1$.  

For $f \in L^2(\T^d)$, we define $$\hat{f}(n) = (2\pi)^{-d}\int f(x)e^{-inx}dx.$$

We have an inner product $\left <f, g\right > = \int_{\T^d} f(x) \overline{g(x)}d\mu(x)$ defined over a Lebesgue measure or Euclidean measure on $\T^d$. 

\begin{thm}[Parseval's Theorem] For all $f \in L^2(\Pi^d)$,
$$\|f\|_{L^2}^2 = (2 \pi)^d \sum_{n \in \Z^d}|\hat{f}(n)|^2,$$
and we have 
$$f = \sum_{n \in \Z^d} \hat{f}(n)e^{inx},$$ in the sense that 
$$\|f - \sum_{n \in \Z^d} \hat{f}(n)e^{inx}\|_L^2 \rightarrow 0.$$ 
\end{thm}
Note: you can usually figure out the constant with the simplest example, $f = 1$.  
\begin{proof}
Take $\T^d, e_n(x) = e^{in\cdot x}$.  The $\{(2\pi)^{-d/2}e^n:n \in \Z^d\}$ is orthonormal(left as an exercise).  Then, for all $f$, $\sum_{n} \left <f, (2\pi)^{-d/2} e_n\right > \le \|f\|_{L^2}^2$, with equality if the set is a basis(Bessel's inequality).  

It suffices to show that $\text{span}\{e_n\}$ is dense in $L^2$.  Take $P = \text{span}\{e_n\}$, and note that $P$ is an algebra of continuous functions on $\Pi^d$,  closed under conjugation, contains $1$, and separates points.  Hence, the Stone-Weierstrass theorem implies that $P$ is dense in $C^o(\Pi^d)$ with respect to $\|\cdot\|_{C^o}$.  Then $C^o \subset L^2$ is dense(general theory about Compact Hausdorff spaces, Radon Measures).  

The statement $\|f - \sum_{n \in \Z^d} \hat{f}(n)e^{inx}\|_L^2 \rightarrow 0$ follows from the general theory of orthonormal systems.
\end{proof}

\subsection{Euclidean Spaces}
We work in $\R^d, (d \ge 1)$.  Take $\xi \in R^d$, and $x \mapsto x\xi \in \R$ is a homomorphism from $\R^d \rightarrow \R$, but if we take $x \mapsto e^{ix\xi}$, we have a homomorphism from $\R^d\mapsto \Gamma$.  We try to define the following:
$$\hat{f}(\xi) = \int_{\R^d} f(x)e^{-ix\xi}dx = \left <f, e_{\xi}\right >_{L^2(\R^d),}$$
where $e_{xi}(x) = e^{ix\xi}$.

Some problems:
\begin{enumerate}
\item $e_\xi \not \in L^2(\R^d)$
\item $f(x)e^{-ix\xi}$ need not be in $L^1$ if $f \in L^2$.
\end{enumerate}
We fix this by imposing extra conditions.
\begin{definition} For $f \in L^1(\R^d)$, we define
$$\hat{f}(\xi) = \int_{\R^d} f(x)e^{-ix\xi}dx.$$
\end{definition} 

Note that $f \in L^1$ implies that $\hat{f}$ is bounded, continuous.  We see this as follows:
$\hat{f}(\xi + u) - \hat{f}(\xi) = \int f(x)e^{-ix\xi}(e^{-ixu}-1)dx$.  If we let $u \rightarrow 0$, the right goes to $0$ pointwise, and $(2|f|) \in L^1$ dominates the integral, it goes to $0$.  
\begin{proposition}If $f \in L^1 \cap L^2(\R^d)$, $\hat{f}\in L^2(\R^d)$, 
$$\|\hat{f}\|_{L^2}^2 = (2\pi)^d \|f\|_{L^2}^2.$$
\end{proposition}

\begin{thm}[Plancherel's Theorem] $\pi: L^1\cap L^2 \rightarrow L^2$ extends uniquely to $\hat{\pi}: L^2(\R^d) \rightarrow L^2(\R^d)$, linear, bounded, 
$\|\hat{\pi}f\|_{L^2}^2 = (2\pi)^d\|f\|_{L^2}^2$,
and for all $f \in L^2$, we have an inverse Fourier Transform,
$\check{f}(y) = \int f(\xi)e^{iy\xi}d\xi$ for $f \in L^1 \cap L^2$, and $\check{\cdot}$ also extends.  

Finally, $$\|f - (2\pi)^{-d}\int_{|\xi| \le R} \hat{f}(\xi)e^{ix\xi}d\xi\|_{L^2} \rightarrow 0.$$
\end{thm}
Note that $\check{f}(y) = \hat{f}(-y)$.
\begin{proof}
We first prove that $\|f\|_{L^2}^2 = (2\pi)^{-d}\|\hat{f}\|_{L^2}^2$ for all $f \in L^1 \cap L^2$.  We prove this for a dense subspace $\mathscr{P} $ of $L^2$.  We will show later that there exists a subspace $V \subset L^2(\R^d)$ so that $V$ is dense in $L^2$, $V \subset L^1$, $\forall f \in V,$ there exists $C_f < \infty$, so for all $\xi \in \R^d$, $|\hat{f}(\xi)| \le C_f(f(\xi))^{-d}$ and $f$ is continuous with compact support.

We are given $f: \R^d \rightarrow \C$ supported where $|x |\le R = R_f < \infty$.  For large $t \ge 0$, define $f_t(x) = f(tx)$(this shrinks the support of $f$), supported where $|x| \le R/t < \pi$.  We can then think of $f_t: \T^d \rightarrow \C$.  

Now, we calculate
\begin{align*}
\hat{f}_{t}(n) & = (2\pi)^d\int_{\T^d} f_t(x)e^{-inx}dx\\
&= t^{-d}(2\pi)^d \int_{R^d} f(x)e^{-in/ty}dy \\
&= t^{-d} (2\pi)^{-d}\hat{f}(t^{-1}n),
\end{align*}
where the first hat is on $\T^d$ and the second is on $\R^d$, so the Fourier coeficients in the euclidean case are scalar multiples of the Fourier coefficients in the Tori case.  

Thus,
$$\|f_t\|_{L^2(\T^d)}^2 = t^{-d}\|f\|_{L^2(\R^d)}^2 = c_d\sum_{n \in \Z^d}|\hat{f}_t(n)|^2 = c_d't^{-2d}\sum_{n} |\hat{f}(t^{-1}n)|^2$$

Hence,
$$\|f\|_{L^2(\R^d)}^2 = c_d't^{-d}\sum_{n} |\hat{f}(t^{-1}n)|^2.$$

This has a nice tiling Riemann sum interpretation: if we take $\R^d$ and tile it with cubes of sidelength $1/t$ where one corner is at $t^{-1}n$ for $n \in \Z^d$, then
$$\|f\|_{L^2(\R^d)}^2 = c_d' t^{-d}\sum_{n}\left |\hat{f}(t^{-1}n)\right |^2 = \int_{\R^d}|g_t|^2 dx,$$
where $g(x) = \hat{f}(t^{-1}n)$.
 
We claim 
$$\int_{\R^d}|g_t|^2 \rightarrow \int_{\R^d}|\hat{f}|^2,$$
which follows from the dominated convergence theorem:  where we take a sequence over $t$ going to infinity, with dominator $C_f^2(1 + |\xi|)^{-2d}$ in $L^1$ and $|\hat{f}(\xi)| \le C_f^2(1 + |\xi|)^{-2d}$. Furthermore, we have $g_t(\xi) \rightarrow \hat{f}(\xi)$ as $t \rightarrow 0$, and $\hat{f}$ is continuous so $g_t$ is pointwise convergent, and we have
$$|g_t(\xi)| = |\hat{f}(t^{-1}n)| \le C_f(1 + |t^{-1}n|)^{-d} \le C'(1 + |\xi|)^{-d}.$$
\end{proof}
\newpage
\section{September 1st, 2020}
\subsection{Proof of Plancherel's Theorem}
Last time
\begin{itemize}
\item $\R^d$, $$\hat{f}(\xi) = \int_{\R^d} e^{-ix \cdot \xi} f(x)dx.$$
\item $V = (f \in L_1 \cap L_2(\R^d)) : |\hat{f}(\xi)|\left <\xi \right >^d$ is a bounded linear function, $\left< x\right > = (1 + |x|^2)^{1/2} \ge 1, = |x|$ for $x$ large.
\item Claim: $V$ is dense in $L^2(\R^d)$.  Then $\|\hat{f}\|_{L^2} = (2\pi)^{d/2}\|f\|_{L^2}$ for all $f \in V$ so there exists a unique bounded linear operator $\mathscr{F}$ on $L^2(\R^d)$, where $\mathscr{F}$ takes a function to it's fourier transform.
\item We discussed some properties of $\mathscr{F}$.
\begin{itemize}
\item $\|\mathscr{F}f\|_2 = (2\pi)^{d/2} \|f\|_2$
\item $\mathscr{F}$ is onto.
\item For all $f \in L^2$, $$\left\|f - (2\pi )^{-d} \int_{|\xi| \le R}e^{ix \cdot \xi}\mathscr{F}(f)(\xi) d\xi \right \|_{L^2}\rightarrow 0,$$
in the limit where $R \rightarrow \infty$.
\end{itemize}
\end{itemize}
First note that $\mathscr{F}$ has closed range(this was an exercise).  It suffices to show: If $g \in L^2, g \perp \mathscr{F}(f) $ for all $f \in V$, then $g = 0$.
\begin{proof} First, note that 
$$0 = \left <g, \mathscr{F}(f)\right >= \left <\mathscr{F}^*(g),f\right >,$$ 
and for all $g \in V$, $$\mathscr{F}^*g(x) = \int g(\xi)e^{ix \cdot \xi}d\xi$$
Therefore, $\mathscr{F}^*(g)(x) = (\mathscr{F}g)(-x)$ for all $g \in V$, which is dense in $L^2$.  Hence, $\mathscr{F}g = 0$, and the Fourier transform preserves norms, so $g = 0$.
\end{proof}
We also claimed the following:
Let $f \in L^2$:
$$\|f(x) - (2\pi)^{-d}\int_{|\xi| \le R} (\mathscr{F}f)(\xi)e^{ix\cdot \xi}d\xi\|_2^2 \rightarrow 0.$$
\begin{proof}
Let $g_r = (2\pi)^{-d}\int_{|\xi| \le R} (\mathscr{F}f)(\xi)e^{ix\cdot \xi}d\xi$.We have to show $\langle f, g_r \rangle \rightarrow \|f\|_2^2$.  Then
$$\|f - g_r\|_2^2 = \|f\|_2^2 + \|g_r\|_2^2 - 2 \text{Re}\langle f, g_r \rangle \rightarrow \|f\|_2^2 + \|f\|_2^2 - 2 \|f\|_2^2.$$
\begin{align*}
\left <f, g_R\right > &= (2\pi)^{-d} \int f(x) \overline{\int_{|\xi| \le R} (\mathscr{F}f)(\xi)e^{ix \cdot \xi}d\xi} dx\\
&= (2\pi)^{-d} \int_{|\xi| \le R} \left (\int f(x)e^{-ix \cdot \xi} dx\right ) \overline{(\mathscr{F}f)(\xi) d\xi}\\
&= (2\pi)^{-d} \int_{|\xi| \le R} |\mathscr{F}f(\xi)|^2 d\xi \rightarrow (2\pi)^{-d} \|\mathscr{F}f\|_2^2 = \|f\|_2^2.
\end{align*}
However, it's not clear that we can use Fubini's theorem.  We would need $f \in L^1 \cap L^2$.  But this is not an issue as $L^1 \cap L^2 \subset L^2$ is dense, so if we let $\epsilon > 0, f = G+h, \|h\|_2 \le \epsilon$ and $G \in L^1 \cap L^2$.  Showing the convergence from here is an exercise.
\end{proof}
We still need $V = (f \in L^1 \cap L^2 : \left <\xi\right>^d (\hat{f}(\xi)) \text{ is bounded})$ is dense in $L^2$.  We'll discuss this in the future.
\subsection{Introduction to Convolution}
Our meta definition is $f * g(x) = \int f(x-y)g(y) dy$, but it will depend on the conditions of the function for the integral to be defined.

 Convolution is generally associated to a group, where
$$\int_G f(xy^{-1}g(y)d\mu(y)),$$
with the Haar measure(done in 202b).

If we substitute $y = x-u$, then $$f *g(x) = \int f(u)g(x-u)du = g * f(x).$$  It is also associative: $(f * g)* g = f * (g * h)$ for all $f, g, h$(involves Fubini's theorem).

We can formally write
$$f * g(x) = \int_{\R^d \times \R^d} f(u)g(v) d\lambda_x(u, v),$$
where $\lambda_x$ is supported on $\Lambda = \{(u, v) : u+v = \lambda \}$(an affline subspace).  If we have a subset $E \subset \Lambda$, $\lambda_x(E) = |\pi_1(E)| = |\pi_2(E)|$, where $\pi_i$ are Lebesgue measure s of projections on the $i$-th factor.  Note the following: suppose that $f, g$ are continuous with compact support.  Then $\text{supp}(f * g) \subset \text{supp}(f) + \text{supp}(g)$, where $A + B = \{a+b : (a, b) \in A \times B\}$.

Let $T : C_0^0(\R^d) \rightarrow C_b^0(\R^d)$ be bounded, linear and $T \circ \tau_y  = \tau_y \circ T$ for all $x \in \R^d$ ($\tau_y f(x) = f(x + y)$, a translation).  Then, there exists a Complex Radon measure $\mu$ on $\R^d$ so that for all $f \in C_{0}^0$, $T(f) = f * \mu$, where 
$$f * \mu(x) = \int f(x-y) d\mu(y).$$

In the case of $\T^1$, $f(x) = \sum_{n = -\infty}^{\infty} \hat{f}(n) e^{inx}$ for all $f \in L^2$. Suppose we wanted to consider the partial sums, 
$$\sum_{n  = -N}^N \hat{f}(n)e^{inx} = S_N(f)(x).$$
In what sense does $S_N f \rightarrow f$, and for which functions $f$ do we have convergence?

$$S_N(f)(x) = \sum_{n = -N}^N e^{inx}(2\pi)^{-1}\int_{-\pi}^{\pi} f(y)e^{-iny}dy = (2\pi)^{-1} \int f(y) \sum_{n = -N}^N e^{in(x-y)} dy$$
$$= (2\pi)^{-1} \int_{-\pi}^{\pi} f(y) D_n(x-y)dy.$$

The Dirichlet Kernels, $D_N(x) = \sum_{n = -N}^N e^{inx} = \frac{\sin{(N+1/2)x}}{\sin{(x/2)}}$ if $x \ne 0$ or $D_N(x) = 2N+1$ if $x = 0$.
\subsection{General Convolution}
\begin{thm} Let $f, g \in L^1(\R^d)$.  Then, the following are true:
\begin{itemize}
\item $y \mapsto f(x-y)g(y) \in L^1(\R^d)$ for almost every $x \in \R^d$.
\item $x \mapsto \int f(x-y)g(y) dy$ is Lebesgue measurable.
\item $f * g \in L^1(\R^d)$ and $\|f * g\|_1 \le \|f\|_1 \|g\|_1$.
\item If $f, g \ge 0$, then $\|f * g\|_1 = \int f * g = \int f \int g$.
\item The operation commutative and associative, so $L^1$ is an algebra, but it no multiplicative identity, so no inverses.
\item For $f, g \in L^1$, $(\widehat{f \star g}) = \widehat{f}\cdot\widehat{g}$.
\end{itemize}
In other words, convolution is a nice bilinear operation.
\end{thm}
\begin{proof}
Let $F(x, y) = f(x-y)g(y)$, $F : \R^{d+d} \rightarrow \C$ is Lebesgue measurable. We claim that $F \in L^1(\R^d \times \R^d)$. It follows from  
$$\int |F(x, y)|dxdy = \int |f(x-y)| |g(y)| dx dy = \int |g(y)| dy  \int |f(x)|dx = \|g\|_1\|f\|_1 < \infty.$$

Now, $F \in L^1$, so by Fubini's theorem, for almost every $x, y \rightarrow f(x-y)g(y) \in L^1$ and $x \mapsto \int f(x-y)g(y) dy$ is Lebesgue measurable.

$$ \|f* g\|_1=\int |f*g(x)|dx = \int \left |\int f(x-y)g(y)dy\right |dx \le \int \int |f(x-y)||g(y)|dy dx = \|f\|_1\|g\|_1.$$

Note that $\int (f * g)(x)dx = \|f\|_1\|g\|_1,$ for non-negative functions.

Finally, 
\begin{align*}
(f*g)^\wedge (\xi) &= \int e^{-ix \cdot \xi} \left (\int f(x-y)g(y)dy\right)dx \\
&= \int \left (\int e^{-ix\cdot \xi}f(x-y)dx\right ) dy, x = u+y\\
&= \int \left (e^{-i(u+y)\cdot \xi}f(u)du\right)g(y)dy \\
&= \int e^{-iy\cdot \xi} \hat{f}(u)g(y)dy\\
&= \hat{f}(\xi) \cdot \hat{g}(\xi).
\end{align*}
\end{proof}
\begin{example}[A Warning] In $\R^1$, $f(x) = |x|^{-2/3} 1_{|x| \le 1}$, which has an asymptote at $0$.  $f \in L^1$, and 
$$(f*f)(0) = \int_{-1}^1 |u|^{-4/3}dy= + \infty.$$
\end{example}
\begin{proposition} Let $p \in [1, \infty]$.  Let $f \in L^1, g \in L^p$.  Then,
\begin{itemize}
\item $y \mapsto f(x-y) g(y) \in L^1$ for almost every $x \in \R^d$.
\item $x \mapsto \int f(x-y)g(y)dy$ is Lebesgue measurable.
\item $f*g \in L^p(\R^d)$, $\|f * g\|_p \le \|f\|_1 \|g\|_p$.
\end{itemize}
\end{proposition}
\begin{proof} 
For $p = \infty$, $\int f(x-y)g(y)dy \in C_0(\R^d)$.  

If $1 < p < \infty$, $L^P \subset L^1 + L^\infty$, as follows:
$$f(x) = f(x)1_{|f(x)| \le 1} + f(x)1_{f(x) > 1}.$$

We can prove the rest with Minkowski's inequality, or a simpler way.  Let $q = p' = \frac{p}{p-1}$ (hence $\frac{1}{q} + \frac{1}{p} = 1$).  
We use the norm definition,
$$\|f * g\|_p = \sup_{\|h\|_q \le 1} \int |g* f| \cdot |h|. $$
$$\int |g*f|\cdot h \le \int (|g|*|f|)\cdot h = \int \int |g(x-y)||f(y)|dy h(x)dx$$
$$= \int |f(y)| \int |g(x-y)|h(x)dxdy \le \int |f(y)| \|g\|_p *1 dy = \|f\|_1 \|g\|_p .$$
\end{proof}
\pagebreak
\section{September 3rd, 2020}
\subsection{Convolution and Continuity}
Recall convolution:
$$f * g(x) = \int f(x-y)g(y)dy, f*\mu(x) = \int_{\R^d}f(x-y)d\mu(y),$$
where $f$ is continuous, bounded, $\mu$ is a complex Radon measure($|\mu|$ is finite) 
\begin{proposition} Let $T : C_0^0 \rightarrow C_b^0$.  Suppose $T$ is translation invariant: $T \circ \tau_y = \tau_y \circ T$ for all $y \in \R^d$.  [There exists $A < \infty : \|Tf\|_{C_0} \le A\|f\|_{C_0}$ for all $f$. Recall $\|f\|_{C_0} = \sup_x |f(x)|,$ and $C_0^0, C_b^0$ are Banach spaces.]  There exists a complex radon measure $\mu$ such that $Tf = f * \mu$ for all $f$.
\end{proposition}
\begin{proof}
Given $T: C_0^0 \rightarrow C_b^0$, consider the map $\ell : \C_0^0 \rightarrow \C$ given by $f \mapsto (Tf)(0)$.  It is clear that $\ell$ is linear.  Furthermore, $\ell$ is bounded, since $$|Tf(0)| \le \|Tf\|_{C_0} \le A\|f\|_{C_0},$$
so $\ell \in (C_0^0)^*.$  Recall the Riesz Representation Theorem, there exists $\nu$, a complex Radon measure, such that for all $f \in C_0^0$
$$\ell(f) = \int f d\nu.$$

Let $y \in \R^d$.  We have 
$$Tf(-y) = Tf(0 - y) = (\tau_yTf)(0) = T(\tau_y f)(0) = \int \tau_y f(x) d\nu(x) = \int f(x-y)d\nu(x).$$
Similarly, for all $x$, $(Tf)(-x) = \int f(y-x)d\nu(y)$.  [See lecture notes for correct algebra, sad].
\end{proof}
\subsection{Convolution and Differentiation}
Informally,
$$\frac{\partial}{\partial x_j} \int f(x-y)g(y)dy = \int \frac{\partial f}{\partial x_j}f(x-y)g(y)dy.$$
\begin{proposition} Assume $f \in C^1(\R^d), g \in L^1$ and $f, \nabla f$ is bounded.    Then
$$f*g \in C^1, \frac{\partial}{\partial x_j}(f \star g) = \left (\frac{\partial f}{\partial x_j}\right) * g.$$
\end{proposition}
\begin{proof}
We assume $d=1$ for clarity. 
$$\frac{(f*g)(x+t) - (f*g)(x)}{t} = \int \frac{f(x+t-y) - f(x-y)}{t}g(y)dy. $$
Let $t \rightarrow 0$.  Use DCT, with dominator $$|g(y)|\cdot \sup_{t, u} \frac{|f(u+t) - f(u)|}{|t|}.$$ The supremum is finite by the mean value theorem.
\end{proof}
\begin{example} Take $g \in L^\infty$, $f \in C_1$, and there exists $a < \infty$ such that for all $x$, $$|f(x)| + |\nabla f(x)| \le A \<x\>^{-\gamma}.$$
Hence, $f, \nabla f \in L^1$.  Then $f * g \in C^1, \nabla(f*g) =(\nabla f)*g.$
\end{example}
We can iterate this: Under appropriate conditions
$$\frac{\partial ^{\alpha}(f*g)}{\partial x^{\alpha}} = \frac{\partial ^{\alpha}f}{\partial x^{\alpha}}*g,$$
$$\frac{\partial ^{\alpha+\beta}(f*g)}{\partial x^{\alpha_\beta}} = \frac{\partial ^{\alpha}f}{\partial x^{\alpha}}*\frac{\partial ^{\beta}g}{\partial x^{\beta}}.$$

\begin{proposition} If $f \in L^1$ and $g \in L^{\infty}$, then $f * g \in C_b^0$.  
\end{proposition}
\begin{proof}
Recall:  If $f \in L^1(\R^d)$, then $y \mapsto \tau_y f \in L^1$ is continuous: As $y\rightarrow 0$, 
$$\|\tau_y f - f\|_1 \rightarrow 0.$$

Then,
$$(f*g)(x) - (f*g)(x') = \int (f(x-y) - f(x' - y))g(y)dy = \int[ f(x-y) - (\tau_u f)(x-y) ]g(y)dy,$$
where $u = x'-x$.  As $u \rightarrow 0$, $\|f - \tau_u f\|_1 \rightarrow 0$, and $g \in \L^\infty$, so the integral approaches $0$, as desired.
\end{proof}
\subsection{Approximation}
\begin{definition}[Approximate Identity Sequence] An approximate identity sequence for $\R^d$ is $(\phi_n)_{n \in \N}, \phi_n \in L^1(\R^d)$ with the following conditions:
\begin{itemize}
\item $\int_{\R^d}\phi_n = 1$.
\item For all $\delta > 0$, $\int_{|x| \ge \delta}|\phi_n|dx \rightarrow 0$ as $n \rightarrow \infty$.
\end{itemize}
\end{definition}
\begin{thm} Let $(\phi_n)$ be an approximate identity sequence in $\R^d$.  \begin{enumerate}
\item Let $f \in C_b^0$ be uniformly continuous.  Then $f * \phi_n \rightarrow f$ uniformly.
\item Let $f \in C_b^0$.  Then $f * \phi_n \rightarrow f$ uniformly on every compact set.
\item If $1 \le p \le \infty$, then for all $f \in L^p$, $\|f * \phi_n - f\|_p \rightarrow 0$.
\end{enumerate}
[All the above limits are taken for $n \rightarrow \infty$.]
\end{thm}
\begin{proof}
\begin{align*}
f*\phi_n(x) - f(x) &= \int f(x-y)\phi_n(y)dy - f(x) \\
&= \int(f(x-y) - f(x))\phi_n(y)dy\\
\end{align*}
Then,
$$|f*\phi_n(x) - f(x)| \le \int |f(x-y)-f(x)||\phi_n(y)|dy.$$
Let $\delta > 0$.   Then,
$$\int |f(x-y)-f(x)||\phi_n(y)|dy = \int_{|y \le \delta|} |f(x-y)-f(x)||\phi_n(y)|dy + \int_{|y \ge \delta|} |f(x-y)-f(x)||\phi_n(y)|dy.$$

\begin{align*}
\int_{|y \le \delta|} |f(x-y)-f(x)||\phi_n(y)|dy  &\le \|\phi_n\|_1 \cdot \sup_{x, |y| \le \delta} |f(x-y) - f(x)|\\
&= \|\phi_n\|_1 \cdot \omega_f(\delta) \\
& \le A \cdot \omega_f(\delta).
\end{align*}
Then
\begin{align*}
\int_{|y \ge \delta|} |f(x-y)-f(x)||\phi_n(y)|dy &\le \int_{|y| \ge \delta} 2\|f\|_{C^0} \cdot |\phi_n(y)| dy \\
&\le 2 \|f\|_{C^0}\int_{|y|\ge \delta}|\phi_n|dy.
\end{align*}
Hence
$$|f*\phi_n(x) - f(y)| \le A\omega_f(\delta) +  2 \|f\|_{C^0}\int_{|y|\ge \delta}|\phi_n|dy.$$

Taking the $\lim \sup$, the second term goes to $0$,
so for all $\delta > 0$,
$$\lim_{n\rightarrow \infty} \sup \|f * \phi_n - f\|_{C^0} \le A \omega_f(\delta).$$
Since $f$ is uniformly continuous, $\lim_{\delta \rightarrow 0} \omega_f(\delta) = 0$, which proves the claim.
\end{proof}
\begin{corollary} $C^\infty \cap L^p$ is dense in $L^p$ for all $1 \le p \le \infty$.
\end{corollary}
\begin{proof}
We want to construct $(\phi_n)$ with $\phi_n \in C_0^{\infty}$.  

We claim there exists a function $\phi \in C_0^{\infty}(\R^d)$ with $\int \phi = 1$ and $\phi \ge 0$.  In $d = 1$, take $h(x) = 1{x > 0} e^{-\|x\|}$.  Then, define $\phi(x) = h(x) h(1-x) \in C_0^\infty$.  Then, we normalize $\phi$.

Now, take $\phi_n(x) = n^d \phi(nx)$.
\end{proof}
\begin{example} Let $\phi \ge 0$, $\int \phi = 1$.  Define $\phi_n(x) = n^{d}\phi(nx)$.  Then $\int \phi_n = 1$.  

Furthermore,
$$\int_{|x| \ge \delta} n^d \phi(nx) dx = \int_{|y| \ge n\delta} \phi(y)dy \rightarrow 0.$$ 
\end{example}
\begin{example} Let $\phi(x) = (2\pi)^{-d/2}e^{-|x^2|/2}, x \in \R^d$.  Let $t > 0$ and $\phi_t(x) = (2\pi)^{-d/2} t^{-d/2} e^{-|x|^2/(2t)}$.  Now $t \rightarrow 0^+$ and 
$$\int_{|x|\ge \delta} \phi_t(x)dx \rightarrow 0.$$ 
This is an approximate identity family.
\end{example}
\begin{example}[Interpretation of $f*g$]
$$f * g = \int \tau_y f(x) \cdot g(y)dy.$$
If $g \ge 0$ and $\int g = 1$, then we have an \textbf{average} of translates of $f$.

As $n \rightarrow \infty$, $g = \phi_n$ so the weight concentrates asymptotically at the origin.
\end{example}

\end{document}

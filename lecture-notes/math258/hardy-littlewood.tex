\documentclass[11pt]{scrartcl}
\usepackage[sexy]{evan}
\usepackage{graphicx}

 %Sets
\newcommand{\N}{\mathbb{N}}
\newcommand{\Z}{\mathbb{Z}}
\newcommand{\F}{\mathbb{F}}
\newcommand{\Q}{\mathbb{Q}}
\newcommand{\R}{\mathbb{R}}
\newcommand{\C}{\mathbb C}
\newcommand{\T}{\mathbb T}
\newcommand{\SRS}{\mathscr {S}}

\let \phi \varphi
\let \hat \widehat

\newcommand{\<}{\langle}
\renewcommand{\>}{\rangle}

%From Topology
\newcommand{\cT}{\mathcal{T}}
\newcommand{\cB}{\mathcal{B}}
\newcommand{\cC}{\mathcal{C}}
\newcommand{\cH}{\mathcal{H}}


\usepackage{answers}
\Newassociation{hint}{hintitem}{all-hints}
\renewcommand{\solutionextension}{out}
\renewenvironment{hintitem}[1]{\item[\bfseries #1.]}{}
\declaretheorem[style=thmbluebox,name={Theorem}]{thm}



\begin{document}
\title{The Hardy-Littlewood Maximal Function}
\author{Vishal Raman}
\maketitle
\begin{abstract}
The Hardy-Littlewood maximal operator is a non-linear operator that takes a locally integrable function and returns another function corresponding to the maximum average value the original function can have on balls cenetered at a given point.  It has several applications in Real Analysis and Harmonic Analysis.  We present the lectures notes and solutions to exercises from Math 258(Christ).
\end{abstract}
\section{Weak $\mathbf{L^p}$ and Distribution Functions}
We work in a measure space $(X, \mu)$ that is $\sigma$-finite.  Let $S(X) = S(X, \mu)$ denote the space of simple functions $f: X \to \C$ and $\mathcal M(X)$ denote the space of measure functions.

\begin{definition} The distribution function $\lambda_f$ of $f \in \mathcal M(X)$ is 
$$\lambda_f(\alpha) = \mu\{x \in X : |f(x)| > \alpha \}.$$
\end{definition}
This gives us a way to think about norms in the measure space.  For example, consider the following lemma:
\begin{lemma} For $p \in (0, \infty)$ and $f \in \mathcal M(X)$,
$$\|f\|_p^p = \int |f|^p d\mu = p\int_{0}^{\infty} \alpha^{p-1} \lambda_f(\alpha) d\alpha.$$
\end{lemma}
\begin{proof}
Denote $E = \{(x, \alpha): |f(x)| > \alpha\}$.  
\begin{align*}
p \int_{0}^{\infty} \alpha^{p-1} \lambda_f(\alpha) d\alpha &= p \int_{0}^{\infty}\alpha^{p-1} \int_X 1_E(x, \alpha)d\mu(x)d\alpha \\
&= \int_X \int_{0}^{\infty} p\alpha^{p-1} 1(\alpha < |f(x)|)d\alpha d\mu(x) \\
&= \int_X |f(x)|^p d\mu(x) = \|f\|_p^p.
\end{align*}
\end{proof}
\begin{exercise} Present an alternate proof for simple functions and use the monotone convergence theorem to pass to general functions.
\end{exercise}
\begin{proof} Let $f = \sum_{i=1}^n c_j 1_{E_j}$ be a simple function.  Then, $\|f\|_p^p = \sum_{i=1}^n |c_j|^p \mu(E_j)$.
Note that
\begin{align*}
p\int_0^\infty \alpha^{p-1}\lambda_f(\alpha)d\alpha &= \int_0^\infty p\alpha^{p-1} \sum_{i=1}^n \mu(E_j)1(|c_j| > \alpha) \\
&= \sum_{i=1}^n \mu(E_j) \int_{0}^\infty p\alpha^{p-1}1(|c_j| > \alpha) \\
&= \sum_{i=1}^n \mu(E_j) \int_{0}^|c_j| p\alpha^{p-1} \\
&= \sum_{i=1}^n \mu(E_j) |c_j|^p \\
&= \|f\|_p^p.
\end{align*}
For a general nonnegative function $f$, we can write $ f_n\uparrow f$, where $f_n = \sum_{i=1}^n c_{in}1_{E_in}$.  By the monotone convergence theorem, it follows that 
$$\int |f|^p = \lim_{n \to \infty} \int |f_n|^p = \lim_{n \to \infty} \int_{0}^\infty p\alpha^{p-1}\lambda_{f_n}(\alpha)d\alpha =  \int_{0}^\infty p\alpha^{p-1}\lambda_{f}(\alpha)d\alpha,$$
by noting that $\lambda_{f_n} \uparrow \lambda_f$ and using the monotone convergence theorem.
\end{proof}
\begin{lemma}[Chebyshev's Inequality] If $p \in (0, \infty)$ and $f \in L^p$, then for $\alpha > 0$,
$$\lambda_f(\alpha) \le \alpha^{-p} \|f\|_p^p.$$

For $p=1$, then gives \textit{Markov's Inequality}:
$$\lambda_f(\ell) \le \ell^{-1} \|f\|_1.$$
\end{lemma}
\begin{proof}
$$\lambda_f(\alpha) = \int_X 1(|f(x)| > \alpha) d\mu(x) \le \int_X \alpha^{-p}|f(x)|^p d\mu(x)= \alpha^{-p} \|f\|_p^p.$$
\end{proof}
Chebyshev's inequality loses information, in the sense that 
$$p\int_0^{\infty} \alpha^{p-1} \lambda_f(\alpha) d\alpha \le p\|f\|_p^p\int_0^\infty \alpha^{p-1} \alpha^{-p}d\alpha,$$
and the latter integral diverges.  However, it does allow us to extract useful information from the finiteness of the $L^p$ norms.

\begin{definition} For $p \in [1, \infty)$, define $L^{p, \infty}(X, \mu)$ as the set of functions $f \in \mathcal M(X)$ for which there exists $C < \infty$ with $\lambda_f(\alpha) \le C^p\alpha^{-p}$.
\end{definition}
Note that $L^{p, \infty}$ is a quasi-normed vector space.  We prove the triangle inequality:
\begin{proof}
\begin{align*}
\|f + g\|_{p, \infty} &= \inf \{C : \lambda_{f + g}(\alpha) \le C^p \alpha^{-p}\} \\
&\le \inf \{C : \lambda_{f }(\alpha/2) + \lambda_g(\alpha/2) \le C^p \alpha^{-p}\} \\
&\le \inf \{C: \lambda_f(\alpha/2) \le C^p \alpha^{-p}/2\} + \inf \{C: \lambda_g(\alpha/2) \le C^p \alpha^{-p}/2\} \\
&\le 
\end{align*}
\end{proof}

\subsection{The Hardy-Littlewood Maximal Operator}
\begin{definition}[Hardy-Littlewood Maximal Operator]
Let $f \in L_{loc}^1(\R^d)$, and define $Mf : \R^d \to [0, \infty]$ by 
$$Mf(x) = \sup_{r > 0} |B(x, r)|^{-1} \int_{B(x, r)} |f(y)|dy.$$
\end{definition}

\end{document}

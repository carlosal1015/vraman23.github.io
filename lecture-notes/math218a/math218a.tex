\documentclass[11pt]{scrartcl}
\usepackage[sexy]{evan}
\usepackage{graphicx}

 %Sets
\newcommand{\N}{\mathbb{N}}
\newcommand{\Z}{\mathbb{Z}}
\newcommand{\F}{\mathbb{F}}
\newcommand{\Q}{\mathbb{Q}}
\newcommand{\R}{\mathbb{R}}
\newcommand{\C}{\mathbb C}

%From Topology
\newcommand{\cT}{\mathcal{T}}
\newcommand{\cB}{\mathcal{B}}
\newcommand{\cC}{\mathcal{C}}
\newcommand{\cU}{\mathcal{U}}


\usepackage{answers}
\Newassociation{hint}{hintitem}{all-hints}
\renewcommand{\solutionextension}{out}
\renewenvironment{hintitem}[1]{\item[\bfseries #1.]}{}
\declaretheorem[style=thmbluebox,name={Theorem}]{thm}


\begin{document}
\title{Math 218a}
\author{Vishal Raman}
\thispagestyle{empty}
$ $
\vfill
\begin{center}

\centerline{\huge \textbf{Math 218a Lecture Notes, Fall 2020}}
\centerline{\Large \textbf{Probability Theory} } 
\centerline{Professor: Shirshendu Ganguly}
\centerline{Vishal Raman}
\end{center}
\vfill
$ $
\newpage
\thispagestyle{empty}
\tableofcontents
\newpage
%\maketitle
\section{August 27th, 2020}
\subsection{Introduction}
Consider a \textbf{random experiment} - this involves a state space $\Omega$ and some "probability" on it.  The outcome of an experiment would be $\omega \in \Omega$.

\begin{example}[Fair Coin Toss] $\Omega = \{0, 1\}, P(0) = 1/2, P(1) = 1/2$ models a fair coin toss.  The outcomes are $\omega \in \Omega, \omega = 0$ or $\omega = 1$.
\end{example}
\begin{example}[Continuous State Space] $\Omega = [0, 1]$, $X$ is the outcome of a random experiment.  Suppose $X$ is uniformly distributed random variable.  $P(X \in [0, \frac{1}{2}]) = 1/2$.  Take $A = \Q \cap [0, 1]$.  $P(x \in A) = 0$, since $A$ has no "volume".  Similarly, taking $A_1 = \R\setminus\Q \cap [0, 1]$, then $P(x \in A_1) = 1 - P(x \in A) = 1$.  Finally, take $E \subset [0, 1]$.  $P(x \in E) = $"volume" of $E$.   
\end{example}
The issue: we need to define some notion of volume.  Some properties we would like are the following:
\begin{itemize}
\item Translation Invariance
\item Countable Additivity: $A_1, A_2, \dots$ disjoint with $A = \bigcup A_i$, then $P(A) = \sum_{i=1}^{\infty}P(A_i)$.
\end{itemize}
\subsection{Nonmeasurable Sets}
Take $I = [-1, 2]$, and define $x \sim y$ iff $x-y \in \Q$. [Exercise: check that $\sim$ is an equivalence relation.]  This decomposes $I$ into equivalence classes $I/\sim$.  Note that the equivalence classes are countable, since any class is $x + A, A \subset Q$.  

For each equivalence class $B$, pick $x_B \in B\cap [0, 1]$.  Define $E = \{x_B \}$ over all the equivalence classes.  Note that $x_B$ is a representative of $B$ in $E$, so $B = \{x_b + q : x_b + q \in I, q \in \Q \}.$  

Now, consider the set $[0, 1] \subset \bigcup_{q \in [-1, 1]}E + q \subset [-1, 2]$.  Equality doesn't hold, because there can be $B$ s. t. $x_b$ is close to $0$.  Then $E + (\Q \cap [-1, 1])$ will only recover elements of $B$ near $1$ and will not go up to $2$.

\begin{proposition} We claim that $E+q$ are disjoint for different values of $q$.
\end{proposition}
\begin{proof}
Suppose $E+q_1 \cap E+q_2 \ne \emptyset$ for some $q_1, q_2$.  Then, there exists $x, y \in E$ such that $x+q_1 = y + q_2$.  This implies that $x-y = q_2 - q_1 \in \Q$, so $x \sim y$, but by definition, there is exactly one member of each equivalence class in $E$.  
\end{proof}

The big question: What is $P(E)$?  Suppose $P(E) > 0$.  Then $\bigcup_{q \in [-1, 1]} E + Q \subset [-1, 2]$ and $P(E + q_1) = P(E + q_2) = P(E)$ for all $q_1, q_2$.  Furthermore, by countable additivity,
$$1 \ge P(\bigcup_{q \in [-1, 1] E + q}) = \sum_{q \in [-1, 1]} P(E + q) = \infty \cdot P(E).$$
This would imply that $P(E) = 0$.  However, $$[0, 1] \subseteq \bigcup_{q \in [-1, 1]}E + q \Rightarrow P([0, 1]) = 1/3 \le \sum_{q \in[-1, 1]} P(E+q) = 0.$$  Hence, $P(E)$ cannot be defined.

The issue is the step where we pick $x_B$, since we need to pick $x_B$ from uncountably many points, which assumes the axiom of choice.  It was proved by Robert M. Solovay that all models of set theory excluding the axiom of choice have the property that all sets are Lebesgue measurable.

Our goal is thus to come up with a general framework where things can be consistently defined for a large class of sets.  
\subsection{Measure Theory Beginnings}  For the definitions, we take $\Omega$ to be the state space.
\begin{definition}[Sigma-Algebra]  Suppose $\Sigma$ follows the following properties:
\begin{enumerate}
\item $\emptyset \in \Sigma$
\item $A \in \Sigma \Rightarrow A^c \in \Sigma$
\item $A_1, A_2, \dots \in \Sigma$, then $\bigcup A_i \in \Sigma$
\end{enumerate}
Note that $2$ and $3$ imply $1$ since $(A \cup A^c)^c = \emptyset$.  Then $\Sigma$ is a sigma-algebra.
\end{definition}
Note that we also have countable intersections(this is an easy exercise).

\newpage
\section{September 1st, 2020}
Last time:
\begin{itemize}
 \item We discussed the notation of a $\Sigma$-algebra, a reasonable class of sets on which we will define measures.
 \item Properties: $\emptyset \in \SA, A \in \SA, A^c \in \SA, \bigcup A_i \in \SA$.
 \end{itemize} 
\subsection{Measures}
We are working in a space $(\Omega, \Sigma)$.  
\begin{definition}[Measure] A measure is a function $\mu: \Sigma \rightarrow [0, \infty]$ with the following properties:
\begin{itemize}
\item $\mu(\emptyset) = 0$
\item "Countable Additivity": $\mu(\bigcup A)i) = \sum \mu(A_i)$ for disjoint $A_i \in \Sigma$.
\end{itemize}
\end{definition}
\begin{example} If $\Omega$ is finite, ${1, 2, \dots, n}$, $\Sigma = 2^{\Omega}$, then all possible measures on $(\Omega, \Sigma)$ are given by fixing $a_1, a_2, \dots, a_n \in [0, \infty]$ and $\mu(A) = \sum_{i \in A} a_i$.
\end{example}
Properties of measures:
\begin{itemize}
\item Monotonicity: $A \subset B$, then $\mu(A) \le \mu(B)$. 
\begin{proof}
$B = A \cup (B\setminus A)$ and $B\setminus A \in \Sigma$, so $$\mu(B) = \mu(A) + \mu(B\setminus A) \ge \mu(A).$$
\end{proof}
\item Countable Subadditivity:  $A \subseteq \bigcup_{i=1}^\infty B_i$, then $\mu(A) \le \sum \mu(B_i)$.
\begin{proof}
We disjointify the $B_i$:  Define $C_1 = B_1, C_i = B_i \setminus B_{i-1}$.  Then
$$\mu(A) \le \mu(\bigcup C_i) = \sum \mu(C_i) \le \sum \mu(B_i).$$
\end{proof}
\item: Continuity from below:  If $A_i \uparrow A$, then $\mu(A_i) \uparrow \mu(A)$.
\begin{proof}
$A = A_1 \cup (A_2 \setminus A_1) \cup (A_3 \setminus A_2) \dots$, so by countable additivity
$$\mu(A) = \sum_{i=1}^{\infty} \mu(C_i) = \lim_{n\rightarrow \infty} \sum_{i=1}^n \mu(C_i) = \lim_{n \rightarrow \infty} \mu(A_n).$$
\end{proof}
\item Continuity from above, if $A_i \downarrow A$,  and $\mu(A_1) < \infty$, then $\mu(A_i) \rightarrow \mu(A)$ 
\begin{proof}
We need the condition $\mu(A_i) < \infty$.  Take $A_i = [i, \infty)$ as a counterexample if we don't have that condition.

Define $A_1 \setminus A_i = B_i$, so $B_i \uparrow A_1 \setminus A$.  Then, use the continuity from below.  
\end{proof}
\end{itemize}
\subsection{Sigma algebras}
\begin{fact} For any $A \subset 2^{\Omega}$, define $$\Sigma(A) = \bigcap_{A \in \Sigma}\Sigma.$$ Then, $\Sigma(A)$ is a sigma-algebra.
\end{fact}
Note that $\Sigma(A)$ is the smallest sigma-algebra containing $A$.  For this reason, we call it the sigma-algebra \textbf{generated} by $A$.

\begin{example} Take $X, Y \subset 2^{\Omega}$.  We want to prove $\Sigma(X) = \Sigma(Y)$.  It suffices to show $X \subseteq \Sigma(Y)$ and $Y \subseteq \Sigma(X)$.  
\end{example}

\begin{definition}[Borel Sigma-Algebra] $(\Omega, \cU)$, a topological space with a family of open sets.  The \textbf{Borel Sigma-Algebra} is $\SB = \Sigma(\cU)$. 
\end{definition}

\begin{example} For $\Omega = \R$, $\SB$ is the sigma algebra generated by open sets in $\R$.  We also have $\SB$ is the sigma-algebra generated by open intervals in $\R$, which follows from the fact that any open set can be written as a countable union of open intervals.  Furthermore,
$$\Sigma((a, b): a, b\in \Q, \R) = \Sigma([a, b]: a, b \in \Q, \R),$$
since $[a, b] = \bigcap (a-1/n, b+1/n)$ and $(a, b) = \bigcup [a+1/n, b-1/n]$.
\end{example}

\subsection{Uniform Measure on the Borel Sets}
We will attempt to define the uniform measure on Borel sets of $\R$.  Broadly, we do it as follows:
\begin{enumerate}
\item Define it on a semi-algebra containing the intervals.
\item Extend the definition to an algebra.
\item Extend it to a sigma-algebra.
\end{enumerate}
\begin{definition}[Semi-algebra] $\Sigma \subset 2^\Omega$ is a semi-algebra if
\begin{itemize}
\item $A_1, A_2 \in \Sigma$ implies $A_1\cap A_2 \in \Sigma$
\item $A_1 \in \Sigma$ implies that $A_1^c = \bigcup_{i=1}^n B_i$ for $B_i \in \Sigma$.
\end{itemize}
\end{definition}
Note:  The set of intervals $\{(a, b): a, b \in \R\}$ is not a semi-algebra.  If $(a, b)^c = [b, \infty)$ which is not finitely coverable by disjoint open sets. Similarly, $\{[a, b]: a, b \in \R\}$ is not a semi-algebra.  

Claim: $\Sigma = \{(a, b]: a, b \in \R\}$ is a semi-algebra.  [This is left as an exercise].

Now, $\mu((a, b]) = b-a$. The proof that $\mu$ is countable additive on $\Sigma$.  If $A = \bigcup_{i=1}^{\infty} B_i$, $B_i$ disjoint, $A, B_i \in \Sigma$, then $\mu(A) = \sum_{i=1}^{\infty} \mu(B_i).$
\begin{proof}
We first show that $\mu(A) \ge \sum_{i=1}^{\infty} \mu(B_i)$.  This is an easy exercise, show $\mu(A) \ge \sum_{i=1}^{n}\mu(B_i)$, and we pass to the limit.

It suffices to show $\mu(A) \le \sum_{i=1}^{\infty} \mu(B_i)$.  We do this by exploiting compactness.

Let $A = (a, b] \supset [a+1/n, b] = A;$, take $B_i = (c_i, d] \subset c_i, d + \frac{\epsilon}{2^i} = B_i'$.  Note that $$A' \subset \bigcup_{i=1}^{\infty}B_i',$$
so there exists a finite subcover $A' \subset \bigcup_{j=1}^{k} B_{i_j}'.$  It is easy to show that $b - (a+1/n) \le \sum_{j=1}^{k} (d_{i_j}' - c_{i_j}')$. But note that 
$$\sum_{j=1}^{k} (d_{i_j}' - c_{i_j}') \le \sum_{j=1}^k d_{i_j} - c_{i_j} + \epsilon,$$
which implies that $$\mu(A) - 1/n \le \sum_{i=1}^{\infty}\mu(B_i) + \epsilon \Rightarrow \mu(A) \le \sum_{i=1}^{\infty} \mu(B_i).$$ 
\end{proof}
\begin{definition} $\SA$ is an algebra if 
\begin{itemize}
\item $\emptyset \in \SA$
\item $A_1 \in \SA$ implies $A^c \in \SA$
\item $A_1, \dots, A_n \in \SA$, then $\bigcup_{i=1}^n A_i \in A$.
\end{itemize}
The algebra generated by a semi-algebra is given by taking all possible disjoint finite unions.
\end{definition}
Claim: $\Sigma_a = \{\bigcup_{i=1}^n A_i\}$ for disjoint $A_i$ semialgebras is an algebra.
\begin{proof}
We show $A, B \in \Sigma_a \Rightarrow A\cup B \in \Sigma_a$ and $A^c \in \Sigma_a$.  
Note that $A = \bigcup_{i=1}^n C_i, B = \bigcup_{j=1}^k D_j$, so $$A \cap B = \bigcup_{i=1}^n \bigcup_{j=1}^k C_i \cap D_j,$$
and $C_i \cap D_j $ are disjoint.  Then $C_i, D_j \in \Sigma$ implies $C_i \cap D_j \in \Sigma$.

Then, if $A = \bigcup_{i=1}^k C_i$, then $A^c = \bigcap_{i=1}^k C_i^c$, and $C_i^c = \bigcup_{j=1}^\ell E_j \in \Sigma_a$.
\end{proof}

We extend $\mu$ to an algebra by $\mu(A) = \sum_{i=1}^k \mu(C_i)$, where $A = \bigcup C_i$ in the semi-algebra.
\end{document}

\documentclass[11pt]{scrartcl}
\usepackage[sexy]{evan}
\usepackage{graphicx}

\usepackage{answers}
\usepackage{listings}
\usepackage{xcolor}
\Newassociation{hint}{hintitem}{all-hints}
\renewcommand{\solutionextension}{out}
\renewenvironment{hintitem}[1]{\item[\bfseries #1.]}{}
\declaretheorem[style=thmbluebox,name={Theorem}]{thm}

\lstset{
    frame=tb, % draw a frame at the top and bottom of the code block
    tabsize=4, % tab space width
    showstringspaces=false, % don't mark spaces in strings
    numbers=left, % display line numbers on the left
    commentstyle=\color{green}, % comment color
    keywordstyle=\color{blue}, % keyword color
    stringstyle=\color{red} % string color
}

\begin{document}
\title{CS 170}
\author{Vishal Raman}
\thispagestyle{empty}
$ $
\vfill
\begin{center}

\centerline{\huge \textbf{CS 170 Lecture Notes, Fall 2020}}
\centerline{\Large \textbf{Algorithms and Intractable Problems} } 
\centerline{Professor: Avishay Tal, Umesh Vazirani}
\centerline{Vishal Raman}
\end{center}
\vfill
$ $
\newpage
\thispagestyle{empty}
\tableofcontents
\newpage
%\maketitle
\section{August 27th, 2020}
\subsection{Example: Fibonacci Numbers} 
Consider the sequence $0, 1, 1, 2, 3, 5, 8, ...$ defined by $F_0 = 0, F_1 = 1, F_n = F_{n-1} + F_{n-2}$ for $n \ge 2$.  We might write a function for calculating the $n$-th Fibonacci number as follows:
\begin{lstlisting}[language=C++, caption={N-th Fibonacci Number}]
int fib(int n)
    if (n == 0) return 0;
    if (n == 1) return 1;
    else return fib(n-1) + fib(n-2);
\end{lstlisting}

\end{document}
